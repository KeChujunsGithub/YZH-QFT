\section{作用量原理}

%%%%%%%%%%%%%%%%%%%%%%%%%%%%%%%%%%%%%%%%%%%%%%%%%%%%%%%%%%%%%%%%%%%%%%%%%%%%%%%%%%%%%%%%%%%%%%%%%%%%%%%%%%
\subsection{推导:的运动方程}
1.
对于
\begin{equation}
     S=\int_{t_1}^{t_2}{\mathrm{d}t}L
\end{equation}
两边同时变分
\begin{equation}
     \delta S=\int_{t_1}^{t_2}{\mathrm{d}t}\delta L
\end{equation}
2.对于拉氏量
\begin{equation}
    L=L[q_i(t),\dot{q}_i(t)]
\end{equation}
变分计算为
\begin{equation}
     \delta L=\frac{\partial L}{\partial q_i}\delta q_i+\frac{\partial L}{\partial \dot{q}_i}\delta \dot{q}_i
\end{equation}
得到
\begin{equation}
    \delta S=\int_{t_1}^{t_2}{\mathrm{d}t}\left( \frac{\partial L}{\partial q_i}\delta q_i+\frac{\partial L}{\partial \dot{q}_i}\delta \dot{q}_i \right) 
\end{equation}
3.因为求导和变分可交换
\begin{equation}
     \delta \dot{q}_i=\delta \left( \frac{\mathrm{d}}{\mathrm{d}t}q_i \right) =\frac{\mathrm{d}}{\mathrm{d}t}\left( \delta q_i \right) 
\end{equation}
得到
\begin{equation}
     \begin{aligned}
         \delta S&=\int_{t_1}^{t_2}{\mathrm{d}t}\left( \frac{\partial L}{\partial q_i}\delta q_i+\frac{\partial L}{\partial \dot{q}_i}\delta \left( \frac{\mathrm{d}}{\mathrm{d}t}q_i \right) \right) 
\\
&=\int_{t_1}^{t_2}{\mathrm{d}t}\left( \frac{\partial L}{\partial q_i}\delta q_i+\frac{\partial L}{\partial \dot{q}_i}\frac{\mathrm{d}}{\mathrm{d}t}\left( \delta q_i \right) \right) 
     \end{aligned}
\end{equation}
4.求导法则
\begin{equation}
    \frac{\mathrm{d}}{\mathrm{d}t}\left( \frac{\partial L}{\partial \dot{q}_i}\delta q_i \right) =\left( \frac{\mathrm{d}}{\mathrm{d}t}\frac{\partial L}{\partial \dot{q}_i} \right) \delta q_i+\frac{\partial L}{\partial \dot{q}_i}\frac{\mathrm{d}}{\mathrm{d}t}\left( \delta q_i \right) 
\end{equation}
写出
\begin{equation}
    \frac{\partial L}{\partial \dot{q}_i}\frac{\mathrm{d}}{\mathrm{d}t}\left( \delta q_i \right) =\frac{\mathrm{d}}{\mathrm{d}t}\left( \frac{\partial L}{\partial \dot{q}_i}\delta q_i \right) -\left( \frac{\mathrm{d}}{\mathrm{d}t}\frac{\partial L}{\partial \dot{q}_i} \right) \delta q_i
\end{equation}
得到
\begin{equation}
    \begin{aligned}
        \delta S&=\int_{t_1}^{t_2}{\mathrm{d}t}\left[ \frac{\partial L}{\partial q_i}\delta q_i+\frac{\mathrm{d}}{\mathrm{d}t}\left( \frac{\partial L}{\partial \dot{q}_i}\delta q_i \right) -\left( \frac{\mathrm{d}}{\mathrm{d}t}\frac{\partial L}{\partial \dot{q}_i} \right) \delta q_i \right] 
\\
&=\int_{t_1}^{t_2}{\mathrm{d}t}\left( \frac{\partial L}{\partial q_i}-\frac{\mathrm{d}}{\mathrm{d}t}\frac{\partial L}{\partial \dot{q}_i} \right) \delta q_i+\int_{t_1}^{t_2}{\mathrm{d}t}\frac{\mathrm{d}}{\mathrm{d}t}\left( \frac{\partial L}{\partial \dot{q}_i}\delta q_i \right) 
    \end{aligned}
\end{equation}
5.牛顿莱布尼兹法则
\begin{equation}
    \int_{t_1}^{t_2}{\mathrm{d}t}\frac{\mathrm{d}}{\mathrm{d}t}\left( \frac{\partial L}{\partial \dot{q}_i}\delta q_i \right) =\int_{t_1}^{t_2}{\mathrm{d}\left( \frac{\partial L}{\partial \dot{q}_i}\delta q_i \right)}=\frac{\partial L}{\partial \dot{q}_i}\delta q_i|_{t_1}^{t_2}
\end{equation}
得到
\begin{equation}
    \delta S=\int_{t_1}^{t_2}{\mathrm{d}t}\left( \frac{\partial L}{\partial q_i}-\frac{\mathrm{d}}{\mathrm{d}t}\frac{\partial L}{\partial \dot{q}_i} \right) \delta q_i+\frac{\partial L}{\partial \dot{q}_i}\delta q_i|_{t_1}^{t_2}
\end{equation}
6.
\begin{equation}
    \delta q_i(t_1)=\delta q_i(t_2)=0
\end{equation}
写出
\begin{equation}
    \frac{\partial L}{\partial \dot{q}_i}\delta q_i|_{t_1}^{t_2}=\frac{\partial L}{\partial \dot{q}_i}\delta q_i(t_2)-\frac{\partial L}{\partial \dot{q}_i}\delta q_i(t_1)=0
\end{equation}
得到
\begin{equation}
    \delta S=\int_{t_1}^{t_2}{\mathrm{d}t}\left( \frac{\partial L}{\partial q_i}-\frac{\mathrm{d}}{\mathrm{d}t}\frac{\partial L}{\partial \dot{q}_i} \right) \delta q_i
\end{equation}
7.利用最小作用量原理
\begin{equation}
    \delta S=0
\end{equation}
等式右边积分为零
\begin{equation}
    \int_{t_1}^{t_2}{\mathrm{d}t}\left( \frac{\partial L}{\partial q_i}-\frac{\mathrm{d}}{\mathrm{d}t}\frac{\partial L}{\partial \dot{q}_i} \right) \delta q_i=0
\end{equation}
被积函数为零
\begin{equation}
    \frac{\partial L}{\partial q_i}-\frac{\mathrm{d}}{\mathrm{d}t}\frac{\partial L}{\partial \dot{q}_i}=0
\end{equation}

\newpage
\subsection{推导:场的运动方程}
1.
对于
\begin{equation}
    S=\int{\mathrm{d}^4x}\mathcal{L} 
\end{equation}
两边同时变分
\begin{equation}
    \delta S=\int{\mathrm{d}^4x}\delta \mathcal{L} 
\end{equation}
2.对于拉氏量
\begin{equation}
    \mathcal{L} =\mathcal{L} (\Phi _a,\partial _{\mu}\Phi _a)
\end{equation}
变分计算为
\begin{equation}
    \delta \mathcal{L} =\frac{\partial \mathcal{L}}{\partial \Phi _a}\delta \Phi _a+\frac{\partial \mathcal{L}}{\partial \left( \partial _{\mu}\Phi _a \right)}\delta \left( \partial _{\mu}\Phi _a \right) 
\end{equation}
得到
\begin{equation}
    \delta S=\int{\mathrm{d}^4x\left[ \frac{\partial \mathcal{L}}{\partial \Phi _a}\delta \Phi _a+\frac{\partial \mathcal{L}}{\partial \left( \partial _{\mu}\Phi _a \right)}\delta \left( \partial _{\mu}\Phi _a \right) \right]}
\end{equation}
3.因为可交换
\begin{equation}
    \delta \left( \partial _{\mu}\Phi _a \right) =\partial _{\mu}\left( \delta \Phi _a \right) 
\end{equation}
得到
\begin{equation}
    \delta S=\int{\mathrm{d}^4x}\left[ \frac{\partial \mathcal{L}}{\partial \Phi _a}\delta \Phi _a+\frac{\partial \mathcal{L}}{\partial \left( \partial _{\mu}\Phi _a \right)}\partial _{\mu}\left( \delta \Phi _a \right) \right] 
\end{equation}
4.求导法则
\begin{equation}
    \partial _{\mu}\left( \frac{\partial \mathcal{L}}{\partial \left( \partial _{\mu}\Phi _a \right)}\delta \Phi _a \right) =\frac{\partial \mathcal{L}}{\partial \left( \partial _{\mu}\Phi _a \right)}\partial _{\mu}\left( \delta \Phi _a \right) +\left( \partial _{\mu}\frac{\partial \mathcal{L}}{\partial \left( \partial _{\mu}\Phi _a \right)} \right) \delta \Phi _a
\end{equation}
写出
\begin{equation}
    \frac{\partial \mathcal{L}}{\partial \left( \partial _{\mu}\Phi _a \right)}\partial _{\mu}\left( \delta \Phi _a \right) =\partial _{\mu}\left( \frac{\partial \mathcal{L}}{\partial \left( \partial _{\mu}\Phi _a \right)}\delta \Phi _a \right) -\left( \partial _{\mu}\frac{\partial \mathcal{L}}{\partial \left( \partial _{\mu}\Phi _a \right)} \right) \delta \Phi _a
\end{equation}
得到
\begin{equation}
    \begin{aligned}
        \delta S&=\int{\mathrm{d}^4x}\left[ \frac{\partial \mathcal{L}}{\partial \Phi _a}\delta \Phi _a+\partial _{\mu}\left( \frac{\partial \mathcal{L}}{\partial \left( \partial _{\mu}\Phi _a \right)}\delta \Phi _a \right) -\left( \partial _{\mu}\frac{\partial \mathcal{L}}{\partial \left( \partial _{\mu}\Phi _a \right)} \right) \delta \Phi _a \right] 
\\
&=\int{\mathrm{d}^4x}\left[ \frac{\partial \mathcal{L}}{\partial \Phi _a}-\partial _{\mu}\frac{\partial \mathcal{L}}{\partial \left( \partial _{\mu}\Phi _a \right)} \right] \delta \Phi _a+\int{\mathrm{d}^4x}\partial _{\mu}\left[ \frac{\partial \mathcal{L}}{\partial \left( \partial _{\mu}\Phi _a \right)}\delta \Phi _a \right] 
    \end{aligned}
\end{equation}
5.广义斯托克斯,将四维体积转化为三维面积
\begin{equation}
    \int_{\mathcal{V}}{\mathrm{d}^4x}\partial _{\mu}\left[ \frac{\partial \mathcal{L}}{\partial \left( \partial _{\mu}\Phi _a \right)}\delta \Phi _a \right] =\int_{\mathcal{S}}{\mathrm{d}\sigma _{\mu}}\frac{\partial \mathcal{L}}{\partial \left( \partial _{\mu}\Phi _a \right)}\delta \Phi _a
\end{equation}
得到
\begin{equation}
    \delta S=\int{\mathrm{d}^4x}\left[ \frac{\partial \mathcal{L}}{\partial \Phi _a}-\partial _{\mu}\frac{\partial \mathcal{L}}{\partial \left( \partial _{\mu}\Phi _a \right)} \right] \delta \Phi _a+\int_{\mathcal{S}}{\mathrm{d}\sigma _{\mu}}\frac{\partial \mathcal{L}}{\partial \left( \partial _{\mu}\Phi _a \right)}\delta \Phi _a
\end{equation}
6.全空间积分为0
\begin{equation}
    \int_{\mathcal{S}}{\mathrm{d}\sigma _{\mu}}\frac{\partial \mathcal{L}}{\partial \left( \partial _{\mu}\Phi _a \right)}\delta \Phi _a=0
\end{equation}
得到
\begin{equation}
    \delta S=\int{\mathrm{d}^4x}\left[ \frac{\partial \mathcal{L}}{\partial \Phi _a}-\partial _{\mu}\frac{\partial \mathcal{L}}{\partial \left( \partial _{\mu}\Phi _a \right)} \right] \delta \Phi _a
\end{equation}
7.利用最小作用量原理
\begin{equation}
    \delta S=0
\end{equation}
等式右边积分为零
\begin{equation}
    \int{\mathrm{d}^4x}\left[ \frac{\partial \mathcal{L}}{\partial \Phi _a}-\partial _{\mu}\frac{\partial \mathcal{L}}{\partial \left( \partial _{\mu}\Phi _a \right)} \right] \delta \Phi _a=0
\end{equation}
被积函数为零
\begin{equation}
    \frac{\partial \mathcal{L}}{\partial \Phi _a}-\partial _{\mu}\frac{\partial \mathcal{L}}{\partial \left( \partial _{\mu}\Phi _a \right)}=0
\end{equation}


第二种写法是分部积分











%%%%%%%%%%%%%%%%%%%%%%%%%%%%%%%%%%%%%%%%%%%%%%%%%%%%%5
\subsection{}

















