\documentclass[a4paper,12pt]{article}
\usepackage[utf8]{inputenc}
\usepackage{xeCJK}  % 中文支持
\usepackage{amsmath,amsthm,amssymb}  % 数学包
\usepackage{graphicx}  % 插入图片
\usepackage{booktabs}  % 美化表格
\usepackage{hyperref}  % 超链接
\usepackage{enumitem}  % 自定义列表
\usepackage[left=2cm,right=2cm,top=2cm,bottom=2cm]{geometry}  % 页面设置
\usepackage{multirow}
\usepackage{makecell} %优化单元格内内容的排版,支持自动换行、对齐和文本格式化。
\usepackage{array}

\usepackage{booktabs}   % 专业表格线
\usepackage{braket}      % Dirac 符号支持
\usepackage{caption}     % 优化标题样式

\usepackage{simplewick}
\usepackage{simpler-wick} %余钊焕,依赖tikz,可能有点重

\usepackage{slashed}

\setcounter{section}{-1} % 设置节从 0 开始

 
\title{余钊焕-量子场论}
\author{余钊焕}
\date{\today}
 
\begin{document}
 
\maketitle
 
\tableofcontents  % 自动生成目录
\section{备忘录}

群的线性表示理论
标量场-恒定表示
矢量场-矢量表示
旋量场-旋量表示



\subsection{协变、逆变}
1.四维坐标
\begin{equation}
    \begin{aligned}
        x^{\mu}&=\left( x^0,x^i \right) =\left( x^0,x^1,x^2,x^3 \right) 
\\
&=\left( t,\mathbf{x} \right) =\left( t,x,y,z \right) =\left( ct,x,y,z \right) 
\\
x_{\mu}&=\left( x_0,x_i \right) =\left( x_0,x_1,x_2,x_3 \right) {\color{gray} =\left( x^0,-x^i \right) =\left( x^0,-x^1,-x^2,-x^3 \right) }
\\
&=\left( t,{\color[RGB]{240, 0, 0} -}\mathbf{x} \right) =\left( t,{\color[RGB]{240, 0, 0} -}x,{\color[RGB]{240, 0, 0} -}y,{\color[RGB]{240, 0, 0} -}z \right) =\left( ct,{\color[RGB]{240, 0, 0} -}x,{\color[RGB]{240, 0, 0} -}y,{\color[RGB]{240, 0, 0} -}z \right) 
    \end{aligned}
\end{equation}
2.四维动量
\begin{equation}
    \begin{aligned}
        p^{\mu}&=\left( p^0,p^i \right) =\left( p^0,p^1,p^2,p^3 \right) 
\\
&=\left( E,\mathbf{p} \right) =\left( E,p_x,p_y,p_z \right) =\left( \frac{E}{c},p_x,p_y,p_z \right) 
\\
p_{\mu}&=\left( p_0,p_i \right) =\left( p_0,p_1,p_2,p_3 \right) {\color{gray} =\left( p^0,-p^i \right) =\left( p^0,-p^1,-p^2,-p^3 \right) }
\\
&=\left( E,{\color[RGB]{240, 0, 0} -}\mathbf{p} \right) =\left( E,{\color[RGB]{240, 0, 0} -}p_x,{\color[RGB]{240, 0, 0} -}p_y,{\color[RGB]{240, 0, 0} -}p_z \right) =\left( \frac{E}{c},{\color[RGB]{240, 0, 0} -}p_x,{\color[RGB]{240, 0, 0} -}p_y,{\color[RGB]{240, 0, 0} -}p_z \right) 
    \end{aligned}
\end{equation}
3.四维电磁矢量
\begin{equation}
    \begin{aligned}
        A^{\mu}&=\left( A^0,A^i \right) =\left( A^0,A^1,A^2,A^3 \right) 
\\
&=\left( V,\mathbf{A} \right) =\left( V,A_x,A_y,A_y \right) 
\\
A_{\mu}&=\left( A_0,A_i \right) =\left( A_0,A_1,A_2,A_3 \right) {\color{gray} =\left( A^0,-A^i \right) =\left( A^0,-A^1,-A^2,-A^3 \right) }
\\
&=\left( V,{\color[RGB]{240, 0, 0} -}\mathbf{A} \right) =\left( V,{\color[RGB]{240, 0, 0} -}A_x,{\color[RGB]{240, 0, 0} -}A_y,{\color[RGB]{240, 0, 0} -}A_z \right) 
    \end{aligned}
\end{equation}
4.四维流
\begin{equation}
    \begin{aligned}
        j^{\mu}&=\left( j^0,j^i \right) =\left( j^0,j^1,j^2,j^3 \right) 
\\
&=\left( \rho ,\mathbf{j} \right) =\left( \rho ,j_x,j_y,j_z \right) 
\\
j_{\mu}&=\left( j_0,j_i \right) =\left( j_0,j_1,j_2,j_3 \right) {\color{gray} =\left( j^0,-j^i \right) =\left( j^0,-j^1,-j^2,-j^3 \right) }
\\
&=\left( \rho ,{\color[RGB]{240, 0, 0} -}\mathbf{j} \right) =\left( \rho ,{\color[RGB]{240, 0, 0} -}j_x,{\color[RGB]{240, 0, 0} -}j_y,{\color[RGB]{240, 0, 0} -}j_z \right) 
    \end{aligned}
\end{equation}
5.时空坐标求导
\begin{equation}
    \begin{aligned}
        \partial _{\mu}&=\frac{\partial}{\partial x^{\mu}}
\\
\partial _{\mu}&=\left( \partial _0,\partial _i \right) =\left( \partial _0,\partial _1,\partial _2,\partial _3 \right) 
\\
\frac{\partial}{\partial x^{\mu}}&=\left( \frac{\partial}{\partial x^0},\frac{\partial}{\partial x^i} \right) =\left( \frac{\partial}{\partial x^0},\frac{\partial}{\partial x^1},\frac{\partial}{\partial x^2},\frac{\partial}{\partial x^3} \right) 
\\
&=\left( \frac{\partial}{\partial t},\nabla \right) =\left( \frac{\partial}{\partial t},\frac{\partial}{\partial x},\frac{\partial}{\partial y},\frac{\partial}{\partial z} \right) =\left( \frac{1}{c}\frac{\partial}{\partial t},\nabla \right) =\left( \frac{1}{c}\frac{\partial}{\partial t},\frac{\partial}{\partial x},\frac{\partial}{\partial y},\frac{\partial}{\partial z} \right) 
\\
\partial ^{\mu}&=\frac{\partial}{\partial x_{\mu}}
\\
\partial ^{\mu}&=\left( \partial ^0,\partial ^i \right) =(\partial ^0,\partial ^1,\partial ^2,\partial ^3)
\\
\frac{\partial}{\partial x_{\mu}}&=\left( \frac{\partial}{\partial x_0},\frac{\partial}{\partial x_i} \right) =\left( \frac{\partial}{\partial x_0},\frac{\partial}{\partial x_1},\frac{\partial}{\partial x_2},\frac{\partial}{\partial x_3} \right) {\color{gray} =\left( \frac{\partial}{\partial x^0},-\frac{\partial}{\partial x^i} \right) =\left( \frac{\partial}{\partial x^0},-\frac{\partial}{\partial x^1},-\frac{\partial}{\partial x^2},-\frac{\partial}{\partial x^3} \right) }
\\
&=\left( \frac{\partial}{\partial t},{\color[RGB]{240, 0, 0} -}\nabla \right) =\left( \frac{\partial}{\partial t},{\color[RGB]{240, 0, 0} -}\frac{\partial}{\partial x},{\color[RGB]{240, 0, 0} -}\frac{\partial}{\partial y},{\color[RGB]{240, 0, 0} -}\frac{\partial}{\partial z} \right) =\left( \frac{1}{c}\frac{\partial}{\partial t},{\color[RGB]{240, 0, 0} -}\nabla \right) =\left( \frac{1}{c}\frac{\partial}{\partial t},{\color[RGB]{240, 0, 0} -}\frac{\partial}{\partial x},{\color[RGB]{240, 0, 0} -}\frac{\partial}{\partial y},{\color[RGB]{240, 0, 0} -}\frac{\partial}{\partial z} \right) 
    \end{aligned}
\end{equation}


\subsection{色散关系}
\begin{equation}
    \begin{aligned}
        p^0=E_{\mathbf{p}}&=\sqrt{\left| \mathbf{p} \right|^2+m^2}
\\
&=\sqrt{\mathbf{p}^2+m^2}
    \end{aligned}
\end{equation}


\subsection{质壳条件}
\begin{equation}
    \begin{aligned}
        p^2&=E^2-\left| \mathbf{p} \right|^2=m^2
\\
&=E^2-\mathbf{p}^2
    \end{aligned}
\end{equation}

\subsection{Maxwell 方程}




\section{预备知识}

量子场论与粒子物理密切相关。

粒子物理研究物质的基本结构和基本相互作用,组成物质的基本单元是粒子。

自然界中存在 4 种基本相互作用,
即引力相互作用 (gravitational interaction)、
电磁相互作用 (electromagnetic interaction) 、 
强相互作用 (strong interaction) 、
弱相互作用 (weak interaction),
支配着基本粒子的运动和转化。

基本粒子指尚未发现内部结构的粒子。
目前已发现 3 代基本费米子 ,
每一代包含带电轻子、中微子(中性轻子)、下型夸克、上型夸克各一种。
\\第1代基本费米子包括电子$(e)$、电子型中微子$(\nu_e)$、下夸克$(d)$和上夸克$(u)$ ;
\\第2代包括$\mu$子$(\mu)$、$\mu$子型中微子$(\nu_\mu)$、奇夸克$(s)$和粲夸克$(c)$;
\\第3代包括$\tau$子$(\tau)$、$\tau$子型中微子$(\nu_\tau)$、 底夸克$(b)$和顶夸克$(t)$。
\\某代某种费米子与它在另一代中相对应的费米子具有相同的量子数,但质量不同。(表格中费米子部分同一列)

夸克的种类也称为味道 (flavor), 6 种味道的夸克$d$、$u$、$s$、$c$、$b$、$t$具有不同的质量。
\\每一味夸克都具有 3 种颜色 (color),
\\同味异色的夸克具有相同质量,严格构成颜色三重态,与描述强相互作用的量子色动力学有关。

多个夸克可以通过强相互作用组成复合粒子、称为强子(hadron),比如,
\\1个介子(meson)由1个正夸克和1个反夸克组成,
\\一个重子(baryon)由3个正夸克或3个反夸克组成。
\\除 3 代中微子以外,其它基本费米子都具有电荷 (electric charge) , 参与电磁相互作用,相应的理论称为量子电动力学。
\\所有基本费米子都参与弱相互作用,它与电磁相互作用统一由电弱规范理论描述,这个理论包含了量子电动力学。电弱规范理论和量子色动力学统称为标准模型 (standard model)。

在标准模型中,费米子的相互作用由一些基本玻色子 (boson) 传递。
\\传递夸克间强相互作用的规范玻色子称为胶子 (gluon),
\\传递电磁相互作用的规范玻色子是光子 (photon),
\\传递弱相互作用的规范玻色子是$W^\pm$和$Z^0$玻色子。
\\此外,还存在一种 Higgs 玻色子,它与电弱规范对称性的自发破缺及基本粒子的质量起源有关。

标准模型是在研究基本粒子如何参与电磁、强、弱相互作用的过程中建立起来的,它的理论
基础就是量子场论。在粒子物理理论和实验发展的过程中,量子场论起着极为关键的作用。反
过来,粒子物理研究也极大地促进了量子场论的发展。

\begin{table}[ht]
\centering
\caption{标准模型粒子分类表}
\begin{tabular}{|l|c|c|c|c|c|}
\hline
 & \textbf{(第一代)费米子} & \textbf{(第二代)费米子} & \textbf{(第三代)费米子} & \textbf{规范玻色子} & \textbf{标量玻色子} \\
\hline
\multirow{4}{*}{\textbf{上型夸克}} 
& \makecell{u夸克 \\ 2.2\,MeV \\ +2/3 \\ 1/2} 
& \makecell{c夸克 \\ 1.27\,GeV \\ +2/3 \\ 1/2} 
& \makecell{t夸克 \\ 173\,GeV \\ +2/3 \\ 1/2} 
& \makecell{光子 ($\gamma$) \\ 0 \\ 0 \\ 1} 
& \multirow{-4}{*}{\makecell{希格斯玻色子 ($H^0$) \\ 125\,GeV \\ 0 \\ 0}} \\
\cline{2-5}

\multirow{4}{*}{\textbf{下型夸克}} 
& \makecell{d夸克 \\ 4.7\,MeV \\ -1/3 \\ 1/2} 
& \makecell{s夸克 \\ 95\,MeV \\ -1/3 \\ 1/2} 
& \makecell{b夸克 \\ 4.18\,GeV \\ -1/3 \\ 1/2} 
& \makecell{$W^\pm$玻色子 \\ 80.4\,GeV \\ $\pm$1 \\ 1} 
& \\ \cline{2-5}

\multirow{4}{*}{\textbf{带电轻子}} 
& \makecell{电子 ($e^-$) \\ 0.511\,MeV \\ -1 \\ 1/2} 
& \makecell{$\mu$子 ($\mu^-$) \\ 105.7\,MeV \\ -1 \\ 1/2} 
& \makecell{$\tau$子 ($\tau^-$) \\ 1.777\,GeV \\ -1 \\ 1/2} 
& \makecell{$Z$玻色子 \\ 91.2\,GeV \\ 0 \\ 1} 
& \\ \cline{2-5}

\multirow{4}{*}{\textbf{中微子}\\ \textbf{中性轻子}} 
& \makecell{$\nu_e$电子中微子 \\ <1\,eV \\ 0 \\ 1/2} 
& \makecell{$\nu_\mu$$\mu$子中微子 \\ <0.1\,eV \\ 0 \\ 1/2} 
& \makecell{$\nu_\tau$$\tau$子中微子 \\ <15.5\,MeV \\ 0 \\ 1/2} 
& \makecell{胶子 (8种) \\ 0 \\ 0 \\ 1} 
& \\ \hline
\end{tabular}
\end{table}

%%%%%%%%%%%%%%%%%%%%%%%%%%%%%%%%%%%%%%%%%%%%%%%%%%%%%5
\subsection{}












%%%%%%%%%%%%%%%%%%%%%%%%%%%%%%%%%%%%%%%%%%%%%%%%%%%%%5
\subsection{}















%%%%%%%%%%%%%%%%%%%%%%%%%%%%%%%%%%%%%%%%%%%%%%%%%%%%%5
\subsection{}












%%%%%%%%%%%%%%%%%%%%%%%%%%%%%%%%%%%%%%%%%%%%%%%%%%%%%5
\subsection{}














%%%%%%%%%%%%%%%%%%%%%%%%%%%%%%%%%%%%%%%%%%%%%%%%%%%%%5
\subsection{}














%%%%%%%%%%%%%%%%%%%%%%%%%%%%%%%%%%%%%%%%%%%%%%%%%%%%%5
\subsection{ Lorentz 矢量}







%%%%%%%%%%%%%%%%%%%%%%%%%%%%%%%%%%%%%%%%%%%%%%%%%%%%%5
\subsection{Lorentz 张量}





















总结:自然单位制下的麦克斯韦方程组
\begin{equation}
    \begin{aligned}
        \nabla \cdot \mathbf{E}&=\rho 
\\
\nabla \times \mathbf{B}&=\mathbf{J}+\frac{\partial \mathbf{E}}{\partial t}
\\
\nabla \cdot \mathbf{B}&=0
\\
\nabla \times \mathbf{E}&=-\frac{\partial \mathbf{B}}{\partial t}
    \end{aligned}
\end{equation}

Gauss 定律
\begin{equation}
    \nabla \cdot \mathbf{E}=\rho 
\end{equation}
Ampère 环路定律对应的Ampère-Maxwell 方程
\begin{equation}
    \nabla \times \mathbf{B}=\mathbf{J}+\frac{\partial \mathbf{E}}{\partial t}
\end{equation}
Gauss 磁定律
\begin{equation}
    \nabla \cdot \mathbf{B}=0
\end{equation}
Faraday 电磁感应定律对应的 MaxwellFaraday 方程
\begin{equation}
    \nabla \cdot \mathbf{B}=0
\end{equation}

补充:推导
1.1推导
\begin{equation}
    \nabla \cdot \mathbf{E}=\rho \Rightarrow \partial _{\mu}F^{\mu 0}=J^0
\end{equation}
过程
\begin{equation}
    \begin{aligned}
        \nabla \cdot \mathbf{E}&=\rho 
\\
\partial _iE^i&=J^0
\\
-\partial _iF^{0i}&=J^0
\\
\partial _iF^{i0}&=J^0
\\
\partial _0F^{00}+\partial _iF^{i0}&=J^0
\\
\partial _{\mu}F^{\mu 0}&=J^0
    \end{aligned}
\end{equation}
其中






2.1推导
\begin{equation}
    \nabla \times \mathbf{B}=\mathbf{J}+\frac{\partial \mathbf{E}}{\partial t}\Rightarrow \partial _{\mu}F^{\mu i}=J^i
\end{equation}
过程
\begin{equation}
    \begin{aligned}
        \nabla \times \mathbf{B}&=\mathbf{J}+\frac{\partial \mathbf{E}}{\partial t}
\\
\varepsilon ^{ijk}\partial _jB^k&=J^i+\partial _0E^i
\\
\partial _j\varepsilon ^{ijk}B^k&=J^i+\partial _0E^i
\\
-\partial _jF^{ij}&=J^i-\partial _0F^{0i}
\\
\partial _0F^{0i}-\partial _jF^{ij}&=J^i
\\
\partial _0F^{0i}+\partial _jF^{ji}&=J^i
\\
\partial _{\mu}F^{\mu i}&=J^i
$$

    \end{aligned}
\end{equation}




3.1推导
\begin{equation}
    
\end{equation}
过程
\begin{equation}
    \begin{aligned}
        \nabla \times \mathbf{E}&=-\frac{\partial \mathbf{B}}{\partial t}
\\
\varepsilon ^{kmn}\partial _mE^n&=-\partial _0B^k
\\
-\varepsilon ^{kmn}\partial _mF^{0n}&=\frac{1}{2}\varepsilon ^{kmn}\partial _0F^{mn}
\\
\varepsilon ^{kmn}\partial _mF^{0n}+\frac{1}{2}\varepsilon ^{kmn}\partial _0F^{mn}&=0
\\
\varepsilon ^{kij}\varepsilon ^{kmn}\left( \partial _mF^{0n}+\frac{1}{2}\partial _0F^{mn} \right) &=0
\\
\left( \delta ^{im}\delta ^{jn}-\delta ^{in}\delta ^{jm} \right) \left( \partial _mF^{0n}+\frac{1}{2}\partial _0F^{mn} \right) &=0
\\
\delta ^{im}\delta ^{jn}\partial _mF^{0n}-\delta ^{in}\delta ^{jm}\partial _mF^{0n}+\frac{1}{2}\left( \delta ^{im}\delta ^{jn}\partial _0F^{mn}-\delta ^{in}\delta ^{jm}\partial _0F^{mn} \right) &=0
\\
\partial _iF^{0j}-\partial _jF^{0i}+\frac{1}{2}\left( \partial _0F^{ij}-\partial _0F^{ji} \right) &=0
\\
-\partial _iF^{j0}-\partial _jF^{0i}+\frac{1}{2}\left( \partial _0F^{ij}+\partial _0F^{ij} \right) &=0
\\
\partial _0F^{ij}-\partial _iF^{j0}-\partial _jF^{0i}&=0
\\
\partial ^0F^{ij}+\partial ^iF^{j0}+\partial ^jF^{0i}&=0
    \end{aligned}
\end{equation}







4.1推导
\begin{equation}
    \nabla \times \mathbf{E}=-\frac{\partial \mathbf{B}}{\partial t}\Rightarrow \partial ^0F^{ij}+\partial ^iF^{j0}+\partial ^jF^{0i}=0
\end{equation}

\begin{equation}
    \begin{aligned}
        
    \end{aligned}
\end{equation}



1.2推导
\begin{equation}
    \partial _{\mu}F^{\mu 0}=J^0\Rightarrow \nabla \cdot \mathbf{E}=\rho 
\end{equation}
过程
\begin{equation}
    \begin{aligned}
        \partial _{\mu}F^{\mu 0}&=J^0
\\
\,\partial _0F^{00}+\partial _iF^{i0}&=J^0
\\
\partial _iF^{i0}&=J^0
\\
-\partial _iF^{0i}&=J^0
\\
\partial _iE^i&=J^0
\\
\nabla \cdot \mathbf{E}&=\rho 
    \end{aligned}
\end{equation}
其中
\begin{equation}
    \begin{aligned}
        F^{00}&=0
\\
\partial _0F^{00}&=0
    \end{aligned}
\end{equation}




2.2推导
\begin{equation}
    \partial _{\mu}F^{\mu i}=J^i\Rightarrow \nabla \times \mathbf{B}=\mathbf{J}+\frac{\partial \mathbf{E}}{\partial t}
\end{equation}
过程
\begin{equation}
    \begin{aligned}
        \partial _{\mu}F^{\mu i}&=J^i
\\
\partial _0F^{0i}+\partial _jF^{ji}&=J^i
\\
-\partial _0E^i-\partial _j\varepsilon ^{jik}B^k&=J^i
\\
-\partial _0E^i-\varepsilon ^{jik}\partial _jB^k&=J^i
\\
-\partial _0E^i+\varepsilon ^{ijk}\partial _jB^k&=J^i
\\
\varepsilon ^{ijk}\partial _jB^k&=J^i+\partial _0E^i
\\
\nabla \times \mathbf{B}&=\mathbf{J}+\frac{\partial \mathbf{E}}{\partial t}
    \end{aligned}
\end{equation}




3.2推导
\begin{equation}
    \partial ^iF^{jk}+\partial ^jF^{ki}+\partial ^kF^{ij}=0\Rightarrow \nabla \cdot \mathbf{B}=0
\end{equation}
过程
\begin{equation}
    \begin{aligned}
        \partial ^iF^{jk}+\partial ^jF^{ki}+\partial ^kF^{ij}&=0
\\
-\partial _iF^{jk}-\partial _jF^{ki}-\partial _kF^{ij}&=0
\\
\partial _iF^{jk}+\partial _jF^{ki}+\partial _kF^{ij}&=0
\\
\partial _i\varepsilon ^{jkl}B^l+\partial _j\varepsilon ^{kil}B^l+\partial _k\varepsilon ^{ijl}B^l&=0
\\
\varepsilon ^{jkl}\partial _iB^l+\varepsilon ^{kil}\partial _jB^l+\varepsilon ^{ijl}\partial _kB^l&=0
\\
\varepsilon ^{ijk}\left( \varepsilon ^{jkl}\partial _iB^l+\varepsilon ^{kil}\partial _jB^l+\varepsilon ^{ijl}\partial _kB^l \right) &=0
\\
\varepsilon ^{jki}\varepsilon ^{jkl}\partial _iB^l+\varepsilon ^{kij}\varepsilon ^{kil}\partial _jB^l+\varepsilon ^{ijk}\varepsilon ^{ijl}\partial _kB^l&=0
\\
\delta ^{il}\partial _iB^l+\delta ^{jl}\partial _jB^l+\delta ^{kl}\partial _kB^l&=0
\\
3\partial _lB^l&=0
\\
\partial _iB^i&=0
\\
\nabla \cdot \mathbf{B}&=0
    \end{aligned}
\end{equation}







4.2推导
\begin{equation}
    \partial ^0F^{ij}+\partial ^iF^{j0}+\partial ^jF^{0i}=0\Rightarrow \nabla \times \mathbf{E}=-\frac{\partial \mathbf{B}}{\partial t}
\end{equation}
过程
\begin{equation}
    \begin{aligned}
        
    \end{aligned}
\end{equation}








%%%%%%%%%%%%%%%%%%%%%%%%%%%%%%%%%%%%%%%%%%%%%%%%%%%%%5
\subsection{}


















\section{2}
\section{3}
\section{4}
\section{量子旋量场}


%%%%%%%%%%%%%%%%%%%%%%%%%%%%%%%%%%%%%%%%%%%%%%%%%%%%%%%%%
\subsection{Dirac旋量场}



\begin{tabular}{lll}
\hline
线性空间名称 & 矢量表示空间 & 旋量表示空间 \\
\hline
维度 & 4 维 & 4 维 \\
空间中元素 & Lorentz 矢量 $A^{\mu}$ & Dirac 旋量 $\psi_{a}$ \\
Lorentz 群生成元 & $(\mathcal{J}^{\mu\nu})^{\alpha}_{\beta} \equiv \mathrm{i}(g^{\mu\alpha}\delta^{\nu}_{\beta} - g^{\nu\alpha}\delta^{\mu}_{\beta})$ & $\mathcal{S}^{\mu\nu} = \frac{\mathrm{i}}{4}[\gamma^{\mu},\gamma^{\nu}]$ \\
固有保时向 Lorentz 变换 & $\Lambda = \exp\left(-\frac{\mathrm{i}}{2}\omega_{\mu\nu}\mathcal{J}^{\mu\nu}\right)$ & $D(\Lambda) = \exp\left(-\frac{\mathrm{i}}{2}\omega_{\mu\nu}\mathcal{S}^{\mu\nu}\right)$ \\
\hline
线性空间名称 & Hilbert 空间 & 场空间 \\
\hline
维度 & 无限维 & 无限维 \\
空间中元素 & 态矢 $|\Psi\rangle$ & 场 $\phi(x), A^{\mu}(x), \psi_{a}(x)$ \\
Lorentz 群生成元 & 算符 $J^{\mu\nu}$ & $\hat{L}^{\mu\nu} = \mathrm{i}(x^{\mu}\partial^{\nu} - x^{\nu}\partial^{\mu})$ \\
固有保时向 Lorentz 变换 & $U(\Lambda) = \exp\left(-\frac{\mathrm{i}}{2}\omega_{\mu\nu}J^{\mu\nu}\right)$ & $\exp\left(-\frac{\mathrm{i}}{2}\omega_{\mu\nu}\hat{L}^{\mu\nu}\right)$ \\
\hline
\end{tabular}


\begin{table}[htbp]
\centering
\caption{与Lorentz变换相关的线性空间}
\begin{tabular}{lll}
\toprule
线性空间名称 & 矢量表示空间 & 旋量表示空间 \\
\midrule
维度 & 4 维 & 4 维 \\
空间中元素 & Lorentz 矢量 $A^{\mu}$ & Dirac 旋量 $\psi_{a}$ \\
Lorentz 群生成元 & $\displaystyle (\mathcal{J}^{\mu\nu})^{\alpha}_{\beta} \equiv \mathrm{i}\big(g^{\mu\alpha}\delta^{\nu}_{\beta} - g^{\nu\alpha}\delta^{\mu}_{\beta}\big)$ & $\displaystyle \mathcal{S}^{\mu\nu} = \frac{\mathrm{i}}{4}[\gamma^{\mu},\gamma^{\nu}]$ \\
固有保时向 Lorentz 变换 & $\displaystyle \Lambda = \exp\left(-\frac{\mathrm{i}}{2}\omega_{\mu\nu}\mathcal{J}^{\mu\nu}\right)$ & $\displaystyle D(\Lambda) = \exp\left(-\frac{\mathrm{i}}{2}\omega_{\mu\nu}\mathcal{S}^{\mu\nu}\right)$ \\
\midrule
线性空间名称 & Hilbert 空间 & 场空间 \\
\midrule
维度 & 无限维 & 无限维 \\
空间中元素 & 态矢 $\ket{\Psi}$ & 场 $\phi(x), A^{\mu}(x), \psi_{a}(x)$ \\
Lorentz 群生成元 & 算符 $J^{\mu\nu}$ & $\displaystyle \hat{L}^{\mu\nu} = \mathrm{i}(x^{\mu}\partial^{\nu} - x^{\nu}\partial^{\mu})$ \\
固有保时向 Lorentz 变换 & $\displaystyle U(\Lambda) = \exp\left(-\frac{\mathrm{i}}{2}\omega_{\mu\nu}J^{\mu\nu}\right)$ & $\displaystyle \exp\left(-\frac{\mathrm{i}}{2}\omega_{\mu\nu}\hat{L}^{\mu\nu}\right)$ \\
\bottomrule
\end{tabular}
\end{table}





\section{量子场的相互作用}
%%%%%%%%%%%%%%%%%%%%%%%%%%%%%%%%%%%%%%%%%%%%%%%%%%%%
\subsection{}





























%%%%%%%%%%%%%%%%%%%%%%%%%%%%%%%%%%%%%%%%%%%%%%%%%%%%
\subsection{}





散射指的是
在时间上,从无穷远时刻来,到发生散射的时刻,再到无穷远时刻去
在空间上,从无穷远位置来,到发生散射的位置,再到无穷远空间去















%%%%%%%%%%%%%%%%%%%%%%%%%%%%%%%%%%%%%%%%%%%%%%%%%%%%
\subsection{}
























%%%%%%%%%%%%%%%%%%%%%%%%%%%%%%%%%%%%%%%%%%%%%%%%%%%%
\subsection{Feynman 传播子}














实标量场的 Feynman 传播子
\begin{equation}
    D_{\mathrm{F}}(x-y)=\varphi (x)\varphi (y)=\langle 0|\mathrm{T}\left[ \varphi (x)\varphi (y) \right] |0\rangle =\int{\frac{\mathrm{d}^4p}{\left( 2\pi \right) ^4}}\frac{\mathrm{i}}{p^2-m^2+\mathrm{i}\epsilon}\mathrm{e}^{-\mathrm{i}p\cdot \left( x-y \right)}
\end{equation}
复标量场的Feynman传播子
\begin{equation}
   D_{\mathrm{F}}(x-y)=\phi (x)\phi ^{\dagger}(y)=\langle 0|\mathrm{T}\left[ \phi (x)\phi ^{\dagger}(y) \right] |0\rangle =\int{\frac{\mathrm{d}^4p}{\left( 2\pi \right) ^4}}\frac{\mathrm{i}}{p^2-m^2+\mathrm{i}\epsilon}\mathrm{e}^{-\mathrm{i}p\cdot \left( x-y \right)}
\end{equation}
有质量矢量场的Feynman传播子
\begin{equation}
   \Delta _{\mathrm{F}}^{\mu \nu}(x-y)=A^{\mu}(x)A^{\nu}(y)=\langle 0|\mathrm{T}\left[ A^{\mu}(x)A^{\nu}(y) \right] |0\rangle =\int{\frac{\mathrm{d}^4p}{\left( 2\pi \right) ^4}}\frac{-\mathrm{i}\left( g^{\mu \nu}-\frac{p^{\mu}p^{\nu}}{m^2} \right)}{p^2-m^2+\mathrm{i}\epsilon}\mathrm{e}^{-\mathrm{i}p\cdot \left( x-y \right)}
\end{equation}
无质量矢量场的Feynman传播子
\begin{equation}
   \Delta _{\mathrm{F}}^{\mu \nu}(x-y)=A^{\mu}(x)A^{\nu}(y)=\langle 0|\mathrm{T}\left[ A^{\mu}(x)A^{\nu}(y) \right] |0\rangle =\int{\frac{\mathrm{d}^4p}{\left( 2\pi \right) ^4}}\frac{-\mathrm{i}g^{\mu \nu}}{p^2+\mathrm{i}\epsilon}\mathrm{e}^{-\mathrm{i}p\cdot \left( x-y \right)}
\end{equation}
Dirac旋量场的Feynman传播子
\begin{equation}
   \begin{aligned}
       S_{\mathrm{F},ab}(x-y)=\psi _a(x)\psi _b(y)=\langle 0|\mathrm{T}\left[ \psi _a(x)\psi _b(y) \right] |0\rangle &=\int{\frac{\mathrm{d}^4p}{\left( 2\pi \right) ^4}}\frac{\mathrm{i}\left( p+m \right)}{p^2-m^2+\mathrm{i}\epsilon}\mathrm{e}^{-\mathrm{i}p\cdot \left( x-y \right)}
       \\
       &=\int{\frac{\mathrm{d}^4p}{\left( 2\pi \right) ^4}}\frac{\mathrm{i}}{p^2-m^2+\mathrm{i}\epsilon}\mathrm{e}^{-\mathrm{i}p\cdot \left( x-y \right)}
   \end{aligned}
\end{equation}

补充:
坐标表象下的Feynman传播子通过傅里叶变换转换到动量表象下的Feynman传播子


%%%%%%%%%%%%%%%%%%%%%%%%%%%%%%%%%%%%%%%%%%%%%%%%%%%%
\subsection{}





















%%%%%%%%%%%%%%%%%%%%%%%%%%%%%%%%%%%%%%%%%%%%%%%%%%%%
\subsection{习题}


\subsubsection{6.2}
\begin{equation}
    \begin{aligned}
        \mathrm{T[}A^{\mu}\bar{\psi}(x)\gamma _{\mu}\psi (x)A^{\nu}(y)\bar{\psi}(y)\gamma _{\nu}\psi (y)]&=\mathrm{N[}A^{\mu}(x)\bar{\psi}(x)\gamma _{\mu}\psi (x)A^{\nu}(y)\bar{\psi}(y)\gamma _{\nu}\psi (y)]
\\
&+\wick{\c{A}^{\mu}(x)\bar{\psi}(x)\gamma _{\mu}\psi (x)\c{A}^{\nu}(y)\bar{\psi}(y)\gamma _{\nu}\psi (y)}
\\
&+\wick{A^{\mu}(x)\bar{\psi}(x)\gamma _{\mu}\c{\psi} (x)A^{\nu}(y)\c{\bar{\psi}}(y)\gamma _{\nu}\psi (y)}
\\
&+\wick{A^{\mu}(x)\c{\bar{\psi}}(x)\gamma _{\mu}\psi (x)A^{\nu}(y)\bar{\psi}(y)\gamma _{\nu}\c{\psi} (y)}
\\
&+\wick{A^{\mu}(x)\c{\bar{\psi}}(x)\gamma _{\mu}\c{\psi} (x)A^{\nu}(y)\bar{\psi}(y)\gamma _{\nu}\psi (y)}
\\
&+\wick{A^{\mu}(x)\bar{\psi}(x)\gamma _{\mu}\psi (x)A^{\nu}(y)\c{\bar{\psi}}(y)\gamma _{\nu}\c{\psi} (y)}
\\
&+\wick{\c1{A}^{\mu}(x)\bar{\psi}(x)\gamma _{\mu}\c2{\psi} (x)\c1{A}^{\nu}(y)\c2{\bar{\psi}}(y)\gamma _{\nu}\psi (y)}
\\
&+\wick{\c1{A}^{\mu}(x)\c2{\bar{\psi}}(x)\gamma _{\mu}\psi (x)\c1{A}^{\nu}(y)\bar{\psi}(y)\gamma _{\nu}\c2{\psi} (y)}
\\
&+\wick{\c1{A}^{\mu}(x)\c2{\bar{\psi}}(x)\gamma _{\mu}\c2{\psi} (x)\c1{A}^{\nu}(y)\bar{\psi}(y)\gamma _{\nu}\psi (y)}
\\
&+\wick{\c1{A}^{\mu}(x)\bar{\psi}(x)\gamma _{\mu}\psi (x)\c1{A}^{\nu}(y)\c2{\bar{\psi}}(y)\gamma _{\nu}\c2{\psi} (y)}
\\
&+\wick{A^{\mu}(x)\c2{\bar{\psi}}(x)\gamma _{\mu}\c1{\psi} (x)A^{\nu}(y)\c1{\bar{\psi}}(y)\gamma _{\nu}\c2{\psi} (y)}
\\
&+\wick{A^{\mu}(x)\c1{\bar{\psi}}(x)\gamma _{\mu}\c1{\psi} (x)A^{\nu}(y)\c2{\bar{\psi}}(y)\gamma _{\nu}\c2{\psi} (y)}
\\
&+\wick{\c3{A}^{\mu}(x)\c2{\bar{\psi}}(x)\gamma _{\mu}\c1{\psi} (x)\c3{A}^{\nu}(y)\c1{\bar{\psi}}(y)\gamma _{\nu}\c2{\psi} (y)}
\\
&+\wick{\c2{A}^{\mu}(x)\c1{\bar{\psi}}(x)\gamma _{\mu}\c1{\psi} (x)\c2{A}^{\nu}(y)\c3{\bar{\psi}}(y)\gamma _{\nu}\c3{\psi} (y)}
    \end{aligned}
\end{equation}






\section{Feynman 图}


%%%%%%%%%%%%%%%%%%%%%%%%%%%%%%%%%%%%%%%%%%%%%%%%%%%%%%%%%%%%%%
\subsection{Yukawa 理论}

















补充推导:

(7.21)
\begin{equation}
    \begin{aligned}
     \langle 0|\mathrm{i}T_{1}^{(1)}|\mathbf{p}^+,\lambda ;\mathbf{q}^-,\lambda ^{\prime};\mathbf{k}\rangle &=-\mathrm{i}\kappa \int{\mathrm{d}^4x}\langle 0|\mathrm{N} [ \phi (x)\bar{\psi}(x)\psi (x) ] |\mathbf{p}^+,\lambda ;\mathbf{q}^-,\lambda ^{\prime};\mathbf{k}\rangle 
\\
\text{缩并}&=\wick{-\mathrm{i}\kappa \int{\mathrm{d}^4x}\langle 0|\mathrm{N} [ \c3{\phi}(x)\c2{\bar{\psi}}(x)\c1{\psi}(x) ] |\c1{\mathbf{p}}^+,\lambda ;\c2{\mathbf{q}}^-,\lambda ^{\prime};\c3{\mathbf{k}}\rangle } 
\\
\text{调整}&=\wick{-\mathrm{i}\kappa \int{\mathrm{d}^4x}\langle 0|\mathrm{N} [ \c3{\phi}(x)\c2{\bar{\psi}}_a(x)\c1{\psi}_a(x) ] |\c1{\mathbf{p}}^+,\lambda ;\c2{\mathbf{q}}^-,\lambda ^{\prime};\c3{\mathbf{k}}\rangle }
\\
&=-\mathrm{i}\kappa \int{\mathrm{d}^4x}\langle 0|\phi ^{(+)}(x)\bar{\psi}_{a}^{(+)}(x)\psi _{a}^{(+)}(x)|\mathbf{p}^+,\lambda ;\mathbf{q}^-,\lambda ^{\prime};\mathbf{k}\rangle 
\\
&=-\mathrm{i}\kappa \int{\mathrm{d}^4x}\langle 0|\mathrm{e}^{-\mathrm{i}k\cdot x}\bar{v}_a(\mathbf{q},\lambda ^{\prime})\mathrm{e}^{-\mathrm{i}q\cdot x}u_a(\mathbf{p},\lambda )\mathrm{e}^{-\mathrm{i}p\cdot x}|0\rangle 
\\
\text{忽略下标}a&=-\mathrm{i}\kappa \int{\mathrm{d}^4x}\bar{v}_a(\mathbf{q},\lambda ^{\prime})u_a(\mathbf{p},\lambda )\mathrm{e}^{-\mathrm{i}\left( p+q+k \right) \cdot x}\langle 0|0\rangle 
\\
&=-\mathrm{i}\kappa \int{\mathrm{d}^4x}\bar{v}(\mathbf{q},\lambda ^{\prime})u(\mathbf{p},\lambda )\mathrm{e}^{-\mathrm{i}\left( p+q+k \right) \cdot x}
\\
\text{带一横在前}&=-\mathrm{i}\kappa \bar{v}(\mathbf{q},\lambda ^{\prime})u(\mathbf{p},\lambda )\left( 2\pi \right) ^4\delta ^{(4)}(p+q+k)
    \end{aligned}
\end{equation}





(7.28)
\begin{equation}
    \begin{aligned}
       \langle \mathbf{p}^+,\lambda ;\mathbf{q}^-,\lambda ^{\prime};\mathbf{k}|\mathrm{i}T_{1}^{(1)}|0\rangle &=-\mathrm{i}\kappa \int{\mathrm{d}^4}x\langle \mathbf{p}^+,\lambda ;\mathbf{q}^-,\lambda ^{\prime};\mathbf{k}|\mathrm{N}[ \phi (x)\bar{\psi}(x)\psi (x) ] |0\rangle 
\\
\text{缩并}&=\wick{-\mathrm{i}\kappa \int{\mathrm{d}^4}x\langle \c3{\mathbf{p}}^+,\lambda ;\c2{\mathbf{q}}^-,\lambda ^{\prime};\c1{\mathbf{k}}|\mathrm{N}[ \c1{\phi}(x)\c3{\bar{\psi}}(x)\c2{\psi}(x) ] |0\rangle }
\\
\text{调整}&=\wick{{\color[RGB]{240, 0, 0} +}\mathrm{i}\kappa \int{\mathrm{d}^4}x\langle \c3{\mathbf{p}}^+,\lambda ;\c2{\mathbf{q}}^-,\lambda ^{\prime};\c1{\mathbf{k}}|\mathrm{N}[ \c1{\phi}(x)\c2{\psi}_a(x)\c3{\bar{\psi}}_a(x) ] |0\rangle }
\\
&=+\mathrm{i}\kappa \int{\mathrm{d}^4}x\langle \mathbf{p}^+,\lambda ;\mathbf{q}^-,\lambda ^{\prime};\mathbf{k}|\phi ^{(-)}(x)\psi _{a}^{(-)}(x)\bar{\psi}_{a}^{(-)}(x)|0\rangle 
\\
&=+\mathrm{i}\kappa \int{\mathrm{d}^4}x\langle 0|\mathrm{e}^{\mathrm{i}k\cdot x}v_a(\mathbf{q},\lambda ^{\prime})\mathrm{e}^{\mathrm{i}q\cdot x}\bar{u}_a(\mathbf{p},\lambda )\mathrm{e}^{\mathrm{i}p\cdot x}|0\rangle 
\\
&=+\mathrm{i}\kappa \int{\mathrm{d}^4}xv_a(\mathbf{q},\lambda ^{\prime})\bar{u}_a(\mathbf{p},\lambda )\mathrm{e}^{\mathrm{i}\left( p+q+k \right) \cdot x}\langle 0|0\rangle 
\\
\text{忽略下标,调整横在前}&=+\mathrm{i}\kappa \int{\mathrm{d}^4}x\bar{u}(\mathbf{p},\lambda )v(\mathbf{q},\lambda ^{\prime})\mathrm{e}^{\mathrm{i}\left( p+q+k \right) \cdot x}
\\
\delta \text{函数性质}&=+\mathrm{i}\kappa \bar{u}(\mathbf{p},\lambda )v(\mathbf{q},\lambda ^{\prime})\left( 2\pi \right) ^4\delta ^{(4)}(p+q+k)
    \end{aligned}
\end{equation}

(7.32)b
\begin{equation}
    \begin{aligned}
        \langle \mathbf{p}^+,\lambda ;\mathbf{q}^-,\lambda ^{\prime}|\mathrm{i}T_{1}^{(1)}|\mathbf{k}\rangle &=-\mathrm{i}\kappa \int{\mathrm{d}^4}x\langle \mathbf{p}^+,\lambda ;\mathbf{q}^-,\lambda ^{\prime}|\mathrm{N} [ \phi (x)\bar{\psi}(x)\psi (x) ] |\mathbf{k}\rangle 
\\
\text{缩并}&=\wick{-\mathrm{i}\kappa \int{\mathrm{d}^4}x\langle \c1{\mathbf{p}}^+,\lambda ;\c2{\mathbf{q}}^-,\lambda ^{\prime}|\mathrm{N} [\c3{\phi}(x)\c1{\bar{\psi}}(x)\c2{\psi}(x) ] |\c3{\mathbf{k}}\rangle }
\\
\text{调整}&=\wick{+\mathrm{i}\kappa \int{\mathrm{d}^4}x\langle \c2{\mathbf{p}}^+,\lambda ;\c1{\mathbf{q}}^-,\lambda ^{\prime}|\mathrm{N} [ \c1{\psi}_a(x)\c2{\bar{\psi}}_a(x)\c3{\phi}(x) ] |\c3{\mathbf{k}}\rangle }
\\
\text{去正规乘积}&=+\mathrm{i}\kappa \int{\mathrm{d}^4}x\langle \mathbf{p}^+,\lambda ;\mathbf{q}^-,\lambda ^{\prime}|\psi _{a}^{(-)}(x)\bar{\psi}_{a}^{(-)}(x)\phi ^{(+)}(x)|\mathbf{k}\rangle 
\\
&=+\mathrm{i}\kappa \int{\mathrm{d}^4}x\langle 0|v_a(\mathbf{q},\lambda ^{\prime})\mathrm{e}^{\mathrm{i}q\cdot x}\bar{u}_a(\mathbf{p},\lambda )\mathrm{e}^{\mathrm{i}p\cdot x}\mathrm{e}^{-\mathrm{i}k\cdot x}|0\rangle 
\\
&=+\mathrm{i}\kappa \int{\mathrm{d}^4}xv_a(\mathbf{q},\lambda ^{\prime})\bar{u}_a(\mathbf{p},\lambda )\mathrm{e}^{-\mathrm{i}\left( k-p-q \right) \cdot x}\langle 0|0\rangle 
\\
&=+\mathrm{i}\kappa \int{\mathrm{d}^4}x\bar{u}(\mathbf{p},\lambda )v(\mathbf{q},\lambda ^{\prime})\mathrm{e}^{-\mathrm{i}\left( k-p-q \right) \cdot x}
\\
\text{有一横在前面}&=+\mathrm{i}\kappa \bar{u}(\mathbf{p},\lambda )v(\mathbf{q},\lambda ^{\prime})\left( 2\pi \right) ^4\delta ^{(4)}(k-p-q)
    \end{aligned}
\end{equation}

(7.33)c
\begin{equation}
    \begin{aligned}
        \langle \mathbf{q}^+,\lambda ^{\prime};\mathbf{k}|\mathrm{i}T_{1}^{(1)}|\mathbf{p}^+,\lambda \rangle &=-\mathrm{i}\kappa \int{\mathrm{d}^4}x\langle \mathbf{q}^+,\lambda ^{\prime};\mathbf{k}|\mathrm{N} [ \phi (x)\bar{\psi}(x)\psi (x) ] |\mathbf{p}^+,\lambda \rangle 
\\
\text{缩并}&=\wick{-\mathrm{i}\kappa \int{\mathrm{d}^4}x\langle \c2{\mathbf{q}}^+,\lambda ^{\prime};\c1{\mathbf{k}}|\mathrm{N} [ \c1{\phi}(x)\c2{\bar{\psi}}(x)\c3{\psi}(x) ] |\c3{\mathbf{p}}^+,\lambda \rangle }
\\
\text{整理}&=\wick{-\mathrm{i}\kappa \int{\mathrm{d}^4}x\langle \c2{\mathbf{q}}^+,\lambda ^{\prime};\c1{\mathbf{k}}|\mathrm{N} [\c1{\phi}(x)\c2{\bar{\psi}}_a(x)\c3{\psi}_a(x) ] |\c3{\mathbf{p}}^+,\lambda \rangle }
\\
\text{去正规乘积}&=-\mathrm{i}\kappa \int{\mathrm{d}^4}x\langle \mathbf{q}^+,\lambda ^{\prime};\mathbf{k}|\phi ^{(-)}(x)\bar{\psi}_{a}^{(-)}(x)\psi _{a}^{(+)}(x)|\mathbf{p}^+,\lambda \rangle 
\\
&=-\mathrm{i}\kappa \int{\mathrm{d}^4}x\langle 0|\mathrm{e}^{\mathrm{i}k\cdot x}\bar{u}_a(\mathbf{q},\lambda ^{\prime})\mathrm{e}^{\mathrm{i}q\cdot x}u_a(\mathbf{p},\lambda )\mathrm{e}^{-\mathrm{i}p\cdot x}|0\rangle 
\\
&=-\mathrm{i}\kappa \int{\mathrm{d}^4}x\bar{u}_a(\mathbf{q},\lambda ^{\prime})u_a(\mathbf{p},\lambda )\mathrm{e}^{-\mathrm{i}\left( p-q-k \right) \cdot x}\langle 0|0\rangle 
\\
&=-\mathrm{i}\kappa \int{\mathrm{d}^4}x\bar{u}(\mathbf{q},\lambda ^{\prime})u(\mathbf{p},\lambda )\mathrm{e}^{-\mathrm{i}\left( p-q-k \right) \cdot x}
\\
&=-\mathrm{i}\kappa \bar{u}(\mathbf{q},\lambda ^{\prime})u(\mathbf{p},\lambda )\left( 2\pi \right) ^4\delta ^{(4)}(p-q-k)
    \end{aligned}
\end{equation}



(7.34)d
\begin{equation}
    \begin{aligned}
        \langle \mathbf{q}^-,\lambda ^{\prime};\mathbf{k}|\mathrm{i}T_{1}^{(1)}|\mathbf{p}^-,\lambda \rangle &=-\mathrm{i}\kappa \int{\mathrm{d}^4x}\langle \mathbf{q}^-,\lambda ^{\prime};\mathbf{k}|\mathrm{N} [ \phi (x)\bar{\psi}_a(x)\psi _a(x) ] |\mathbf{p}^-,\lambda \rangle 
\\
\text{缩并}&=\wick{-\mathrm{i}\kappa \int{\mathrm{d}^4}x\langle \c3{\mathbf{q}}^-,\lambda ^{\prime};\c1{\mathbf{k}}|\mathrm{N} [ \c1{\phi}(x)\c2{\bar{\psi}}(x)\c3{\psi}(x) ] |\c2{\mathbf{p}}^-,\lambda \rangle }
\\
\text{调整}&=\wick{+\mathrm{i}\kappa \int{\mathrm{d}^4}x\langle \c2{\mathbf{q}}^-,\lambda ^{\prime};\c1{\mathbf{k}}|\mathrm{N} [ \c1{\phi}(x)\c2{\bar{\psi}}_a(x)\c3{\psi}_a(x) ] |\c3{\mathbf{p}}^-,\lambda \rangle }
\\
&=+\mathrm{i}\kappa \int{\mathrm{d}^4}x\langle \mathbf{q}^-,\lambda ^{\prime};\mathbf{k}|\phi ^{(-)}(x)\psi _{a}^{(-)}(x)\bar{\psi}_{a}^{(+)}(x)|\mathbf{p}^-,\lambda \rangle 
\\
&=+\mathrm{i}\kappa \int{\mathrm{d}^4}x\langle 0|\mathrm{e}^{\mathrm{i}k\cdot x}v_a(\mathbf{q},\lambda ^{\prime})\mathrm{e}^{\mathrm{i}q\cdot x}\bar{v}_a(\mathbf{p},\lambda )\mathrm{e}^{-\mathrm{i}p\cdot x}|0\rangle 
\\
&=+\mathrm{i}\kappa \int{\mathrm{d}^4}xv_a(\mathbf{q},\lambda ^{\prime})\bar{v}_a(\mathbf{p},\lambda )\mathrm{e}^{-\mathrm{i}\left( p-q-k \right) \cdot x}\langle 0|0\rangle 
\\
&=+\mathrm{i}\kappa \int{\mathrm{d}^4}x\bar{v}(\mathbf{p},\lambda )v(\mathbf{q},\lambda ^{\prime})\mathrm{e}^{-\mathrm{i}\left( p-q-k \right) \cdot x}
\\
&=+\mathrm{i}\kappa \bar{v}(\mathbf{p},\lambda )v(\mathbf{q},\lambda ^{\prime})\left( 2\pi \right) ^4\delta ^{(4)}(p-q-k)
    \end{aligned}
\end{equation}

(7.35)f
\begin{equation}
    \begin{aligned}
        \langle \mathbf{k}|\mathrm{i}T_{1}^{(1)}|\mathbf{p}^+,\lambda ;\mathbf{q}^-,\lambda ^{\prime}\rangle &=-\mathrm{i}\kappa \int{\mathrm{d}^4}x\langle \mathbf{k}|\mathrm{N}[ \phi (x)\bar{\psi}(x)\psi (x) ] |\mathbf{p}^+,\lambda ;\mathbf{q}^-,\lambda ^{\prime}\rangle 
\\
\text{缩并}&=\wick{-\mathrm{i}\kappa \int{\mathrm{d}^4}x\langle \c1{\mathbf{k}}|\mathrm{N} [ \c1{\phi}(x)\c3{\bar{\psi}}(x)\c2{\psi}(x) ] |\c2{\mathbf{p}}^+,\lambda ;\c3{\mathbf{q}}^-,\lambda ^{\prime}\rangle }
\\
\text{调整}&=\wick{-\mathrm{i}\kappa \int{\mathrm{d}^4}x\langle \c1{\mathbf{k}}|\mathrm{N} [ \c1{\phi}(x)\c3{\bar{\psi}}_a(x)\c2{\psi}_a(x) ] |\c2{\mathbf{p}}^+,\lambda ;\c3{\mathbf{q}}^-,\lambda ^{\prime}\rangle }
\\
\text{去正规乘积}&=-\mathrm{i}\kappa \int{\mathrm{d}^4}x\langle \mathbf{k}|\phi ^{(-)}(x)\bar{\psi}_{a}^{(+)}(x)\psi _{a}^{(+)}(x)|\mathbf{p}^+,\lambda ;\mathbf{q}^-,\lambda ^{\prime}\rangle 
\\
&=-\mathrm{i}\kappa \int{\mathrm{d}^4}x\langle 0|\mathrm{e}^{\mathrm{i}k\cdot x}\bar{v}_a(\mathbf{q},\lambda ^{\prime})\mathrm{e}^{-\mathrm{i}q\cdot x}u_a(\mathbf{p},\lambda )\mathrm{e}^{-\mathrm{i}p\cdot x}|0\rangle 
\\
&=-\mathrm{i}\kappa \int{\mathrm{d}^4}x\bar{v}_a(\mathbf{q},\lambda ^{\prime})u_a(\mathbf{p},\lambda )\mathrm{e}^{-\mathrm{i}\left( p+q-k \right) \cdot x}\langle 0|0\rangle 
\\
&=-\mathrm{i}\kappa \int{\mathrm{d}^4}x\bar{v}(\mathbf{q},\lambda ^{\prime})u(\mathbf{p},\lambda )\mathrm{e}^{-\mathrm{i}\left( p+q-k \right) \cdot x}
\\
\delta \text{函数性质}&=-\mathrm{i}\kappa \bar{v}(\mathbf{q},\lambda ^{\prime})u(\mathbf{p},\lambda )\left( 2\pi \right) ^4\delta ^{(4)}(p+q-k)
    \end{aligned}
\end{equation}


(7.36)g
\begin{equation}
    \begin{aligned}
        \langle \mathbf{q}^+,\lambda ^{\prime}|\mathrm{i}T_{1}^{(1)}|\mathbf{p}^+,\lambda ;\mathbf{k}\rangle &=-\mathrm{i}\kappa \int{\mathrm{d}^4}x\langle \mathbf{q}^+,\lambda ^{\prime}|\mathrm{N}[ \phi (x)\bar{\psi}(x)\psi (x) ] |\mathbf{p}^+,\lambda ;\mathbf{k}\rangle 
\\
\text{缩并}&=\wick{-\mathrm{i}\kappa \int{\mathrm{d}^4}x\langle \c1{\mathbf{q}}^+,\lambda ^{\prime}|\mathrm{N}[ \c3{\phi}(x)\c1{\bar{\psi}}(x)\c2{\psi}(x) ] |\c2{\mathbf{p}}^+,\lambda ;\c3{\mathbf{k}}\rangle }
\\
\text{调整}&=\wick{-\mathrm{i}\kappa \int{\mathrm{d}^4}x\langle \c1{\mathbf{q}}^+,\lambda ^{\prime}|\mathrm{N}[ \c1{\bar{\psi}}_a(x)\c3{\phi}(x)\c2{\psi}_a(x) ] |\c2{\mathbf{p}}^+,\lambda ;\c3{\mathbf{k}}\rangle }
\\
\text{去正规乘积}&=-\mathrm{i}\kappa \int{\mathrm{d}^4}x\langle \mathbf{q}^+,\lambda ^{\prime}|\bar{\psi}_{a}^{(-)}(x)\phi ^{(+)}(x)\psi _{a}^{(+)}(x)|\mathbf{p}^+,\lambda ;\mathbf{k}\rangle 
\\
&=-\mathrm{i}\kappa \int{\mathrm{d}^4}x\langle 0|\bar{u}_a(\mathbf{q},\lambda ^{\prime})\mathrm{e}^{\mathrm{i}q\cdot x}\mathrm{e}^{-\mathrm{i}k\cdot x}u_a(\mathbf{p},\lambda )\mathrm{e}^{-\mathrm{i}p\cdot x}|0\rangle 
\\
&=-\mathrm{i}\kappa \int{\mathrm{d}^4}x\bar{u}_a(\mathbf{q},\lambda ^{\prime})u_a(\mathbf{p},\lambda )\mathrm{e}^{-\mathrm{i}\left( p+k-q \right) \cdot x}\langle 0|0\rangle 
\\
&=-\mathrm{i}\kappa \int{\mathrm{d}^4}x\bar{u}(\mathbf{q},\lambda ^{\prime})u(\mathbf{p},\lambda )\mathrm{e}^{-\mathrm{i}\left( p+k-q \right) \cdot x}
\\
\text{函数性质}&=-\mathrm{i}\kappa \bar{u}(\mathbf{q},\lambda ^{\prime})u(\mathbf{p},\lambda )\left( 2\pi \right) ^4\delta ^{(4)}(p+k-q)
    \end{aligned}
\end{equation}


(7.37)h
\begin{equation}
    \begin{aligned}
        \langle \mathbf{q}^-,\lambda ^{\prime}|\mathrm{i}T_{1}^{(1)}|\mathbf{p}^-,\lambda ;\mathbf{k}\rangle &=-\mathrm{i}\kappa \int{\mathrm{d}^4}x\langle \mathbf{q}^-,\lambda ^{\prime}|\mathrm{N}[ \phi (x)\bar{\psi}(x)\psi (x) ] |\mathbf{p}^-,\lambda ;\mathbf{k}\rangle 
\\
\text{缩并}&=\wick{-\mathrm{i}\kappa \int{\mathrm{d}^4}x\langle \c1{\mathbf{q}}^-,\lambda ^{\prime}|\mathrm{N}[ \c2{\phi}(x)\c3{\bar{\psi}}(x)\c1{\psi}(x) ] |\c3{\mathbf{p}}^-,\lambda ;\c2{\mathbf{k}}\rangle }
\\
\text{调整}&=\wick{+\mathrm{i}\kappa \int{\mathrm{d}^4}x\langle \c1{\mathbf{q}}^-,\lambda ^{\prime}|\mathrm{N} [ \c1{\psi}_a(x)\c3{\phi}(x)\c2{\bar{\psi}}_a(x) ] |\c2{\mathbf{p}}^-,\lambda ;\c3{\mathbf{k}}\rangle }
\\
\text{正负能解}&=+\mathrm{i}\kappa \int{\mathrm{d}^4}x\langle \mathbf{q}^-,\lambda ^{\prime}|\psi _{a}^{(-)}(x)\phi (x)\bar{\psi}_{a}^{(+)}(x)|\mathbf{p}^-,\lambda ;\mathbf{k}\rangle 
\\
&=+\mathrm{i}\kappa \int{\mathrm{d}^4}x\langle 0|v_a(\mathbf{q},\lambda ^{\prime})\mathrm{e}^{\mathrm{i}q\cdot x}\mathrm{e}^{-\mathrm{i}k\cdot x}\bar{v}_a(\mathbf{p},\lambda )\mathrm{e}^{-\mathrm{i}p\cdot x}|0\rangle 
\\
&=+\mathrm{i}\kappa \int{\mathrm{d}^4}x\bar{v}_a(\mathbf{p},\lambda )v_a(\mathbf{q},\lambda ^{\prime})\mathrm{e}^{-\mathrm{i}\left( p+k-q \right) \cdot x}\langle 0|0\rangle 
\\
\text{横在前}&=+\mathrm{i}\kappa \int{\mathrm{d}^4}x\bar{v}(\mathbf{p},\lambda )v(\mathbf{q},\lambda ^{\prime})\mathrm{e}^{-\mathrm{i}\left( p+k-q \right) \cdot x}
\\
\text{函数性质}&=+\mathrm{i}\kappa \bar{v}(\mathbf{p},\lambda )v(\mathbf{q},\lambda ^{\prime})\left( 2\pi \right) ^4\delta ^{(4)}(p+k-q)
    \end{aligned}
\end{equation}


\newpage




\newpage
(7.58)无缩并
\begin{equation}
    \mathrm{i}T_{1}^{(2)}=\frac{\left( -\mathrm{i}\kappa \right) ^2}{2!}\int{\mathrm{d}^4x\mathrm{d}^4y\mathrm{N[}\phi (x)\bar{\psi}(x)\psi (x)\phi (y)\bar{\psi}(y)\psi (y)]}
\end{equation}

一次
\begin{equation}
    \begin{aligned}
        \mathrm{i}T_{2}^{(2)}&=\wick{ \frac{\left( -\mathrm{i}\kappa \right) ^2}{2!}\int{\mathrm{d}^4x\mathrm{d}^4y\mathrm{N} [ \c{\phi}(x)\bar{\psi}(x)\psi(x)\c{\phi}(y)\bar{\psi}(y)\psi (y)]} }
\\
\mathrm{i}T_{3}^{(2)}&=\frac{\left( -\mathrm{i}\kappa \right) ^2}{2!}\int{\mathrm{d}^4x\mathrm{d}^4y\mathrm{N} [ \phi (x)\bar{\psi}(x)\psi (x)\phi (y)\bar{\psi}(y)\psi (y)]}
\\
\mathrm{i}T_{4}^{(2)}&=\frac{\left( -\mathrm{i}\kappa \right) ^2}{2!}\int{\mathrm{d}^4x\mathrm{d}^4y\mathrm{N} [ \phi (x)\bar{\psi}(x)\psi (x)\phi (y)\bar{\psi}(y)\psi (y)]}
\\
\mathrm{i}T_{5}^{(2)}&=\frac{\left( -\mathrm{i}\kappa \right) ^2}{2!}\int{\mathrm{d}^4x\mathrm{d}^4y\mathrm{N} [ \phi (x)\bar{\psi}(x)\psi (x)\phi (y)\bar{\psi}(y)\psi (y)]}
\\
\mathrm{i}T_{6}^{(2)}&=\frac{\left( -\mathrm{i}\kappa \right) ^2}{2!}\int{\mathrm{d}^4x\mathrm{d}^4y\mathrm{N} [ \phi (x)\bar{\psi}(x)\psi (x)\phi (y)\bar{\psi}(y)\psi (y)]}
    \end{aligned}
\end{equation}

二次缩并





三次缩并




、




























\section{8}



\subsection{8.1}





\subsection{8.2}


1.对于Feynman图,根据Feynman 规则,写出散射过程的不变振幅

根据,


求解双线性型的复共轭为

得到,iM 的复共轭为


根据上面的结论

不变振幅的模方为


Casimir 技巧


计算非极化不变振幅模方


\subsection{8.4}


在高能极限下,忽略质量,
\\左手Dirac旋量场$\psi_\mathrm{L}$/左手Weyl旋量场$\eta_\mathrm{L}$ :描述 左旋极化的正费米子 和 右旋极化的反费米子,
\\右手Dirac旋量场$\psi_\mathrm{R}$/右手Weyl旋量场$\eta_\mathrm{R}$ :描述 右旋极化的正费米子 和 左旋极化的反费米子,
\\$\psi_\mathrm{L}$ 和$\psi_\mathrm{R}$成为两个相互独立的场。
\\左手Dirac旋量场$\psi_\mathrm{L}$ 等价于左手Weyl旋量场$\eta_\mathrm{L}$ 
\\右手Dirac旋量场$\psi_\mathrm{R}$ 等价于右手Weyl旋量场$\eta_\mathrm{R}$

左旋极化 $\lambda=-$
右旋极化 $\lambda=+$


\subsection{8.5}

交叉对称性
一个过程包含一个四维动量为$p^\mathrm{\mu}$的粒子$\Phi$的初态,
一个过程包含一个四维动量为$k^\mathrm{\mu}$的反粒子$\bar{\Phi}$的末态,
则这两个过程的不变振幅可以通过动量替换$k^\mu=-p^\mu$联系起来。

一个粒子沿着时间方向运动等价于它的反粒子逆着时间方向运动,这样的反粒子具有负能量和相反动量


\subsection{8.6}

\subsection{$e^{-}\gamma\to e^{-}\gamma$}
Compton 散射:电子与光子的散射过程

s通道的

u通道的

得到总的



\subsection{$e^+e^-\to\gamma\gamma$}






\section{9}
%%%%%%%%%%%%%%%%%%%%%%%%%%%%%%%%%%%%%%%%%%%%%%%%%%%%%%%
\subsection{9.1}

宇称变换
\begin{equation}
    {\mathcal{P} ^{\mu}}_{\nu}=(\mathcal{P} ^{-1}{)^{\mu}}_{\nu}=\left( \begin{matrix}
	+1&		&		&		\\
	&		-1&		&		\\
	&		&		-1&		\\
	&		&		&		-1\\
\end{matrix} \right) 
\end{equation}
时空坐标变换
\begin{equation}
    x^{\mu}=\left( t,\mathbf{x} \right) \Rightarrow x^{\prime \mu}={\mathcal{P} ^{\mu}}_{\nu}x^{\nu}=(\mathcal{P} x)^{\mu}=\left( t,-\mathbf{x} \right) 
\end{equation}
四维动量变换
\begin{equation}
    p^{\mu}=\left( E,\mathbf{p} \right) \Rightarrow p^{\prime \mu}={\mathcal{P} ^{\mu}}_{\nu}p^{\nu}=(\mathcal{P} p)^{\mu}=\left( E,-\mathbf{p} \right) 
\end{equation}
时空导数变换
\begin{equation}
    \partial _{\mu}^{\prime}=(\mathcal{P} ^{-1}{)^{\nu}}_{\mu}\partial _{\nu}
\end{equation}
保持时空体积元不变
\begin{equation}
    \mathrm{d}^4x^{\prime}=\left| \det \left( \mathcal{P} \right) \right|\mathrm{d}^4x=\mathrm{d}^4x
\end{equation}

如果场论系统的作用量S在宇称变换下不变,则运动方程的形式也在宇称变换下不变,此时称系统是宇称守恒的,即具有空间反射对称性。


在宇称守恒的量子理论中,宇称变换在 Hilbert 空间中诱导出态矢$|\Psi \rangle$的线性幺正变换
\begin{equation}
    |\Psi ^{\prime}\rangle =U(\mathcal{P} )|\Psi \rangle =P|\Psi \rangle 
\end{equation}





推导:
根据
\begin{equation}
    \phi (x)=\int{\frac{\mathrm{d}^3p}{\left( 2\pi \right) ^3}}\frac{1}{\sqrt{2E_{\mathbf{p}}}}\left( a_{\mathbf{p}}\mathrm{e}^{-\mathrm{i}p\cdot x}+b_{\mathbf{p}}^{\dagger}\mathrm{e}^{\mathrm{i}p\cdot x} \right) 
\end{equation}
得到
\begin{equation}
    \phi (\mathcal{P} x)=\int{\frac{\mathrm{d}^3p}{\left( 2\pi \right) ^3}}\frac{1}{\sqrt{2E_{\mathbf{p}}}}\left( a_{\mathbf{p}}\mathrm{e}^{-\mathrm{i}p\cdot \left( \mathcal{P} x \right)}+b_{\mathbf{p}}^{\dagger}\mathrm{e}^{\mathrm{i}p\cdot \left( \mathcal{P} x \right)} \right) 
\end{equation}
计算
\begin{equation}
    \begin{aligned}
        P^{-1}\phi (x)P&=\int{\frac{\mathrm{d}^3p}{\left( 2\pi \right) ^3}}\frac{1}{\sqrt{2E_{\mathbf{p}}}}\left( P^{-1}a_{\mathbf{p}}P\mathrm{e}^{-\mathrm{i}p\cdot x}+P^{-1}b_{\mathbf{p}}^{\dagger}P\mathrm{e}^{\mathrm{i}p\cdot x} \right) 
\\
&=\int{\frac{\mathrm{d}^3p}{\left( 2\pi \right) ^3}}\frac{1}{\sqrt{2E_{\mathbf{p}}}}\left( \eta _{P}^{*}a_{-\mathbf{p}}\mathrm{e}^{-\mathrm{i}p\cdot x}+\tilde{\eta}_Pb_{-\mathbf{p}}^{\dagger}\mathrm{e}^{\mathrm{i}p\cdot x} \right) 
\\
&=\int{\frac{\mathrm{d}^3p}{\left( 2\pi \right) ^3}}\frac{1}{\sqrt{2E_{\mathbf{p}}}}\left( \eta _{P}^{*}a_{\mathbf{p}}\mathrm{e}^{-\mathrm{i}\left( \mathcal{P} p \right) \cdot x}+\tilde{\eta}_Pb_{\mathbf{p}}^{\dagger}\mathrm{e}^{\mathrm{i}\left( \mathcal{P} p \right) \cdot x} \right) 
\\
&=\int{\frac{\mathrm{d}^3p}{\left( 2\pi \right) ^3}}\frac{1}{\sqrt{2E_{\mathbf{p}}}}\left( \eta _{P}^{*}a_{\mathbf{p}}\mathrm{e}^{-\mathrm{i}p\cdot \left( \mathcal{P} x \right)}+\tilde{\eta}_Pb_{\mathbf{p}}^{\dagger}\mathrm{e}^{\mathrm{i}p\cdot \left( \mathcal{P} x \right)} \right) 
\\
&=\int{\frac{\mathrm{d}^3p}{\left( 2\pi \right) ^3}}\frac{1}{\sqrt{2E_{\mathbf{p}}}}\left( \eta _{P}^{*}a_{\mathbf{p}}\mathrm{e}^{-\mathrm{i}p\cdot \left( \mathcal{P} x \right)}+\eta _{P}^{*}b_{\mathbf{p}}^{\dagger}\mathrm{e}^{\mathrm{i}p\cdot \left( \mathcal{P} x \right)} \right) 
\\
&=\eta _{P}^{*}\int{\frac{\mathrm{d}^3p}{\left( 2\pi \right) ^3}}\frac{1}{\sqrt{2E_{\mathbf{p}}}}\left( a_{\mathbf{p}}\mathrm{e}^{-\mathrm{i}p\cdot \left( \mathcal{P} x \right)}+b_{\mathbf{p}}^{\dagger}\mathrm{e}^{\mathrm{i}p\cdot \left( \mathcal{P} x \right)} \right) 
    \end{aligned}
\end{equation}
得到
\begin{equation}
    P^{-1}\phi (x)P=\eta _{P}^{*}\phi (\mathcal{P} x)
\end{equation}


推导:
根据
\begin{equation}
    \phi ^{\dagger}(x)=\int{\frac{\mathrm{d}^3p}{\left( 2\pi \right) ^3}}\frac{1}{\sqrt{2E_{\mathbf{p}}}}\left( b_{\mathbf{p}}\mathrm{e}^{-\mathrm{i}p\cdot x}+a_{\mathbf{p}}^{\dagger}\mathrm{e}^{\mathrm{i}p\cdot x} \right) 
\end{equation}
得到
\begin{equation}
    \phi ^{\dagger}(\mathcal{P} x)=\int{\frac{\mathrm{d}^3p}{\left( 2\pi \right) ^3}}\frac{1}{\sqrt{2E_{\mathbf{p}}}}\left( b_{\mathbf{p}}\mathrm{e}^{-\mathrm{i}p\cdot \left( \mathcal{P} x \right)}+a_{\mathbf{p}}^{\dagger}\mathrm{e}^{\mathrm{i}p\cdot \left( \mathcal{P} x \right)} \right) 
\end{equation}
计算
\begin{equation}
    \begin{aligned}
        P^{-1}\phi ^{\dagger}(x)P&=\int{\frac{\mathrm{d}^3p}{\left( 2\pi \right) ^3}}\frac{1}{\sqrt{2E_{\mathbf{p}}}}\left( P^{-1}b_{\mathbf{p}}P\mathrm{e}^{-\mathrm{i}p\cdot x}+P^{-1}a_{\mathbf{p}}^{\dagger}P\mathrm{e}^{\mathrm{i}p\cdot x} \right) 
\\
&=\int{\frac{\mathrm{d}^3p}{\left( 2\pi \right) ^3}}\frac{1}{\sqrt{2E_{\mathbf{p}}}}\left( \tilde{\eta}_{P}^{*}b_{-\mathbf{p}}\mathrm{e}^{-\mathrm{i}p\cdot x}+\eta _Pa_{-\mathbf{p}}^{\dagger}\mathrm{e}^{\mathrm{i}p\cdot x} \right) 
\\
&=\int{\frac{\mathrm{d}^3p}{\left( 2\pi \right) ^3}}\frac{1}{\sqrt{2E_{\mathbf{p}}}}\left( \tilde{\eta}_{P}^{*}b_{\mathbf{p}}\mathrm{e}^{-\mathrm{i}\left( \mathcal{P} p \right) \cdot x}+\eta _Pa_{\mathbf{p}}^{\dagger}\mathrm{e}^{\mathrm{i}\left( \mathcal{P} p \right) \cdot x} \right) 
\\
&=\int{\frac{\mathrm{d}^3p}{\left( 2\pi \right) ^3}}\frac{1}{\sqrt{2E_{\mathbf{p}}}}\left( \tilde{\eta}_{P}^{*}b_{\mathbf{p}}\mathrm{e}^{-\mathrm{i}p\cdot \left( \mathcal{P} x \right)}+\eta _Pa_{\mathbf{p}}^{\dagger}\mathrm{e}^{\mathrm{i}p\cdot \left( \mathcal{P} x \right)} \right) 
\\
&=\int{\frac{\mathrm{d}^3p}{\left( 2\pi \right) ^3}}\frac{1}{\sqrt{2E_{\mathbf{p}}}}\left( \eta _Pb_{\mathbf{p}}\mathrm{e}^{-\mathrm{i}p\cdot \left( \mathcal{P} x \right)}+\eta _Pa_{\mathbf{p}}^{\dagger}\mathrm{e}^{\mathrm{i}p\cdot \left( \mathcal{P} x \right)} \right) 
\\
&=\eta _P\int{\frac{\mathrm{d}^3p}{\left( 2\pi \right) ^3}}\frac{1}{\sqrt{2E_{\mathbf{p}}}}\left( b_{\mathbf{p}}\mathrm{e}^{-\mathrm{i}p\cdot \left( \mathcal{P} x \right)}+a_{\mathbf{p}}^{\dagger}\mathrm{e}^{\mathrm{i}p\cdot \left( \mathcal{P} x \right)} \right) 
    \end{aligned}
\end{equation}
得到
\begin{equation}
    P^{-1}\phi ^{\dagger}(x)P=\eta _P\phi ^{\dagger}(\mathcal{P} x)
\end{equation}










\include{余钊焕-10}


\end{document}