\section{习题2}

\newpage
\subsection{2.1}
设算符 $a$ 与其厄米共轭 $a^\dagger$ 满足反对易关系
$$\{a, a^\dagger\} = 1, \quad \{a, a\} = \{a^\dagger, a^\dagger\} = 0, \tag{2.221}$$
其中反对易子定义为 $\{A, B\} \equiv AB + BA$。记算符 $N \equiv a^\dagger a$ 的本征值为 $n$,本征态为 $|n\rangle$,即 $N |n\rangle = n |n\rangle$,归一化为 $\langle n|n\rangle = 1$。

(a) 证明 $[N, a^\dagger] = a^\dagger$ 和 $[N, a] = -a$。

(b) 证明本征值 $n$ 只能取 0 和 1,而且
$$a^\dagger |n\rangle = \sqrt{1 - n}|n + 1\rangle, \quad a |n\rangle = \sqrt{n}|n - 1\rangle. \tag{2.222}$$

\newpage
\subsection{2.2}
已知产生湮灭算符的对易关系 (2.122),以及 $\phi(x,t)$ 和 $\pi(x,t)$ 的平面波展开式 (2.103) 和 (2.105),推出等时对易关系 (2.87)。

\newpage
\subsection{2.3}
用实标量场的单粒子态 (2.152) 构造波包,设
$$|\Psi_p\rangle = \int \frac{d^3q}{(2\pi)^3} \frac{F_p(q)}{\sqrt{2E_q}} |q\rangle, \tag{2.223}$$
其中函数 $F_p(q)$ 满足
$$\int \frac{d^3q}{(2\pi)^3} |F_p(q)|^2 = 1, \quad \int \frac{d^3q}{(2\pi)^3} |F_p(q)|^2 q = p, \tag{2.224}$$
求内积 $\langle \Psi_p |\Psi_p\rangle$ 和总动量算符期待值 $\langle \Psi_p |P|\Psi_p\rangle$。

\newpage
\subsection{2.4}
将实标量场 $\phi(x)$ 的平面波展开式 (2.103) 代入对易关系 (2.129),推出
$$[P^\mu, a_p] = -p^\mu a_p, \quad [P^\mu, a_p^\dagger] = p^\mu a_p^\dagger. \tag{2.225}$$

\newpage
\subsection{2.5}
对于自由实标量场 $\phi(x)$,根据1.7节关于Noether定理的讨论,Lorentz对称性给出的守恒荷算符为
$$J^{\mu\nu} = \int d^3x (T^{0\nu}x^\mu - T^{0\mu}x^\nu),$$
其中
$$T^{00} = \mathcal{H} = \frac{1}{2}[\pi^2 + (\nabla \phi)^2 + m^2 \phi^2], \quad T^{0i} = \pi \partial^i \phi.$$
利用等时对易关系(2.87)推出
$$[\phi(x), J^{\mu\nu}] = i(x^\mu \partial^\nu - x^\nu \partial^\mu)\phi(x). \tag{2.226}$$

\newpage
\subsection{2.6}
复标量场 $\phi(x)$ 可以按(2.182)式分解为两个实标量场 $\phi_1(x)$ 和 $\phi_2(x)$ 的线性组合。设 $\phi_1(x)$ 和 $\phi_2(x)$ 的平面波展开式为
$$\phi_1(x) = \int \frac{d^3p}{(2\pi)^3} \frac{1}{\sqrt{2E_p}} (c_p e^{-ip\cdot x} + c_p^\dagger e^{ip\cdot x}),$$
$$\phi_2(x) = \int \frac{d^3p}{(2\pi)^3} \frac{1}{\sqrt{2E_p}} (d_p e^{-ip\cdot x} + d_p^\dagger e^{ip\cdot x}). \tag{2.229}$$
(a) 推导复标量场平面波展开式(2.187)和(2.189)中使用的产生湮灭算符 $(a_p, a_p^\dagger, b_p, b_p^\dagger)$ 与实标量场产生湮灭算符 $(c_p, c_p^\dagger, d_p, d_p^\dagger)$ 之间的关系。
(b) 根据上述关系及对易关系
$$[c_p, c_p^\dagger] = (2\pi)^3 \delta^{(3)} (p-q), \quad [d_p, d_q^\dagger] = (2\pi)^3 \delta^{(3)} (p-q),$$
$$[c_p, c_q] = [c_p^\dagger, c_q^\dagger] = [d_p, d_q] = [d_p^\dagger, d_q^\dagger] = 0,$$
$$[c_p, d_q] = [c_p^\dagger, d_q^\dagger] = [c_p, d_q^\dagger] = [c_p^\dagger, d_q] = 0,$$
验证 $(a_p, a_p^\dagger, b_p, b_p^\dagger)$ 满足对易关系(2.193)。

\newpage
\subsection{2.7}
复标量场 $\phi(x)$ 的守恒荷算符 $Q$ 可以用产生湮灭算符表达成(2.212)式。
(a) 证明
$$[Q, \phi] = -q\phi, \quad [Q, \phi^\dagger] = q\phi^\dagger. \tag{2.232}$$
(b) 设 $|Q'\rangle$ 是 $Q$ 的本征态,本征值为 $Q'$, 即 $Q | Q' \rangle = Q' | Q' \rangle$。论证 $\phi | Q' \rangle$ 和 $\phi^\dagger | Q' \rangle$ 的 $Q$ 本征值分别为 $Q' - q$ 和 $Q' + q$。

\newpage
\subsection{2.8}
对于复标量场 $\phi(x)$, 真空态 $|0\rangle$ 满足 $a_p |0\rangle = b_p |0\rangle = 0$ 和 $\langle 0|0\rangle = 1$, 引入动量为 $p$ 的正标量玻色子态 $|p^+ \rangle \equiv \sqrt{2E_p} a_p^\dagger |0\rangle$ 和反标量玻色子态 $|p^- \rangle \equiv \sqrt{2E_p} b_p^\dagger |0\rangle$。
(a) 求内积 $\langle q^+ | p^+ \rangle, \langle q^- | p^- \rangle$ 和 $\langle q^- | p^+ \rangle$。
(b) 求 $\langle 0| \phi(x) | p^+ \rangle, \langle 0| \phi^\dagger(x) | p^- \rangle, \langle p^+ | \phi^\dagger(x) | 0\rangle$ 和 $\langle p^- | \phi(x) | 0\rangle$。
(c) 根据守恒荷算符 $Q$ 的表达式 (2.212),推出
$$Q \left| p^+ \right\rangle = (Q_{vac} + q) \left| p^+ \right\rangle, \quad Q \left| p^- \right\rangle = (Q_{vac} - q) \left| p^- \right\rangle,$$
其中
$$Q_{vac} \equiv -(2\pi)^3 \delta^{(3)} (0) \int \frac{d^3p}{(2\pi)^3} q. \tag{2.233}$$

\newpage
\subsection{2.9}
根据复标量场守恒荷算符 $Q$ 和哈密顿量算符 $H$ 的表达式 (2.212) 和 (2.215),证明
$$[H, Q] = 0. \tag{2.235}$$

\newpage
\subsection{2.10}
依照 (2.182) 式将复标量场 $\phi(x)$ 分解为实标量场 $\phi_1(x)$ 和 $\phi_2(x)$ 的线性组合。
(a) 论证复标量场的 U(1) 整体变换 (2.208) 等价于实标量场的整体变换
$$\begin{pmatrix}
\phi_1'(x) \\
\phi_2'(x)
\end{pmatrix}
=
\begin{pmatrix}
\cos q\theta & -\sin q\theta \\
\sin q\theta & \cos q\theta
\end{pmatrix}
\begin{pmatrix}
\phi_1(x) \\
\phi_2(x)
\end{pmatrix}. \tag{2.236}$$
将 $(\phi_1, \phi_2)^T$ 看作一个二维线性空间中的矢量,则上述是此空间中的一个 SO(2) 整体变换,即转动角为 $q\theta$ 的二维旋转变换。
(b) 证明以 (2.183) 式表达的拉氏量
$$\mathcal{L} = \frac{1}{2} (\partial^\mu \phi_1) \partial_\mu \phi_1 - \frac{1}{2} m^2 \phi_1^2 + \frac{1}{2} (\partial^\mu \phi_2) \partial_\mu \phi_2 - \frac{1}{2} m^2 \phi_2^2. \tag{2.237}$$
在上述 SO(2) 整体变换下不变。
