\section{9.1 标量场的分立变换}
%%%%%%%%%%%%%%%%%%%%%%%%%%%%%%%%%%%%%%%%%%%%%%%%%%%%%%%
\subsection{标量场的P变换}

\subsubsection{笔记:宇称变换}
宇称变换
\begin{equation}
    {\mathcal{P} ^{\mu}}_{\nu}=(\mathcal{P} ^{-1}{)^{\mu}}_{\nu}=\left( \begin{matrix}
	+1&		&		&		\\
	&		-1&		&		\\
	&		&		-1&		\\
	&		&		&		-1\\
\end{matrix} \right) 
\end{equation}
时空坐标变换
\begin{equation}
    x^{\mu}=\left( t,\mathbf{x} \right) \Rightarrow x^{\prime \mu}={\mathcal{P} ^{\mu}}_{\nu}x^{\nu}=(\mathcal{P} x)^{\mu}=\left( t,-\mathbf{x} \right) 
\end{equation}
四维动量变换
\begin{equation}
    p^{\mu}=\left( E,\mathbf{p} \right) \Rightarrow p^{\prime \mu}={\mathcal{P} ^{\mu}}_{\nu}p^{\nu}=(\mathcal{P} p)^{\mu}=\left( E,-\mathbf{p} \right) 
\end{equation}
时空导数变换
\begin{equation}
    \partial _{\mu}^{\prime}=(\mathcal{P} ^{-1}{)^{\nu}}_{\mu}\partial _{\nu}
\end{equation}
保持时空体积元不变
\begin{equation}
    \mathrm{d}^4x^{\prime}=\left| \det \left( \mathcal{P} \right) \right|\mathrm{d}^4x=\mathrm{d}^4x
\end{equation}

如果场论系统的作用量S在宇称变换下不变,则运动方程的形式也在宇称变换下不变,此时称系统是宇称守恒的,即具有空间反射对称性。


在宇称守恒的量子理论中,宇称变换在 Hilbert 空间中诱导出态矢$|\Psi \rangle$的线性幺正变换
\begin{equation}
    |\Psi ^{\prime}\rangle =U(\mathcal{P} )|\Psi \rangle =P|\Psi \rangle 
\end{equation}


\subsubsection{推导:9.19}
1.由标量场
\begin{equation}
    \phi (x)=\int{\frac{\mathrm{d}^3p}{\left( 2\pi \right) ^3}}\frac{1}{\sqrt{2E_{\mathbf{p}}}}\left( a_{\mathbf{p}}\mathrm{e}^{-\mathrm{i}p\cdot x}+b_{\mathbf{p}}^{\dagger}\mathrm{e}^{\mathrm{i}p\cdot x} \right) 
\end{equation}
以及复共轭
\begin{equation}
    \phi ^{\dagger}(x)=\int{\frac{\mathrm{d}^3p}{\left( 2\pi \right) ^3}}\frac{1}{\sqrt{2E_{\mathbf{p}}}}\left( b_{\mathbf{p}}\mathrm{e}^{-\mathrm{i}p\cdot x}+a_{\mathbf{p}}^{\dagger}\mathrm{e}^{\mathrm{i}p\cdot x} \right) 
\end{equation}
得到宇称变换后的标量场
\begin{equation}
    \phi (\mathcal{P} x)=\int{\frac{\mathrm{d}^3p}{\left( 2\pi \right) ^3}}\frac{1}{\sqrt{2E_{\mathbf{p}}}}\left( a_{\mathbf{p}}\mathrm{e}^{-\mathrm{i}p\cdot \left( \mathcal{P} x \right)}+b_{\mathbf{p}}^{\dagger}\mathrm{e}^{\mathrm{i}p\cdot \left( \mathcal{P} x \right)} \right) 
\end{equation}
以及
\begin{equation}
    \phi ^{\dagger}(\mathcal{P} x)=\int{\frac{\mathrm{d}^3p}{\left( 2\pi \right) ^3}}\frac{1}{\sqrt{2E_{\mathbf{p}}}}\left( b_{\mathbf{p}}\mathrm{e}^{-\mathrm{i}p\cdot \left( \mathcal{P} x \right)}+a_{\mathbf{p}}^{\dagger}\mathrm{e}^{\mathrm{i}p\cdot \left( \mathcal{P} x \right)} \right) 
\end{equation}

2.计算
\begin{equation}
    \begin{aligned}
        P^{-1}\phi (x)P&=\int{\frac{\mathrm{d}^3p}{\left( 2\pi \right) ^3}}\frac{1}{\sqrt{2E_{\mathbf{p}}}}\left( P^{-1}a_{\mathbf{p}}P\mathrm{e}^{-\mathrm{i}p\cdot x}+P^{-1}b_{\mathbf{p}}^{\dagger}P\mathrm{e}^{\mathrm{i}p\cdot x} \right) 
\\
&=\int{\frac{\mathrm{d}^3p}{\left( 2\pi \right) ^3}}\frac{1}{\sqrt{2E_{\mathbf{p}}}}\left( \eta _{P}^{*}a_{-\mathbf{p}}\mathrm{e}^{-\mathrm{i}p\cdot x}+\tilde{\eta}_Pb_{-\mathbf{p}}^{\dagger}\mathrm{e}^{\mathrm{i}p\cdot x} \right) 
\\
&=\int{\frac{\mathrm{d}^3p}{\left( 2\pi \right) ^3}}\frac{1}{\sqrt{2E_{\mathbf{p}}}}\left( \eta _{P}^{*}a_{\mathbf{p}}\mathrm{e}^{-\mathrm{i}\left( \mathcal{P} p \right) \cdot x}+\tilde{\eta}_Pb_{\mathbf{p}}^{\dagger}\mathrm{e}^{\mathrm{i}\left( \mathcal{P} p \right) \cdot x} \right) 
\\
&=\int{\frac{\mathrm{d}^3p}{\left( 2\pi \right) ^3}}\frac{1}{\sqrt{2E_{\mathbf{p}}}}\left( \eta _{P}^{*}a_{\mathbf{p}}\mathrm{e}^{-\mathrm{i}p\cdot \left( \mathcal{P} x \right)}+\tilde{\eta}_Pb_{\mathbf{p}}^{\dagger}\mathrm{e}^{\mathrm{i}p\cdot \left( \mathcal{P} x \right)} \right) 
\\
&=\int{\frac{\mathrm{d}^3p}{\left( 2\pi \right) ^3}}\frac{1}{\sqrt{2E_{\mathbf{p}}}}\left( \eta _{P}^{*}a_{\mathbf{p}}\mathrm{e}^{-\mathrm{i}p\cdot \left( \mathcal{P} x \right)}+\eta _{P}^{*}b_{\mathbf{p}}^{\dagger}\mathrm{e}^{\mathrm{i}p\cdot \left( \mathcal{P} x \right)} \right) 
\\
&=\eta _{P}^{*}\int{\frac{\mathrm{d}^3p}{\left( 2\pi \right) ^3}}\frac{1}{\sqrt{2E_{\mathbf{p}}}}\left( a_{\mathbf{p}}\mathrm{e}^{-\mathrm{i}p\cdot \left( \mathcal{P} x \right)}+b_{\mathbf{p}}^{\dagger}\mathrm{e}^{\mathrm{i}p\cdot \left( \mathcal{P} x \right)} \right) 
    \end{aligned}
\end{equation}
得到
\begin{equation}
    P^{-1}\phi (x)P=\eta _{P}^{*}\phi (\mathcal{P} x)
\end{equation}
同样地
\begin{equation}
    \begin{aligned}
        P^{-1}\phi ^{\dagger}(x)P&=\int{\frac{\mathrm{d}^3p}{\left( 2\pi \right) ^3}}\frac{1}{\sqrt{2E_{\mathbf{p}}}}\left( P^{-1}b_{\mathbf{p}}P\mathrm{e}^{-\mathrm{i}p\cdot x}+P^{-1}a_{\mathbf{p}}^{\dagger}P\mathrm{e}^{\mathrm{i}p\cdot x} \right) 
\\
&=\int{\frac{\mathrm{d}^3p}{\left( 2\pi \right) ^3}}\frac{1}{\sqrt{2E_{\mathbf{p}}}}\left( \tilde{\eta}_{P}^{*}b_{-\mathbf{p}}\mathrm{e}^{-\mathrm{i}p\cdot x}+\eta _Pa_{-\mathbf{p}}^{\dagger}\mathrm{e}^{\mathrm{i}p\cdot x} \right) 
\\
&=\int{\frac{\mathrm{d}^3p}{\left( 2\pi \right) ^3}}\frac{1}{\sqrt{2E_{\mathbf{p}}}}\left( \tilde{\eta}_{P}^{*}b_{\mathbf{p}}\mathrm{e}^{-\mathrm{i}\left( \mathcal{P} p \right) \cdot x}+\eta _Pa_{\mathbf{p}}^{\dagger}\mathrm{e}^{\mathrm{i}\left( \mathcal{P} p \right) \cdot x} \right) 
\\
&=\int{\frac{\mathrm{d}^3p}{\left( 2\pi \right) ^3}}\frac{1}{\sqrt{2E_{\mathbf{p}}}}\left( \tilde{\eta}_{P}^{*}b_{\mathbf{p}}\mathrm{e}^{-\mathrm{i}p\cdot \left( \mathcal{P} x \right)}+\eta _Pa_{\mathbf{p}}^{\dagger}\mathrm{e}^{\mathrm{i}p\cdot \left( \mathcal{P} x \right)} \right) 
\\
&=\int{\frac{\mathrm{d}^3p}{\left( 2\pi \right) ^3}}\frac{1}{\sqrt{2E_{\mathbf{p}}}}\left( \eta _Pb_{\mathbf{p}}\mathrm{e}^{-\mathrm{i}p\cdot \left( \mathcal{P} x \right)}+\eta _Pa_{\mathbf{p}}^{\dagger}\mathrm{e}^{\mathrm{i}p\cdot \left( \mathcal{P} x \right)} \right) 
\\
&=\eta _P\int{\frac{\mathrm{d}^3p}{\left( 2\pi \right) ^3}}\frac{1}{\sqrt{2E_{\mathbf{p}}}}\left( b_{\mathbf{p}}\mathrm{e}^{-\mathrm{i}p\cdot \left( \mathcal{P} x \right)}+a_{\mathbf{p}}^{\dagger}\mathrm{e}^{\mathrm{i}p\cdot \left( \mathcal{P} x \right)} \right) 
    \end{aligned}
\end{equation}
得到
\begin{equation}
    P^{-1}\phi ^{\dagger}(x)P=\eta _P\phi ^{\dagger}(\mathcal{P} x)
\end{equation}

\subsubsection{推导:9.23,9.24}
1
\begin{equation}
    \begin{aligned}
        \partial _{x,\mu}&=\frac{\partial}{\partial x^{\mu}}
\\
&=\frac{\partial \left( {\mathcal{P} ^{\nu}}_{\rho}x^{\rho} \right)}{\partial x^{\mu}}\frac{\partial}{\partial \left( {\mathcal{P} ^{\nu}}_{\rho}x^{\rho} \right)}
\\
&={\mathcal{P} ^{\nu}}_{\rho}\frac{\partial x^{\rho}}{\partial x^{\mu}}\frac{\partial}{\partial (\mathcal{P} x)^{\nu}}
\\
&={\mathcal{P} ^{\nu}}_{\rho}{\delta ^{\rho}}_{\mu}\frac{\partial}{\partial (\mathcal{P} x)^{\nu}}
\\
&={\mathcal{P} ^{\nu}}_{\mu}\frac{\partial}{\partial (\mathcal{P} x)^{\nu}}
\\
&=(\mathcal{P} ^{-1}{)^{\nu}}_{\mu}\frac{\partial}{\partial (\mathcal{P} x)^{\nu}}
\\
&=(\mathcal{P} ^{-1}{)^{\nu}}_{\mu}\partial _{\mathcal{P} x,\nu}
    \end{aligned}
\end{equation}
2
\begin{equation}
    \begin{aligned}
        \partial _{x}^{\mu}&=\frac{\partial}{\partial x_{\mu}}
\\
&=\frac{\partial \left( (\mathcal{P} ^{-1}{)^{\rho}}_{\nu}x_{\rho} \right)}{\partial x_{\mu}}\frac{\partial}{\partial \left( (\mathcal{P} ^{-1}{)^{\rho}}_{\nu}x_{\rho} \right)}
\\
&=(\mathcal{P} ^{-1}{)^{\rho}}_{\nu}\frac{\partial x_{\rho}}{\partial x_{\mu}}\frac{\partial}{\partial (\mathcal{P} x)_{\nu}}
\\
&=(\mathcal{P} ^{-1}{)^{\rho}}_{\nu}{\delta ^{\mu}}_{\rho}\frac{\partial}{\partial (\mathcal{P} x)_{\nu}}
\\
&=(\mathcal{P} ^{-1}{)^{\mu}}_{\nu}\frac{\partial}{\partial (\mathcal{P} x)_{\nu}}
\\
&={\mathcal{P} ^{\mu}}_{\nu}\frac{\partial}{\partial (\mathcal{P} x)_{\nu}}
\\
&={\mathcal{P} ^{\mu}}_{\nu}\partial _{\mathcal{P} x}^{\nu}
    \end{aligned}
\end{equation}



%%%%%%%%%%%%%%%%%%%%%%%%%%%%%%%%%%%%%%%%%%%%%%%%%%%
\subsection{标量场的T变换}

\subsubsection{笔记:时间反演变换}
时间反演变换
\begin{equation}
    {\mathcal{T} ^{\mu}}_{\nu}=(\mathcal{T} ^{-1}{)^{\mu}}_{\nu}=\left( \begin{matrix}
	-1&		&		&		\\
	&		+1&		&		\\
	&		&		+1&		\\
	&		&		&		+1\\
\end{matrix} \right) 
\end{equation}
时空坐标变换
\begin{equation}
    x^{\mu}=\left( t,\mathbf{x} \right) \Rightarrow \,\,x^{\prime \mu}={\mathcal{T} ^{\mu}}_{\nu}x^{\nu}=(\mathcal{T} x)^{\mu}=\left( -t,\mathbf{x} \right) 
\end{equation}
四维动量变换
\begin{equation}
    p^{\mu}=\left( E,\mathbf{p} \right) \Rightarrow \,\,p^{\prime \mu}={\mathcal{T} ^{\mu}}_{\nu}p^{\nu}=(\mathcal{T} p)^{\mu}=\left( -E,\mathbf{p} \right) 
\end{equation}
时空导数变换
\begin{equation}
    \partial _{\mu}^{\prime}=(\mathcal{T} ^{-1}{)^{\nu}}_{\mu}\partial _{\nu}
\end{equation}
时间反演变换保持时空体积元不变
\begin{equation}
    \mathrm{d}^4x^{\prime}=\left| \det\mathrm{(}\mathcal{T} ) \right|\mathrm{d}^4x=\mathrm{d}^4x
\end{equation}


\subsubsection{推导:9.63}
1.由标量场
\begin{equation}
    \phi (x)=\int{\frac{\mathrm{d}^3p}{\left( 2\pi \right) ^3}\frac{1}{\sqrt{2E_{\mathbf{p}}}}\left( a_{\mathbf{p}}\mathrm{e}^{-\mathrm{i}p\cdot x}+b_{\mathbf{p}}^{\dagger}\mathrm{e}^{\mathrm{i}p\cdot x} \right)}
\end{equation}
以及复共轭
\begin{equation}
    \phi ^{\dagger}(x)=\int{\frac{\mathrm{d}^3p}{\left( 2\pi \right) ^3}}\frac{1}{\sqrt{2E_{\mathbf{p}}}}\left( b_{\mathbf{p}}\mathrm{e}^{-\mathrm{i}p\cdot x}+a_{\mathbf{p}}^{\dagger}\mathrm{e}^{\mathrm{i}p\cdot x} \right) 
\end{equation}
得到时间反演变换后的标量场
\begin{equation}
    \phi (\mathcal{T} x)=\int{\frac{\mathrm{d}^3p}{\left( 2\pi \right) ^3}\frac{1}{\sqrt{2E_{\mathbf{p}}}}\left( a_{\mathbf{p}}\mathrm{e}^{-\mathrm{i}p\cdot \left( \mathcal{T} x \right)}+b_{\mathbf{p}}^{\dagger}\mathrm{e}^{\mathrm{i}p\cdot \left( \mathcal{T} x \right)} \right)}
\end{equation}
以及
\begin{equation}
    \phi ^{\dagger}(\mathcal{T} x)=\int{\frac{\mathrm{d}^3p}{\left( 2\pi \right) ^3}}\frac{1}{\sqrt{2E_{\mathbf{p}}}}\left( b_{\mathbf{p}}\mathrm{e}^{-\mathrm{i}p\cdot \left( \mathcal{T} x \right)}+a_{\mathbf{p}}^{\dagger}\mathrm{e}^{\mathrm{i}p\cdot \left( \mathcal{T} x \right)} \right) 
\end{equation}
2.计算
\begin{equation}
    \begin{aligned}
        T^{-1}\phi \left( x \right) T&=\int{\frac{\mathrm{d}^3p}{\left( 2\pi \right) ^3}\frac{1}{\sqrt{2E_{\mathbf{p}}}}T^{-1}\left( a_{\mathbf{p}}\mathrm{e}^{-\mathrm{i}p\cdot x}+b_{\mathbf{p}}^{\dagger}\mathrm{e}^{\mathrm{i}p\cdot x} \right)}T
\\
&=\int{\frac{\mathrm{d}^3p}{\left( 2\pi \right) ^3}}\frac{1}{\sqrt{2E_{\mathbf{p}}}}\left( T^{-1}a_{\mathbf{p}}TT^{-1}\mathrm{e}^{-\mathrm{i}p\cdot x}T+T^{-1}b_{\mathbf{p}}^{\dagger}TT^{-1}\mathrm{e}^{\mathrm{i}p\cdot x}T \right) 
\\
&=\int{\frac{\mathrm{d}^3p}{\left( 2\pi \right) ^3}}\frac{1}{\sqrt{2E_{\mathbf{p}}}}\left( \eta _{T}^{*}a_{-\mathbf{p}}\mathrm{e}^{\mathrm{i}p\cdot x}+\tilde{\eta}_Tb_{-\mathbf{p}}^{\dagger}\mathrm{e}^{-\mathrm{i}p\cdot x} \right) 
\\
&=\int{\frac{\mathrm{d}^3p}{\left( 2\pi \right) ^3}}\frac{1}{\sqrt{2E_{\mathbf{p}}}}\left( \eta _{T}^{*}a_{\mathbf{p}}\mathrm{e}^{\mathrm{i}\left( \mathcal{P} p \right) \cdot x}+\tilde{\eta}_Tb_{\mathbf{p}}^{\dagger}\mathrm{e}^{-\mathrm{i}\left( \mathcal{P} p \right) \cdot x} \right) 
\\
&=\int{\frac{\mathrm{d}^3p}{\left( 2\pi \right) ^3}}\frac{1}{\sqrt{2E_{\mathbf{p}}}}\left( \eta _{T}^{*}a_{\mathbf{p}}\mathrm{e}^{\mathrm{i}p\cdot \left( \mathcal{P} x \right)}+\tilde{\eta}_Tb_{\mathbf{p}}^{\dagger}\mathrm{e}^{-\mathrm{i}p\cdot \left( \mathcal{P} x \right)} \right) 
\\
&=\int{\frac{\mathrm{d}^3p}{\left( 2\pi \right) ^3}}\frac{1}{\sqrt{2E_{\mathbf{p}}}}\left( \eta _{T}^{*}a_{\mathbf{p}}\mathrm{e}^{-\mathrm{i}p\cdot (\mathcal{T} x)}+\tilde{\eta}_Tb_{\mathbf{p}}^{\dagger}\mathrm{e}^{\mathrm{i}p\cdot (\mathcal{T} x)} \right) 
\\
&=\int{\frac{\mathrm{d}^3p}{\left( 2\pi \right) ^3}}\frac{1}{\sqrt{2E_{\mathbf{p}}}}\left( \eta _{T}^{*}a_{\mathbf{p}}\mathrm{e}^{-\mathrm{i}p\cdot (\mathcal{T} x)}+\eta _{T}^{*}b_{\mathbf{p}}^{\dagger}\mathrm{e}^{\mathrm{i}p\cdot (\mathcal{T} x)} \right) 
\\
&=\eta _{T}^{*}\int{\frac{\mathrm{d}^3p}{\left( 2\pi \right) ^3}}\frac{1}{\sqrt{2E_{\mathbf{p}}}}\left( a_{\mathbf{p}}\mathrm{e}^{-\mathrm{i}p\cdot (\mathcal{T} x)}+b_{\mathbf{p}}^{\dagger}\mathrm{e}^{\mathrm{i}p\cdot (\mathcal{T} x)} \right) 
    \end{aligned}
\end{equation}
对比得到

同样地
计算
\begin{equation}
    \begin{aligned}
        T^{-1}\phi ^{\dagger}\left( x \right) T&=\int{\frac{\mathrm{d}^3p}{\left( 2\pi \right) ^3}}\frac{1}{\sqrt{2E_{\mathbf{p}}}}T^{-1}\left( b_{\mathbf{p}}\mathrm{e}^{-\mathrm{i}p\cdot x}+a_{\mathbf{p}}^{\dagger}\mathrm{e}^{\mathrm{i}p\cdot x} \right) T
\\
&=\int{\frac{\mathrm{d}^3p}{\left( 2\pi \right) ^3}}\frac{1}{\sqrt{2E_{\mathbf{p}}}}\left( T^{-1}b_{\mathbf{p}}TT^{-1}\mathrm{e}^{-\mathrm{i}p\cdot x}T+T^{-1}a_{\mathbf{p}}^{\dagger}TT^{-1}\mathrm{e}^{\mathrm{i}p\cdot x}T \right) 
\\
&=\int{\frac{\mathrm{d}^3p}{\left( 2\pi \right) ^3}}\frac{1}{\sqrt{2E_{\mathbf{p}}}}\left( \tilde{\eta}_{T}^{*}b_{-\mathbf{p}}\mathrm{e}^{\mathrm{i}p\cdot x}+\eta _Ta_{-\mathbf{p}}^{\dagger}\mathrm{e}^{-\mathrm{i}p\cdot x} \right) 
\\
&=\int{\frac{\mathrm{d}^3p}{\left( 2\pi \right) ^3}}\frac{1}{\sqrt{2E_{\mathbf{p}}}}\left( \tilde{\eta}_{T}^{*}b_{\mathbf{p}}\mathrm{e}^{\mathrm{i}\left( \mathcal{P} p \right) \cdot x}+\eta _Ta_{\mathbf{p}}^{\dagger}\mathrm{e}^{-\mathrm{i}\left( \mathcal{P} p \right) \cdot x} \right) 
\\
&=\int{\frac{\mathrm{d}^3p}{\left( 2\pi \right) ^3}}\frac{1}{\sqrt{2E_{\mathbf{p}}}}\left( \tilde{\eta}_{T}^{*}b_{\mathbf{p}}\mathrm{e}^{\mathrm{i}p\cdot \left( \mathcal{P} x \right)}+\eta _Ta_{\mathbf{p}}^{\dagger}\mathrm{e}^{-\mathrm{i}p\cdot \left( \mathcal{P} x \right)} \right) 
\\
&=\int{\frac{\mathrm{d}^3p}{\left( 2\pi \right) ^3}}\frac{1}{\sqrt{2E_{\mathbf{p}}}}\left( \tilde{\eta}_{T}^{*}b_{\mathbf{p}}\mathrm{e}^{-\mathrm{i}p\cdot (\mathcal{T} x)}+\eta _Ta_{\mathbf{p}}^{\dagger}\mathrm{e}^{\mathrm{i}p\cdot (\mathcal{T} x)} \right) 
\\
&=\int{\frac{\mathrm{d}^3p}{\left( 2\pi \right) ^3}}\frac{1}{\sqrt{2E_{\mathbf{p}}}}\left( \eta _Tb_{\mathbf{p}}\mathrm{e}^{-\mathrm{i}p\cdot (\mathcal{T} x)}+\eta _Ta_{\mathbf{p}}^{\dagger}\mathrm{e}^{\mathrm{i}p\cdot (\mathcal{T} x)} \right) 
\\
&=\eta _T\int{\frac{\mathrm{d}^3p}{\left( 2\pi \right) ^3}}\frac{1}{\sqrt{2E_{\mathbf{p}}}}\left( b_{\mathbf{p}}\mathrm{e}^{-\mathrm{i}p\cdot (\mathcal{T} x)}+a_{\mathbf{p}}^{\dagger}\mathrm{e}^{\mathrm{i}p\cdot (\mathcal{T} x)} \right) 
    \end{aligned}
\end{equation}
对比得到
\begin{equation}
    T^{-1}\phi ^{\dagger}\left( x \right) T=\eta _T\phi ^{\dagger}(\mathcal{T} x)
\end{equation}




%%%%%%%%%%%%%%%%%%%%%%%%%%%%%%%%%%%%%%%%%%%%%%%%%%%
\subsection{标量场的C变换}



\subsubsection{推导:}
1.由标量场
\begin{equation}
    \phi \left( x \right) =\int{\frac{\mathrm{d}^3p}{\left( 2\pi \right) ^3}}\frac{1}{\sqrt{2E_{\mathbf{p}}}}\left( a_{\mathbf{p}}\mathrm{e}^{-\mathrm{i}p\cdot x}+b_{\mathbf{p}}^{\dagger}\mathrm{e}^{\mathrm{i}p\cdot x} \right) 
\end{equation}
和复共轭
\begin{equation}
    \phi ^{\dagger}(x)=\int{\frac{\mathrm{d}^3p}{\left( 2\pi \right) ^3}}\frac{1}{\sqrt{2E_{\mathbf{p}}}}\left( b_{\mathbf{p}}\mathrm{e}^{-\mathrm{i}p\cdot x}+a_{\mathbf{p}}^{\dagger}\mathrm{e}^{\mathrm{i}p\cdot x} \right) 
\end{equation}

2.计算
\begin{equation}
    \begin{aligned}
        C^{-1}\phi C&=\int{\frac{\mathrm{d}^3p}{\left( 2\pi \right) ^3}}\frac{1}{\sqrt{2E_{\mathbf{p}}}}C^{-1}\left( a_{\mathbf{p}}\mathrm{e}^{-\mathrm{i}p\cdot x}+b_{\mathbf{p}}^{\dagger}\mathrm{e}^{\mathrm{i}p\cdot x} \right) C
\\
&=\int{\frac{\mathrm{d}^3p}{\left( 2\pi \right) ^3}}\frac{1}{\sqrt{2E_{\mathbf{p}}}}\left( C^{-1}a_{\mathbf{p}}C\mathrm{e}^{-\mathrm{i}p\cdot x}+C^{-1}b_{\mathbf{p}}^{\dagger}C\mathrm{e}^{\mathrm{i}p\cdot x} \right) 
\\
&=\int{\frac{\mathrm{d}^3p}{\left( 2\pi \right) ^3}}\frac{1}{\sqrt{2E_{\mathbf{p}}}}\left( \eta _{C}^{*}b_{\mathbf{p}}\mathrm{e}^{-\mathrm{i}p\cdot x}+\eta _{C}^{*}a_{\mathbf{p}}^{\dagger}\mathrm{e}^{\mathrm{i}p\cdot x} \right) 
\\
&=\eta _{C}^{*}\int{\frac{\mathrm{d}^3p}{\left( 2\pi \right) ^3}}\frac{1}{\sqrt{2E_{\mathbf{p}}}}\left( b_{\mathbf{p}}\mathrm{e}^{-\mathrm{i}p\cdot x}+a_{\mathbf{p}}^{\dagger}\mathrm{e}^{\mathrm{i}p\cdot x} \right) 
    \end{aligned}
\end{equation}
对比得到
\begin{equation}
    C^{-1}\phi (x)C=\eta _{C}^{*}\phi ^{\dagger}(x)
\end{equation}
同样地
\begin{equation}
    \begin{aligned}
        C^{-1}\phi ^{\dagger}(x)C&=\int{\frac{\mathrm{d}^3p}{\left( 2\pi \right) ^3}}\frac{1}{\sqrt{2E_{\mathbf{p}}}}C^{-1}\left( b_{\mathbf{p}}\mathrm{e}^{-\mathrm{i}p\cdot x}+a_{\mathbf{p}}^{\dagger}\mathrm{e}^{\mathrm{i}p\cdot x} \right) C
\\
&=\int{\frac{\mathrm{d}^3p}{\left( 2\pi \right) ^3}}\frac{1}{\sqrt{2E_{\mathbf{p}}}}\left( C^{-1}b_{\mathbf{p}}C\mathrm{e}^{-\mathrm{i}p\cdot x}+C^{-1}a_{\mathbf{p}}^{\dagger}C\mathrm{e}^{\mathrm{i}p\cdot x} \right) 
\\
&=\int{\frac{\mathrm{d}^3p}{\left( 2\pi \right) ^3}}\frac{1}{\sqrt{2E_{\mathbf{p}}}}\left( \eta _Ca_{\mathbf{p}}\mathrm{e}^{-\mathrm{i}p\cdot x}+\eta _Cb_{\mathbf{p}}^{\dagger}\mathrm{e}^{\mathrm{i}p\cdot x} \right) 
\\
&=\eta _C\int{\frac{\mathrm{d}^3p}{\left( 2\pi \right) ^3}}\frac{1}{\sqrt{2E_{\mathbf{p}}}}\left( a_{\mathbf{p}}\mathrm{e}^{-\mathrm{i}p\cdot x}+b_{\mathbf{p}}^{\dagger}\mathrm{e}^{\mathrm{i}p\cdot x} \right) 
    \end{aligned}
\end{equation}
对比得到
\begin{equation}
    C^{-1}\phi ^{\dagger}(x)C=\eta _C\phi (x)
\end{equation}







