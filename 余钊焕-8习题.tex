\section{习题8}

\newpage
\subsection{8.1}
推出以下公式。

(a) $\text{tr}(\slash{p}\gamma^\mu) = 4p^\mu$。

(b) $\text{tr}(\slash{p}\slash{k}\slash{q}\gamma^\mu) = 4[q^\mu(p \cdot k) - k^\mu(p \cdot q) + p^\mu(k \cdot q)]$。

(c)
$$
\text{tr}(\gamma^\mu\gamma^\nu\gamma^\rho\gamma^\sigma\gamma^\tau\gamma^\phi) = 4g^{\mu\nu}(g^{\rho\sigma}g^{\tau\phi} - g^{\rho\tau}g^{\sigma\phi} + g^{\rho\phi}g^{\sigma\tau}) - 4g^{\mu\rho}(g^{\nu\sigma}g^{\tau\phi} - g^{\nu\tau}g^{\sigma\phi} + g^{\nu\phi}g^{\sigma\tau})
$$
$$
+ 4g^{\mu\sigma}(g^{\nu\rho}g^{\tau\phi} - g^{\nu\tau}g^{\rho\phi} + g^{\nu\phi}g^{\rho\tau}) - 4g^{\mu\tau}(g^{\nu\rho}g^{\sigma\phi} - g^{\nu\sigma}g^{\rho\phi} + g^{\nu\phi}g^{\rho\sigma})
$$
$$
+ 4g^{\mu\phi}(g^{\nu\rho}g^{\sigma\tau} - g^{\nu\sigma}g^{\rho\tau} + g^{\nu\tau}g^{\rho\sigma})。
$$

(d)
$$
\varepsilon^{\alpha\mu\nu\rho}\varepsilon_{\alpha\beta\gamma\delta} = -\delta^\mu_\beta\delta^\nu_\gamma\delta^\rho_\delta - \delta^\mu_\gamma\delta^\nu_\delta\delta^\rho_\beta - \delta^\mu_\delta\delta^\nu_\beta\delta^\rho_\gamma + \delta^\mu_\gamma\delta^\nu_\beta\delta^\rho_\delta + \delta^\mu_\beta\delta^\nu_\delta\delta^\rho_\gamma + \delta^\mu_\delta\delta^\nu_\gamma\delta^\rho_\beta。
$$

\newpage
\subsection{8.2}
在QED 领头阶,Bhabha 散射$e^+e^- \to e^+e^-$和Møller 散射$e^-e^- \to e^-e^-$的Feynman 图分别如(8.312) 和(8.313) 所示。

(a) 忽略电子质量,证明Bhabha 散射的非极化振幅模方为
$$
|\mathcal{M}_B|^2 = 32\pi^2\alpha^2 \left[ u^2 \left( \frac{1}{s} + \frac{1}{t} \right)^2 + \frac{t^2}{s^2} + \frac{s^2}{t^2} \right]。
$$

(b) 利用交叉对称性,求出Møller 散射的非极化振幅模方$|\mathcal{M}_M|^2$。

\newpage
\subsection{8.3}
验证用(8.334) 式表达的$M^{\mu\nu}$ 满足Ward 恒等式$k_{2\nu}M^{\mu\nu}$。

\newpage
\subsection{8.4}
考虑另一种形式的Yukawa 理论,拉氏量为
$$
\mathcal{L} = \frac{1}{2}(\partial_\mu\phi)\partial^\mu\phi - \frac{1}{2} m_\phi^2 \phi^2 + i \bar{\psi}\gamma^\mu\partial_\mu\psi - m_\psi \bar{\psi}\psi - \kappa \phi \bar{\psi} i\gamma^5 \psi,
$$
其中$\phi$ 是实标量场,$\psi$ 是Dirac 旋量场,$\kappa$ 是实耦合常数。

(a) 写出动量空间中的顶点 Feynman 规则。

(b) 设 $m_\phi > 2m_\psi$,画出衰变过程 $\phi \to \psi\bar{\psi}$ 的领头阶 Feynman 图,计算非极化振幅模方 $|\mathcal{M}|^2$ 和衰变宽度 $\Gamma$。

(c) 画出湮灭过程 $\psi\bar{\psi} \to \phi\phi$ 的领头阶 Feynman 图。设 $m_\psi = m_\phi = 0$,计算非极化振幅模方 $|\mathcal{M}|^2$,并在质心系中求出微分散射截面 $d\sigma/d\Omega$。

\newpage
\subsection{8.5}
考虑四费米子相互作用理论,拉氏量为
$$
\mathcal{L} = \sum_{f=\mu, e,\nu_{\mu}, \nu_e} (i\bar{\psi}_f \gamma^\mu \partial_\mu \psi_f - m_f \bar{\psi}_f \psi_f)
$$
$$
- \frac{G_F}{\sqrt{2}} [\bar{\psi}_{\nu_{\mu}} \gamma^\rho (1-\gamma^5) \psi_{\mu} \bar{\psi}_e \gamma_\rho (1-\gamma^5) \psi_{\nu_e} + \bar{\psi}_{\mu} \gamma^\rho (1-\gamma^5) \psi_{\nu_{\mu}} \bar{\psi}_{\nu_e} \gamma_\rho (1-\gamma^5) \psi_e],
$$
其中 Dirac 旋量场 $\psi_{\mu}$、$\psi_e$、$\psi_{\nu_{\mu}}$ 和 $\psi_{\nu_e}$ 分别描述 $\mu$ 子、电子、$\mu$ 子型中微子和电子型中微子,Fermi 常数 $G_F = 1.166 \times 10^{-5} \, \text{GeV}^{-2}$,方括号中两项互为厄米共轭。

(a) 写出动量空间中所有顶点的 Feynman 规则。

(b) 画出三体衰变过程 $\mu^- \to e^- \bar{\nu}_e \nu_{\mu}$ 的领头阶 Feynman 图。设 $m_e = m_{\nu_e} = m_{\nu_{\mu}} = 0$,计算非极化振幅模方 $|\mathcal{M}|^2$ 和衰变宽度 $\Gamma$。

(c) 计算 $\mu$ 子寿命 $\tau = 1/\Gamma$ 的数值,以秒为单位。

\newpage
\subsection{8.6}
考虑标准模型里面电中性矢量玻色子 $Z$ 和带电矢量玻色子 $W^{\pm}$ 与第一代轻子的相互作用,拉氏量为
$$
\mathcal{L}_{ZW} = -\frac{1}{4} Z_{\mu \nu} Z^{\mu \nu} + \frac{1}{2} m_Z^2 Z_{\mu} Z^{\mu} - \frac{1}{2} W_{\mu \nu} W^{+ \mu \nu} + m_W^2 W_{\mu}^{-} W^{+ \mu} + i \bar{\psi}_e \gamma^\mu \partial_\mu \psi_e - m_e \bar{\psi}_e \psi_e + i \bar{\psi}_{\nu_e} \gamma^\mu \partial_\mu \psi_{\nu_e}
$$
$$
- \frac{g}{2 \cos \theta_W} Z_{\mu} [\bar{\psi}_e \gamma^\mu (g_V^e - g_A^e \gamma^5) \psi_e + \bar{\psi}_{\nu_e} \gamma^\mu (g_V^{\nu_e} - g_A^{\nu_e} \gamma^5) \psi_{\nu_e}] - \frac{g}{\sqrt{2}} (W_{\mu}^+ \bar{\psi}_{\nu_e} \gamma^\mu P_L \psi_e + \text{H.c.})
$$
其中,实矢量场 $Z^{\mu}$ 描述质量为 $m_Z = 91.19 \, \text{GeV}$ 的 $Z$ 玻色子;复矢量场 $W^{+ \mu}$ 描述质量为 $m_W = 80.38 \, \text{GeV}$ 的 $W^{\pm}$ 玻色子,且 $W^{- \mu} = (W^{+ \mu})^*$,相应的内外线 Feynman 规则见习题 7.5。场强张量 $Z_{\mu \nu} = \partial_\mu Z_{\nu} - \partial_\nu Z_{\mu}$,$W_{\mu\nu}^{\pm} = \partial_\mu W_{\nu}^{\pm} - \partial_\nu W_{\mu}^{\pm}$。Dirac 旋量场 $\psi_e$ 和 $\psi_{\nu_e}$ 分别描述电子和电子型中微子。$g$ 是实的弱耦合常数,$\theta_w$ 是弱混合角。$g_V^e$、$g_A^e$、$g_V^{\nu_e}$、$g_A^{\nu_e}$ 都是实的无量纲常数。H.c. 代表厄米共轭。

(a) 写出动量空间中所有顶点的 Feynman 规则。

(b) 画出衰变过程 $Z \to e^+ e^-$ 和 $Z \to \nu_e \bar{\nu}_e$ 的领头阶 Feynman 图。计算 $Z \to e^+ e^-$ 的非极化振幅模方 $|\mathcal{M}|^2$ 和衰变分宽度 $\Gamma (Z \to e^+ e^-)$。类比给出衰变分宽度 $\Gamma (Z \to \nu_e \bar{\nu}_e)$。

(c) 画出衰变过程 $W^+ \to e^+ \nu_e$ 的领头阶 Feynman 图,计算非极化振幅模方 $|\mathcal{M}|^2$ 和衰变分宽度 $\Gamma (W^+ \to e^+ \nu_e)$。

(d) 已知
$$
\cos^2 \theta_W = \frac{m_W^2}{m_Z^2}, \quad g = \frac{e}{\sin \theta_W}, \quad g_V^e = -\frac{1}{2} + 2 \sin^2 \theta_W, \quad g_A^e = -\frac{1}{2}, \quad g_V^{\nu_e} = g_A^{\nu_e} = \frac{1}{2}。
$$
计算上述三个衰变分宽度的数值,以 GeV 为单位。

\newpage
\subsection{8.7}
以实标量场 $H(x)$ 描述质量为 $m_H$ 的标准模型 Higgs 玻色子 $H$,考虑拉氏量
$$
\mathcal{L} = \mathcal{L}_{ZW} + \frac{1}{2} (\partial^\mu H) \partial_\mu H - \frac{1}{2} m_H^2 H^2 + \frac{m_Z^2}{v} H Z_\mu Z^\mu + \frac{2m_W^2}{v} H W_\mu^{-} W^{+,\mu},
$$
其中 $\mathcal{L}_{ZW}$ 由 (8.469) 式给出,而 $v = (\sqrt{2} G_F)^{-1/2} = 246.2 \, \text{GeV}$ 是 Higgs 场的真空期望值。

(a) 写出动量空间中 $HZZ$ 和 $HWW$ 顶点的 Feynman 规则。

(b) 设 $m_H > 2m_Z$,画出衰变过程 $H \to ZZ$ 的领头阶 Feynman 图,计算相应的非极化振幅模方 $|\mathcal{M}|^2$ 和衰变分宽度 $\Gamma (H \to ZZ)$。

(c) 设 $m_H > 2m_W$,画出衰变过程 $H \to W^+ W^-$ 的领头阶 Feynman 图,计算相应的非极化振幅模方 $|\mathcal{M}|^2$ 和衰变分宽度 $\Gamma (H \to W^+ W^-)$。

(d) 画出散射过程 $e^+ e^- \to ZH$ 的领头阶 Feynman 图,忽略电子质量 $m_e$,计算相应的非极化振幅模方 $|\mathcal{M}|^2$ 和散射截面 $\sigma$,将 $\sigma$ 表达为质心能 $\sqrt{s}$ 的函数。取 $m_H = 125 \, \text{GeV}$ 和 $\sqrt{s} = 240 \, \text{GeV}$,利用 (8.470) 式计算 $\sigma$ 的数值,以 pb 为单位。