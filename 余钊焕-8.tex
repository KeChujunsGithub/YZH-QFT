\section{8}



\subsection{8.1}





\subsection{8.2}


1.对于Feynman图,根据Feynman 规则,写出散射过程的不变振幅

根据,


求解双线性型的复共轭为

得到,iM 的复共轭为


根据上面的结论

不变振幅的模方为


Casimir 技巧


计算非极化不变振幅模方


\subsection{8.4}


在高能极限下,忽略质量,
\\左手Dirac旋量场$\psi_\mathrm{L}$/左手Weyl旋量场$\eta_\mathrm{L}$ :描述 左旋极化的正费米子 和 右旋极化的反费米子,
\\右手Dirac旋量场$\psi_\mathrm{R}$/右手Weyl旋量场$\eta_\mathrm{R}$ :描述 右旋极化的正费米子 和 左旋极化的反费米子,
\\$\psi_\mathrm{L}$ 和$\psi_\mathrm{R}$成为两个相互独立的场。
\\左手Dirac旋量场$\psi_\mathrm{L}$ 等价于左手Weyl旋量场$\eta_\mathrm{L}$ 
\\右手Dirac旋量场$\psi_\mathrm{R}$ 等价于右手Weyl旋量场$\eta_\mathrm{R}$

左旋极化 $\lambda=-$
右旋极化 $\lambda=+$


\subsection{8.5}

交叉对称性
一个过程包含一个四维动量为$p^\mathrm{\mu}$的粒子$\Phi$的初态,
一个过程包含一个四维动量为$k^\mathrm{\mu}$的反粒子$\bar{\Phi}$的末态,
则这两个过程的不变振幅可以通过动量替换$k^\mu=-p^\mu$联系起来。

一个粒子沿着时间方向运动等价于它的反粒子逆着时间方向运动,这样的反粒子具有负能量和相反动量


\subsection{8.6}

\subsection{$e^{-}\gamma\to e^{-}\gamma$}
Compton 散射:电子与光子的散射过程

s通道的

u通道的

得到总的



\subsection{$e^+e^-\to\gamma\gamma$}





