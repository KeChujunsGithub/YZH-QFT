\section{5.3}







\subsection{笔记}
具有四分量的Dirac旋量场分解为两个二分量Weyl旋量场和

四分量的Dirac旋量场
二分量的Weyl旋量场
左手Weyl旋量场
右手Weyl旋量场

\begin{equation}
    \psi =\left( \begin{array}{c}
	\psi _1\\
	\psi _2\\
	\psi _3\\
	\psi _4\\
\end{array} \right) \Rightarrow \psi =\left( \begin{array}{c}
	\eta _{\mathrm{L}}\\
	\eta _{\mathrm{R}}\\
\end{array} \right) \begin{array}{c}
	\eta _{\mathrm{L}}=\left( \begin{array}{c}
	\eta _{\mathrm{L}1}\\
	\eta _{\mathrm{L}2}\\
\end{array} \right)\\
	\eta _{\mathrm{R}}=\left( \begin{array}{c}
	\eta _{\mathrm{R}1}\\
	\eta _{\mathrm{R}2}\\
\end{array} \right)\\
\end{array}
\end{equation}


\subsection{笔记}



\subsection{推导}
由
\begin{equation}
    \begin{aligned}
        \mathrm{i}\bar{\sigma}^{\mu}\partial _{\mu}\eta _{\mathrm{L}}-m\eta _{\mathrm{R}}&=0
\\
\mathrm{i}\sigma ^{\mu}\partial _{\mu}\eta _{\mathrm{R}}-m\eta _{\mathrm{L}}&=0
    \end{aligned}
\end{equation}
当$$m=0$$
得到
\begin{equation}
    \begin{aligned}
        \mathrm{i}\bar{\sigma}^{\mu}\partial _{\mu}\eta _{\mathrm{L}}&=0
\\
\mathrm{i}\sigma ^{\mu}\partial _{\mu}\eta _{\mathrm{R}}&=0
    \end{aligned}
\end{equation}

