\section{预备知识}

量子场论与粒子物理密切相关。

粒子物理研究物质的基本结构和基本相互作用,组成物质的基本单元是粒子。

自然界中存在 4 种基本相互作用,
\\引力相互作用 (gravitational interaction)、
\\电磁相互作用 (electromagnetic interaction) 、 
\\强相互作用 (strong interaction) 、
\\弱相互作用 (weak interaction),
支配着基本粒子的运动和转化。

基本粒子指尚未发现内部结构的粒子。
目前已发现 3 代基本费米子 ,
每一代包含带电轻子、中微子(中性轻子)、下型夸克、上型夸克各一种。
\\第1代包括电子$(e)$、电子型中微子$(\nu_e)$、下夸克$(d)$和上夸克$(u)$ ;
\\第2代包括$\mu$子$(\mu)$、$\mu$子型中微子$(\nu_\mu)$、奇夸克$(s)$和粲夸克$(c)$;
\\第3代包括$\tau$子$(\tau)$、$\tau$子型中微子$(\nu_\tau)$、 底夸克$(b)$和顶夸克$(t)$。
\\某代某种费米子与它在另一代中相对应的费米子具有相同的量子数,但质量不同。(表格中费米子部分同一列)

夸克的种类也称为味道 (flavor), 6 种味道的夸克$d$、$u$、$s$、$c$、$b$、$t$具有不同的质量。
\\每一味夸克都具有 3 种颜色 (color),
\\同味异色的夸克具有相同质量,严格构成颜色三重态,与描述强相互作用的量子色动力学有关。

多个夸克可以通过强相互作用组成复合粒子、称为强子(hadron),比如,
\\1个介子(meson)由1个正夸克和1个反夸克组成,
\\一个重子(baryon)由3个正夸克或3个反夸克组成。
\\除 3 代中微子以外,其它基本费米子都具有电荷 (electric charge) , 参与电磁相互作用,相应的理论称为量子电动力学。
\\所有基本费米子都参与弱相互作用,它与电磁相互作用统一由电弱规范理论描述,这个理论包含了量子电动力学。电弱规范理论和量子色动力学统称为标准模型 (standard model)。

在标准模型中,费米子的相互作用由一些基本玻色子 (boson) 传递。
\\传递夸克间强相互作用的规范玻色子称为胶子 (gluon),
\\传递电磁相互作用的规范玻色子是光子 (photon),
\\传递弱相互作用的规范玻色子是$W^\pm$和$Z^0$玻色子。
\\此外,还存在一种 Higgs 玻色子,它与电弱规范对称性的自发破缺及基本粒子的质量起源有关。

标准模型是在研究基本粒子如何参与电磁、强、弱相互作用的过程中建立起来的,它的理论
基础就是量子场论。在粒子物理理论和实验发展的过程中,量子场论起着极为关键的作用。反
过来,粒子物理研究也极大地促进了量子场论的发展。

\begin{table}[ht]
\centering
\caption{标准模型粒子分类表}
\begin{tabular}{|l|c|c|c|c|c|}
\hline
 & \textbf{(第一代)费米子} & \textbf{(第二代)费米子} & \textbf{(第三代)费米子} & \textbf{规范玻色子} & \textbf{标量玻色子} \\
\hline
\multirow{4}{*}{\textbf{上型夸克}} 
& \makecell{u夸克 \\ 2.2\,MeV \\ +2/3 \\ 1/2} 
& \makecell{c夸克 \\ 1.27\,GeV \\ +2/3 \\ 1/2} 
& \makecell{t夸克 \\ 173\,GeV \\ +2/3 \\ 1/2} 
& \makecell{光子 ($\gamma$) \\ 0 \\ 0 \\ 1} 
& \multirow{-4}{*}{\makecell{希格斯玻色子 ($H^0$) \\ 125\,GeV \\ 0 \\ 0}} \\
\cline{2-5}

\multirow{4}{*}{\textbf{下型夸克}} 
& \makecell{d夸克 \\ 4.7\,MeV \\ -1/3 \\ 1/2} 
& \makecell{s夸克 \\ 95\,MeV \\ -1/3 \\ 1/2} 
& \makecell{b夸克 \\ 4.18\,GeV \\ -1/3 \\ 1/2} 
& \makecell{$W^\pm$玻色子 \\ 80.4\,GeV \\ $\pm$1 \\ 1} 
& \\ \cline{2-5}

\multirow{4}{*}{\textbf{带电轻子}} 
& \makecell{电子 ($e^-$) \\ 0.511\,MeV \\ -1 \\ 1/2} 
& \makecell{$\mu$子 ($\mu^-$) \\ 105.7\,MeV \\ -1 \\ 1/2} 
& \makecell{$\tau$子 ($\tau^-$) \\ 1.777\,GeV \\ -1 \\ 1/2} 
& \makecell{$Z$玻色子 \\ 91.2\,GeV \\ 0 \\ 1} 
& \\ \cline{2-5}

\multirow{4}{*}{\textbf{中微子}\\ \textbf{中性轻子}} 
& \makecell{$\nu_e$电子中微子 \\ <1\,eV \\ 0 \\ 1/2} 
& \makecell{$\nu_\mu$$\mu$子中微子 \\ <0.1\,eV \\ 0 \\ 1/2} 
& \makecell{$\nu_\tau$$\tau$子中微子 \\ <15.5\,MeV \\ 0 \\ 1/2} 
& \makecell{胶子 (8种) \\ 0 \\ 0 \\ 1} 
& \\ \hline
\end{tabular}
\end{table}