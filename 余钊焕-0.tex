\section{备忘录}

群的线性表示理论
标量场-恒定表示
矢量场-矢量表示
旋量场-旋量表示



\subsection{协变、逆变}
1.四维坐标
\begin{equation}
    \begin{aligned}
        x^{\mu}&=\left( x^0,x^i \right) =\left( x^0,x^1,x^2,x^3 \right) 
\\
&=\left( t,\mathbf{x} \right) =\left( t,x,y,z \right) =\left( ct,x,y,z \right) 
\\
x_{\mu}&=\left( x_0,x_i \right) =\left( x_0,x_1,x_2,x_3 \right) {\color{gray} =\left( x^0,-x^i \right) =\left( x^0,-x^1,-x^2,-x^3 \right) }
\\
&=\left( t,{\color[RGB]{240, 0, 0} -}\mathbf{x} \right) =\left( t,{\color[RGB]{240, 0, 0} -}x,{\color[RGB]{240, 0, 0} -}y,{\color[RGB]{240, 0, 0} -}z \right) =\left( ct,{\color[RGB]{240, 0, 0} -}x,{\color[RGB]{240, 0, 0} -}y,{\color[RGB]{240, 0, 0} -}z \right) 
    \end{aligned}
\end{equation}
2.四维动量
\begin{equation}
    \begin{aligned}
        p^{\mu}&=\left( p^0,p^i \right) =\left( p^0,p^1,p^2,p^3 \right) 
\\
&=\left( E,\mathbf{p} \right) =\left( E,p_x,p_y,p_z \right) =\left( \frac{E}{c},p_x,p_y,p_z \right) 
\\
p_{\mu}&=\left( p_0,p_i \right) =\left( p_0,p_1,p_2,p_3 \right) {\color{gray} =\left( p^0,-p^i \right) =\left( p^0,-p^1,-p^2,-p^3 \right) }
\\
&=\left( E,{\color[RGB]{240, 0, 0} -}\mathbf{p} \right) =\left( E,{\color[RGB]{240, 0, 0} -}p_x,{\color[RGB]{240, 0, 0} -}p_y,{\color[RGB]{240, 0, 0} -}p_z \right) =\left( \frac{E}{c},{\color[RGB]{240, 0, 0} -}p_x,{\color[RGB]{240, 0, 0} -}p_y,{\color[RGB]{240, 0, 0} -}p_z \right) 
    \end{aligned}
\end{equation}
3.四维电磁矢量
\begin{equation}
    \begin{aligned}
        A^{\mu}&=\left( A^0,A^i \right) =\left( A^0,A^1,A^2,A^3 \right) 
\\
&=\left( V,\mathbf{A} \right) =\left( V,A_x,A_y,A_y \right) 
\\
A_{\mu}&=\left( A_0,A_i \right) =\left( A_0,A_1,A_2,A_3 \right) {\color{gray} =\left( A^0,-A^i \right) =\left( A^0,-A^1,-A^2,-A^3 \right) }
\\
&=\left( V,{\color[RGB]{240, 0, 0} -}\mathbf{A} \right) =\left( V,{\color[RGB]{240, 0, 0} -}A_x,{\color[RGB]{240, 0, 0} -}A_y,{\color[RGB]{240, 0, 0} -}A_z \right) 
    \end{aligned}
\end{equation}
4.四维流
\begin{equation}
    \begin{aligned}
        j^{\mu}&=\left( j^0,j^i \right) =\left( j^0,j^1,j^2,j^3 \right) 
\\
&=\left( \rho ,\mathbf{j} \right) =\left( \rho ,j_x,j_y,j_z \right) 
\\
j_{\mu}&=\left( j_0,j_i \right) =\left( j_0,j_1,j_2,j_3 \right) {\color{gray} =\left( j^0,-j^i \right) =\left( j^0,-j^1,-j^2,-j^3 \right) }
\\
&=\left( \rho ,{\color[RGB]{240, 0, 0} -}\mathbf{j} \right) =\left( \rho ,{\color[RGB]{240, 0, 0} -}j_x,{\color[RGB]{240, 0, 0} -}j_y,{\color[RGB]{240, 0, 0} -}j_z \right) 
    \end{aligned}
\end{equation}
5.时空坐标求导
\begin{equation}
    \begin{aligned}
        \partial _{\mu}&=\frac{\partial}{\partial x^{\mu}}
\\
\partial _{\mu}&=\left( \partial _0,\partial _i \right) =\left( \partial _0,\partial _1,\partial _2,\partial _3 \right) 
\\
\frac{\partial}{\partial x^{\mu}}&=\left( \frac{\partial}{\partial x^0},\frac{\partial}{\partial x^i} \right) =\left( \frac{\partial}{\partial x^0},\frac{\partial}{\partial x^1},\frac{\partial}{\partial x^2},\frac{\partial}{\partial x^3} \right) 
\\
&=\left( \frac{\partial}{\partial t},\nabla \right) =\left( \frac{\partial}{\partial t},\frac{\partial}{\partial x},\frac{\partial}{\partial y},\frac{\partial}{\partial z} \right) =\left( \frac{1}{c}\frac{\partial}{\partial t},\nabla \right) =\left( \frac{1}{c}\frac{\partial}{\partial t},\frac{\partial}{\partial x},\frac{\partial}{\partial y},\frac{\partial}{\partial z} \right) 
\\
\partial ^{\mu}&=\frac{\partial}{\partial x_{\mu}}
\\
\partial ^{\mu}&=\left( \partial ^0,\partial ^i \right) =(\partial ^0,\partial ^1,\partial ^2,\partial ^3)
\\
\frac{\partial}{\partial x_{\mu}}&=\left( \frac{\partial}{\partial x_0},\frac{\partial}{\partial x_i} \right) =\left( \frac{\partial}{\partial x_0},\frac{\partial}{\partial x_1},\frac{\partial}{\partial x_2},\frac{\partial}{\partial x_3} \right) {\color{gray} =\left( \frac{\partial}{\partial x^0},-\frac{\partial}{\partial x^i} \right) =\left( \frac{\partial}{\partial x^0},-\frac{\partial}{\partial x^1},-\frac{\partial}{\partial x^2},-\frac{\partial}{\partial x^3} \right) }
\\
&=\left( \frac{\partial}{\partial t},{\color[RGB]{240, 0, 0} -}\nabla \right) =\left( \frac{\partial}{\partial t},{\color[RGB]{240, 0, 0} -}\frac{\partial}{\partial x},{\color[RGB]{240, 0, 0} -}\frac{\partial}{\partial y},{\color[RGB]{240, 0, 0} -}\frac{\partial}{\partial z} \right) =\left( \frac{1}{c}\frac{\partial}{\partial t},{\color[RGB]{240, 0, 0} -}\nabla \right) =\left( \frac{1}{c}\frac{\partial}{\partial t},{\color[RGB]{240, 0, 0} -}\frac{\partial}{\partial x},{\color[RGB]{240, 0, 0} -}\frac{\partial}{\partial y},{\color[RGB]{240, 0, 0} -}\frac{\partial}{\partial z} \right) 
    \end{aligned}
\end{equation}


\subsection{色散关系}
\begin{equation}
    \begin{aligned}
        p^0=E_{\mathbf{p}}&=\sqrt{\left| \mathbf{p} \right|^2+m^2}
\\
&=\sqrt{\mathbf{p}^2+m^2}
    \end{aligned}
\end{equation}


\subsection{质壳条件}
\begin{equation}
    \begin{aligned}
        p^2&=E^2-\left| \mathbf{p} \right|^2=m^2
\\
&=E^2-\mathbf{p}^2
    \end{aligned}
\end{equation}

\subsection{Maxwell 方程}



