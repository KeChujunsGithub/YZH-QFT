\section{9}
%%%%%%%%%%%%%%%%%%%%%%%%%%%%%%%%%%%%%%%%%%%%%%%%%%%%%%%
\subsection{9.1}

宇称变换
\begin{equation}
    {\mathcal{P} ^{\mu}}_{\nu}=(\mathcal{P} ^{-1}{)^{\mu}}_{\nu}=\left( \begin{matrix}
	+1&		&		&		\\
	&		-1&		&		\\
	&		&		-1&		\\
	&		&		&		-1\\
\end{matrix} \right) 
\end{equation}
时空坐标变换
\begin{equation}
    x^{\mu}=\left( t,\mathbf{x} \right) \Rightarrow x^{\prime \mu}={\mathcal{P} ^{\mu}}_{\nu}x^{\nu}=(\mathcal{P} x)^{\mu}=\left( t,-\mathbf{x} \right) 
\end{equation}
四维动量变换
\begin{equation}
    p^{\mu}=\left( E,\mathbf{p} \right) \Rightarrow p^{\prime \mu}={\mathcal{P} ^{\mu}}_{\nu}p^{\nu}=(\mathcal{P} p)^{\mu}=\left( E,-\mathbf{p} \right) 
\end{equation}
时空导数变换
\begin{equation}
    \partial _{\mu}^{\prime}=(\mathcal{P} ^{-1}{)^{\nu}}_{\mu}\partial _{\nu}
\end{equation}
保持时空体积元不变
\begin{equation}
    \mathrm{d}^4x^{\prime}=\left| \det \left( \mathcal{P} \right) \right|\mathrm{d}^4x=\mathrm{d}^4x
\end{equation}

如果场论系统的作用量S在宇称变换下不变,则运动方程的形式也在宇称变换下不变,此时称系统是宇称守恒的,即具有空间反射对称性。


在宇称守恒的量子理论中,宇称变换在 Hilbert 空间中诱导出态矢$|\Psi \rangle$的线性幺正变换
\begin{equation}
    |\Psi ^{\prime}\rangle =U(\mathcal{P} )|\Psi \rangle =P|\Psi \rangle 
\end{equation}





推导:
根据
\begin{equation}
    \phi (x)=\int{\frac{\mathrm{d}^3p}{\left( 2\pi \right) ^3}}\frac{1}{\sqrt{2E_{\mathbf{p}}}}\left( a_{\mathbf{p}}\mathrm{e}^{-\mathrm{i}p\cdot x}+b_{\mathbf{p}}^{\dagger}\mathrm{e}^{\mathrm{i}p\cdot x} \right) 
\end{equation}
得到
\begin{equation}
    \phi (\mathcal{P} x)=\int{\frac{\mathrm{d}^3p}{\left( 2\pi \right) ^3}}\frac{1}{\sqrt{2E_{\mathbf{p}}}}\left( a_{\mathbf{p}}\mathrm{e}^{-\mathrm{i}p\cdot \left( \mathcal{P} x \right)}+b_{\mathbf{p}}^{\dagger}\mathrm{e}^{\mathrm{i}p\cdot \left( \mathcal{P} x \right)} \right) 
\end{equation}
计算
\begin{equation}
    \begin{aligned}
        P^{-1}\phi (x)P&=\int{\frac{\mathrm{d}^3p}{\left( 2\pi \right) ^3}}\frac{1}{\sqrt{2E_{\mathbf{p}}}}\left( P^{-1}a_{\mathbf{p}}P\mathrm{e}^{-\mathrm{i}p\cdot x}+P^{-1}b_{\mathbf{p}}^{\dagger}P\mathrm{e}^{\mathrm{i}p\cdot x} \right) 
\\
&=\int{\frac{\mathrm{d}^3p}{\left( 2\pi \right) ^3}}\frac{1}{\sqrt{2E_{\mathbf{p}}}}\left( \eta _{P}^{*}a_{-\mathbf{p}}\mathrm{e}^{-\mathrm{i}p\cdot x}+\tilde{\eta}_Pb_{-\mathbf{p}}^{\dagger}\mathrm{e}^{\mathrm{i}p\cdot x} \right) 
\\
&=\int{\frac{\mathrm{d}^3p}{\left( 2\pi \right) ^3}}\frac{1}{\sqrt{2E_{\mathbf{p}}}}\left( \eta _{P}^{*}a_{\mathbf{p}}\mathrm{e}^{-\mathrm{i}\left( \mathcal{P} p \right) \cdot x}+\tilde{\eta}_Pb_{\mathbf{p}}^{\dagger}\mathrm{e}^{\mathrm{i}\left( \mathcal{P} p \right) \cdot x} \right) 
\\
&=\int{\frac{\mathrm{d}^3p}{\left( 2\pi \right) ^3}}\frac{1}{\sqrt{2E_{\mathbf{p}}}}\left( \eta _{P}^{*}a_{\mathbf{p}}\mathrm{e}^{-\mathrm{i}p\cdot \left( \mathcal{P} x \right)}+\tilde{\eta}_Pb_{\mathbf{p}}^{\dagger}\mathrm{e}^{\mathrm{i}p\cdot \left( \mathcal{P} x \right)} \right) 
\\
&=\int{\frac{\mathrm{d}^3p}{\left( 2\pi \right) ^3}}\frac{1}{\sqrt{2E_{\mathbf{p}}}}\left( \eta _{P}^{*}a_{\mathbf{p}}\mathrm{e}^{-\mathrm{i}p\cdot \left( \mathcal{P} x \right)}+\eta _{P}^{*}b_{\mathbf{p}}^{\dagger}\mathrm{e}^{\mathrm{i}p\cdot \left( \mathcal{P} x \right)} \right) 
\\
&=\eta _{P}^{*}\int{\frac{\mathrm{d}^3p}{\left( 2\pi \right) ^3}}\frac{1}{\sqrt{2E_{\mathbf{p}}}}\left( a_{\mathbf{p}}\mathrm{e}^{-\mathrm{i}p\cdot \left( \mathcal{P} x \right)}+b_{\mathbf{p}}^{\dagger}\mathrm{e}^{\mathrm{i}p\cdot \left( \mathcal{P} x \right)} \right) 
    \end{aligned}
\end{equation}
得到
\begin{equation}
    P^{-1}\phi (x)P=\eta _{P}^{*}\phi (\mathcal{P} x)
\end{equation}


推导:
根据
\begin{equation}
    \phi ^{\dagger}(x)=\int{\frac{\mathrm{d}^3p}{\left( 2\pi \right) ^3}}\frac{1}{\sqrt{2E_{\mathbf{p}}}}\left( b_{\mathbf{p}}\mathrm{e}^{-\mathrm{i}p\cdot x}+a_{\mathbf{p}}^{\dagger}\mathrm{e}^{\mathrm{i}p\cdot x} \right) 
\end{equation}
得到
\begin{equation}
    \phi ^{\dagger}(\mathcal{P} x)=\int{\frac{\mathrm{d}^3p}{\left( 2\pi \right) ^3}}\frac{1}{\sqrt{2E_{\mathbf{p}}}}\left( b_{\mathbf{p}}\mathrm{e}^{-\mathrm{i}p\cdot \left( \mathcal{P} x \right)}+a_{\mathbf{p}}^{\dagger}\mathrm{e}^{\mathrm{i}p\cdot \left( \mathcal{P} x \right)} \right) 
\end{equation}
计算
\begin{equation}
    \begin{aligned}
        P^{-1}\phi ^{\dagger}(x)P&=\int{\frac{\mathrm{d}^3p}{\left( 2\pi \right) ^3}}\frac{1}{\sqrt{2E_{\mathbf{p}}}}\left( P^{-1}b_{\mathbf{p}}P\mathrm{e}^{-\mathrm{i}p\cdot x}+P^{-1}a_{\mathbf{p}}^{\dagger}P\mathrm{e}^{\mathrm{i}p\cdot x} \right) 
\\
&=\int{\frac{\mathrm{d}^3p}{\left( 2\pi \right) ^3}}\frac{1}{\sqrt{2E_{\mathbf{p}}}}\left( \tilde{\eta}_{P}^{*}b_{-\mathbf{p}}\mathrm{e}^{-\mathrm{i}p\cdot x}+\eta _Pa_{-\mathbf{p}}^{\dagger}\mathrm{e}^{\mathrm{i}p\cdot x} \right) 
\\
&=\int{\frac{\mathrm{d}^3p}{\left( 2\pi \right) ^3}}\frac{1}{\sqrt{2E_{\mathbf{p}}}}\left( \tilde{\eta}_{P}^{*}b_{\mathbf{p}}\mathrm{e}^{-\mathrm{i}\left( \mathcal{P} p \right) \cdot x}+\eta _Pa_{\mathbf{p}}^{\dagger}\mathrm{e}^{\mathrm{i}\left( \mathcal{P} p \right) \cdot x} \right) 
\\
&=\int{\frac{\mathrm{d}^3p}{\left( 2\pi \right) ^3}}\frac{1}{\sqrt{2E_{\mathbf{p}}}}\left( \tilde{\eta}_{P}^{*}b_{\mathbf{p}}\mathrm{e}^{-\mathrm{i}p\cdot \left( \mathcal{P} x \right)}+\eta _Pa_{\mathbf{p}}^{\dagger}\mathrm{e}^{\mathrm{i}p\cdot \left( \mathcal{P} x \right)} \right) 
\\
&=\int{\frac{\mathrm{d}^3p}{\left( 2\pi \right) ^3}}\frac{1}{\sqrt{2E_{\mathbf{p}}}}\left( \eta _Pb_{\mathbf{p}}\mathrm{e}^{-\mathrm{i}p\cdot \left( \mathcal{P} x \right)}+\eta _Pa_{\mathbf{p}}^{\dagger}\mathrm{e}^{\mathrm{i}p\cdot \left( \mathcal{P} x \right)} \right) 
\\
&=\eta _P\int{\frac{\mathrm{d}^3p}{\left( 2\pi \right) ^3}}\frac{1}{\sqrt{2E_{\mathbf{p}}}}\left( b_{\mathbf{p}}\mathrm{e}^{-\mathrm{i}p\cdot \left( \mathcal{P} x \right)}+a_{\mathbf{p}}^{\dagger}\mathrm{e}^{\mathrm{i}p\cdot \left( \mathcal{P} x \right)} \right) 
    \end{aligned}
\end{equation}
得到
\begin{equation}
    P^{-1}\phi ^{\dagger}(x)P=\eta _P\phi ^{\dagger}(\mathcal{P} x)
\end{equation}









