\documentclass[a4paper,12pt]{article}
\usepackage[utf8]{inputenc}
\usepackage{xeCJK}  % 中文支持
\usepackage{amsmath,amsthm,amssymb}  % 数学包
\usepackage{graphicx}  % 插入图片
\usepackage{booktabs}  % 美化表格
\usepackage{hyperref}  % 超链接
\usepackage{enumitem}  % 自定义列表
\usepackage[left=2cm,right=2cm,top=2cm,bottom=2cm]{geometry}  % 页面设置
\usepackage{multirow}
\usepackage{makecell} %优化单元格内内容的排版,支持自动换行、对齐和文本格式化。
\usepackage{array}

\usepackage{booktabs}   % 专业表格线
\usepackage{braket}      % Dirac 符号支持
\usepackage{caption}     % 优化标题样式

\usepackage{simplewick}
\usepackage{simpler-wick} %余钊焕,依赖tikz,可能有点重

\usepackage{slashed}

\setcounter{section}{-1} % 设置节从 0 开始

 
\title{余钊焕-量子场论}
\author{余钊焕}
\date{\today}
 
\begin{document}
 
\maketitle
 
\tableofcontents  % 自动生成目录
\section{备忘录}

群的线性表示理论
标量场-恒定表示
矢量场-矢量表示
旋量场-旋量表示


\subsection{Maxwell 方程}




\section{预备知识}

量子场论与粒子物理密切相关。

粒子物理研究物质的基本结构和基本相互作用,组成物质的基本单元是粒子。

自然界中存在 4 种基本相互作用,
\\引力相互作用 (gravitational interaction)、
\\电磁相互作用 (electromagnetic interaction) 、 
\\强相互作用 (strong interaction) 、
\\弱相互作用 (weak interaction),
支配着基本粒子的运动和转化。

基本粒子指尚未发现内部结构的粒子。
目前已发现 3 代基本费米子 ,
每一代包含带电轻子、中微子(中性轻子)、下型夸克、上型夸克各一种。
\\第1代包括电子$(e)$、电子型中微子$(\nu_e)$、下夸克$(d)$和上夸克$(u)$ ;
\\第2代包括$\mu$子$(\mu)$、$\mu$子型中微子$(\nu_\mu)$、奇夸克$(s)$和粲夸克$(c)$;
\\第3代包括$\tau$子$(\tau)$、$\tau$子型中微子$(\nu_\tau)$、 底夸克$(b)$和顶夸克$(t)$。
\\某代某种费米子与它在另一代中相对应的费米子具有相同的量子数,但质量不同。(表格中费米子部分同一列)

夸克的种类也称为味道 (flavor), 6 种味道的夸克$d$、$u$、$s$、$c$、$b$、$t$具有不同的质量。
\\每一味夸克都具有 3 种颜色 (color),
\\同味异色的夸克具有相同质量,严格构成颜色三重态,与描述强相互作用的量子色动力学有关。

多个夸克可以通过强相互作用组成复合粒子、称为强子(hadron),比如,
\\1个介子(meson)由1个正夸克和1个反夸克组成,
\\一个重子(baryon)由3个正夸克或3个反夸克组成。
\\除 3 代中微子以外,其它基本费米子都具有电荷 (electric charge) , 参与电磁相互作用,相应的理论称为量子电动力学。
\\所有基本费米子都参与弱相互作用,它与电磁相互作用统一由电弱规范理论描述,这个理论包含了量子电动力学。电弱规范理论和量子色动力学统称为标准模型 (standard model)。

在标准模型中,费米子的相互作用由一些基本玻色子 (boson) 传递。
\\传递夸克间强相互作用的规范玻色子称为胶子 (gluon),
\\传递电磁相互作用的规范玻色子是光子 (photon),
\\传递弱相互作用的规范玻色子是$W^\pm$和$Z^0$玻色子。
\\此外,还存在一种 Higgs 玻色子,它与电弱规范对称性的自发破缺及基本粒子的质量起源有关。

标准模型是在研究基本粒子如何参与电磁、强、弱相互作用的过程中建立起来的,它的理论
基础就是量子场论。在粒子物理理论和实验发展的过程中,量子场论起着极为关键的作用。反
过来,粒子物理研究也极大地促进了量子场论的发展。

\begin{table}[ht]
\centering
\caption{标准模型粒子分类表}
\begin{tabular}{|l|c|c|c|c|c|}
\hline
 & \textbf{(第一代)费米子} & \textbf{(第二代)费米子} & \textbf{(第三代)费米子} & \textbf{规范玻色子} & \textbf{标量玻色子} \\
\hline
\multirow{4}{*}{\textbf{上型夸克}} 
& \makecell{u夸克 \\ 2.2\,MeV \\ +2/3 \\ 1/2} 
& \makecell{c夸克 \\ 1.27\,GeV \\ +2/3 \\ 1/2} 
& \makecell{t夸克 \\ 173\,GeV \\ +2/3 \\ 1/2} 
& \makecell{光子 ($\gamma$) \\ 0 \\ 0 \\ 1} 
& \multirow{-4}{*}{\makecell{希格斯玻色子 ($H^0$) \\ 125\,GeV \\ 0 \\ 0}} \\
\cline{2-5}

\multirow{4}{*}{\textbf{下型夸克}} 
& \makecell{d夸克 \\ 4.7\,MeV \\ -1/3 \\ 1/2} 
& \makecell{s夸克 \\ 95\,MeV \\ -1/3 \\ 1/2} 
& \makecell{b夸克 \\ 4.18\,GeV \\ -1/3 \\ 1/2} 
& \makecell{$W^\pm$玻色子 \\ 80.4\,GeV \\ $\pm$1 \\ 1} 
& \\ \cline{2-5}

\multirow{4}{*}{\textbf{带电轻子}} 
& \makecell{电子 ($e^-$) \\ 0.511\,MeV \\ -1 \\ 1/2} 
& \makecell{$\mu$子 ($\mu^-$) \\ 105.7\,MeV \\ -1 \\ 1/2} 
& \makecell{$\tau$子 ($\tau^-$) \\ 1.777\,GeV \\ -1 \\ 1/2} 
& \makecell{$Z$玻色子 \\ 91.2\,GeV \\ 0 \\ 1} 
& \\ \cline{2-5}

\multirow{4}{*}{\textbf{中微子}\\ \textbf{中性轻子}} 
& \makecell{$\nu_e$电子中微子 \\ <1\,eV \\ 0 \\ 1/2} 
& \makecell{$\nu_\mu$$\mu$子中微子 \\ <0.1\,eV \\ 0 \\ 1/2} 
& \makecell{$\nu_\tau$$\tau$子中微子 \\ <15.5\,MeV \\ 0 \\ 1/2} 
& \makecell{胶子 (8种) \\ 0 \\ 0 \\ 1} 
& \\ \hline
\end{tabular}
\end{table}
\include{余钊焕-1.2}
\section{1.3}
\section{1.4}


\subsection{笔记}



\section{1.5}

%%%%%%%%%%%%%%%%%%%%%%%%%%%%%%%%%%%%%%%%%%%%%%%%%%%%%5
\subsection{自然单位制下的麦克斯韦方程组}

总结:自然单位制下的麦克斯韦方程组
\begin{equation}
    \begin{aligned}
        \nabla \cdot \mathbf{E}&=\rho 
\\
\nabla \times \mathbf{B}&=\mathbf{J}+\frac{\partial \mathbf{E}}{\partial t}
\\
\nabla \cdot \mathbf{B}&=0
\\
\nabla \times \mathbf{E}&=-\frac{\partial \mathbf{B}}{\partial t}
    \end{aligned}
\end{equation}

Gauss 定律
\begin{equation}
    \nabla \cdot \mathbf{E}=\rho 
\end{equation}
Ampère 环路定律对应的Ampère-Maxwell 方程
\begin{equation}
    \nabla \times \mathbf{B}=\mathbf{J}+\frac{\partial \mathbf{E}}{\partial t}
\end{equation}
Gauss 磁定律
\begin{equation}
    \nabla \cdot \mathbf{B}=0
\end{equation}
Faraday 电磁感应定律对应的 MaxwellFaraday 方程
\begin{equation}
    \nabla \cdot \mathbf{B}=0
\end{equation}

补充:推导
1.1推导
\begin{equation}
    \nabla \cdot \mathbf{E}=\rho \Rightarrow \partial _{\mu}F^{\mu 0}=J^0
\end{equation}
过程
\begin{equation}
    \begin{aligned}
        \nabla \cdot \mathbf{E}&=\rho 
\\
\partial _iE^i&=J^0
\\
-\partial _iF^{0i}&=J^0
\\
\partial _iF^{i0}&=J^0
\\
\partial _0F^{00}+\partial _iF^{i0}&=J^0
\\
\partial _{\mu}F^{\mu 0}&=J^0
    \end{aligned}
\end{equation}
其中






2.1推导
\begin{equation}
    \nabla \times \mathbf{B}=\mathbf{J}+\frac{\partial \mathbf{E}}{\partial t}\Rightarrow \partial _{\mu}F^{\mu i}=J^i
\end{equation}
过程
\begin{equation}
    \begin{aligned}
        \nabla \times \mathbf{B}&=\mathbf{J}+\frac{\partial \mathbf{E}}{\partial t}
\\
\varepsilon ^{ijk}\partial _jB^k&=J^i+\partial _0E^i
\\
\partial _j\varepsilon ^{ijk}B^k&=J^i+\partial _0E^i
\\
-\partial _jF^{ij}&=J^i-\partial _0F^{0i}
\\
\partial _0F^{0i}-\partial _jF^{ij}&=J^i
\\
\partial _0F^{0i}+\partial _jF^{ji}&=J^i
\\
\partial _{\mu}F^{\mu i}&=J^i
$$

    \end{aligned}
\end{equation}




3.1推导
\begin{equation}
    
\end{equation}
过程
\begin{equation}
    \begin{aligned}
        \nabla \times \mathbf{E}&=-\frac{\partial \mathbf{B}}{\partial t}
\\
\varepsilon ^{kmn}\partial _mE^n&=-\partial _0B^k
\\
-\varepsilon ^{kmn}\partial _mF^{0n}&=\frac{1}{2}\varepsilon ^{kmn}\partial _0F^{mn}
\\
\varepsilon ^{kmn}\partial _mF^{0n}+\frac{1}{2}\varepsilon ^{kmn}\partial _0F^{mn}&=0
\\
\varepsilon ^{kij}\varepsilon ^{kmn}\left( \partial _mF^{0n}+\frac{1}{2}\partial _0F^{mn} \right) &=0
\\
\left( \delta ^{im}\delta ^{jn}-\delta ^{in}\delta ^{jm} \right) \left( \partial _mF^{0n}+\frac{1}{2}\partial _0F^{mn} \right) &=0
\\
\delta ^{im}\delta ^{jn}\partial _mF^{0n}-\delta ^{in}\delta ^{jm}\partial _mF^{0n}+\frac{1}{2}\left( \delta ^{im}\delta ^{jn}\partial _0F^{mn}-\delta ^{in}\delta ^{jm}\partial _0F^{mn} \right) &=0
\\
\partial _iF^{0j}-\partial _jF^{0i}+\frac{1}{2}\left( \partial _0F^{ij}-\partial _0F^{ji} \right) &=0
\\
-\partial _iF^{j0}-\partial _jF^{0i}+\frac{1}{2}\left( \partial _0F^{ij}+\partial _0F^{ij} \right) &=0
\\
\partial _0F^{ij}-\partial _iF^{j0}-\partial _jF^{0i}&=0
\\
\partial ^0F^{ij}+\partial ^iF^{j0}+\partial ^jF^{0i}&=0
    \end{aligned}
\end{equation}







4.1推导
\begin{equation}
    \nabla \times \mathbf{E}=-\frac{\partial \mathbf{B}}{\partial t}\Rightarrow \partial ^0F^{ij}+\partial ^iF^{j0}+\partial ^jF^{0i}=0
\end{equation}

\begin{equation}
    \begin{aligned}
        
    \end{aligned}
\end{equation}



1.2推导
\begin{equation}
    \partial _{\mu}F^{\mu 0}=J^0\Rightarrow \nabla \cdot \mathbf{E}=\rho 
\end{equation}
过程
\begin{equation}
    \begin{aligned}
        \partial _{\mu}F^{\mu 0}&=J^0
\\
\,\partial _0F^{00}+\partial _iF^{i0}&=J^0
\\
\partial _iF^{i0}&=J^0
\\
-\partial _iF^{0i}&=J^0
\\
\partial _iE^i&=J^0
\\
\nabla \cdot \mathbf{E}&=\rho 
    \end{aligned}
\end{equation}
其中
\begin{equation}
    \begin{aligned}
        F^{00}&=0
\\
\partial _0F^{00}&=0
    \end{aligned}
\end{equation}




2.2推导
\begin{equation}
    \partial _{\mu}F^{\mu i}=J^i\Rightarrow \nabla \times \mathbf{B}=\mathbf{J}+\frac{\partial \mathbf{E}}{\partial t}
\end{equation}
过程
\begin{equation}
    \begin{aligned}
        \partial _{\mu}F^{\mu i}&=J^i
\\
\partial _0F^{0i}+\partial _jF^{ji}&=J^i
\\
-\partial _0E^i-\partial _j\varepsilon ^{jik}B^k&=J^i
\\
-\partial _0E^i-\varepsilon ^{jik}\partial _jB^k&=J^i
\\
-\partial _0E^i+\varepsilon ^{ijk}\partial _jB^k&=J^i
\\
\varepsilon ^{ijk}\partial _jB^k&=J^i+\partial _0E^i
\\
\nabla \times \mathbf{B}&=\mathbf{J}+\frac{\partial \mathbf{E}}{\partial t}
    \end{aligned}
\end{equation}




3.2推导
\begin{equation}
    \partial ^iF^{jk}+\partial ^jF^{ki}+\partial ^kF^{ij}=0\Rightarrow \nabla \cdot \mathbf{B}=0
\end{equation}
过程
\begin{equation}
    \begin{aligned}
        \partial ^iF^{jk}+\partial ^jF^{ki}+\partial ^kF^{ij}&=0
\\
-\partial _iF^{jk}-\partial _jF^{ki}-\partial _kF^{ij}&=0
\\
\partial _iF^{jk}+\partial _jF^{ki}+\partial _kF^{ij}&=0
\\
\partial _i\varepsilon ^{jkl}B^l+\partial _j\varepsilon ^{kil}B^l+\partial _k\varepsilon ^{ijl}B^l&=0
\\
\varepsilon ^{jkl}\partial _iB^l+\varepsilon ^{kil}\partial _jB^l+\varepsilon ^{ijl}\partial _kB^l&=0
\\
\varepsilon ^{ijk}\left( \varepsilon ^{jkl}\partial _iB^l+\varepsilon ^{kil}\partial _jB^l+\varepsilon ^{ijl}\partial _kB^l \right) &=0
\\
\varepsilon ^{jki}\varepsilon ^{jkl}\partial _iB^l+\varepsilon ^{kij}\varepsilon ^{kil}\partial _jB^l+\varepsilon ^{ijk}\varepsilon ^{ijl}\partial _kB^l&=0
\\
\delta ^{il}\partial _iB^l+\delta ^{jl}\partial _jB^l+\delta ^{kl}\partial _kB^l&=0
\\
3\partial _lB^l&=0
\\
\partial _iB^i&=0
\\
\nabla \cdot \mathbf{B}&=0
    \end{aligned}
\end{equation}




4.2推导
\begin{equation}
    \partial ^0F^{ij}+\partial ^iF^{j0}+\partial ^jF^{0i}=0\Rightarrow \nabla \times \mathbf{E}=-\frac{\partial \mathbf{B}}{\partial t}
\end{equation}
过程
\begin{equation}
    \begin{aligned}
        
    \end{aligned}
\end{equation}



\section{1.6作用量原理}

%%%%%%%%%%%%%%%%%%%%%%%%%%%%%%%%%%%%%%%%%%%%%%%%%%%%%%%%%%%%%%%%%%%%%%%%%%%%%%%%%%%%%%%%%%%%%%%%%%%%%%%%%%
\subsection{推导:经典力学的运动方程}
1.
对于
\begin{equation}
     S=\int_{t_1}^{t_2}{\mathrm{d}t}L
\end{equation}
两边同时变分
\begin{equation}
     \delta S=\int_{t_1}^{t_2}{\mathrm{d}t}\delta L
\end{equation}
2.对于拉氏量
\begin{equation}
    L=L[q_i(t),\dot{q}_i(t)]
\end{equation}
变分计算为
\begin{equation}
     \delta L=\frac{\partial L}{\partial q_i}\delta q_i+\frac{\partial L}{\partial \dot{q}_i}\delta \dot{q}_i
\end{equation}
得到
\begin{equation}
    \delta S=\int_{t_1}^{t_2}{\mathrm{d}t}\left( \frac{\partial L}{\partial q_i}\delta q_i+\frac{\partial L}{\partial \dot{q}_i}\delta \dot{q}_i \right) 
\end{equation}
3.因为求导和变分可交换
\begin{equation}
     \delta \dot{q}_i=\delta \left( \frac{\mathrm{d}}{\mathrm{d}t}q_i \right) =\frac{\mathrm{d}}{\mathrm{d}t}\left( \delta q_i \right) 
\end{equation}
得到
\begin{equation}
     \begin{aligned}
         \delta S&=\int_{t_1}^{t_2}{\mathrm{d}t}\left( \frac{\partial L}{\partial q_i}\delta q_i+\frac{\partial L}{\partial \dot{q}_i}\delta \left( \frac{\mathrm{d}}{\mathrm{d}t}q_i \right) \right) 
\\
&=\int_{t_1}^{t_2}{\mathrm{d}t}\left( \frac{\partial L}{\partial q_i}\delta q_i+\frac{\partial L}{\partial \dot{q}_i}\frac{\mathrm{d}}{\mathrm{d}t}\left( \delta q_i \right) \right) 
     \end{aligned}
\end{equation}
4.求导法则
\begin{equation}
    \frac{\mathrm{d}}{\mathrm{d}t}\left( \frac{\partial L}{\partial \dot{q}_i}\delta q_i \right) =\left( \frac{\mathrm{d}}{\mathrm{d}t}\frac{\partial L}{\partial \dot{q}_i} \right) \delta q_i+\frac{\partial L}{\partial \dot{q}_i}\frac{\mathrm{d}}{\mathrm{d}t}\left( \delta q_i \right) 
\end{equation}
写出
\begin{equation}
    \frac{\partial L}{\partial \dot{q}_i}\frac{\mathrm{d}}{\mathrm{d}t}\left( \delta q_i \right) =\frac{\mathrm{d}}{\mathrm{d}t}\left( \frac{\partial L}{\partial \dot{q}_i}\delta q_i \right) -\left( \frac{\mathrm{d}}{\mathrm{d}t}\frac{\partial L}{\partial \dot{q}_i} \right) \delta q_i
\end{equation}
得到
\begin{equation}
    \begin{aligned}
        \delta S&=\int_{t_1}^{t_2}{\mathrm{d}t}\left[ \frac{\partial L}{\partial q_i}\delta q_i+\frac{\mathrm{d}}{\mathrm{d}t}\left( \frac{\partial L}{\partial \dot{q}_i}\delta q_i \right) -\left( \frac{\mathrm{d}}{\mathrm{d}t}\frac{\partial L}{\partial \dot{q}_i} \right) \delta q_i \right] 
\\
&=\int_{t_1}^{t_2}{\mathrm{d}t}\left( \frac{\partial L}{\partial q_i}-\frac{\mathrm{d}}{\mathrm{d}t}\frac{\partial L}{\partial \dot{q}_i} \right) \delta q_i+\int_{t_1}^{t_2}{\mathrm{d}t}\frac{\mathrm{d}}{\mathrm{d}t}\left( \frac{\partial L}{\partial \dot{q}_i}\delta q_i \right) 
    \end{aligned}
\end{equation}
5.牛顿莱布尼兹法则
\begin{equation}
    \int_{t_1}^{t_2}{\mathrm{d}t}\frac{\mathrm{d}}{\mathrm{d}t}\left( \frac{\partial L}{\partial \dot{q}_i}\delta q_i \right) =\int_{t_1}^{t_2}{\mathrm{d}\left( \frac{\partial L}{\partial \dot{q}_i}\delta q_i \right)}=\frac{\partial L}{\partial \dot{q}_i}\delta q_i|_{t_1}^{t_2}
\end{equation}
得到
\begin{equation}
    \delta S=\int_{t_1}^{t_2}{\mathrm{d}t}\left( \frac{\partial L}{\partial q_i}-\frac{\mathrm{d}}{\mathrm{d}t}\frac{\partial L}{\partial \dot{q}_i} \right) \delta q_i+\frac{\partial L}{\partial \dot{q}_i}\delta q_i|_{t_1}^{t_2}
\end{equation}
6.
\begin{equation}
    \delta q_i(t_1)=\delta q_i(t_2)=0
\end{equation}
写出
\begin{equation}
    \frac{\partial L}{\partial \dot{q}_i}\delta q_i|_{t_1}^{t_2}=\frac{\partial L}{\partial \dot{q}_i}\delta q_i(t_2)-\frac{\partial L}{\partial \dot{q}_i}\delta q_i(t_1)=0
\end{equation}
得到
\begin{equation}
    \delta S=\int_{t_1}^{t_2}{\mathrm{d}t}\left( \frac{\partial L}{\partial q_i}-\frac{\mathrm{d}}{\mathrm{d}t}\frac{\partial L}{\partial \dot{q}_i} \right) \delta q_i
\end{equation}
7.利用最小作用量原理
\begin{equation}
    \delta S=0
\end{equation}
等式右边积分为零
\begin{equation}
    \int_{t_1}^{t_2}{\mathrm{d}t}\left( \frac{\partial L}{\partial q_i}-\frac{\mathrm{d}}{\mathrm{d}t}\frac{\partial L}{\partial \dot{q}_i} \right) \delta q_i=0
\end{equation}
被积函数为零
\begin{equation}
    \frac{\partial L}{\partial q_i}-\frac{\mathrm{d}}{\mathrm{d}t}\frac{\partial L}{\partial \dot{q}_i}=0
\end{equation}

\newpage
\subsection{推导:经典场论的运动方程}
1.
对于
\begin{equation}
    S=\int{\mathrm{d}^4x}\mathcal{L} 
\end{equation}
两边同时变分
\begin{equation}
    \delta S=\int{\mathrm{d}^4x}\delta \mathcal{L} 
\end{equation}
2.对于拉氏量
\begin{equation}
    \mathcal{L} =\mathcal{L} (\Phi _a,\partial _{\mu}\Phi _a)
\end{equation}
变分计算为
\begin{equation}
    \delta \mathcal{L} =\frac{\partial \mathcal{L}}{\partial \Phi _a}\delta \Phi _a+\frac{\partial \mathcal{L}}{\partial \left( \partial _{\mu}\Phi _a \right)}\delta \left( \partial _{\mu}\Phi _a \right) 
\end{equation}
得到
\begin{equation}
    \delta S=\int{\mathrm{d}^4x\left[ \frac{\partial \mathcal{L}}{\partial \Phi _a}\delta \Phi _a+\frac{\partial \mathcal{L}}{\partial \left( \partial _{\mu}\Phi _a \right)}\delta \left( \partial _{\mu}\Phi _a \right) \right]}
\end{equation}
3.因为可交换
\begin{equation}
    \delta \left( \partial _{\mu}\Phi _a \right) =\partial _{\mu}\left( \delta \Phi _a \right) 
\end{equation}
得到
\begin{equation}
    \delta S=\int{\mathrm{d}^4x}\left[ \frac{\partial \mathcal{L}}{\partial \Phi _a}\delta \Phi _a+\frac{\partial \mathcal{L}}{\partial \left( \partial _{\mu}\Phi _a \right)}\partial _{\mu}\left( \delta \Phi _a \right) \right] 
\end{equation}
4.求导法则
\begin{equation}
    \partial _{\mu}\left( \frac{\partial \mathcal{L}}{\partial \left( \partial _{\mu}\Phi _a \right)}\delta \Phi _a \right) =\frac{\partial \mathcal{L}}{\partial \left( \partial _{\mu}\Phi _a \right)}\partial _{\mu}\left( \delta \Phi _a \right) +\left( \partial _{\mu}\frac{\partial \mathcal{L}}{\partial \left( \partial _{\mu}\Phi _a \right)} \right) \delta \Phi _a
\end{equation}
写出
\begin{equation}
    \frac{\partial \mathcal{L}}{\partial \left( \partial _{\mu}\Phi _a \right)}\partial _{\mu}\left( \delta \Phi _a \right) =\partial _{\mu}\left( \frac{\partial \mathcal{L}}{\partial \left( \partial _{\mu}\Phi _a \right)}\delta \Phi _a \right) -\left( \partial _{\mu}\frac{\partial \mathcal{L}}{\partial \left( \partial _{\mu}\Phi _a \right)} \right) \delta \Phi _a
\end{equation}
得到
\begin{equation}
    \begin{aligned}
        \delta S&=\int{\mathrm{d}^4x}\left[ \frac{\partial \mathcal{L}}{\partial \Phi _a}\delta \Phi _a+\partial _{\mu}\left( \frac{\partial \mathcal{L}}{\partial \left( \partial _{\mu}\Phi _a \right)}\delta \Phi _a \right) -\left( \partial _{\mu}\frac{\partial \mathcal{L}}{\partial \left( \partial _{\mu}\Phi _a \right)} \right) \delta \Phi _a \right] 
\\
&=\int{\mathrm{d}^4x}\left[ \frac{\partial \mathcal{L}}{\partial \Phi _a}-\partial _{\mu}\frac{\partial \mathcal{L}}{\partial \left( \partial _{\mu}\Phi _a \right)} \right] \delta \Phi _a+\int{\mathrm{d}^4x}\partial _{\mu}\left[ \frac{\partial \mathcal{L}}{\partial \left( \partial _{\mu}\Phi _a \right)}\delta \Phi _a \right] 
    \end{aligned}
\end{equation}
5.广义斯托克斯,将四维体积转化为三维面积
\begin{equation}
    \int_{\mathcal{V}}{\mathrm{d}^4x}\partial _{\mu}\left[ \frac{\partial \mathcal{L}}{\partial \left( \partial _{\mu}\Phi _a \right)}\delta \Phi _a \right] =\int_{\mathcal{S}}{\mathrm{d}\sigma _{\mu}}\frac{\partial \mathcal{L}}{\partial \left( \partial _{\mu}\Phi _a \right)}\delta \Phi _a
\end{equation}
得到
\begin{equation}
    \delta S=\int{\mathrm{d}^4x}\left[ \frac{\partial \mathcal{L}}{\partial \Phi _a}-\partial _{\mu}\frac{\partial \mathcal{L}}{\partial \left( \partial _{\mu}\Phi _a \right)} \right] \delta \Phi _a+\int_{\mathcal{S}}{\mathrm{d}\sigma _{\mu}}\frac{\partial \mathcal{L}}{\partial \left( \partial _{\mu}\Phi _a \right)}\delta \Phi _a
\end{equation}
6.全空间积分为0
\begin{equation}
    \int_{\mathcal{S}}{\mathrm{d}\sigma _{\mu}}\frac{\partial \mathcal{L}}{\partial \left( \partial _{\mu}\Phi _a \right)}\delta \Phi _a=0
\end{equation}
得到
\begin{equation}
    \delta S=\int{\mathrm{d}^4x}\left[ \frac{\partial \mathcal{L}}{\partial \Phi _a}-\partial _{\mu}\frac{\partial \mathcal{L}}{\partial \left( \partial _{\mu}\Phi _a \right)} \right] \delta \Phi _a
\end{equation}
7.利用最小作用量原理
\begin{equation}
    \delta S=0
\end{equation}
等式右边积分为零
\begin{equation}
    \int{\mathrm{d}^4x}\left[ \frac{\partial \mathcal{L}}{\partial \Phi _a}-\partial _{\mu}\frac{\partial \mathcal{L}}{\partial \left( \partial _{\mu}\Phi _a \right)} \right] \delta \Phi _a=0
\end{equation}
被积函数为零
\begin{equation}
    \frac{\partial \mathcal{L}}{\partial \Phi _a}-\partial _{\mu}\frac{\partial \mathcal{L}}{\partial \left( \partial _{\mu}\Phi _a \right)}=0
\end{equation}


第二种写法是分部积分











%%%%%%%%%%%%%%%%%%%%%%%%%%%%%%%%%%%%%%%%%%%%%%%%%%%%%5
\subsection{}


















\include{余钊焕-1.7}
\section{习题1}


\newpage
\subsection{1.1}
在自然单位制中,1 GeV$^{-2}$ 等于多少 cm$^2$,也等于多少 cm$^3$/s?

\newpage
\subsection{1.2}
推出绕 $z$ 轴转动 $\theta$ 角的变换公式 (1.35)。验证相应的变换矩阵 (1.36) 满足保度规条件 (1.44)。

\newpage
\subsection{1.3}
设四维动量 $p^\mu$ 满足质壳条件 $p^2 = m^2 \geq 0$,沿 $x$ 轴方向对它作增速变换,得
$$p^0 = \gamma (p^0 - \beta p^1), \quad p^1 = \gamma (p^1 - \beta p^0), \quad p^2 = p^2, \quad p^3 = p^3.$$
(1.262)

证明无论 $p^0 > 0$ 还是 $p^0 < 0$,必有
$$\frac{p^0}{p^0} > 0,$$
(1.263)

即增速变换不能改变 $p^0$ 的符号。

\newpage
\subsection{1.4}
设四维动量 $p^\mu$ 和 $k^\mu$ 满足质壳条件 $p^2 = m_1^2$ 和 $k^2 = m_2^2$,其中 $m_1, m_2, p^0, k^0 > 0$,证明
$$(p + k)^2 \geq (m_1 + m_2)^2, \quad (p - k)^2 \leq (m_1 - m_2)^2.$$

\newpage
\subsection{1.5}
已知绕 $x$ 轴旋转变换 $R_x(\theta)$、绕 $y$ 轴旋转变换 $R_y(\theta)$、沿 $y$ 轴增速变换 $B_y(\xi)$ 和沿 $z$ 轴增速变换 $B_z(\xi)$ 的具体形式为
$$R_x(\theta) =
\begin{pmatrix}
1 & & \\
& 1 & \\
& & \cos\theta & \sin\theta \\
& & -\sin\theta & \cos\theta 
\end{pmatrix},
\quad R_y(\theta) =
\begin{pmatrix}
1 & & \\
& \cos\theta & -\sin\theta \\
& & 1 \\
& & \sin\theta & \cos\theta 
\end{pmatrix},$$
$$B_y(\xi) =
\begin{pmatrix}
\cosh\xi & -\sinh\xi \\
& 1 & \\
-\sinh\xi & \cosh\xi \\
& & 1 
\end{pmatrix},
\quad B_z(\xi) =
\begin{pmatrix}
\cosh\xi & -\sinh\xi \\
& 1 & \\
-\sinh\xi & \cosh\xi 
\end{pmatrix}.$$

(a) 分别推出这四个变换的无穷小变换参数 $\omega_{\mu\nu}$ 的矩阵形式。
(b) 证明 $R_y(\theta_1)R_y(\theta_2) = R_y(\theta_1 + \theta_2)$ 和 $B_y(\xi_1)B_y(\xi_2) = B_y(\xi_1 + \xi_2)$。

\newpage
\subsection{1.6}
证明
$$g^{\mu\nu} g_{\nu\sigma}(\Lambda^{-1})^\sigma = \Lambda^\mu_\nu.$$

\newpage
\subsection{1.7}
设 $S^{\mu\nu}$ 是对称的 Lorentz 张量,$A_{\mu\nu}$ 是反对称的 Lorentz 张量,证明
$$S^{\mu\nu} A_{\mu\nu} = 0.$$

\newpage
\subsection{1.8}
用 $E$ 和 $B$ 将 $F_{\mu\nu}F^{\mu\nu}$ 和 $F_{\mu\nu}\tilde{F}^{\mu\nu}$ 表示出来。

\newpage
\subsection{1.9}
根据 Euler-Lagrange 方程(1.167),从下列拉氏量导出场 $\phi(x)$ 或 $A^\mu(x)$ 的经典运动方程。
(a) $$L = \frac{1}{2}(\partial^\mu \phi)\partial_\mu \phi - \frac{1}{2}m^2 \phi^2 + \frac{\lambda}{3!} \phi^3,$$
其中 $m$ 和 $\lambda$ 是常数。
(b) $$L = -\frac{1}{2}(\partial^\nu A^\mu)\partial_\nu A_\mu + \frac{1}{2}m^2 A^\mu A_\mu,$$
其中 $m$ 是常数。
(c) $$L = -\frac{a}{2}(\partial^\nu A^\mu)\partial_\nu A_\mu - \frac{b}{2}(\partial^\mu A^\nu)\partial_\nu A_\mu,$$
其中 $a$ 和 $b$ 是常数。
(d) 对于上一小题,取 $a = 1$ 且 $b = -1$,然后用 $F^{\mu\nu} = \partial^\mu A^\nu - \partial^\nu A^\mu$ 表达经典运动方程。

\newpage
\subsection{1.10}
将(1.237)式改写为
$$\mathbb{J}^{\mu\nu\rho} = T^{\mu\nu}x^\nu - T^{\mu\nu}x^\rho + \mathbb{S}^{\mu\nu\rho},$$
其中
$$\mathbb{S}^{\mu\nu\rho} \equiv -i\frac{\partial L}{\partial (\partial_\mu \Phi_a)} (T^\rho)_{ab}\Phi_b$$

满足 $\mathbb{S}^{\mu\nu\rho} = -\mathbb{S}^{\mu\nu\rho}$, 引入 Belinfante-Rosenfeld 能动张量
$$\Theta^{\mu\nu} \equiv T^{\mu\nu} + \frac{1}{2}\partial_\rho (\mathbb{S}^{\mu\nu\rho} + \mathbb{S}^{\mu\nu\rho} - \mathbb{S}^{\mu\nu\rho}).$$

(a) 由 (1.213) 和 (1.238) 式推出
$$\partial_\mu S^{\mu\nu\rho} = T^{\rho\nu} - T^{\nu\rho}.$$    (1.271)

(b) 证明
$$\Theta^{\mu\nu} = \Theta^{\mu\nu}.$$    (1.272)

(c) 证明
$$\partial_\mu \Theta^{\mu\nu} = 0.$$    (1.273)

(d) 证明
$$\int d^3x \Theta^{00} = H, \quad \int d^3x \Theta^{0i} = P^i.$$    (1.274)

可见,Belinfante-Rosenfeld 能动张量是满足守恒流方程的对称张量,且其 00 分量和 0i 分量的空间积分分别是场的总能量和总动量,符合作为对称能动张量的要求,可将它放入广义相对论的 Einstein 方程以充当引力场的源项。








\include{余钊焕-2.1}
\include{余钊焕-2.2}
\include{余钊焕-2.3}
\include{余钊焕-2.4}
\section{习题2}

\newpage
\subsection{2.1}
设算符 $a$ 与其厄米共轭 $a^\dagger$ 满足反对易关系
$$\{a, a^\dagger\} = 1, \quad \{a, a\} = \{a^\dagger, a^\dagger\} = 0, \tag{2.221}$$
其中反对易子定义为 $\{A, B\} \equiv AB + BA$。记算符 $N \equiv a^\dagger a$ 的本征值为 $n$,本征态为 $|n\rangle$,即 $N |n\rangle = n |n\rangle$,归一化为 $\langle n|n\rangle = 1$。

(a) 证明 $[N, a^\dagger] = a^\dagger$ 和 $[N, a] = -a$。

(b) 证明本征值 $n$ 只能取 0 和 1,而且
$$a^\dagger |n\rangle = \sqrt{1 - n}|n + 1\rangle, \quad a |n\rangle = \sqrt{n}|n - 1\rangle. \tag{2.222}$$

\newpage
\subsection{2.2}
已知产生湮灭算符的对易关系 (2.122),以及 $\phi(x,t)$ 和 $\pi(x,t)$ 的平面波展开式 (2.103) 和 (2.105),推出等时对易关系 (2.87)。

\newpage
\subsection{2.3}
用实标量场的单粒子态 (2.152) 构造波包,设
$$|\Psi_p\rangle = \int \frac{d^3q}{(2\pi)^3} \frac{F_p(q)}{\sqrt{2E_q}} |q\rangle, \tag{2.223}$$
其中函数 $F_p(q)$ 满足
$$\int \frac{d^3q}{(2\pi)^3} |F_p(q)|^2 = 1, \quad \int \frac{d^3q}{(2\pi)^3} |F_p(q)|^2 q = p, \tag{2.224}$$
求内积 $\langle \Psi_p |\Psi_p\rangle$ 和总动量算符期待值 $\langle \Psi_p |P|\Psi_p\rangle$。

\newpage
\subsection{2.4}
将实标量场 $\phi(x)$ 的平面波展开式 (2.103) 代入对易关系 (2.129),推出
$$[P^\mu, a_p] = -p^\mu a_p, \quad [P^\mu, a_p^\dagger] = p^\mu a_p^\dagger. \tag{2.225}$$

\newpage
\subsection{2.5}
对于自由实标量场 $\phi(x)$,根据1.7节关于Noether定理的讨论,Lorentz对称性给出的守恒荷算符为
$$J^{\mu\nu} = \int d^3x (T^{0\nu}x^\mu - T^{0\mu}x^\nu),$$
其中
$$T^{00} = \mathcal{H} = \frac{1}{2}[\pi^2 + (\nabla \phi)^2 + m^2 \phi^2], \quad T^{0i} = \pi \partial^i \phi.$$
利用等时对易关系(2.87)推出
$$[\phi(x), J^{\mu\nu}] = i(x^\mu \partial^\nu - x^\nu \partial^\mu)\phi(x). \tag{2.226}$$

\newpage
\subsection{2.6}
复标量场 $\phi(x)$ 可以按(2.182)式分解为两个实标量场 $\phi_1(x)$ 和 $\phi_2(x)$ 的线性组合。设 $\phi_1(x)$ 和 $\phi_2(x)$ 的平面波展开式为
$$\phi_1(x) = \int \frac{d^3p}{(2\pi)^3} \frac{1}{\sqrt{2E_p}} (c_p e^{-ip\cdot x} + c_p^\dagger e^{ip\cdot x}),$$
$$\phi_2(x) = \int \frac{d^3p}{(2\pi)^3} \frac{1}{\sqrt{2E_p}} (d_p e^{-ip\cdot x} + d_p^\dagger e^{ip\cdot x}). \tag{2.229}$$
(a) 推导复标量场平面波展开式(2.187)和(2.189)中使用的产生湮灭算符 $(a_p, a_p^\dagger, b_p, b_p^\dagger)$ 与实标量场产生湮灭算符 $(c_p, c_p^\dagger, d_p, d_p^\dagger)$ 之间的关系。
(b) 根据上述关系及对易关系
$$[c_p, c_p^\dagger] = (2\pi)^3 \delta^{(3)} (p-q), \quad [d_p, d_q^\dagger] = (2\pi)^3 \delta^{(3)} (p-q),$$
$$[c_p, c_q] = [c_p^\dagger, c_q^\dagger] = [d_p, d_q] = [d_p^\dagger, d_q^\dagger] = 0,$$
$$[c_p, d_q] = [c_p^\dagger, d_q^\dagger] = [c_p, d_q^\dagger] = [c_p^\dagger, d_q] = 0,$$
验证 $(a_p, a_p^\dagger, b_p, b_p^\dagger)$ 满足对易关系(2.193)。

\newpage
\subsection{2.7}
复标量场 $\phi(x)$ 的守恒荷算符 $Q$ 可以用产生湮灭算符表达成(2.212)式。
(a) 证明
$$[Q, \phi] = -q\phi, \quad [Q, \phi^\dagger] = q\phi^\dagger. \tag{2.232}$$
(b) 设 $|Q'\rangle$ 是 $Q$ 的本征态,本征值为 $Q'$, 即 $Q | Q' \rangle = Q' | Q' \rangle$。论证 $\phi | Q' \rangle$ 和 $\phi^\dagger | Q' \rangle$ 的 $Q$ 本征值分别为 $Q' - q$ 和 $Q' + q$。

\newpage
\subsection{2.8}
对于复标量场 $\phi(x)$, 真空态 $|0\rangle$ 满足 $a_p |0\rangle = b_p |0\rangle = 0$ 和 $\langle 0|0\rangle = 1$, 引入动量为 $p$ 的正标量玻色子态 $|p^+ \rangle \equiv \sqrt{2E_p} a_p^\dagger |0\rangle$ 和反标量玻色子态 $|p^- \rangle \equiv \sqrt{2E_p} b_p^\dagger |0\rangle$。
(a) 求内积 $\langle q^+ | p^+ \rangle, \langle q^- | p^- \rangle$ 和 $\langle q^- | p^+ \rangle$。
(b) 求 $\langle 0| \phi(x) | p^+ \rangle, \langle 0| \phi^\dagger(x) | p^- \rangle, \langle p^+ | \phi^\dagger(x) | 0\rangle$ 和 $\langle p^- | \phi(x) | 0\rangle$。
(c) 根据守恒荷算符 $Q$ 的表达式 (2.212),推出
$$Q \left| p^+ \right\rangle = (Q_{vac} + q) \left| p^+ \right\rangle, \quad Q \left| p^- \right\rangle = (Q_{vac} - q) \left| p^- \right\rangle,$$
其中
$$Q_{vac} \equiv -(2\pi)^3 \delta^{(3)} (0) \int \frac{d^3p}{(2\pi)^3} q. \tag{2.233}$$

\newpage
\subsection{2.9}
根据复标量场守恒荷算符 $Q$ 和哈密顿量算符 $H$ 的表达式 (2.212) 和 (2.215),证明
$$[H, Q] = 0. \tag{2.235}$$

\newpage
\subsection{2.10}
依照 (2.182) 式将复标量场 $\phi(x)$ 分解为实标量场 $\phi_1(x)$ 和 $\phi_2(x)$ 的线性组合。
(a) 论证复标量场的 U(1) 整体变换 (2.208) 等价于实标量场的整体变换
$$\begin{pmatrix}
\phi_1'(x) \\
\phi_2'(x)
\end{pmatrix}
=
\begin{pmatrix}
\cos q\theta & -\sin q\theta \\
\sin q\theta & \cos q\theta
\end{pmatrix}
\begin{pmatrix}
\phi_1(x) \\
\phi_2(x)
\end{pmatrix}. \tag{2.236}$$
将 $(\phi_1, \phi_2)^T$ 看作一个二维线性空间中的矢量,则上述是此空间中的一个 SO(2) 整体变换,即转动角为 $q\theta$ 的二维旋转变换。
(b) 证明以 (2.183) 式表达的拉氏量
$$\mathcal{L} = \frac{1}{2} (\partial^\mu \phi_1) \partial_\mu \phi_1 - \frac{1}{2} m^2 \phi_1^2 + \frac{1}{2} (\partial^\mu \phi_2) \partial_\mu \phi_2 - \frac{1}{2} m^2 \phi_2^2. \tag{2.237}$$
在上述 SO(2) 整体变换下不变。



\include{余钊焕-3.1}
\include{余钊焕-3.2}
\include{余钊焕-3.3}
\section{习题3}


\newpage
\subsection{3.1}
将 Lorentz 群的空间旋转生成元 $J^i$ 和增速生成元 $K^i$ 线性组合成
$$ J_+^i \equiv \frac{1}{2}(J^i + iK^i), \quad J_-^i \equiv \frac{1}{2}(J^i - iK^i). \tag{3.208} $$

通过对易关系 (3.62) 证明
$$ [J_+^i, J_+^j] = i e^{ijk} J_+^k, \quad [J_-^i, J_-^j] = i e^{ijk} J_-^k, \quad [J_+^i, J_-^j] = 0. \tag{3.209} $$

因此,$J_+^i$ 和 $J_-^i$ 是两套彼此独立的 SU(2) 群生成元,而 Lorentz 代数是两个 SU(2) 代数的直和。

\newpage
\subsection{3.2}
根据对易关系 (3.71) 和 (3.72),证明算符 $\mathbf{J} \cdot \mathbf{P} \equiv J^i P^i$ 满足
$$ [H, \mathbf{J} \cdot \mathbf{P}] = 0, \quad [P^i, \mathbf{J} \cdot \mathbf{P}] = 0. \tag{3.210} $$

\newpage
\subsection{3.3}
将 SU(2) 群的任意元素 $U$ 表达为
$$ U = \begin{pmatrix} a & b \\ c & d \end{pmatrix}. \tag{3.211} $$

(a) 由 $U^\dagger U = 1$ 和 $\det(U) = 1$ 推出
$$ |a|^2 + |c|^2 = |b|^2 + |d|^2 = 1, \quad a^* b + c^* d = 0, \quad ad - bc = 1, \tag{3.212} $$

并证明满足这些方程的解为
$$ a = d^*, \quad b = -c^*, \quad |c|^2 + |d|^2 = 1. \tag{3.213} $$

令 $ d = u_0 + iu_3 $,$ c = u_2 - iu_1 $,则 $ U $ 表达为
$$ U = \begin{pmatrix} u_0 - iu_3 & -u_2 - iu_1 \\ u_2 - iu_1 & u_0 + iu_3 \end{pmatrix}. \tag{3.214} $$

实参数 $ u_0 $、$ u_1 $、$ u_2 $ 和 $ u_3 $ 必须满足
$$ u_0^2 + u_1^2 + u_2^2 + u_3^2 = |c|^2 + |d|^2 = 1, \tag{3.215} $$

因此,它们之中只有三个是独立的。将 $ u_0 $、$ u_1 $、$ u_2 $ 和 $ u_3 $ 当作四维空间的直角坐标,则约束条件 (3.215) 表明 SU(2) 的群空间是四维空间中的三维球面 $ S^3 $。

记三维矢量 $ \omega $ 的球坐标为 $ (\omega, \theta, \phi) $,则 $ \omega = \omega \mathbf{n} $,其中单位矢量 $ \mathbf{n}(\theta, \phi) $ 是 $ \omega $ 的方向矢量,相应的直角坐标为 $ \mathbf{n} = (\sin \theta \cos \phi, \sin \theta \sin \phi, \cos \theta) $。设
$$ u_0 = \cos \frac{\omega}{2}, \quad u_1 = \sin \frac{\omega}{2} \sin \theta \cos \phi, \quad u_2 = \sin \frac{\omega}{2} \sin \theta \sin \phi, \quad u_3 = \sin \frac{\omega}{2} \cos \theta, \tag{3.216} $$

则条件 (3.215) 得到满足,从而可以用 $ (\omega, \theta, \phi) $ 作为描述任意 $ U(\mathbf{n}, \omega) \in SU(2) $ 的三个独立实参数。

(b) 证明
$$ U(\mathbf{n}, \omega) = \cos \frac{\omega}{2} \mathbf{1} - i \sin \frac{\omega}{2} (\mathbf{n} \cdot \boldsymbol{\sigma}) \tag{3.217} $$

和
$$ (\mathbf{n} \cdot \boldsymbol{\sigma})^2 = \mathbf{1}. \tag{3.218} $$

(c) 证明
$$ U(\mathbf{n}, \omega_1) U(\mathbf{n}, \omega_2) = U(\mathbf{n}, \omega_1 + \omega_2), \tag{3.219} $$
$$ U(\mathbf{n}, 2\pi) = -\mathbf{1}, \tag{3.220} $$
$$ U(\mathbf{n}, 4\pi) = \mathbf{1}, \tag{3.221} $$
$$ U(\mathbf{n}, \omega) = -U(-\mathbf{n}, 2\pi - \omega), \tag{3.222} $$
$$ U(\mathbf{n}, \omega) = U(-\mathbf{n}, 4\pi - \omega). \tag{3.223} $$

\newpage
\subsection{3.4}
Pauli 矩阵 (3.53) 是无迹厄米矩阵,而任意 $ 2 \times 2 $ 无迹厄米矩阵 $ X $ 只包含三个独立实参数,因此必定可以将 $ X $ 展开为三个 Pauli 矩阵的实线性组合。将组合系数取为三维空间中任意一点 $ P $ 的三个直角坐标 $ x^i $,得
$$ X = x^i \sigma^i = \begin{pmatrix} x^3 & x^1 - i x^2 \\ x^1 + i x^2 & -x^3 \end{pmatrix}. \tag{3.224} $$

可见,无迹厄米矩阵 $ X $ 与 $ P $ 点的位置矢量 $ \mathbf{x} = (x^1, x^2, x^3) $ 是一一对应的。

(a) 证明
$$ x^i = \frac{1}{2} \mathrm{tr}(X \sigma^i), \quad \det(X) = -|\mathbf{x}|^2. \tag{3.225} $$

(b) 设 $ 2 \times 2 $ 矩阵 $ U \in SU(2) $,即满足 $ U^\dagger U = U U^\dagger = \mathbf{1} $ 和 $ \det(U) = 1 $。对 $ X $ 作相似变换 
$$ X' = U X U^\dagger, $$

利用 $ \mathrm{tr}(AB) = \mathrm{tr}(BA) $ 和 $ \det(AB) = \det(BA) $ 证明
$$ X'^\dagger = X', \quad \mathrm{tr}(X') = 0, \quad \det(X') = \det(X). \tag{3.226} $$

可见,$ X' $ 也是无迹厄米矩阵,因而可表达为 $ X' = x'^i \sigma^i $,即对应于三维空间另一点 $ P' $ 的位置矢量 $ \mathbf{x}' = (x'^1, x'^2, x'^3) $。$ \det(X') = \det(X) $ 表明 $ |\mathbf{x}'|^2 = |\mathbf{x}|^2 $,因而 $ \mathbf{x}' $ 与 $ \mathbf{x} $ 可以用 3 阶实正交矩阵 $ R $ 联系起来,
$$ x'^i = R^i_j x^j. \tag{3.227} $$

当 $ U = \mathbf{1} $ 时 $ X' = X $,故 $ R = \mathbf{1} $。因为任意 $ U $ 可以在连通的 $ SU(2) $ 群空间中由恒元连续变化得到,所以 $ R $ 也可以在 $ O(3) $ 群空间中由恒元连续变化得到。这意味着 $ R $ 属于 $ O(3) $ 群的连通子群 $ SO(3) $,满足 $ \det(R) = 1 $。

任意两个行列式相等的 $ 2 \times 2 $ 无迹厄米矩阵 $ X $ 和 $ X' $ 可以用 $ U \in SU(2) $ 作相似变换联系起来,但这样的 $ U $ 不是唯一的。

(c) 设 $ U_1, U_2 \in SU(2) $ 满足 $ U_1 X U_1^\dagger = U_2 X U_2^\dagger = X' $,证明
$$ [U_2^\dagger U_1, X] = 0, \tag{3.228} $$

即 $ U_2^\dagger U_1 $ 与任意 $ X $ 对易。因此 $ U_2^\dagger U_1 $ 与单位矩阵只相差一个常数因子 $ \lambda $,即 $ U_2^\dagger U_1 = \lambda \mathbf{I} $,故 $ U_1 = \lambda U_2 $。利用 $ \det(U_1) = \det(U_2) = 1 $ 证明
$$ \lambda = \pm 1. \tag{3.229} $$

于是,用 $ U $ 和 $ -U $ 对 $ X $ 作相似变换将得到的相同 $ X' $。由 $ R^i_j x^j \sigma^i = x'^i \sigma^i = X' = U X U^\dagger = x^j U \sigma^j U^\dagger $ 和 $ \mathbf{x} $ 的任意性得到
$$ U \sigma^j U^\dagger = \sigma^i R^i_j, \quad (-U) \sigma^j (-U)^\dagger = \sigma^i R^i_j. \tag{3.230} $$

这给出 $ SU(2) $ 群元素 $ U $ 和 $ -U $ 与 $ SO(3) $ 群元素 $ R $ 之间二对一的对应关系。

(d) 设 $ U_1 \sigma^j U_1^\dagger = \sigma^i (R_1)^i_j $ 和 $ U_2 \sigma^j U_2^\dagger = \sigma^i (R_2)^i_j $,证明以上对应关系对群乘积保持不变,即
$$ (U_2 U_1) \sigma^j (U_2 U_1)^\dagger = \sigma^i (R_2 R_1)^i_j. \tag{3.231} $$

因此,这种对应关系是 $ SU(2) $ 群与 $ SO(3) $ 群之间的 $ 2:1 $ 同态关系。

\newpage
\subsection{3.5}
本题将验证 (3.217) 表达的 $ SU(2) $ 群元 $ U(\mathbf{n}, \omega) $ 满足对应关系 (3.230)。如图 3.9 所示,将任意位置矢量 $ \mathbf{x} $ 分解为平行于 $ \mathbf{n} $ 的分量 $ a \mathbf{n} $ 和垂直于 $ \mathbf{n} $ 的分量 $ b \mathbf{m} $,即
$$ \mathbf{x} = a \mathbf{n} + b \mathbf{m}, \tag{3.232} $$

其中单位矢量 $ \mathbf{n} $ 和 $ \mathbf{m} $ 满足 $ |\mathbf{n}|^2 = |\mathbf{m}|^2 = 1 $ 和 $ \mathbf{n} \cdot \mathbf{m} = 0 $。

(a) 利用 (3.218) 式证明
$$ U(\mathbf{n}, \omega) (\mathbf{n} \cdot \boldsymbol{\sigma}) U^\dagger(\mathbf{n}, \omega) = \mathbf{n} \cdot \boldsymbol{\sigma}. $$

(b) 对于任意三维矢量 $ \mathbf{p} $ 和 $ \mathbf{q} $,利用 (3.56) 式证明
$$ (\mathbf{p} \cdot \boldsymbol{\sigma}) (\mathbf{q} \cdot \boldsymbol{\sigma}) = (\mathbf{p} \cdot \mathbf{q}) \mathbf{1} + i (\mathbf{p} \times \mathbf{q}) \cdot \boldsymbol{\sigma}. $$

以此推出
$$ (\mathbf{n} \cdot \boldsymbol{\sigma}) (\mathbf{m} \cdot \boldsymbol{\sigma}) = i (\mathbf{n} \times \mathbf{m}) \cdot \boldsymbol{\sigma}. $$

利用 (1.118) 式证明
$$ (\mathbf{n} \times \mathbf{m}) \times \mathbf{n} = \mathbf{m}, $$

进而推出
$$ (\mathbf{n} \cdot \boldsymbol{\sigma}) (\mathbf{m} \cdot \boldsymbol{\sigma}) (\mathbf{n} \cdot \boldsymbol{\sigma}) = -\mathbf{m} \cdot \boldsymbol{\sigma}. $$

(c) 证明
$$ U(\mathbf{n}, \omega) (\mathbf{m} \cdot \boldsymbol{\sigma}) U^\dagger(\mathbf{n}, \omega) = \mathbf{m}' \cdot \boldsymbol{\sigma}, $$

其中
$$ \mathbf{m}' = \cos \omega \mathbf{m} + \sin \omega (\mathbf{n} \times \mathbf{m}) $$

是把 $ \mathbf{m} $ 绕 $ \mathbf{n} $ 方向转动 $ \omega $ 角得到的单位矢量。

(d) 令
$$ \mathbf{x}' \cdot \boldsymbol{\sigma} = U(\mathbf{n}, \omega) (\mathbf{x} \cdot \boldsymbol{\sigma}) U^\dagger(\mathbf{n}, \omega), $$

证明
$$ \mathbf{x}' = a \mathbf{n} + b \mathbf{m}', $$

即
$$ \mathbf{x}' = a \mathbf{n} + b \cos \omega \mathbf{m} + b \sin \omega (\mathbf{n} \times \mathbf{m}). $$

\newpage
\subsection{3.6}
设 $ \mathbf{n}_1 $ 是一个与 $ \hat{\mathbf{p}} = \mathbf{p}/|\mathbf{p}| $ 垂直的单位矢量,而单位矢量 $ \mathbf{n}_2 = \hat{\mathbf{p}} \times \mathbf{n}_1 $,则 $ \hat{\mathbf{p}} $、$ \mathbf{n}_1 $ 和 $ \mathbf{n}_2 $ 两两之间相互垂直,如图 3.10 所示。引入自旋角动量算符 $ \mathbf{S} $ 在 $ \mathbf{n}_1 $ 和 $ \mathbf{n}_2 $ 上的投影
$$ S_{T,1} = \mathbf{n}_1 \cdot \mathbf{S}, \quad S_{T,2} = \mathbf{n}_2 \cdot \mathbf{S}, \tag{3.244} $$

以及它们的线性组合
$$ S_\pm = S_{T,1} \pm i S_{T,2}, \tag{3.245} $$

证明
$$ \mathbf{S}^2 = S_{T,1}^2 + S_{T,2}^2 + S_p^2, \tag{3.246} $$

和对易关系
$$ [S_{T,1}, S_{T,2}] = i S_p, \quad [\mathbf{S}^2, S_\pm] = 0, \quad [S_p, S_\pm] = \pm S_\pm, \tag{3.247} $$

其中 $ S_p = \hat{\mathbf{p}} \cdot \mathbf{S} $ 是螺旋度算符。由此可见,$ \mathbf{S}^2 $、$ S_{T,1} $、$ S_{T,2} $、$ S_p $ 和 $ S_\pm $ 之间的关系与 $ \mathbf{J}^2 $、$ J^1 $、$ J^2 $、$ J^3 $ 和 $ J^\pm $ 之间的关系形式相同,从而螺旋度 $ \lambda $ 与磁量子数 $ \sigma $ 的取值情况也相同。

\newpage
\subsection{3.7}
引入 $ \sigma^\mu = (1, \boldsymbol{\sigma}) $ 和 $ \bar{\sigma}^\mu = (1, -\boldsymbol{\sigma}) $,则任意时空坐标 $ x^\mu $ 一一对应于 $ 2 \times 2 $ 厄米矩阵
$$ \tilde{X} = x^\mu \sigma_\mu = x^0 \mathbf{1} - x^i \sigma^i = \begin{pmatrix} x^0 - x^3 & -x^1 + i x^2 \\ -x^1 - i x^2 & x^0 + x^3 \end{pmatrix}. \tag{3.248} $$

(a) 证明
$$ x^\mu = \frac{1}{2} \mathrm{tr}(\tilde{X} \sigma^\mu), \quad \det(\tilde{X}) = x^\mu x_\mu. \tag{3.249} $$

(b) 设 $ \lambda $ 是满足 $ |\det(\lambda)| = 1 $ 的任意 $ 2 \times 2 $ 可逆复矩阵,对 $ \tilde{X} $ 作变换得到厄米矩阵
$$ \tilde{X}' = \lambda \tilde{X} \lambda^\dagger, \tag{3.250} $$

证明
$$ \det(\tilde{X}') = \det(\tilde{X}). \tag{3.251} $$

(c) 将 $ \tilde{X}' $ 表达为 $ \tilde{X}' = x'^\mu \sigma_\mu $,则 (3.251) 表明 $ x'^\mu x'_\mu = x^\mu x_\mu $。因此,$ \lambda $ 对应于一个 Lorentz 变换 $ \Lambda(\lambda) $,满足
$$ \lambda x^\mu \sigma_\mu \lambda^\dagger = x'^\mu \sigma_\mu = \Lambda^\mu_\nu (\lambda) x^\nu \sigma_\mu. \tag{3.252} $$

对于满足 $ |\det(\lambda)| = 1 $ 的任意 $ 2 \times 2 $ 可逆复矩阵 $ \lambda_1 $ 和 $ \lambda_2 $,证明同态关系
$$ \Lambda^\mu_\nu (\lambda_2 \lambda_1) = \Lambda^\mu_\rho (\lambda_2) \Lambda^\rho_\nu (\lambda_1). \tag{3.253} $$

如果 $ \lambda_1 $ 和 $ \lambda_2 $ 只相差一个整体相位因子,即 $ \lambda_2 = \eta \lambda_1 $,其中 $ |\eta| = 1 $,则 $ \lambda_2 \tilde{X} \lambda_2^\dagger = |\eta|^2 \lambda_1 \tilde{X} \lambda_1^\dagger = \lambda_1 \tilde{X} \lambda_1^\dagger $,因而 $ \lambda_1 $ 和 $ \lambda_2 $ 对应于同一个 Lorentz 变换。为消除这样的重复性,可以适当选取相位因子,使得 $ \det(\lambda) = 1 $。因此只需要讨论 $ \det(\lambda) = 1 $ 的任意 $ 2 \times 2 $ 可逆复矩阵 $ \lambda $,所有这样的矩阵 $ \{\lambda\} $ 构成复数域上的 2 阶特殊线性群 SL(2, C)。

(d) 根据线性代数的极分解定理,任意 $ \lambda \in SL(2, C) $ 可以分解为
$$ \lambda = U \exp(h) \tag{3.254} $$

其中 $ U \in SU(2) $,而 $ h $ 是一个 $ 2 \times 2 $ 无迹厄米矩阵。对于 $ \tilde{X}' = U \tilde{X} U^\dagger $,证明
$$ x'^0 = x^0. \tag{3.255} $$

因此 $ U \in SU(2) $ 对应的 $ \Lambda(U) $ 是空间旋转变换。

(e) 设 $ \theta $ 为实数,令
$$ \lambda(\theta) = \begin{pmatrix} e^{i\theta/2} & 0 \\ 0 & e^{-i\theta/2} \end{pmatrix}, \tag{3.256} $$

证明 $ \tilde{X}' = \lambda(\theta) \tilde{X} \lambda^\dagger(\theta) $ 和 $ \tilde{X} $ 对应的时空坐标满足
$$ x'^0 = x^0, \quad x'^1 = x^1 \cos \theta + x^2 \sin \theta, \quad x'^2 = -x^1 \sin \theta + x^2 \cos \theta, \quad x'^3 = x^3. \tag{3.257} $$

可见,$ \lambda(\theta) $ 对应的 Lorentz 变换 $ \Lambda[\lambda(\theta)] $ 是绕 z 轴转动 $ \theta $ 角的空间旋转变换,而 $ \lambda(0) = \mathbf{1} $ 和 $ \lambda(2\pi) = -\mathbf{1} $ 对应于同一个 Lorentz 变换 $ \Lambda[\lambda(\theta)] = \mathbf{1} $。

\newpage
\subsection{3.8}
考虑 (3.173) 式表达的 Lorentz 变换 $ S(\alpha, \beta) $ 和 (1.36) 式表达的旋转变换 $ R_z(\theta) $。

(a) 验证 $ g = S^T g S $ 和 $ \det(S) = 1 $。

(b) 推出 (3.178) 式。

(c) 推出 (3.180) 式。

\newpage
\subsection{3.9}
用 Poincaré 群的生成元算符 $ P^\mu $ 和 $ J^{\mu\nu} $ 定义 Pauli-Lubanski 赝矢量算符
$$ W^\mu = -\frac{1}{2} \varepsilon^{\mu\nu\rho\sigma} J_{\nu\rho} P_\sigma \tag{3.258} $$

和两个标量算符
$$ I = \frac{i}{8} \varepsilon_{\mu\nu\rho\sigma} J^{\mu\nu} J^{\rho\sigma}, \tag{3.259} $$
$$ W^2 = W^\mu W_\mu. \tag{3.260} $$

(a) 证明
$$ W^\mu P_\mu = 0, \quad [P^\mu, W^\nu] = 0. \tag{3.261} $$

(b) 证明
$$ [I, P^\mu] = -W^\mu, \quad U^{-1}(\Lambda) I U(\Lambda) = I, \quad [J^{\mu\nu}, I] = 0. \tag{3.262} $$

(c) 对于任意算符 $ A, B $ 和 $ C $,推出 Jacobi 恒等式
$$ [A, [B, C]] + [B, [C, A]] + [C, [A, B]] = 0. \tag{3.263} $$

进而证明
$$ [J^{\mu\nu}, W^\rho] = i (g^{\nu\rho} W^\mu - g^{\mu\rho} W^\nu). \tag{3.264} $$

(d) 证明
$$ [P^\mu, W^2] = 0, \quad [J^{\mu\nu}, W^2] = 0. \tag{3.265} $$

可见,$ W^2 $ 与 Poincaré 群的所有生成元算符对易,即它是 Poincaré 群的 Casimir 算符,因而 $ W^2 $ 的本征值在任意 Poincaré 变换下不变。Poincaré 群的不可约么正表示可以用 $ W^2 $ 和另一个 Casimir 算符 $ P^2 $ 的本征值来刻画。

(e) 推出
$$ W^0 = \mathbf{J} \cdot \mathbf{P}, \quad \mathbf{W} = \mathbf{J} H + \mathbf{K} \times \mathbf{P}. \tag{3.266} $$

(f) 已知质量为 $ m > 0 $,自旋为 $ s $ 的单粒子态 $ |\Psi_{s,\sigma}(p^\mu)\rangle $ 满足本征方程
$$ P^\mu |\Psi_{s,\sigma}(p^\mu)\rangle = p^\mu |\Psi_{s,\sigma}(p^\mu)\rangle, \quad P^2 |\Psi_{s,\sigma}(p^\mu)\rangle = m^2 |\Psi_{s,\sigma}(p^\mu)\rangle, \tag{3.267} $$
$$ \mathbf{J}^2 |\Psi_{s,\sigma}(p^\mu)\rangle = s(s+1) |\Psi_{s,\sigma}(p^\mu)\rangle, \quad J^3 |\Psi_{s,\sigma}(p^\mu)\rangle = \sigma |\Psi_{s,\sigma}(p^\mu)\rangle. \tag{3.268} $$

取 $ p^\mu = (m, \mathbf{0}) $,推出
$$ W^0 |\Psi_{s,\sigma}(p^\mu)\rangle = 0 |\Psi_{s,\sigma}(p^\mu)\rangle, \quad \mathbf{W} |\Psi_{s,\sigma}(p^\mu)\rangle = m \mathbf{J} |\Psi_{s,\sigma}(p^\mu)\rangle, \tag{3.269} $$

和
$$ W^2 |\Psi_{s,\sigma}(p^\mu)\rangle = -m^2 s(s+1) |\Psi_{s,\sigma}(p^\mu)\rangle. \tag{3.270} $$

(g) 已知螺旋度为 $ \lambda $ 的无质量单粒子态 $ |\Psi_\lambda(p^\mu)\rangle $ 满足本征方程
$$ P^\mu |\Psi_\lambda(p^\mu)\rangle = p^\mu |\Psi_\lambda(p^\mu)\rangle, \quad \frac{\mathbf{P} \cdot \mathbf{J}}{|\mathbf{p}|} |\Psi_\lambda(p^\mu)\rangle = \lambda |\Psi_\lambda(p^\mu)\rangle, \quad W^2 |\Psi_\lambda(p^\mu)\rangle = 0 |\Psi_\lambda(p^\mu)\rangle, \tag{3.271} $$

其中四维动量 $ p^\mu $ 满足 $ p^2 = 0 $。设 $ W^\mu |\Psi_\lambda(p^\mu)\rangle = w^\mu |\Psi_\lambda(p^\mu)\rangle $,论证
$$ w^\mu = \lambda p^\mu. \tag{3.272} $$

\newpage
\subsection{3.10}
取 $ j = 3/2 $,根据 (3.142) 和 (3.147) 式推出 SU(2) 群 4 维表示的生成元矩阵 $ \tau_{(3/2)}^1, \tau_{(3/2)}^2 $ 和 $ \tau_{(3/2)}^3 $。





\include{余钊焕-4.1}
\include{余钊焕-4.2}
\include{余钊焕-4.3}
\include{余钊焕-4.4}
\section{习题4}

\newpage
\subsection{4.1}
定义 Lorentz 矢量表示的增速生成元
$$\mathcal{K}^{i} \equiv \mathcal{J}^{0i}。$$
(a) 根据 (4.2) 式,写出 $(\mathcal{K}^{1})^{\mu}_{\nu}, (\mathcal{K}^{2})^{\mu}_{\nu}$ 和 $(\mathcal{K}^{3})^{\mu}_{\nu}$ 的矩阵表达式。
(b) 根据 $ \mathcal{J}^i $ 的定义 (4.36) 以及 $ \theta^i $ 和 $ \xi^i $ 的定义 (3.41),证明有限 Lorentz 变换 (4.14) 可以表示为
$$\Lambda = \exp(i\theta^i \mathcal{J}^i + i\xi^i \mathcal{K}^i).$$

\newpage
\subsection{4.2}
根据 (4.39)、(4.40) 和 (4.41) 式,将 $ \mathcal{J}^i $ 的纯空间部分记为
$$\hat{\tau}_{(1)}^i = 
\begin{pmatrix}
0 & -i \\
i & 0
\end{pmatrix}, \quad \hat{\tau}_{(1)}^i = 
\begin{pmatrix}
i & 0 \\
-i & 0
\end{pmatrix}, \quad \hat{\tau}_{(1)}^i = 
\begin{pmatrix}
-i & 0 \\
i & 0
\end{pmatrix}.$$
(a) 验证 $\hat{\tau}_{(1)}^i$ 满足 SU(2) 代数关系
$$[\hat{\tau}_{(1)}^i, \hat{\tau}_{(1)}^j] = i e^{ijk} \hat{\tau}_{(1)}^k.$$
(b) 验证 $\hat{\tau}_{(1)}^i$ 与 (3.156) 式表达的 $\hat{\tau}_{(1)}^i$ 之间的关系为
$$U^{\dagger} \hat{\tau}_{(1)}^i U = \hat{\tau}_{(1)}^i,$$
其中幺正矩阵
$$U = \frac{1}{\sqrt{2}} 
\begin{pmatrix}
1 & 0 & 1 \\
i & 0 & -i \\
0 & \sqrt{2} & 0
\end{pmatrix}.$$
可见,$\hat{\tau}_{(1)}^i$ 与 $\hat{\tau}_{(1)}^i$ 等价,也可作为 SU(2) 群 3 维线性表示 $D^{(1)}$ 的生成元矩阵。

\newpage
\subsection{4.3}
证明 (4.23) 式定义的微分算符 $\hat{L}^{\mu\nu}$ 满足 Lorentz 代数关系
$$[\hat{L}^{\mu\nu}, \hat{L}^{\rho\sigma}] = i(g^{\nu\rho} \hat{L}^{\mu\sigma} - g^{\mu\rho} \hat{L}^{\nu\sigma} - g^{\nu\sigma} \hat{L}^{\mu\rho} + g^{\mu\sigma} \hat{L}^{\nu\rho}).$$
实际上,$\hat{L}^{\mu\nu}$ 是 Lorentz 群在场空间上的生成元,它们生成一个无限维幺正表示。

\newpage
\subsection{4.4}
比较 (4.111) 和 (4.112) 式中负能解的系数,得到
$$e^{\mu*}(p, \lambda)[a_{p,\lambda}^\dagger, J] = [-\delta^\mu, x \times p + (\mathcal{J})^\mu] e^{\mu*}(p, \lambda) a_{p,\lambda}^\dagger,$$
以此推出 $[p, J, a_{p,\lambda}^\dagger] = \lambda a_{p,\lambda}^\dagger$。

\newpage
\subsection{4.5}
验证有质量矢量场 $A^\mu(x,t)$ 和 $\pi_i(x,t)$ 的平面波展开式 (4.110) 和 (4.126) 满足 (4.63) 式
$$A^0 = -\nabla \cdot \pi / m^2.$$

\newpage
\subsection{4.6}
设有质量矢量场 $A^\mu(x)$ 对应的拉氏量为
$$\mathcal{L} = -\frac{1}{2} F_{\mu\nu}^\dagger F^{\mu\nu} + m^2 A_\mu^\dagger A^\mu,$$
其中 $F^{\mu\nu} = \partial^\mu A^\nu - \partial^\nu A^\mu, \, m > 0$。
(a) 由 Euler-Lagrange 方程 (1.167) 推出 $A^\mu(x)$ 的经典运动方程。
(b) 将 $A^\mu(x)$ 分解成两个实矢量场 $B^\mu(x)$ 和 $C^\mu(x)$ 的线性组合,
$$A^\mu = \frac{1}{\sqrt{2}} (B^\mu + iC^\mu),$$
证明拉氏量可化为
$$\mathcal{L} = -\frac{1}{4} B_{\mu\nu} B^{\mu\nu} + \frac{1}{2} m^2 B_\mu B^\mu - \frac{1}{4} C_{\mu\nu} C^{\mu\nu} + \frac{1}{2} m^2 C_\mu C^\mu,$$
其中 $B^{\mu\nu} \equiv \partial^\mu B^\nu - \partial^\nu B^\mu$,而 $C^{\mu\nu} \equiv \partial^\mu C^\nu - \partial^\nu C^\mu$。因此,复矢量场的拉氏量相当于两个质量相同的实矢量场的拉氏量。
(c) 证明 $A^\mu(x)$ 的平面波展开式为
$$A^\mu(x) = \int \frac{d^3 p}{(2\pi)^3} \sum_{\lambda=\pm,0} \left[ e^{i\lambda}(p,\lambda)a_{p,\lambda}e^{-ip\cdot x} + e^{i\mu}(p,\lambda)b_{p,\lambda}^{\dagger}e^{ip\cdot x} \right],$$
且产生湮灭算符满足对易关系
$$[a_{p,\lambda}, a_{q,\lambda}] = (2\pi)^3 \delta_{\lambda \lambda'} \delta^{(3)}(p-q), \quad [a_{p,\lambda}, a_{q,\lambda'}] = [a_{p,\lambda}, a_{q,\lambda'}] = 0,$$
$$[b_{p,\lambda}, b_{q,\lambda'}] = (2\pi)^3 \delta_{\lambda \lambda'} \delta^{(3)}(p-q), \quad [b_{p,\lambda}, b_{q,\lambda'}] = [b_{p,\lambda}, b_{q,\lambda'}] = 0,$$
$$[a_{p,\lambda}, b_{q,\lambda'}] = [b_{p,\lambda}, a_{q,\lambda'}] = [a_{p,\lambda}, b_{q,\lambda'}] = [a_{p,\lambda}, b_{q,\lambda'}] = 0.$$
(d) 作 U(1) 整体变换 $A^\mu(x) = e^{iq\theta} A^\mu(x)$,证明拉氏量 $\mathcal{L}(x)$ 在此变换下不变,并推出相应的 U(1) 守恒流算符
$$J^\mu = ig(F^{\mu\nu\dagger}A_\nu - A_\nu^\dagger F^{\mu\nu}).$$
(e) 证明 U(1) 守恒荷算符 $Q = \int d^3 x J^0$ 表达为
$$Q = \int \frac{d^3 p}{(2\pi)^3} \sum_{\lambda=\pm,0} (q a_{p,\lambda}^\dagger a_{p,\lambda} - q b_{p,\lambda}^\dagger b_{p,\lambda}) - 3\delta^{(3)}(0) \int d^3 p q.$$
可见,$(a_{p,\lambda}, a_{p,\lambda}^\dagger)$ 描述正矢量玻色子,$(b_{p,\lambda}, b_{p,\lambda}^\dagger)$ 描述反矢量玻色子。

\newpage
\subsection{4.7}
考虑无质量情况下的极化矢量。
(a) 设纵向极化矢量 $e^\mu(p,3)$ 满足归一关系 $e^\mu(p,3)e_\mu(p,3) = -1$,且其空间分量正比于 $p$,论证 $e^\mu(p,3)$ 不能满足四维横向条件 $p_\mu e^\mu(p,3) = 0$。
(b) 论证满足归一关系 $e^\mu(p,0)e_\mu(p,0) = 1$ 的类时极化矢量 $e^\mu(p,0)$ 不能满足四维横向条件 $p_\mu e^\mu(p,0) = 0$。

\newpage
\subsection{4.8}
由极化求和关系 (4.102) 和 (4.124) 式推出
$$\sum_{\lambda=\pm,0} \varepsilon_i(p,\lambda)\varepsilon_j^\ast(p,\lambda) = -g_{ij},$$
$$\sum_{\lambda = \pm 0} \varepsilon_i (p, \lambda) \varepsilon_j^* (p, \lambda) = -g_{ij} - \frac{p_ip_j}{p_0^2}.$$
再利用以上两式,产生湮灭算符的对易关系(4.128)以及 $A^\mu (x, t)$ 和 $\pi_i (x, t)$ 的平面波展开式(4.110)和(4.126),推出有质量矢量场的等时对易关系(4.59)。

\newpage
\subsection{4.9}
利用完备性关系(4.67)、产生湮灭算符的对易关系(4.233)以及 $A^\mu (x, t)$ 和 $\pi_\mu (x, t)$ 的平面波展开式(4.229)和(4.231),推出无质量矢量场的等时对易关系(4.218)。

\newpage
\subsection{4.10}
设 $h_{\mu \nu}$ 是对称的二阶 Lorentz 张量场,考虑 Pauli-Fierz 作用量
$$S_{PF} = \int d^4 x \mathcal{L}_{PF}(x),$$
其中
$$\mathcal{L}_{PF} = \frac{1}{2} (\partial^\mu h^{\mu \nu}) \partial_\rho h_{\mu \nu} - (\partial^\mu h^{\nu \rho}) \partial_\rho h_{\mu \nu} + (\partial^\nu h_{\mu \nu}) \partial^\mu h - \frac{1}{2} (\partial^\mu h) \partial_\mu h,$$
而 $h \equiv g^{\mu \nu} h_{\mu \nu}$。
(a) 对 $h_{\mu \nu}$ 作规范变换
$$h'_{\mu \nu} (x) = h_{\mu \nu} (x) + \partial_\mu \chi_\nu (x) + \partial_\nu \chi_\mu (x),$$
其中 $\chi_\mu (x)$ 是任意 Lorentz 矢量函数,证明 $S_{PF}$ 在规范变换下不变。
(b) 根据 Euler-Lagrange 方程(1.167),证明 $h_{\mu \nu}$ 的经典运动方程是
$$\partial^2 h_{\mu \nu} - \partial_\mu \partial^\rho h_{\nu \rho} - \partial_\nu \partial^\rho h_{\mu \rho} + g_{\mu \nu} \partial^\rho \partial^\sigma h_{\rho \sigma} + \partial_\mu \partial_\nu h - g_{\mu \nu} \partial^2 h = 0.$$
注意上式对 $\mu$ 和 $\nu$ 对称。
(c) 令
$$\bar{h}_{\mu \nu} \equiv h_{\mu \nu} - \frac{1}{2} g_{\mu \nu} h,$$
取 Lorenz 规范
$$\partial^\nu \bar{h}_{\mu \nu} = 0,$$
证明 $\bar{h}_{\mu \nu}$ 的运动方程为
$$\partial^2 \bar{h}_{\mu \nu} = 0.$$





\section{量子旋量场}


%%%%%%%%%%%%%%%%%%%%%%%%%%%%%%%%%%%%%%%%%%%%%%%%%%%%%%%%%
\subsection{Dirac旋量场}



\begin{tabular}{lll}
\hline
线性空间名称 & 矢量表示空间 & 旋量表示空间 \\
\hline
维度 & 4 维 & 4 维 \\
空间中元素 & Lorentz 矢量 $A^{\mu}$ & Dirac 旋量 $\psi_{a}$ \\
Lorentz 群生成元 & $(\mathcal{J}^{\mu\nu})^{\alpha}_{\beta} \equiv \mathrm{i}(g^{\mu\alpha}\delta^{\nu}_{\beta} - g^{\nu\alpha}\delta^{\mu}_{\beta})$ & $\mathcal{S}^{\mu\nu} = \frac{\mathrm{i}}{4}[\gamma^{\mu},\gamma^{\nu}]$ \\
固有保时向 Lorentz 变换 & $\Lambda = \exp\left(-\frac{\mathrm{i}}{2}\omega_{\mu\nu}\mathcal{J}^{\mu\nu}\right)$ & $D(\Lambda) = \exp\left(-\frac{\mathrm{i}}{2}\omega_{\mu\nu}\mathcal{S}^{\mu\nu}\right)$ \\
\hline
线性空间名称 & Hilbert 空间 & 场空间 \\
\hline
维度 & 无限维 & 无限维 \\
空间中元素 & 态矢 $|\Psi\rangle$ & 场 $\phi(x), A^{\mu}(x), \psi_{a}(x)$ \\
Lorentz 群生成元 & 算符 $J^{\mu\nu}$ & $\hat{L}^{\mu\nu} = \mathrm{i}(x^{\mu}\partial^{\nu} - x^{\nu}\partial^{\mu})$ \\
固有保时向 Lorentz 变换 & $U(\Lambda) = \exp\left(-\frac{\mathrm{i}}{2}\omega_{\mu\nu}J^{\mu\nu}\right)$ & $\exp\left(-\frac{\mathrm{i}}{2}\omega_{\mu\nu}\hat{L}^{\mu\nu}\right)$ \\
\hline
\end{tabular}


\begin{table}[htbp]
\centering
\caption{与Lorentz变换相关的线性空间}
\begin{tabular}{lll}
\toprule
线性空间名称 & 矢量表示空间 & 旋量表示空间 \\
\midrule
维度 & 4 维 & 4 维 \\
空间中元素 & Lorentz 矢量 $A^{\mu}$ & Dirac 旋量 $\psi_{a}$ \\
Lorentz 群生成元 & $\displaystyle (\mathcal{J}^{\mu\nu})^{\alpha}_{\beta} \equiv \mathrm{i}\big(g^{\mu\alpha}\delta^{\nu}_{\beta} - g^{\nu\alpha}\delta^{\mu}_{\beta}\big)$ & $\displaystyle \mathcal{S}^{\mu\nu} = \frac{\mathrm{i}}{4}[\gamma^{\mu},\gamma^{\nu}]$ \\
固有保时向 Lorentz 变换 & $\displaystyle \Lambda = \exp\left(-\frac{\mathrm{i}}{2}\omega_{\mu\nu}\mathcal{J}^{\mu\nu}\right)$ & $\displaystyle D(\Lambda) = \exp\left(-\frac{\mathrm{i}}{2}\omega_{\mu\nu}\mathcal{S}^{\mu\nu}\right)$ \\
\midrule
线性空间名称 & Hilbert 空间 & 场空间 \\
\midrule
维度 & 无限维 & 无限维 \\
空间中元素 & 态矢 $\ket{\Psi}$ & 场 $\phi(x), A^{\mu}(x), \psi_{a}(x)$ \\
Lorentz 群生成元 & 算符 $J^{\mu\nu}$ & $\displaystyle \hat{L}^{\mu\nu} = \mathrm{i}(x^{\mu}\partial^{\nu} - x^{\nu}\partial^{\mu})$ \\
固有保时向 Lorentz 变换 & $\displaystyle U(\Lambda) = \exp\left(-\frac{\mathrm{i}}{2}\omega_{\mu\nu}J^{\mu\nu}\right)$ & $\displaystyle \exp\left(-\frac{\mathrm{i}}{2}\omega_{\mu\nu}\hat{L}^{\mu\nu}\right)$ \\
\bottomrule
\end{tabular}
\end{table}





\section{量子旋量场}


%%%%%%%%%%%%%%%%%%%%%%%%%%%%%%%%%%%%%%%%%%%%%%%%%%%%%%%%%
\subsection{Dirac旋量场}



\begin{tabular}{lll}
\hline
线性空间名称 & 矢量表示空间 & 旋量表示空间 \\
\hline
维度 & 4 维 & 4 维 \\
空间中元素 & Lorentz 矢量 $A^{\mu}$ & Dirac 旋量 $\psi_{a}$ \\
Lorentz 群生成元 & $(\mathcal{J}^{\mu\nu})^{\alpha}_{\beta} \equiv \mathrm{i}(g^{\mu\alpha}\delta^{\nu}_{\beta} - g^{\nu\alpha}\delta^{\mu}_{\beta})$ & $\mathcal{S}^{\mu\nu} = \frac{\mathrm{i}}{4}[\gamma^{\mu},\gamma^{\nu}]$ \\
固有保时向 Lorentz 变换 & $\Lambda = \exp\left(-\frac{\mathrm{i}}{2}\omega_{\mu\nu}\mathcal{J}^{\mu\nu}\right)$ & $D(\Lambda) = \exp\left(-\frac{\mathrm{i}}{2}\omega_{\mu\nu}\mathcal{S}^{\mu\nu}\right)$ \\
\hline
线性空间名称 & Hilbert 空间 & 场空间 \\
\hline
维度 & 无限维 & 无限维 \\
空间中元素 & 态矢 $|\Psi\rangle$ & 场 $\phi(x), A^{\mu}(x), \psi_{a}(x)$ \\
Lorentz 群生成元 & 算符 $J^{\mu\nu}$ & $\hat{L}^{\mu\nu} = \mathrm{i}(x^{\mu}\partial^{\nu} - x^{\nu}\partial^{\mu})$ \\
固有保时向 Lorentz 变换 & $U(\Lambda) = \exp\left(-\frac{\mathrm{i}}{2}\omega_{\mu\nu}J^{\mu\nu}\right)$ & $\exp\left(-\frac{\mathrm{i}}{2}\omega_{\mu\nu}\hat{L}^{\mu\nu}\right)$ \\
\hline
\end{tabular}


\begin{table}[htbp]
\centering
\caption{与Lorentz变换相关的线性空间}
\begin{tabular}{lll}
\toprule
线性空间名称 & 矢量表示空间 & 旋量表示空间 \\
\midrule
维度 & 4 维 & 4 维 \\
空间中元素 & Lorentz 矢量 $A^{\mu}$ & Dirac 旋量 $\psi_{a}$ \\
Lorentz 群生成元 & $\displaystyle (\mathcal{J}^{\mu\nu})^{\alpha}_{\beta} \equiv \mathrm{i}\big(g^{\mu\alpha}\delta^{\nu}_{\beta} - g^{\nu\alpha}\delta^{\mu}_{\beta}\big)$ & $\displaystyle \mathcal{S}^{\mu\nu} = \frac{\mathrm{i}}{4}[\gamma^{\mu},\gamma^{\nu}]$ \\
固有保时向 Lorentz 变换 & $\displaystyle \Lambda = \exp\left(-\frac{\mathrm{i}}{2}\omega_{\mu\nu}\mathcal{J}^{\mu\nu}\right)$ & $\displaystyle D(\Lambda) = \exp\left(-\frac{\mathrm{i}}{2}\omega_{\mu\nu}\mathcal{S}^{\mu\nu}\right)$ \\
\midrule
线性空间名称 & Hilbert 空间 & 场空间 \\
\midrule
维度 & 无限维 & 无限维 \\
空间中元素 & 态矢 $\ket{\Psi}$ & 场 $\phi(x), A^{\mu}(x), \psi_{a}(x)$ \\
Lorentz 群生成元 & 算符 $J^{\mu\nu}$ & $\displaystyle \hat{L}^{\mu\nu} = \mathrm{i}(x^{\mu}\partial^{\nu} - x^{\nu}\partial^{\mu})$ \\
固有保时向 Lorentz 变换 & $\displaystyle U(\Lambda) = \exp\left(-\frac{\mathrm{i}}{2}\omega_{\mu\nu}J^{\mu\nu}\right)$ & $\displaystyle \exp\left(-\frac{\mathrm{i}}{2}\omega_{\mu\nu}\hat{L}^{\mu\nu}\right)$ \\
\bottomrule
\end{tabular}
\end{table}





\section{5.3}







\subsection{笔记}
具有四分量的Dirac旋量场分解为两个二分量Weyl旋量场和

四分量的Dirac旋量场
二分量的Weyl旋量场
左手Weyl旋量场
右手Weyl旋量场

\begin{equation}
    \psi =\left( \begin{array}{c}
	\psi _1\\
	\psi _2\\
	\psi _3\\
	\psi _4\\
\end{array} \right) \Rightarrow \psi =\left( \begin{array}{c}
	\eta _{\mathrm{L}}\\
	\eta _{\mathrm{R}}\\
\end{array} \right) \begin{array}{c}
	\eta _{\mathrm{L}}=\left( \begin{array}{c}
	\eta _{\mathrm{L}1}\\
	\eta _{\mathrm{L}2}\\
\end{array} \right)\\
	\eta _{\mathrm{R}}=\left( \begin{array}{c}
	\eta _{\mathrm{R}1}\\
	\eta _{\mathrm{R}2}\\
\end{array} \right)\\
\end{array}
\end{equation}


\subsection{笔记}



\subsection{推导}
由
\begin{equation}
    \begin{aligned}
        \mathrm{i}\bar{\sigma}^{\mu}\partial _{\mu}\eta _{\mathrm{L}}-m\eta _{\mathrm{R}}&=0
\\
\mathrm{i}\sigma ^{\mu}\partial _{\mu}\eta _{\mathrm{R}}-m\eta _{\mathrm{L}}&=0
    \end{aligned}
\end{equation}
当$$m=0$$
得到
\begin{equation}
    \begin{aligned}
        \mathrm{i}\bar{\sigma}^{\mu}\partial _{\mu}\eta _{\mathrm{L}}&=0
\\
\mathrm{i}\sigma ^{\mu}\partial _{\mu}\eta _{\mathrm{R}}&=0
    \end{aligned}
\end{equation}


\section{}



\subsection{}

\subsection{推导:}
记得利用AI修改括号的形式

1.计算
\begin{equation}
    \begin{aligned}
        \bar{u}\left( \mathbf{p},\lambda \right) &=u^{\dagger}\left( \mathbf{p},\lambda \right) \gamma ^0
\\
&=\left( \begin{matrix}
	\omega _{-\lambda}\left( \mathbf{p} \right) \xi _{\lambda}^{\dagger}\left( \mathbf{p} \right)&		\omega _{\lambda}\left( \mathbf{p} \right) \xi _{\lambda}^{\dagger}\left( \mathbf{p} \right)\\
\end{matrix} \right) \left( \begin{matrix}
	&		1\\
	1&		\\
\end{matrix} \right) 
\\
&=\left( \begin{matrix}
	\omega _{\lambda}\left( \mathbf{p} \right) \xi _{\lambda}^{\dagger}\left( \mathbf{p} \right)&		\omega _{-\lambda}\left( \mathbf{p} \right) \xi _{\lambda}^{\dagger}\left( \mathbf{p} \right)\\
\end{matrix} \right) 
    \end{aligned}
\end{equation}
且
\begin{equation}
    \begin{aligned}
        \bar{v}\left( \mathbf{p},\lambda \right) &=v^{\dagger}\left( \mathbf{p},\lambda \right) \gamma ^0
\\
&=\left( \begin{matrix}
	\lambda \omega _{\lambda}\left( \mathbf{p} \right) \xi _{-\lambda}^{\dagger}\left( \mathbf{p} \right)&		-\lambda \omega _{-\lambda}\left( \mathbf{p} \right) \xi _{-\lambda}^{\dagger}\left( \mathbf{p} \right)\\
\end{matrix} \right) \left( \begin{matrix}
	&		1\\
	1&		\\
\end{matrix} \right) 
\\
&=\left( \begin{matrix}
	-\lambda \omega _{-\lambda}\left( \mathbf{p} \right) \xi _{-\lambda}^{\dagger}\left( \mathbf{p} \right)&		\lambda \omega _{\lambda}\left( \mathbf{p} \right) \xi _{-\lambda}^{\dagger}\left( \mathbf{p} \right)\\
\end{matrix} \right) 
    \end{aligned}
\end{equation}
2.计算
\begin{equation}
    \begin{aligned}
        \bar{u}\left( \mathbf{p},\lambda \right) u\left( \mathbf{p},\lambda \prime \right) &=\left( \begin{matrix}
	\omega _{\lambda}\left( \mathbf{p} \right) \xi _{\lambda}^{\dagger}\left( \mathbf{p} \right)&		\omega _{-\lambda}\left( \mathbf{p} \right) \xi _{\lambda}^{\dagger}\left( \mathbf{p} \right)\\
\end{matrix} \right) \left( \begin{array}{c}
	\omega _{-\lambda \prime}\left( \mathbf{p} \right) \xi _{\lambda \prime}\left( \mathbf{p} \right)\\
	\omega _{\lambda \prime}\left( \mathbf{p} \right) \xi _{\lambda \prime}\left( \mathbf{p} \right)\\
\end{array} \right) 
\\
&=\omega _{\lambda}\left( \mathbf{p} \right) \xi _{\lambda}^{\dagger}\left( \mathbf{p} \right) \omega _{-\lambda \prime}\left( \mathbf{p} \right) \xi _{\lambda \prime}\left( \mathbf{p} \right) +\omega _{-\lambda}\left( \mathbf{p} \right) \xi _{\lambda}^{\dagger}\left( \mathbf{p} \right) \omega _{\lambda \prime}\left( \mathbf{p} \right) \xi _{\lambda \prime}\left( \mathbf{p} \right) 
\\
&=\omega _{\lambda}\left( \mathbf{p} \right) \omega _{-\lambda \prime}\left( \mathbf{p} \right) \xi _{\lambda}^{\dagger}\left( \mathbf{p} \right) \xi _{\lambda \prime}\left( \mathbf{p} \right) +\omega _{-\lambda}\left( \mathbf{p} \right) \omega _{\lambda \prime}\left( \mathbf{p} \right) \xi _{\lambda}^{\dagger}\left( \mathbf{p} \right) \xi _{\lambda \prime}\left( \mathbf{p} \right) 
\\
&=\left[ \omega _{\lambda}\left( \mathbf{p} \right) \omega _{-\lambda \prime}\left( \mathbf{p} \right) +\omega _{-\lambda}\left( \mathbf{p} \right) \omega _{\lambda \prime}\left( \mathbf{p} \right) \right] \xi _{\lambda}^{\dagger}\left( \mathbf{p} \right) \xi _{\lambda \prime}\left( \mathbf{p} \right) 
\\
&=\left[ \omega _{\lambda}\left( \mathbf{p} \right) \omega _{-\lambda \prime}\left( \mathbf{p} \right) +\omega _{-\lambda}\left( \mathbf{p} \right) \omega _{\lambda \prime}\left( \mathbf{p} \right) \right] \delta _{\lambda \lambda \prime}
\\
&=\left[ \omega _{\lambda}\left( \mathbf{p} \right) \omega _{-\lambda}\left( \mathbf{p} \right) +\omega _{-\lambda}\left( \mathbf{p} \right) \omega _{\lambda}\left( \mathbf{p} \right) \right] \delta _{\lambda \lambda \prime}
\\
&=\left[ \omega _{\lambda}\left( \mathbf{p} \right) \omega _{-\lambda}\left( \mathbf{p} \right) +\omega _{\lambda}\left( \mathbf{p} \right) \omega _{-\lambda}\left( \mathbf{p} \right) \right] \delta _{\lambda \lambda \prime}
\\
&=2\omega _{\lambda}\left( \mathbf{p} \right) \omega _{-\lambda}\left( \mathbf{p} \right) \delta _{\lambda \lambda \prime}
\\
&=2m\delta _{\lambda \lambda \prime}
    \end{aligned}
\end{equation}
且
\begin{equation}
    \begin{aligned}
        \bar{v}\left( \mathbf{p},\lambda \right) v\left( \mathbf{p},\lambda ^{\prime} \right) &=\left( \begin{matrix}
	-\lambda \omega _{-\lambda}\left( \mathbf{p} \right) \xi _{-\lambda}^{\dagger}\left( \mathbf{p} \right)&		\lambda \omega _{\lambda}\left( \mathbf{p} \right) \xi _{-\lambda}^{\dagger}\left( \mathbf{p} \right)\\
\end{matrix} \right) \left( \begin{array}{c}
	\lambda \prime \omega _{\lambda \prime}\left( \mathbf{p} \right) \xi _{-\lambda \prime}\left( \mathbf{p} \right)\\
	-\lambda \prime \omega _{-\lambda \prime}\left( \mathbf{p} \right) \xi _{-\lambda \prime}\left( \mathbf{p} \right)\\
\end{array} \right) 
\\
&=-\lambda \omega _{-\lambda}\left( \mathbf{p} \right) \xi _{-\lambda}^{\dagger}\left( \mathbf{p} \right) \cdot \lambda \prime \omega _{\lambda \prime}\left( \mathbf{p} \right) \xi _{-\lambda \prime}\left( \mathbf{p} \right) -\lambda \omega _{\lambda}\left( \mathbf{p} \right) \xi _{-\lambda}^{\dagger}\left( \mathbf{p} \right) \cdot \lambda \prime \omega _{-\lambda \prime}\left( \mathbf{p} \right) \xi _{-\lambda \prime}\left( \mathbf{p} \right) 
\\
&=-\lambda \lambda \prime \omega _{-\lambda}\left( \mathbf{p} \right) \omega _{\lambda \prime}\left( \mathbf{p} \right) \cdot \xi _{-\lambda}^{\dagger}\left( \mathbf{p} \right) \xi _{-\lambda \prime}\left( \mathbf{p} \right) -\lambda \lambda \prime \omega _{\lambda}\left( \mathbf{p} \right) \omega _{-\lambda \prime}\left( \mathbf{p} \right) \cdot \xi _{-\lambda}^{\dagger}\left( \mathbf{p} \right) \xi _{-\lambda \prime}\left( \mathbf{p} \right) 
\\
&=-\lambda \lambda \prime \left[ \omega _{-\lambda}\left( \mathbf{p} \right) \omega _{\lambda \prime}\left( \mathbf{p} \right) +\omega _{\lambda}\left( \mathbf{p} \right) \omega _{-\lambda \prime}\left( \mathbf{p} \right) \right] \xi _{-\lambda}^{\dagger}\left( \mathbf{p} \right) \xi _{-\lambda \prime}\left( \mathbf{p} \right) 
\\
&=-\lambda \lambda \prime \left[ \omega _{-\lambda}\left( \mathbf{p} \right) \omega _{\lambda \prime}\left( \mathbf{p} \right) +\omega _{\lambda}\left( \mathbf{p} \right) \omega _{-\lambda \prime}\left( \mathbf{p} \right) \right] \delta _{\lambda \lambda \prime}
\\
&=-\lambda ^2\left[ \omega _{-\lambda}\left( \mathbf{p} \right) \omega _{\lambda}\left( \mathbf{p} \right) +\omega _{\lambda}\left( \mathbf{p} \right) \omega _{-\lambda}\left( \mathbf{p} \right) \right] \delta _{\lambda \lambda \prime}
\\
&=-\lambda ^2\left[ \omega _{\lambda}\left( \mathbf{p} \right) \omega _{-\lambda}\left( \mathbf{p} \right) +\omega _{\lambda}\left( \mathbf{p} \right) \omega _{-\lambda}\left( \mathbf{p} \right) \right] \delta _{\lambda \lambda \prime}
\\
&=-2\lambda ^2\omega _{\lambda}\left( \mathbf{p} \right) \omega _{-\lambda}\left( \mathbf{p} \right) \delta _{\lambda \lambda \prime}
\\
&=-2m\delta _{\lambda \lambda ^{\prime}}
    \end{aligned}
\end{equation}
且
\begin{equation}
    \begin{aligned}
        \begin{aligned}
            \bar{u}\left( \mathbf{p},\lambda \right) v\left( \mathbf{p},\lambda ^{\prime} \right) &=\left( \begin{matrix}
	\omega _{\lambda}\left( \mathbf{p} \right) \xi _{\lambda}^{\dagger}\left( \mathbf{p} \right)&		\omega _{-\lambda}\left( \mathbf{p} \right) \xi _{\lambda}^{\dagger}\left( \mathbf{p} \right)\\
\end{matrix} \right) \left( \begin{array}{c}
	\lambda \prime \omega _{\lambda \prime}\left( \mathbf{p} \right) \xi _{-\lambda \prime}\left( \mathbf{p} \right)\\
	-\lambda \prime \omega _{-\lambda \prime}\left( \mathbf{p} \right) \xi _{-\lambda \prime}\left( \mathbf{p} \right)\\
\end{array} \right) 
\\
&=\omega _{\lambda}\left( \mathbf{p} \right) \xi _{\lambda}^{\dagger}\left( \mathbf{p} \right) \cdot \lambda \prime \omega _{\lambda \prime}\left( \mathbf{p} \right) \xi _{-\lambda \prime}\left( \mathbf{p} \right) -\omega _{-\lambda}\left( \mathbf{p} \right) \xi _{\lambda}^{\dagger}\left( \mathbf{p} \right) \cdot \lambda \prime \omega _{-\lambda \prime}\left( \mathbf{p} \right) \xi _{-\lambda \prime}\left( \mathbf{p} \right) 
\\
&=\lambda \prime \omega _{\lambda}\left( \mathbf{p} \right) \omega _{\lambda \prime}\left( \mathbf{p} \right) \cdot \xi _{\lambda}^{\dagger}\left( \mathbf{p} \right) \xi _{-\lambda \prime}\left( \mathbf{p} \right) -\lambda \prime \omega _{-\lambda}\left( \mathbf{p} \right) \omega _{-\lambda \prime}\left( \mathbf{p} \right) \cdot \xi _{\lambda}^{\dagger}\left( \mathbf{p} \right) \xi _{-\lambda \prime}\left( \mathbf{p} \right) 
\\
&=\lambda \prime \left[ \omega _{\lambda}\left( \mathbf{p} \right) \omega _{\lambda \prime}\left( \mathbf{p} \right) -\omega _{-\lambda}\left( \mathbf{p} \right) \omega _{-\lambda \prime}\left( \mathbf{p} \right) \right] \xi _{\lambda}^{\dagger}\left( \mathbf{p} \right) \xi _{-\lambda \prime}\left( \mathbf{p} \right) 
\\
&=\lambda ^{\prime}\left[ \omega _{\lambda}\left( \mathbf{p} \right) \omega _{\lambda ^{\prime}}\left( \mathbf{p} \right) -\omega _{-\lambda}\left( \mathbf{p} \right) \omega _{-\lambda ^{\prime}}\left( \mathbf{p} \right) \right] \delta _{\lambda ,-\lambda ^{\prime}}
\\
&=-\lambda \left[ \omega _{\lambda}\left( \mathbf{p} \right) \omega _{-\lambda}\left( \mathbf{p} \right) -\omega _{-\lambda}\left( \mathbf{p} \right) \omega _{\lambda}\left( \mathbf{p} \right) \right] \delta _{\lambda ,-\lambda ^{\prime}}
\\
&=-\lambda \left[ \omega _{\lambda}\left( \mathbf{p} \right) \omega _{-\lambda}\left( \mathbf{p} \right) -\omega _{\lambda}\left( \mathbf{p} \right) \omega _{-\lambda}\left( \mathbf{p} \right) \right] \delta _{\lambda ,-\lambda ^{\prime}}
\\
&=0
        \end{aligned}
    \end{aligned}
\end{equation}
且
\begin{equation}
    \begin{aligned}
        \bar{v}\left( \mathbf{p},\lambda \right) u\left( \mathbf{p},\lambda ^{\prime} \right) &=\left( \begin{matrix}
	-\lambda \omega _{-\lambda}\left( \mathbf{p} \right) \xi _{-\lambda}^{\dagger}\left( \mathbf{p} \right)&		\lambda \omega _{\lambda}\left( \mathbf{p} \right) \xi _{-\lambda}^{\dagger}\left( \mathbf{p} \right)\\
\end{matrix} \right) \left( \begin{array}{c}
	\omega _{-\lambda \prime}\left( \mathbf{p} \right) \xi _{\lambda \prime}\left( \mathbf{p} \right)\\
	\omega _{\lambda \prime}\left( \mathbf{p} \right) \xi _{\lambda \prime}\left( \mathbf{p} \right)\\
\end{array} \right) 
\\
&=-\lambda \omega _{-\lambda}\left( \mathbf{p} \right) \xi _{-\lambda}^{\dagger}\left( \mathbf{p} \right) \cdot \omega _{-\lambda \prime}\left( \mathbf{p} \right) \xi _{\lambda \prime}\left( \mathbf{p} \right) +\lambda \omega _{\lambda}\left( \mathbf{p} \right) \xi _{-\lambda}^{\dagger}\left( \mathbf{p} \right) \cdot \omega _{\lambda \prime}\left( \mathbf{p} \right) \xi _{\lambda \prime}\left( \mathbf{p} \right) 
\\
&=-\lambda \omega _{-\lambda}\left( \mathbf{p} \right) \omega _{-\lambda \prime}\left( \mathbf{p} \right) \cdot \xi _{-\lambda}^{\dagger}\left( \mathbf{p} \right) \xi _{\lambda \prime}\left( \mathbf{p} \right) +\lambda \omega _{\lambda}\left( \mathbf{p} \right) \omega _{\lambda \prime}\left( \mathbf{p} \right) \cdot \xi _{-\lambda}^{\dagger}\left( \mathbf{p} \right) \xi _{\lambda \prime}\left( \mathbf{p} \right) 
\\
&=\lambda \left[ -\omega _{-\lambda}\left( \mathbf{p} \right) \omega _{-\lambda ^{\prime}}\left( \mathbf{p} \right) +\omega _{\lambda}\left( \mathbf{p} \right) \omega _{\lambda ^{\prime}}\left( \mathbf{p} \right) \right] \xi _{-\lambda}^{\dagger}\left( \mathbf{p} \right) \xi _{\lambda ^{\prime}}\left( \mathbf{p} \right) 
\\
&=\lambda \left[ -\omega _{-\lambda}\left( \mathbf{p} \right) \omega _{-\lambda ^{\prime}}\left( \mathbf{p} \right) +\omega _{\lambda}\left( \mathbf{p} \right) \omega _{\lambda ^{\prime}}\left( \mathbf{p} \right) \right] \delta _{-\lambda ,\lambda ^{\prime}}
\\
&=\lambda \left[ -\omega _{-\lambda}\left( \mathbf{p} \right) \omega _{\lambda}\left( \mathbf{p} \right) +\omega _{\lambda}\left( \mathbf{p} \right) \omega _{-\lambda}\left( \mathbf{p} \right) \right] \delta _{-\lambda ,\lambda ^{\prime}}
\\
&=\lambda \left[ -\omega _{\lambda}\left( \mathbf{p} \right) \omega _{-\lambda}\left( \mathbf{p} \right) +\omega _{\lambda}\left( \mathbf{p} \right) \omega _{-\lambda}\left( \mathbf{p} \right) \right] \delta _{-\lambda ,\lambda ^{\prime}}
\\
&=0
    \end{aligned}
\end{equation}
汇总结论




\subsection{推导:螺旋度求和关系=自旋求和关系}
记得利用AI修改括号的形式
1.由
\begin{equation}
    \begin{aligned}
        \left( p\cdot \bar{\sigma} \right) \xi _{\lambda}\left( \mathbf{p} \right) &=\omega _{\lambda}^{2}\left( \mathbf{p} \right) \xi _{\lambda}\left( \mathbf{p} \right) 
\\
\left( p\cdot \sigma \right) \xi _{\lambda}\left( \mathbf{p} \right) &=\omega _{-\lambda}^{2}\left( \mathbf{p} \right) \xi _{\lambda}\left( \mathbf{p} \right) 
    \end{aligned}
\end{equation}
写出
\begin{equation}
    \begin{aligned}
        \left( p\cdot \bar{\sigma} \right) \xi _{-\lambda}\left( \mathbf{p} \right) &=\omega _{-\lambda}^{2}\left( \mathbf{p} \right) \xi _{-\lambda}\left( \mathbf{p} \right) 
\\
\left( p\cdot \sigma \right) \xi _{-\lambda}\left( \mathbf{p} \right) &=\omega _{\lambda}^{2}\left( \mathbf{p} \right) \xi _{-\lambda}\left( \mathbf{p} \right) 
    \end{aligned}
\end{equation}
2.由
\begin{equation}
    \sum_{\lambda =\pm}{\xi _{\lambda}}\left( \mathbf{p} \right) \xi _{\lambda}^{\dagger}\left( \mathbf{p} \right) =1
\end{equation}
写出
\begin{equation}
    \sum_{\lambda =\pm}{\xi _{-\lambda}}\left( \mathbf{p} \right) \xi _{-\lambda}^{\dagger}\left( \mathbf{p} \right) =1
\end{equation}
2.计算
\begin{equation}
    \begin{aligned}
        \sum_{\lambda =\pm}{u}\left( \mathbf{p},\lambda \right) \bar{u}\left( \mathbf{p},\lambda \right) &=\sum_{\lambda =\pm}{\left( \begin{array}{c}
	\omega _{-\lambda}\left( \mathbf{p} \right) \xi _{\lambda}\left( \mathbf{p} \right)\\
	\omega _{\lambda}\left( \mathbf{p} \right) \xi _{\lambda}\left( \mathbf{p} \right)\\
\end{array} \right)}\left( \begin{matrix}
	\omega _{\lambda}\left( \mathbf{p} \right) \xi _{\lambda}^{\dagger}\left( \mathbf{p} \right)&		\omega _{-\lambda}\left( \mathbf{p} \right) \xi _{\lambda}^{\dagger}\left( \mathbf{p} \right)\\
\end{matrix} \right) 
\\
&=\sum_{\lambda =\pm}{\left( \begin{matrix}
	\omega _{-\lambda}\left( \mathbf{p} \right) \omega _{\lambda}\left( \mathbf{p} \right) \xi _{\lambda}\left( \mathbf{p} \right) \xi _{\lambda}^{\dagger}\left( \mathbf{p} \right)&		\omega _{-\lambda}^{2}\left( \mathbf{p} \right) \xi _{\lambda}\left( \mathbf{p} \right) \xi _{\lambda}^{\dagger}\left( \mathbf{p} \right)\\
	\omega _{\lambda}^{2}\left( \mathbf{p} \right) \xi _{\lambda}\left( \mathbf{p} \right) \xi _{\lambda}^{\dagger}\left( \mathbf{p} \right)&		\omega _{\lambda}\left( \mathbf{p} \right) \omega _{-\lambda}\left( \mathbf{p} \right) \xi _{\lambda}\left( \mathbf{p} \right) \xi _{\lambda}^{\dagger}\left( \mathbf{p} \right)\\
\end{matrix} \right)}
\\
&=\sum_{\lambda =\pm}{\left( \begin{matrix}
	m\xi _{\lambda}\left( \mathbf{p} \right) \xi _{\lambda}^{\dagger}\left( \mathbf{p} \right)&		\left( p\cdot \sigma \right) \xi _{\lambda}\left( \mathbf{p} \right) \xi _{\lambda}^{\dagger}\left( \mathbf{p} \right)\\
	\left( p\cdot \bar{\sigma} \right) \xi _{\lambda}\left( \mathbf{p} \right) \xi _{\lambda}^{\dagger}\left( \mathbf{p} \right)&		m\xi _{\lambda}\left( \mathbf{p} \right) \xi _{\lambda}^{\dagger}\left( \mathbf{p} \right)\\
\end{matrix} \right)}
\\
&=\left( \begin{matrix}
	m&		p\cdot \sigma\\
	p\cdot \bar{\sigma}&		m\\
\end{matrix} \right) 
\\
&=\left( \begin{matrix}
	&		p\cdot \sigma\\
	p\cdot \bar{\sigma}&		\\
\end{matrix} \right) +\left( \begin{matrix}
	m&		\\
	&		m\\
\end{matrix} \right) 
\\
&=p_{\mu}\gamma ^{\mu}+m
    \end{aligned}
\end{equation}
同样地
\begin{equation}
    \begin{aligned}
        \sum_{\lambda =\pm}{v}\left( \mathbf{p},\lambda \right) \bar{v}\left( \mathbf{p},\lambda \right) &=\sum_{\lambda =\pm}{\left( \begin{array}{c}
	\lambda \omega _{\lambda}\left( \mathbf{p} \right) \xi _{-\lambda}\left( \mathbf{p} \right)\\
	-\lambda \omega _{-\lambda}\left( \mathbf{p} \right) \xi _{-\lambda}\left( \mathbf{p} \right)\\
\end{array} \right)}\left( \begin{matrix}
	-\lambda \omega _{-\lambda}\left( \mathbf{p} \right) \xi _{-\lambda}^{\dagger}\left( \mathbf{p} \right)&		\lambda \omega _{\lambda}\left( \mathbf{p} \right) \xi _{-\lambda}^{\dagger}\left( \mathbf{p} \right)\\
\end{matrix} \right) 
\\
&=\sum_{\lambda =\pm}{\left( \begin{matrix}
	-\lambda ^2\omega _{\lambda}\left( \mathbf{p} \right) \omega _{-\lambda}\left( \mathbf{p} \right) \xi _{-\lambda}\left( \mathbf{p} \right) \xi _{-\lambda}^{\dagger}\left( \mathbf{p} \right)&		\lambda ^2\omega _{\lambda}^{2}\left( \mathbf{p} \right) \xi _{-\lambda}\left( \mathbf{p} \right) \xi _{-\lambda}^{\dagger}\left( \mathbf{p} \right)\\
	\lambda ^2\omega _{-\lambda}^{2}\left( \mathbf{p} \right) \xi _{-\lambda}\left( \mathbf{p} \right) \xi _{-\lambda}^{\dagger}\left( \mathbf{p} \right)&		-\lambda ^2\omega _{-\lambda}\left( \mathbf{p} \right) \omega _{\lambda}\left( \mathbf{p} \right) \xi _{-\lambda}\left( \mathbf{p} \right) \xi _{-\lambda}^{\dagger}\left( \mathbf{p} \right)\\
\end{matrix} \right)}
\\
&=\sum_{\lambda =\pm}{\left( \begin{matrix}
	-m\xi _{-\lambda}\left( \mathbf{p} \right) \xi _{-\lambda}^{\dagger}\left( \mathbf{p} \right)&		\left( p\cdot \sigma \right) \xi _{-\lambda}\left( \mathbf{p} \right) \xi _{-\lambda}^{\dagger}\left( \mathbf{p} \right)\\
	\left( p\cdot \bar{\sigma} \right) \xi _{-\lambda}\left( \mathbf{p} \right) \xi _{-\lambda}^{\dagger}\left( \mathbf{p} \right)&		-m\xi _{-\lambda}\left( \mathbf{p} \right) \xi _{-\lambda}^{\dagger}\left( \mathbf{p} \right)\\
\end{matrix} \right)}
\\
&=\left( \begin{matrix}
	-m&		p\cdot \sigma\\
	p\cdot \bar{\sigma}&		-m\\
\end{matrix} \right) 
\\
&=\left( \begin{matrix}
	&		p\cdot \sigma\\
	p\cdot \bar{\sigma}&		\\
\end{matrix} \right) +\left( \begin{matrix}
	-m&		\\
	&		-m\\
\end{matrix} \right) 
\\
&=p_{\mu}\gamma ^{\mu}-m
    \end{aligned}
\end{equation}
其中
\begin{equation}
    \begin{aligned}
        \left( \begin{matrix}
	&		p\cdot \sigma\\
	p\cdot \bar{\sigma}&		\\
\end{matrix} \right) &=\left( \begin{matrix}
	&		p^0\sigma ^0-p^i\sigma ^i\\
	p^0\sigma ^0+p^i\sigma ^i&		\\
\end{matrix} \right) 
\\
&=\left( \begin{matrix}
	&		p^0\sigma ^0\\
	p^0\sigma ^0&		\\
\end{matrix} \right) +\left( \begin{matrix}
	&		-p^i\sigma ^i\\
	p^i\sigma ^i&		\\
\end{matrix} \right) 
\\
&=p^0\left( \begin{matrix}
	&		1\\
	1&		\\
\end{matrix} \right) -p^i\left( \begin{matrix}
	&		\sigma ^i\\
	-\sigma ^i&		\\
\end{matrix} \right) 
\\
&=p^0\gamma ^0-p^i\gamma ^i
\\
&=p_{\mu}\gamma ^{\mu}
    \end{aligned}
\end{equation}







\section{量子旋量场}


%%%%%%%%%%%%%%%%%%%%%%%%%%%%%%%%%%%%%%%%%%%%%%%%%%%%%%%%%
\subsection{Dirac旋量场}



\begin{tabular}{lll}
\hline
线性空间名称 & 矢量表示空间 & 旋量表示空间 \\
\hline
维度 & 4 维 & 4 维 \\
空间中元素 & Lorentz 矢量 $A^{\mu}$ & Dirac 旋量 $\psi_{a}$ \\
Lorentz 群生成元 & $(\mathcal{J}^{\mu\nu})^{\alpha}_{\beta} \equiv \mathrm{i}(g^{\mu\alpha}\delta^{\nu}_{\beta} - g^{\nu\alpha}\delta^{\mu}_{\beta})$ & $\mathcal{S}^{\mu\nu} = \frac{\mathrm{i}}{4}[\gamma^{\mu},\gamma^{\nu}]$ \\
固有保时向 Lorentz 变换 & $\Lambda = \exp\left(-\frac{\mathrm{i}}{2}\omega_{\mu\nu}\mathcal{J}^{\mu\nu}\right)$ & $D(\Lambda) = \exp\left(-\frac{\mathrm{i}}{2}\omega_{\mu\nu}\mathcal{S}^{\mu\nu}\right)$ \\
\hline
线性空间名称 & Hilbert 空间 & 场空间 \\
\hline
维度 & 无限维 & 无限维 \\
空间中元素 & 态矢 $|\Psi\rangle$ & 场 $\phi(x), A^{\mu}(x), \psi_{a}(x)$ \\
Lorentz 群生成元 & 算符 $J^{\mu\nu}$ & $\hat{L}^{\mu\nu} = \mathrm{i}(x^{\mu}\partial^{\nu} - x^{\nu}\partial^{\mu})$ \\
固有保时向 Lorentz 变换 & $U(\Lambda) = \exp\left(-\frac{\mathrm{i}}{2}\omega_{\mu\nu}J^{\mu\nu}\right)$ & $\exp\left(-\frac{\mathrm{i}}{2}\omega_{\mu\nu}\hat{L}^{\mu\nu}\right)$ \\
\hline
\end{tabular}


\begin{table}[htbp]
\centering
\caption{与Lorentz变换相关的线性空间}
\begin{tabular}{lll}
\toprule
线性空间名称 & 矢量表示空间 & 旋量表示空间 \\
\midrule
维度 & 4 维 & 4 维 \\
空间中元素 & Lorentz 矢量 $A^{\mu}$ & Dirac 旋量 $\psi_{a}$ \\
Lorentz 群生成元 & $\displaystyle (\mathcal{J}^{\mu\nu})^{\alpha}_{\beta} \equiv \mathrm{i}\big(g^{\mu\alpha}\delta^{\nu}_{\beta} - g^{\nu\alpha}\delta^{\mu}_{\beta}\big)$ & $\displaystyle \mathcal{S}^{\mu\nu} = \frac{\mathrm{i}}{4}[\gamma^{\mu},\gamma^{\nu}]$ \\
固有保时向 Lorentz 变换 & $\displaystyle \Lambda = \exp\left(-\frac{\mathrm{i}}{2}\omega_{\mu\nu}\mathcal{J}^{\mu\nu}\right)$ & $\displaystyle D(\Lambda) = \exp\left(-\frac{\mathrm{i}}{2}\omega_{\mu\nu}\mathcal{S}^{\mu\nu}\right)$ \\
\midrule
线性空间名称 & Hilbert 空间 & 场空间 \\
\midrule
维度 & 无限维 & 无限维 \\
空间中元素 & 态矢 $\ket{\Psi}$ & 场 $\phi(x), A^{\mu}(x), \psi_{a}(x)$ \\
Lorentz 群生成元 & 算符 $J^{\mu\nu}$ & $\displaystyle \hat{L}^{\mu\nu} = \mathrm{i}(x^{\mu}\partial^{\nu} - x^{\nu}\partial^{\mu})$ \\
固有保时向 Lorentz 变换 & $\displaystyle U(\Lambda) = \exp\left(-\frac{\mathrm{i}}{2}\omega_{\mu\nu}J^{\mu\nu}\right)$ & $\displaystyle \exp\left(-\frac{\mathrm{i}}{2}\omega_{\mu\nu}\hat{L}^{\mu\nu}\right)$ \\
\bottomrule
\end{tabular}
\end{table}





\section{习题5}

\newpage
\subsection{5.1}
证明下列等式。
(a) $\gamma^\mu \psi = 2p^\mu - \psi \gamma^\mu$.
(b) $\psi \psi = p^2$.
(c) $\{\psi \psi, \gamma^\mu \} = 2p^\mu k \psi - 2k^\mu \psi \psi + 2q^\mu \psi k$.
(d) $\gamma^\mu \gamma_\mu = 4$.
(e) $\sigma^\mu \sigma_\mu = 12$.
(f) $\epsilon_{\mu \nu \rho \sigma} \sigma^{\mu \nu} \sigma^{\rho \sigma} = -24i \gamma^5$.
(g) $\gamma_\mu \gamma^5 = -\frac{i}{6} \epsilon_{\mu \nu \rho \sigma} \gamma^\mu \gamma^\rho \gamma^\sigma$.
(h) $[\gamma_\mu, \gamma_\nu] \gamma^5 = i \epsilon_{\mu \nu \rho \sigma} \gamma^\rho \gamma^\sigma$.
(i) $\epsilon_{\mu \nu \rho \sigma} \sigma^{\rho \sigma} = -2i \sigma_{\mu \nu} \gamma^5$.

\newpage
\subsection{5.2}
设自由 Dirac 旋量场 $\psi(x)$ 的拉氏量为
$$\mathcal{L} = \frac{i}{2} \bar{\psi} \gamma^\mu \partial_\mu \psi - m \bar{\psi} \psi, \tag{5.295}$$
证明由 Euler-Lagrange 方程(1.167)推出的经典运动方程也是 Dirac 方程(5.107)。

\newpage
\subsection{5.3}
对于平面波旋量系数 $u(p, \lambda)$ 和 $v(k, \lambda')$,证明下列等式。
(a) $(u \gamma^\mu v)^* = \bar{v} \gamma^\mu u$.
(b) $(\bar{u} \gamma^5 v)^* = -\bar{v} \gamma^5 u$.
(c) $(\bar{u} \gamma^\mu \gamma^5 v)^* = \bar{v} \gamma^\mu \gamma^5 u$.
(d) $(\bar{u} \sigma^\mu v)^* = \bar{v} \sigma^\mu v$.
(e) $(\bar{u} \gamma^5 \sigma^\mu v)^* = -\bar{v} \gamma^5 \sigma^\mu v$.

\newpage
\subsection{5.4}
证明 Gordon 恒等式
$$\bar{u}(p, \lambda) \gamma^\mu u(k, \lambda') = \bar{u}(p, \lambda) \left( \frac{p^\mu + k^\mu}{2m} + \frac{i\sigma^\mu v}{2m} \right) u(k, \lambda'),$$
其中 $q^\mu \equiv p^\mu - k^\mu$。

\newpage
\subsection{5.5}
在球坐标系中,动量表达为 $p = |p| \hat{p} = |p| (s_\theta c_\phi, s_\theta s_\phi, c_\theta)$,其中 $s_\theta \equiv \sin \theta$,$c_\theta \equiv \cos \theta$。
(a) 推出
$$\hat{p} \cdot \sigma = 
\begin{pmatrix}
c_\theta & e^{-i\phi} s_\theta \\
e^{i\phi} s_\theta & -c_\theta
\end{pmatrix}.$$
(b) 推出螺旋态表达式
$$\xi_+ (p) = 
\begin{pmatrix}
c_{\theta/2} \\
e^{i\phi} s_{\theta/2}
\end{pmatrix},
\quad \xi_- (p) = 
\begin{pmatrix}
-e^{-i\phi} s_{\theta/2} \\
c_{\theta/2}
\end{pmatrix}.$$
(c) 根据以上两步结果验证 $(\hat{p} \cdot \sigma) \xi_+ (p) = + \xi_+ (p)$ 和 $(\hat{p} \cdot \sigma) \xi_- (p) = - \xi_- (p)$。
(d) 证明
$$\exp(i\alpha \hat{p} \cdot \sigma) = \cos \alpha + i (\hat{p} \cdot \sigma) \sin \alpha.$$

\newpage
\subsection{5.6}
在 Dirac 表象(也称为标准表象)中,$\gamma$ 矩阵表达为
$$\gamma^0 = 
\begin{pmatrix}
1 & 0 \\
-1 & 0
\end{pmatrix},
\quad \gamma^i = 
\begin{pmatrix}
\sigma^i \\
-\sigma^i
\end{pmatrix}.$$
将平面波旋量系数表达为
$$u(p, \sigma) = \sqrt{E_p + m} \begin{pmatrix}
\zeta_\sigma & 0 \\
\frac{\sigma \cdot p}{E_p + m} \zeta_\sigma
\end{pmatrix},
\quad v(p, \sigma) = \sqrt{E_p + m} \begin{pmatrix}
\frac{\sigma \cdot p}{E_p + m} \eta_{-\sigma} \\
\eta_{-\sigma}
\end{pmatrix}.$$
其中 $\zeta_\sigma$ 是某个固定方向上的二分量自旋本征态,不依赖于动量 $p$,$\sigma = \pm 1/2$ 是磁量子数。记这个固定方向的单位矢量为 $n$,则 $\zeta_\sigma$ 满足的本征方程、正交归一关系和完备性关系为
$$\frac{1}{2} (n \cdot \sigma) \zeta_\sigma = \sigma \zeta_\sigma,
\quad \zeta_\sigma^T \zeta_\sigma' = \delta_{\sigma \sigma'},
\quad \sum_{\sigma = \pm 1/2} \zeta_\sigma \zeta_\sigma^T = 1.$$
另一方面,$\eta_\sigma$ 定义为
$$\eta_\sigma \equiv i \sigma^2 \zeta_{-\sigma}^*.$$
(a) 验证 (5.300) 式表达的 $\gamma^\mu$ 满足反对易关系 (5.1),并推出
$$\gamma^5 = \begin{pmatrix}
1 \\
1
\end{pmatrix} \tag{5.304}$$
和
$$\mathcal{S}^{0i} = \frac{i}{2} \begin{pmatrix}
\sigma^i \\
\sigma^i
\end{pmatrix}, \quad \mathcal{S}^{ij} = \frac{1}{2} e^{ijk} \begin{pmatrix}
\sigma^k \\
\sigma^k
\end{pmatrix}. \tag{5.305}$$
(b) 证明 Pauli 矩阵 (3.53) 满足
$$\sigma^i \sigma^2 = -\sigma^2 (\sigma^i)^T, \tag{5.306}$$
进而证明
$$\frac{1}{2} (\mathbf{n} \cdot \sigma) \eta_\sigma = \sigma \eta_\sigma, \quad \eta_\sigma^{\dagger} \eta_{\sigma'} = \delta_{\sigma \sigma'}, \quad \sum_{\sigma = \pm 1/2} \eta_\sigma \eta_\sigma^{\dagger} = 1. \tag{5.307}$$
这说明 $\eta_\sigma$ 也是本征值为 $\sigma$ 的自旋本征态,跟 $\zeta_\sigma$ 至多相差一个相位因子 $\tau_\sigma$,即
$$\eta_\sigma = \tau_\sigma \zeta_\sigma. \tag{5.308}$$
(c) 设 $\mathbf{n} = (s_0 c_\phi, s_0 s_\phi, c_\theta)$,其中 $s_0 \equiv \sin \theta$,$c_\theta \equiv \cos \theta$。类似于 (5.298) 式,可将 $\zeta_\sigma$ 取为
$$\zeta_{+1/2} = \begin{pmatrix}
c_{\theta/2} \\
e^{i\phi} s_{\theta/2}
\end{pmatrix}, \quad \zeta_{-1/2} = \begin{pmatrix}
-e^{-i\phi} s_{\theta/2} \\
c_{\theta/2}
\end{pmatrix}. \tag{5.309}$$
由此推出 $\eta_\sigma$ 的具体形式,证明
$$\tau_\sigma = 2\sigma. \tag{5.310}$$
(d) 证明 $u(\mathbf{p}, \sigma)$ 和 $v(\mathbf{p}, \sigma)$ 满足运动方程
$$(\psi - m) u(\mathbf{p}, \sigma) = 0, \quad (\psi + m) v(\mathbf{p}, \sigma) = 0, \tag{5.311}$$
正交归一关系
$$u^\dagger (\mathbf{p}, \sigma) u(\mathbf{p}, \sigma') = 2E_p \delta_{\sigma \sigma'}, \quad v^\dagger (\mathbf{p}, \sigma) v(\mathbf{p}, \sigma') = 2E_p \delta_{\sigma \sigma'}, \quad u^\dagger (\mathbf{p}, \sigma) v(-\mathbf{p}, \sigma') = 0, \tag{5.312}$$
和自旋求和关系
$$\sum_{\sigma = \pm 1/2} u(\mathbf{p}, \sigma) \bar{u}(\mathbf{p}, \sigma) = \psi + m, \quad \sum_{\sigma = \pm 1/2} v(\mathbf{p}, \sigma) \bar{v}(\mathbf{p}, \sigma) = \psi - m. \tag{5.313}$$
于是,将 Dirac 旋量场的平面波展开式写成
$$\psi(x) = \int \frac{d^3 p}{(2\pi)^3} \frac{1}{\sqrt{2E_p}} \sum_{\sigma = \pm 1/2} [u(\mathbf{p}, \sigma) c_{\mathbf{p}, \sigma} e^{-ip \cdot x} + v(\mathbf{p}, \sigma) d_{\mathbf{p}, \sigma}^\dagger e^{ip \cdot x}], \tag{5.314}$$

\newpage
\subsection{5.7}
将 Weyl 表象中的 $\gamma$ 矩阵(5.68)记为 $\gamma_W^H$, Dirac 表象中的 $\gamma$ 矩阵(5.300)记为 $\gamma_D^H$, 寻找么正矩阵 $U$, 使得 $\gamma_D^H = U^\dagger \gamma_W^H U$。

\newpage
\subsection{5.8}
对于自由 Dirac 旋量场 $\psi(x)$, 根据 1.7 节关于 Noether 定理的讨论,Lorentz 对称性给出的守恒荷算符(1.239)表达为
$$J^{\mu \nu} = \int d^3 x [T^{0 \mu} x^{\mu} - T^{0 \mu} x^{\nu} - i \pi_a (\mathcal{S}^{\mu \nu})_{ab} \psi_b],$$
其中 $T^{0 \mu} = \pi_a \partial^{\mu} \psi_a$, 利用等时反对易关系(5.238)推出(5.65)式。

\newpage
\subsection{5.9}
自旋求和关系(5.214)等价于
$$\sum_{\lambda = \pm} u(p, \lambda) u^\dagger (p, \lambda) = (\psi + m) \gamma^0, \quad \sum_{\lambda = \pm} v(p, \lambda) v^\dagger (p, \lambda) = (\psi - m) \gamma^0.$$
(5.317)
利用上述、产生湮灭算符的反对易关系(5.246)以及 $\psi(x,t)$ 和 $\psi^\dagger(x,t)$ 的平面波展开式(5.216)和(5.217),推出等时反对易关系(5.239)。

\newpage
\subsection{5.10}
假如采用等时对易关系量子化 Dirac 旋量场,利用得到的产生湮灭算符对易关系(5.236)和哈密顿量表达式(5.237)推出
$$[H, b_{p, \lambda}^\dagger] = E_p b_{p, \lambda}^\dagger.$$
(5.318)
真空态 $|0\rangle$ 满足
$$a_{p, \lambda}|0\rangle = b_{p, \lambda}|0\rangle = 0, \quad \langle 0|0\rangle = 1, \quad H|0\rangle = E_{vac}|0\rangle, \quad E_{vac} = 2\delta^{(3)}(0) \int d^3 p E_p,$$
(5.319)
引入单粒子态 $|p^- , \lambda\rangle \equiv \sqrt{2E_p} b_{p, \lambda}^\dagger |0\rangle$, 推出非物理的结果
$$\langle p^- , \lambda | p^- , \lambda\rangle = -2E_p (2\pi)^3 \delta^{(3)}(0) < 0,$$
(5.320)
$$\langle p^- , \lambda | H | p^- , \lambda\rangle = -2E_p (E_{vac} + E_p)(2\pi)^3 \delta^{(3)}(0) < 0.$$
(5.321)

\section{量子场的相互作用}
%%%%%%%%%%%%%%%%%%%%%%%%%%%%%%%%%%%%%%%%%%%%%%%%%%%%
\subsection{}





























%%%%%%%%%%%%%%%%%%%%%%%%%%%%%%%%%%%%%%%%%%%%%%%%%%%%
\subsection{}





散射指的是
在时间上,从无穷远时刻来,到发生散射的时刻,再到无穷远时刻去
在空间上,从无穷远位置来,到发生散射的位置,再到无穷远空间去















%%%%%%%%%%%%%%%%%%%%%%%%%%%%%%%%%%%%%%%%%%%%%%%%%%%%
\subsection{}
























%%%%%%%%%%%%%%%%%%%%%%%%%%%%%%%%%%%%%%%%%%%%%%%%%%%%
\subsection{Feynman 传播子}














实标量场的 Feynman 传播子
\begin{equation}
    D_{\mathrm{F}}(x-y)=\varphi (x)\varphi (y)=\langle 0|\mathrm{T}\left[ \varphi (x)\varphi (y) \right] |0\rangle =\int{\frac{\mathrm{d}^4p}{\left( 2\pi \right) ^4}}\frac{\mathrm{i}}{p^2-m^2+\mathrm{i}\epsilon}\mathrm{e}^{-\mathrm{i}p\cdot \left( x-y \right)}
\end{equation}
复标量场的Feynman传播子
\begin{equation}
   D_{\mathrm{F}}(x-y)=\phi (x)\phi ^{\dagger}(y)=\langle 0|\mathrm{T}\left[ \phi (x)\phi ^{\dagger}(y) \right] |0\rangle =\int{\frac{\mathrm{d}^4p}{\left( 2\pi \right) ^4}}\frac{\mathrm{i}}{p^2-m^2+\mathrm{i}\epsilon}\mathrm{e}^{-\mathrm{i}p\cdot \left( x-y \right)}
\end{equation}
有质量矢量场的Feynman传播子
\begin{equation}
   \Delta _{\mathrm{F}}^{\mu \nu}(x-y)=A^{\mu}(x)A^{\nu}(y)=\langle 0|\mathrm{T}\left[ A^{\mu}(x)A^{\nu}(y) \right] |0\rangle =\int{\frac{\mathrm{d}^4p}{\left( 2\pi \right) ^4}}\frac{-\mathrm{i}\left( g^{\mu \nu}-\frac{p^{\mu}p^{\nu}}{m^2} \right)}{p^2-m^2+\mathrm{i}\epsilon}\mathrm{e}^{-\mathrm{i}p\cdot \left( x-y \right)}
\end{equation}
无质量矢量场的Feynman传播子
\begin{equation}
   \Delta _{\mathrm{F}}^{\mu \nu}(x-y)=A^{\mu}(x)A^{\nu}(y)=\langle 0|\mathrm{T}\left[ A^{\mu}(x)A^{\nu}(y) \right] |0\rangle =\int{\frac{\mathrm{d}^4p}{\left( 2\pi \right) ^4}}\frac{-\mathrm{i}g^{\mu \nu}}{p^2+\mathrm{i}\epsilon}\mathrm{e}^{-\mathrm{i}p\cdot \left( x-y \right)}
\end{equation}
Dirac旋量场的Feynman传播子
\begin{equation}
   \begin{aligned}
       S_{\mathrm{F},ab}(x-y)=\psi _a(x)\psi _b(y)=\langle 0|\mathrm{T}\left[ \psi _a(x)\psi _b(y) \right] |0\rangle &=\int{\frac{\mathrm{d}^4p}{\left( 2\pi \right) ^4}}\frac{\mathrm{i}\left( p+m \right)}{p^2-m^2+\mathrm{i}\epsilon}\mathrm{e}^{-\mathrm{i}p\cdot \left( x-y \right)}
       \\
       &=\int{\frac{\mathrm{d}^4p}{\left( 2\pi \right) ^4}}\frac{\mathrm{i}}{p^2-m^2+\mathrm{i}\epsilon}\mathrm{e}^{-\mathrm{i}p\cdot \left( x-y \right)}
   \end{aligned}
\end{equation}

补充:
坐标表象下的Feynman传播子通过傅里叶变换转换到动量表象下的Feynman传播子


%%%%%%%%%%%%%%%%%%%%%%%%%%%%%%%%%%%%%%%%%%%%%%%%%%%%
\subsection{}





















%%%%%%%%%%%%%%%%%%%%%%%%%%%%%%%%%%%%%%%%%%%%%%%%%%%%
\subsection{习题}


\subsubsection{6.2}
\begin{equation}
    \begin{aligned}
        \mathrm{T[}A^{\mu}\bar{\psi}(x)\gamma _{\mu}\psi (x)A^{\nu}(y)\bar{\psi}(y)\gamma _{\nu}\psi (y)]&=\mathrm{N[}A^{\mu}(x)\bar{\psi}(x)\gamma _{\mu}\psi (x)A^{\nu}(y)\bar{\psi}(y)\gamma _{\nu}\psi (y)]
\\
&+\wick{\c{A}^{\mu}(x)\bar{\psi}(x)\gamma _{\mu}\psi (x)\c{A}^{\nu}(y)\bar{\psi}(y)\gamma _{\nu}\psi (y)}
\\
&+\wick{A^{\mu}(x)\bar{\psi}(x)\gamma _{\mu}\c{\psi} (x)A^{\nu}(y)\c{\bar{\psi}}(y)\gamma _{\nu}\psi (y)}
\\
&+\wick{A^{\mu}(x)\c{\bar{\psi}}(x)\gamma _{\mu}\psi (x)A^{\nu}(y)\bar{\psi}(y)\gamma _{\nu}\c{\psi} (y)}
\\
&+\wick{A^{\mu}(x)\c{\bar{\psi}}(x)\gamma _{\mu}\c{\psi} (x)A^{\nu}(y)\bar{\psi}(y)\gamma _{\nu}\psi (y)}
\\
&+\wick{A^{\mu}(x)\bar{\psi}(x)\gamma _{\mu}\psi (x)A^{\nu}(y)\c{\bar{\psi}}(y)\gamma _{\nu}\c{\psi} (y)}
\\
&+\wick{\c1{A}^{\mu}(x)\bar{\psi}(x)\gamma _{\mu}\c2{\psi} (x)\c1{A}^{\nu}(y)\c2{\bar{\psi}}(y)\gamma _{\nu}\psi (y)}
\\
&+\wick{\c1{A}^{\mu}(x)\c2{\bar{\psi}}(x)\gamma _{\mu}\psi (x)\c1{A}^{\nu}(y)\bar{\psi}(y)\gamma _{\nu}\c2{\psi} (y)}
\\
&+\wick{\c1{A}^{\mu}(x)\c2{\bar{\psi}}(x)\gamma _{\mu}\c2{\psi} (x)\c1{A}^{\nu}(y)\bar{\psi}(y)\gamma _{\nu}\psi (y)}
\\
&+\wick{\c1{A}^{\mu}(x)\bar{\psi}(x)\gamma _{\mu}\psi (x)\c1{A}^{\nu}(y)\c2{\bar{\psi}}(y)\gamma _{\nu}\c2{\psi} (y)}
\\
&+\wick{A^{\mu}(x)\c2{\bar{\psi}}(x)\gamma _{\mu}\c1{\psi} (x)A^{\nu}(y)\c1{\bar{\psi}}(y)\gamma _{\nu}\c2{\psi} (y)}
\\
&+\wick{A^{\mu}(x)\c1{\bar{\psi}}(x)\gamma _{\mu}\c1{\psi} (x)A^{\nu}(y)\c2{\bar{\psi}}(y)\gamma _{\nu}\c2{\psi} (y)}
\\
&+\wick{\c3{A}^{\mu}(x)\c2{\bar{\psi}}(x)\gamma _{\mu}\c1{\psi} (x)\c3{A}^{\nu}(y)\c1{\bar{\psi}}(y)\gamma _{\nu}\c2{\psi} (y)}
\\
&+\wick{\c2{A}^{\mu}(x)\c1{\bar{\psi}}(x)\gamma _{\mu}\c1{\psi} (x)\c2{A}^{\nu}(y)\c3{\bar{\psi}}(y)\gamma _{\nu}\c3{\psi} (y)}
    \end{aligned}
\end{equation}






\section{量子场的相互作用}
%%%%%%%%%%%%%%%%%%%%%%%%%%%%%%%%%%%%%%%%%%%%%%%%%%%%
\subsection{}





























%%%%%%%%%%%%%%%%%%%%%%%%%%%%%%%%%%%%%%%%%%%%%%%%%%%%
\subsection{}





散射指的是
在时间上,从无穷远时刻来,到发生散射的时刻,再到无穷远时刻去
在空间上,从无穷远位置来,到发生散射的位置,再到无穷远空间去















%%%%%%%%%%%%%%%%%%%%%%%%%%%%%%%%%%%%%%%%%%%%%%%%%%%%
\subsection{}
























%%%%%%%%%%%%%%%%%%%%%%%%%%%%%%%%%%%%%%%%%%%%%%%%%%%%
\subsection{Feynman 传播子}














实标量场的 Feynman 传播子
\begin{equation}
    D_{\mathrm{F}}(x-y)=\varphi (x)\varphi (y)=\langle 0|\mathrm{T}\left[ \varphi (x)\varphi (y) \right] |0\rangle =\int{\frac{\mathrm{d}^4p}{\left( 2\pi \right) ^4}}\frac{\mathrm{i}}{p^2-m^2+\mathrm{i}\epsilon}\mathrm{e}^{-\mathrm{i}p\cdot \left( x-y \right)}
\end{equation}
复标量场的Feynman传播子
\begin{equation}
   D_{\mathrm{F}}(x-y)=\phi (x)\phi ^{\dagger}(y)=\langle 0|\mathrm{T}\left[ \phi (x)\phi ^{\dagger}(y) \right] |0\rangle =\int{\frac{\mathrm{d}^4p}{\left( 2\pi \right) ^4}}\frac{\mathrm{i}}{p^2-m^2+\mathrm{i}\epsilon}\mathrm{e}^{-\mathrm{i}p\cdot \left( x-y \right)}
\end{equation}
有质量矢量场的Feynman传播子
\begin{equation}
   \Delta _{\mathrm{F}}^{\mu \nu}(x-y)=A^{\mu}(x)A^{\nu}(y)=\langle 0|\mathrm{T}\left[ A^{\mu}(x)A^{\nu}(y) \right] |0\rangle =\int{\frac{\mathrm{d}^4p}{\left( 2\pi \right) ^4}}\frac{-\mathrm{i}\left( g^{\mu \nu}-\frac{p^{\mu}p^{\nu}}{m^2} \right)}{p^2-m^2+\mathrm{i}\epsilon}\mathrm{e}^{-\mathrm{i}p\cdot \left( x-y \right)}
\end{equation}
无质量矢量场的Feynman传播子
\begin{equation}
   \Delta _{\mathrm{F}}^{\mu \nu}(x-y)=A^{\mu}(x)A^{\nu}(y)=\langle 0|\mathrm{T}\left[ A^{\mu}(x)A^{\nu}(y) \right] |0\rangle =\int{\frac{\mathrm{d}^4p}{\left( 2\pi \right) ^4}}\frac{-\mathrm{i}g^{\mu \nu}}{p^2+\mathrm{i}\epsilon}\mathrm{e}^{-\mathrm{i}p\cdot \left( x-y \right)}
\end{equation}
Dirac旋量场的Feynman传播子
\begin{equation}
   \begin{aligned}
       S_{\mathrm{F},ab}(x-y)=\psi _a(x)\psi _b(y)=\langle 0|\mathrm{T}\left[ \psi _a(x)\psi _b(y) \right] |0\rangle &=\int{\frac{\mathrm{d}^4p}{\left( 2\pi \right) ^4}}\frac{\mathrm{i}\left( p+m \right)}{p^2-m^2+\mathrm{i}\epsilon}\mathrm{e}^{-\mathrm{i}p\cdot \left( x-y \right)}
       \\
       &=\int{\frac{\mathrm{d}^4p}{\left( 2\pi \right) ^4}}\frac{\mathrm{i}}{p^2-m^2+\mathrm{i}\epsilon}\mathrm{e}^{-\mathrm{i}p\cdot \left( x-y \right)}
   \end{aligned}
\end{equation}

补充:
坐标表象下的Feynman传播子通过傅里叶变换转换到动量表象下的Feynman传播子


%%%%%%%%%%%%%%%%%%%%%%%%%%%%%%%%%%%%%%%%%%%%%%%%%%%%
\subsection{}





















%%%%%%%%%%%%%%%%%%%%%%%%%%%%%%%%%%%%%%%%%%%%%%%%%%%%
\subsection{习题}


\subsubsection{6.2}
\begin{equation}
    \begin{aligned}
        \mathrm{T[}A^{\mu}\bar{\psi}(x)\gamma _{\mu}\psi (x)A^{\nu}(y)\bar{\psi}(y)\gamma _{\nu}\psi (y)]&=\mathrm{N[}A^{\mu}(x)\bar{\psi}(x)\gamma _{\mu}\psi (x)A^{\nu}(y)\bar{\psi}(y)\gamma _{\nu}\psi (y)]
\\
&+\wick{\c{A}^{\mu}(x)\bar{\psi}(x)\gamma _{\mu}\psi (x)\c{A}^{\nu}(y)\bar{\psi}(y)\gamma _{\nu}\psi (y)}
\\
&+\wick{A^{\mu}(x)\bar{\psi}(x)\gamma _{\mu}\c{\psi} (x)A^{\nu}(y)\c{\bar{\psi}}(y)\gamma _{\nu}\psi (y)}
\\
&+\wick{A^{\mu}(x)\c{\bar{\psi}}(x)\gamma _{\mu}\psi (x)A^{\nu}(y)\bar{\psi}(y)\gamma _{\nu}\c{\psi} (y)}
\\
&+\wick{A^{\mu}(x)\c{\bar{\psi}}(x)\gamma _{\mu}\c{\psi} (x)A^{\nu}(y)\bar{\psi}(y)\gamma _{\nu}\psi (y)}
\\
&+\wick{A^{\mu}(x)\bar{\psi}(x)\gamma _{\mu}\psi (x)A^{\nu}(y)\c{\bar{\psi}}(y)\gamma _{\nu}\c{\psi} (y)}
\\
&+\wick{\c1{A}^{\mu}(x)\bar{\psi}(x)\gamma _{\mu}\c2{\psi} (x)\c1{A}^{\nu}(y)\c2{\bar{\psi}}(y)\gamma _{\nu}\psi (y)}
\\
&+\wick{\c1{A}^{\mu}(x)\c2{\bar{\psi}}(x)\gamma _{\mu}\psi (x)\c1{A}^{\nu}(y)\bar{\psi}(y)\gamma _{\nu}\c2{\psi} (y)}
\\
&+\wick{\c1{A}^{\mu}(x)\c2{\bar{\psi}}(x)\gamma _{\mu}\c2{\psi} (x)\c1{A}^{\nu}(y)\bar{\psi}(y)\gamma _{\nu}\psi (y)}
\\
&+\wick{\c1{A}^{\mu}(x)\bar{\psi}(x)\gamma _{\mu}\psi (x)\c1{A}^{\nu}(y)\c2{\bar{\psi}}(y)\gamma _{\nu}\c2{\psi} (y)}
\\
&+\wick{A^{\mu}(x)\c2{\bar{\psi}}(x)\gamma _{\mu}\c1{\psi} (x)A^{\nu}(y)\c1{\bar{\psi}}(y)\gamma _{\nu}\c2{\psi} (y)}
\\
&+\wick{A^{\mu}(x)\c1{\bar{\psi}}(x)\gamma _{\mu}\c1{\psi} (x)A^{\nu}(y)\c2{\bar{\psi}}(y)\gamma _{\nu}\c2{\psi} (y)}
\\
&+\wick{\c3{A}^{\mu}(x)\c2{\bar{\psi}}(x)\gamma _{\mu}\c1{\psi} (x)\c3{A}^{\nu}(y)\c1{\bar{\psi}}(y)\gamma _{\nu}\c2{\psi} (y)}
\\
&+\wick{\c2{A}^{\mu}(x)\c1{\bar{\psi}}(x)\gamma _{\mu}\c1{\psi} (x)\c2{A}^{\nu}(y)\c3{\bar{\psi}}(y)\gamma _{\nu}\c3{\psi} (y)}
    \end{aligned}
\end{equation}






\section{量子场的相互作用}
%%%%%%%%%%%%%%%%%%%%%%%%%%%%%%%%%%%%%%%%%%%%%%%%%%%%
\subsection{}





























%%%%%%%%%%%%%%%%%%%%%%%%%%%%%%%%%%%%%%%%%%%%%%%%%%%%
\subsection{}





散射指的是
在时间上,从无穷远时刻来,到发生散射的时刻,再到无穷远时刻去
在空间上,从无穷远位置来,到发生散射的位置,再到无穷远空间去















%%%%%%%%%%%%%%%%%%%%%%%%%%%%%%%%%%%%%%%%%%%%%%%%%%%%
\subsection{}
























%%%%%%%%%%%%%%%%%%%%%%%%%%%%%%%%%%%%%%%%%%%%%%%%%%%%
\subsection{Feynman 传播子}














实标量场的 Feynman 传播子
\begin{equation}
    D_{\mathrm{F}}(x-y)=\varphi (x)\varphi (y)=\langle 0|\mathrm{T}\left[ \varphi (x)\varphi (y) \right] |0\rangle =\int{\frac{\mathrm{d}^4p}{\left( 2\pi \right) ^4}}\frac{\mathrm{i}}{p^2-m^2+\mathrm{i}\epsilon}\mathrm{e}^{-\mathrm{i}p\cdot \left( x-y \right)}
\end{equation}
复标量场的Feynman传播子
\begin{equation}
   D_{\mathrm{F}}(x-y)=\phi (x)\phi ^{\dagger}(y)=\langle 0|\mathrm{T}\left[ \phi (x)\phi ^{\dagger}(y) \right] |0\rangle =\int{\frac{\mathrm{d}^4p}{\left( 2\pi \right) ^4}}\frac{\mathrm{i}}{p^2-m^2+\mathrm{i}\epsilon}\mathrm{e}^{-\mathrm{i}p\cdot \left( x-y \right)}
\end{equation}
有质量矢量场的Feynman传播子
\begin{equation}
   \Delta _{\mathrm{F}}^{\mu \nu}(x-y)=A^{\mu}(x)A^{\nu}(y)=\langle 0|\mathrm{T}\left[ A^{\mu}(x)A^{\nu}(y) \right] |0\rangle =\int{\frac{\mathrm{d}^4p}{\left( 2\pi \right) ^4}}\frac{-\mathrm{i}\left( g^{\mu \nu}-\frac{p^{\mu}p^{\nu}}{m^2} \right)}{p^2-m^2+\mathrm{i}\epsilon}\mathrm{e}^{-\mathrm{i}p\cdot \left( x-y \right)}
\end{equation}
无质量矢量场的Feynman传播子
\begin{equation}
   \Delta _{\mathrm{F}}^{\mu \nu}(x-y)=A^{\mu}(x)A^{\nu}(y)=\langle 0|\mathrm{T}\left[ A^{\mu}(x)A^{\nu}(y) \right] |0\rangle =\int{\frac{\mathrm{d}^4p}{\left( 2\pi \right) ^4}}\frac{-\mathrm{i}g^{\mu \nu}}{p^2+\mathrm{i}\epsilon}\mathrm{e}^{-\mathrm{i}p\cdot \left( x-y \right)}
\end{equation}
Dirac旋量场的Feynman传播子
\begin{equation}
   \begin{aligned}
       S_{\mathrm{F},ab}(x-y)=\psi _a(x)\psi _b(y)=\langle 0|\mathrm{T}\left[ \psi _a(x)\psi _b(y) \right] |0\rangle &=\int{\frac{\mathrm{d}^4p}{\left( 2\pi \right) ^4}}\frac{\mathrm{i}\left( p+m \right)}{p^2-m^2+\mathrm{i}\epsilon}\mathrm{e}^{-\mathrm{i}p\cdot \left( x-y \right)}
       \\
       &=\int{\frac{\mathrm{d}^4p}{\left( 2\pi \right) ^4}}\frac{\mathrm{i}}{p^2-m^2+\mathrm{i}\epsilon}\mathrm{e}^{-\mathrm{i}p\cdot \left( x-y \right)}
   \end{aligned}
\end{equation}

补充:
坐标表象下的Feynman传播子通过傅里叶变换转换到动量表象下的Feynman传播子


%%%%%%%%%%%%%%%%%%%%%%%%%%%%%%%%%%%%%%%%%%%%%%%%%%%%
\subsection{}





















%%%%%%%%%%%%%%%%%%%%%%%%%%%%%%%%%%%%%%%%%%%%%%%%%%%%
\subsection{习题}


\subsubsection{6.2}
\begin{equation}
    \begin{aligned}
        \mathrm{T[}A^{\mu}\bar{\psi}(x)\gamma _{\mu}\psi (x)A^{\nu}(y)\bar{\psi}(y)\gamma _{\nu}\psi (y)]&=\mathrm{N[}A^{\mu}(x)\bar{\psi}(x)\gamma _{\mu}\psi (x)A^{\nu}(y)\bar{\psi}(y)\gamma _{\nu}\psi (y)]
\\
&+\wick{\c{A}^{\mu}(x)\bar{\psi}(x)\gamma _{\mu}\psi (x)\c{A}^{\nu}(y)\bar{\psi}(y)\gamma _{\nu}\psi (y)}
\\
&+\wick{A^{\mu}(x)\bar{\psi}(x)\gamma _{\mu}\c{\psi} (x)A^{\nu}(y)\c{\bar{\psi}}(y)\gamma _{\nu}\psi (y)}
\\
&+\wick{A^{\mu}(x)\c{\bar{\psi}}(x)\gamma _{\mu}\psi (x)A^{\nu}(y)\bar{\psi}(y)\gamma _{\nu}\c{\psi} (y)}
\\
&+\wick{A^{\mu}(x)\c{\bar{\psi}}(x)\gamma _{\mu}\c{\psi} (x)A^{\nu}(y)\bar{\psi}(y)\gamma _{\nu}\psi (y)}
\\
&+\wick{A^{\mu}(x)\bar{\psi}(x)\gamma _{\mu}\psi (x)A^{\nu}(y)\c{\bar{\psi}}(y)\gamma _{\nu}\c{\psi} (y)}
\\
&+\wick{\c1{A}^{\mu}(x)\bar{\psi}(x)\gamma _{\mu}\c2{\psi} (x)\c1{A}^{\nu}(y)\c2{\bar{\psi}}(y)\gamma _{\nu}\psi (y)}
\\
&+\wick{\c1{A}^{\mu}(x)\c2{\bar{\psi}}(x)\gamma _{\mu}\psi (x)\c1{A}^{\nu}(y)\bar{\psi}(y)\gamma _{\nu}\c2{\psi} (y)}
\\
&+\wick{\c1{A}^{\mu}(x)\c2{\bar{\psi}}(x)\gamma _{\mu}\c2{\psi} (x)\c1{A}^{\nu}(y)\bar{\psi}(y)\gamma _{\nu}\psi (y)}
\\
&+\wick{\c1{A}^{\mu}(x)\bar{\psi}(x)\gamma _{\mu}\psi (x)\c1{A}^{\nu}(y)\c2{\bar{\psi}}(y)\gamma _{\nu}\c2{\psi} (y)}
\\
&+\wick{A^{\mu}(x)\c2{\bar{\psi}}(x)\gamma _{\mu}\c1{\psi} (x)A^{\nu}(y)\c1{\bar{\psi}}(y)\gamma _{\nu}\c2{\psi} (y)}
\\
&+\wick{A^{\mu}(x)\c1{\bar{\psi}}(x)\gamma _{\mu}\c1{\psi} (x)A^{\nu}(y)\c2{\bar{\psi}}(y)\gamma _{\nu}\c2{\psi} (y)}
\\
&+\wick{\c3{A}^{\mu}(x)\c2{\bar{\psi}}(x)\gamma _{\mu}\c1{\psi} (x)\c3{A}^{\nu}(y)\c1{\bar{\psi}}(y)\gamma _{\nu}\c2{\psi} (y)}
\\
&+\wick{\c2{A}^{\mu}(x)\c1{\bar{\psi}}(x)\gamma _{\mu}\c1{\psi} (x)\c2{A}^{\nu}(y)\c3{\bar{\psi}}(y)\gamma _{\nu}\c3{\psi} (y)}
    \end{aligned}
\end{equation}






\section{量子场的相互作用}
%%%%%%%%%%%%%%%%%%%%%%%%%%%%%%%%%%%%%%%%%%%%%%%%%%%%
\subsection{}





























%%%%%%%%%%%%%%%%%%%%%%%%%%%%%%%%%%%%%%%%%%%%%%%%%%%%
\subsection{}





散射指的是
在时间上,从无穷远时刻来,到发生散射的时刻,再到无穷远时刻去
在空间上,从无穷远位置来,到发生散射的位置,再到无穷远空间去















%%%%%%%%%%%%%%%%%%%%%%%%%%%%%%%%%%%%%%%%%%%%%%%%%%%%
\subsection{}
























%%%%%%%%%%%%%%%%%%%%%%%%%%%%%%%%%%%%%%%%%%%%%%%%%%%%
\subsection{Feynman 传播子}














实标量场的 Feynman 传播子
\begin{equation}
    D_{\mathrm{F}}(x-y)=\varphi (x)\varphi (y)=\langle 0|\mathrm{T}\left[ \varphi (x)\varphi (y) \right] |0\rangle =\int{\frac{\mathrm{d}^4p}{\left( 2\pi \right) ^4}}\frac{\mathrm{i}}{p^2-m^2+\mathrm{i}\epsilon}\mathrm{e}^{-\mathrm{i}p\cdot \left( x-y \right)}
\end{equation}
复标量场的Feynman传播子
\begin{equation}
   D_{\mathrm{F}}(x-y)=\phi (x)\phi ^{\dagger}(y)=\langle 0|\mathrm{T}\left[ \phi (x)\phi ^{\dagger}(y) \right] |0\rangle =\int{\frac{\mathrm{d}^4p}{\left( 2\pi \right) ^4}}\frac{\mathrm{i}}{p^2-m^2+\mathrm{i}\epsilon}\mathrm{e}^{-\mathrm{i}p\cdot \left( x-y \right)}
\end{equation}
有质量矢量场的Feynman传播子
\begin{equation}
   \Delta _{\mathrm{F}}^{\mu \nu}(x-y)=A^{\mu}(x)A^{\nu}(y)=\langle 0|\mathrm{T}\left[ A^{\mu}(x)A^{\nu}(y) \right] |0\rangle =\int{\frac{\mathrm{d}^4p}{\left( 2\pi \right) ^4}}\frac{-\mathrm{i}\left( g^{\mu \nu}-\frac{p^{\mu}p^{\nu}}{m^2} \right)}{p^2-m^2+\mathrm{i}\epsilon}\mathrm{e}^{-\mathrm{i}p\cdot \left( x-y \right)}
\end{equation}
无质量矢量场的Feynman传播子
\begin{equation}
   \Delta _{\mathrm{F}}^{\mu \nu}(x-y)=A^{\mu}(x)A^{\nu}(y)=\langle 0|\mathrm{T}\left[ A^{\mu}(x)A^{\nu}(y) \right] |0\rangle =\int{\frac{\mathrm{d}^4p}{\left( 2\pi \right) ^4}}\frac{-\mathrm{i}g^{\mu \nu}}{p^2+\mathrm{i}\epsilon}\mathrm{e}^{-\mathrm{i}p\cdot \left( x-y \right)}
\end{equation}
Dirac旋量场的Feynman传播子
\begin{equation}
   \begin{aligned}
       S_{\mathrm{F},ab}(x-y)=\psi _a(x)\psi _b(y)=\langle 0|\mathrm{T}\left[ \psi _a(x)\psi _b(y) \right] |0\rangle &=\int{\frac{\mathrm{d}^4p}{\left( 2\pi \right) ^4}}\frac{\mathrm{i}\left( p+m \right)}{p^2-m^2+\mathrm{i}\epsilon}\mathrm{e}^{-\mathrm{i}p\cdot \left( x-y \right)}
       \\
       &=\int{\frac{\mathrm{d}^4p}{\left( 2\pi \right) ^4}}\frac{\mathrm{i}}{p^2-m^2+\mathrm{i}\epsilon}\mathrm{e}^{-\mathrm{i}p\cdot \left( x-y \right)}
   \end{aligned}
\end{equation}

补充:
坐标表象下的Feynman传播子通过傅里叶变换转换到动量表象下的Feynman传播子


%%%%%%%%%%%%%%%%%%%%%%%%%%%%%%%%%%%%%%%%%%%%%%%%%%%%
\subsection{}





















%%%%%%%%%%%%%%%%%%%%%%%%%%%%%%%%%%%%%%%%%%%%%%%%%%%%
\subsection{习题}


\subsubsection{6.2}
\begin{equation}
    \begin{aligned}
        \mathrm{T[}A^{\mu}\bar{\psi}(x)\gamma _{\mu}\psi (x)A^{\nu}(y)\bar{\psi}(y)\gamma _{\nu}\psi (y)]&=\mathrm{N[}A^{\mu}(x)\bar{\psi}(x)\gamma _{\mu}\psi (x)A^{\nu}(y)\bar{\psi}(y)\gamma _{\nu}\psi (y)]
\\
&+\wick{\c{A}^{\mu}(x)\bar{\psi}(x)\gamma _{\mu}\psi (x)\c{A}^{\nu}(y)\bar{\psi}(y)\gamma _{\nu}\psi (y)}
\\
&+\wick{A^{\mu}(x)\bar{\psi}(x)\gamma _{\mu}\c{\psi} (x)A^{\nu}(y)\c{\bar{\psi}}(y)\gamma _{\nu}\psi (y)}
\\
&+\wick{A^{\mu}(x)\c{\bar{\psi}}(x)\gamma _{\mu}\psi (x)A^{\nu}(y)\bar{\psi}(y)\gamma _{\nu}\c{\psi} (y)}
\\
&+\wick{A^{\mu}(x)\c{\bar{\psi}}(x)\gamma _{\mu}\c{\psi} (x)A^{\nu}(y)\bar{\psi}(y)\gamma _{\nu}\psi (y)}
\\
&+\wick{A^{\mu}(x)\bar{\psi}(x)\gamma _{\mu}\psi (x)A^{\nu}(y)\c{\bar{\psi}}(y)\gamma _{\nu}\c{\psi} (y)}
\\
&+\wick{\c1{A}^{\mu}(x)\bar{\psi}(x)\gamma _{\mu}\c2{\psi} (x)\c1{A}^{\nu}(y)\c2{\bar{\psi}}(y)\gamma _{\nu}\psi (y)}
\\
&+\wick{\c1{A}^{\mu}(x)\c2{\bar{\psi}}(x)\gamma _{\mu}\psi (x)\c1{A}^{\nu}(y)\bar{\psi}(y)\gamma _{\nu}\c2{\psi} (y)}
\\
&+\wick{\c1{A}^{\mu}(x)\c2{\bar{\psi}}(x)\gamma _{\mu}\c2{\psi} (x)\c1{A}^{\nu}(y)\bar{\psi}(y)\gamma _{\nu}\psi (y)}
\\
&+\wick{\c1{A}^{\mu}(x)\bar{\psi}(x)\gamma _{\mu}\psi (x)\c1{A}^{\nu}(y)\c2{\bar{\psi}}(y)\gamma _{\nu}\c2{\psi} (y)}
\\
&+\wick{A^{\mu}(x)\c2{\bar{\psi}}(x)\gamma _{\mu}\c1{\psi} (x)A^{\nu}(y)\c1{\bar{\psi}}(y)\gamma _{\nu}\c2{\psi} (y)}
\\
&+\wick{A^{\mu}(x)\c1{\bar{\psi}}(x)\gamma _{\mu}\c1{\psi} (x)A^{\nu}(y)\c2{\bar{\psi}}(y)\gamma _{\nu}\c2{\psi} (y)}
\\
&+\wick{\c3{A}^{\mu}(x)\c2{\bar{\psi}}(x)\gamma _{\mu}\c1{\psi} (x)\c3{A}^{\nu}(y)\c1{\bar{\psi}}(y)\gamma _{\nu}\c2{\psi} (y)}
\\
&+\wick{\c2{A}^{\mu}(x)\c1{\bar{\psi}}(x)\gamma _{\mu}\c1{\psi} (x)\c2{A}^{\nu}(y)\c3{\bar{\psi}}(y)\gamma _{\nu}\c3{\psi} (y)}
    \end{aligned}
\end{equation}






\section{量子场的相互作用}
%%%%%%%%%%%%%%%%%%%%%%%%%%%%%%%%%%%%%%%%%%%%%%%%%%%%
\subsection{}





























%%%%%%%%%%%%%%%%%%%%%%%%%%%%%%%%%%%%%%%%%%%%%%%%%%%%
\subsection{}





散射指的是
在时间上,从无穷远时刻来,到发生散射的时刻,再到无穷远时刻去
在空间上,从无穷远位置来,到发生散射的位置,再到无穷远空间去















%%%%%%%%%%%%%%%%%%%%%%%%%%%%%%%%%%%%%%%%%%%%%%%%%%%%
\subsection{}
























%%%%%%%%%%%%%%%%%%%%%%%%%%%%%%%%%%%%%%%%%%%%%%%%%%%%
\subsection{Feynman 传播子}














实标量场的 Feynman 传播子
\begin{equation}
    D_{\mathrm{F}}(x-y)=\varphi (x)\varphi (y)=\langle 0|\mathrm{T}\left[ \varphi (x)\varphi (y) \right] |0\rangle =\int{\frac{\mathrm{d}^4p}{\left( 2\pi \right) ^4}}\frac{\mathrm{i}}{p^2-m^2+\mathrm{i}\epsilon}\mathrm{e}^{-\mathrm{i}p\cdot \left( x-y \right)}
\end{equation}
复标量场的Feynman传播子
\begin{equation}
   D_{\mathrm{F}}(x-y)=\phi (x)\phi ^{\dagger}(y)=\langle 0|\mathrm{T}\left[ \phi (x)\phi ^{\dagger}(y) \right] |0\rangle =\int{\frac{\mathrm{d}^4p}{\left( 2\pi \right) ^4}}\frac{\mathrm{i}}{p^2-m^2+\mathrm{i}\epsilon}\mathrm{e}^{-\mathrm{i}p\cdot \left( x-y \right)}
\end{equation}
有质量矢量场的Feynman传播子
\begin{equation}
   \Delta _{\mathrm{F}}^{\mu \nu}(x-y)=A^{\mu}(x)A^{\nu}(y)=\langle 0|\mathrm{T}\left[ A^{\mu}(x)A^{\nu}(y) \right] |0\rangle =\int{\frac{\mathrm{d}^4p}{\left( 2\pi \right) ^4}}\frac{-\mathrm{i}\left( g^{\mu \nu}-\frac{p^{\mu}p^{\nu}}{m^2} \right)}{p^2-m^2+\mathrm{i}\epsilon}\mathrm{e}^{-\mathrm{i}p\cdot \left( x-y \right)}
\end{equation}
无质量矢量场的Feynman传播子
\begin{equation}
   \Delta _{\mathrm{F}}^{\mu \nu}(x-y)=A^{\mu}(x)A^{\nu}(y)=\langle 0|\mathrm{T}\left[ A^{\mu}(x)A^{\nu}(y) \right] |0\rangle =\int{\frac{\mathrm{d}^4p}{\left( 2\pi \right) ^4}}\frac{-\mathrm{i}g^{\mu \nu}}{p^2+\mathrm{i}\epsilon}\mathrm{e}^{-\mathrm{i}p\cdot \left( x-y \right)}
\end{equation}
Dirac旋量场的Feynman传播子
\begin{equation}
   \begin{aligned}
       S_{\mathrm{F},ab}(x-y)=\psi _a(x)\psi _b(y)=\langle 0|\mathrm{T}\left[ \psi _a(x)\psi _b(y) \right] |0\rangle &=\int{\frac{\mathrm{d}^4p}{\left( 2\pi \right) ^4}}\frac{\mathrm{i}\left( p+m \right)}{p^2-m^2+\mathrm{i}\epsilon}\mathrm{e}^{-\mathrm{i}p\cdot \left( x-y \right)}
       \\
       &=\int{\frac{\mathrm{d}^4p}{\left( 2\pi \right) ^4}}\frac{\mathrm{i}}{p^2-m^2+\mathrm{i}\epsilon}\mathrm{e}^{-\mathrm{i}p\cdot \left( x-y \right)}
   \end{aligned}
\end{equation}

补充:
坐标表象下的Feynman传播子通过傅里叶变换转换到动量表象下的Feynman传播子


%%%%%%%%%%%%%%%%%%%%%%%%%%%%%%%%%%%%%%%%%%%%%%%%%%%%
\subsection{}





















%%%%%%%%%%%%%%%%%%%%%%%%%%%%%%%%%%%%%%%%%%%%%%%%%%%%
\subsection{习题}


\subsubsection{6.2}
\begin{equation}
    \begin{aligned}
        \mathrm{T[}A^{\mu}\bar{\psi}(x)\gamma _{\mu}\psi (x)A^{\nu}(y)\bar{\psi}(y)\gamma _{\nu}\psi (y)]&=\mathrm{N[}A^{\mu}(x)\bar{\psi}(x)\gamma _{\mu}\psi (x)A^{\nu}(y)\bar{\psi}(y)\gamma _{\nu}\psi (y)]
\\
&+\wick{\c{A}^{\mu}(x)\bar{\psi}(x)\gamma _{\mu}\psi (x)\c{A}^{\nu}(y)\bar{\psi}(y)\gamma _{\nu}\psi (y)}
\\
&+\wick{A^{\mu}(x)\bar{\psi}(x)\gamma _{\mu}\c{\psi} (x)A^{\nu}(y)\c{\bar{\psi}}(y)\gamma _{\nu}\psi (y)}
\\
&+\wick{A^{\mu}(x)\c{\bar{\psi}}(x)\gamma _{\mu}\psi (x)A^{\nu}(y)\bar{\psi}(y)\gamma _{\nu}\c{\psi} (y)}
\\
&+\wick{A^{\mu}(x)\c{\bar{\psi}}(x)\gamma _{\mu}\c{\psi} (x)A^{\nu}(y)\bar{\psi}(y)\gamma _{\nu}\psi (y)}
\\
&+\wick{A^{\mu}(x)\bar{\psi}(x)\gamma _{\mu}\psi (x)A^{\nu}(y)\c{\bar{\psi}}(y)\gamma _{\nu}\c{\psi} (y)}
\\
&+\wick{\c1{A}^{\mu}(x)\bar{\psi}(x)\gamma _{\mu}\c2{\psi} (x)\c1{A}^{\nu}(y)\c2{\bar{\psi}}(y)\gamma _{\nu}\psi (y)}
\\
&+\wick{\c1{A}^{\mu}(x)\c2{\bar{\psi}}(x)\gamma _{\mu}\psi (x)\c1{A}^{\nu}(y)\bar{\psi}(y)\gamma _{\nu}\c2{\psi} (y)}
\\
&+\wick{\c1{A}^{\mu}(x)\c2{\bar{\psi}}(x)\gamma _{\mu}\c2{\psi} (x)\c1{A}^{\nu}(y)\bar{\psi}(y)\gamma _{\nu}\psi (y)}
\\
&+\wick{\c1{A}^{\mu}(x)\bar{\psi}(x)\gamma _{\mu}\psi (x)\c1{A}^{\nu}(y)\c2{\bar{\psi}}(y)\gamma _{\nu}\c2{\psi} (y)}
\\
&+\wick{A^{\mu}(x)\c2{\bar{\psi}}(x)\gamma _{\mu}\c1{\psi} (x)A^{\nu}(y)\c1{\bar{\psi}}(y)\gamma _{\nu}\c2{\psi} (y)}
\\
&+\wick{A^{\mu}(x)\c1{\bar{\psi}}(x)\gamma _{\mu}\c1{\psi} (x)A^{\nu}(y)\c2{\bar{\psi}}(y)\gamma _{\nu}\c2{\psi} (y)}
\\
&+\wick{\c3{A}^{\mu}(x)\c2{\bar{\psi}}(x)\gamma _{\mu}\c1{\psi} (x)\c3{A}^{\nu}(y)\c1{\bar{\psi}}(y)\gamma _{\nu}\c2{\psi} (y)}
\\
&+\wick{\c2{A}^{\mu}(x)\c1{\bar{\psi}}(x)\gamma _{\mu}\c1{\psi} (x)\c2{A}^{\nu}(y)\c3{\bar{\psi}}(y)\gamma _{\nu}\c3{\psi} (y)}
    \end{aligned}
\end{equation}






\section{习题6}

\newpage
\subsection{6.1}
考虑 $d$ 维时空中的作用量
$$ S = \int d^d x  \mathcal{L}(x), \tag{6.412} $$
其中拉氏量
$$ \mathcal{L} = \frac{1}{2}(\partial^\mu \phi) \partial_\mu \phi - \frac{1}{2}(\partial_\mu A_\nu) \partial^\mu A^\nu + \frac{1}{2}(\partial_\nu A_\mu) \partial^\mu A^\nu + i \bar{\psi} \gamma^\mu \partial_\mu \psi \tag{6.413} $$
由实标量场 $\phi(x)$、实矢量场 $A^\mu(x)$ 和 Dirac 旋量场 $\psi(x)$ 构成的。在自然单位制下,依然有 $[S] = [E]^0$ 和 $[x^\mu] = [E]^{-1}$,据此分析 $\mathcal{L}$、$\phi$、$A^\mu$ 和 $\psi$ 的量纲。

\newpage
\subsection{6.2}
根据有质量实矢量场的平面波展开式 (4.110),证明
$$ [A^\mu(x), A^\nu(y)] = 0, \quad (x - y)^2 < 0. \tag{6.414} $$

\newpage
\subsection{6.3}
证明级数 (6.106) 中 $n = 3$ 的项
$$ I = \int_{t_0}^t dt_1 \int_{t_0}^{t_1} dt_2 \int_{t_0}^{t_2} dt_3 H_1(t_1)H_1(t_2)H_1(t_3) \tag{6.415} $$
满足
$$ 3!I = \int_{t_0}^t dt_1 \int_{t_0}^t dt_2 \int_{t_0}^t dt_3 T[H_1(t_1)H_1(t_2)H_1(t_3)]. \tag{6.416} $$

\newpage
\subsection{6.4}
对于 Dirac 旋量场 $\psi(x)$ 和实矢量场 $A^\mu(x)$,根据 Wick 定理写出
$$ T[A^\mu(x)\bar{\psi}(x)\gamma_\mu \psi(x)A^\nu(y)\bar{\psi}(y)\gamma_\nu \psi(y)] \tag{6.417} $$
的正规乘积表达式,只需包含非零缩并。

\newpage
\subsection{6.5}
对于拉氏量 (4.290) 描述的有质量复矢量场 $A^\mu(x)$,推出 Feynman 传播子
$$ \bar{A^\mu}(x)A^{\nu\dagger}(y) = \langle 0|T[A^\mu(x)A^{\nu\dagger}(y)|0\rangle \tag{6.418} $$
的表达式
$$ \bar{A^\mu}(x)A^{\nu\dagger}(y) = \int \frac{d^4p}{(2\pi)^4} \frac{-i(g^{\mu\nu}-p^\mu p^\nu/m^2)}{p^2-m^2+i\epsilon} e^{-ip\cdot(x-y)} - \frac{i}{m^2} g^{\mu 0}g^{\nu 0}\delta^{(4)}(x-y). \tag{6.419} $$

\newpage
\subsection{6.6}
在质心系中考虑 $2 \rightarrow n$ 散射过程,将入射流因子表达为
$$ E_A E_B |\mathbf{v}_A - \mathbf{v}_B| = \frac{E_{GM}^2}{2} \lambda^{1/2} \left( 1, \frac{m_A^2}{E_{GM}^2}, \frac{m_B^2}{E_{GM}^2} \right). \tag{6.420} $$

\newpage
\subsection{6.7}
一个粒子的质量为 $m$ ,四维动量为 $p^\mu = (E, p_x, p_y, p_z)$ 。将 $z$ 轴方向视作纵向,则快度
$$ \xi = \tanh^{-1} \frac{p_z}{E} \tag{6.421} $$
对应于沿纵向的 Lorentz 增速变换。定义赝快度 (pseudorapidity)
$$ \eta \equiv -\ln \tan \frac{\theta}{2}, \tag{6.422} $$
其中 $\theta$ 是动量 $\mathbf{p}$ 与 $z$ 轴之间的夹角,如图 6.8 所示。横向动量表达为 $\mathbf{p}_T = (p_x, p_y, 0)$ ,定义横向能量
$$ E_T \equiv \sqrt{m^2 + |\mathbf{p}_T|^2} = \sqrt{m^2 + p_x^2 + p_y^2}. \tag{6.423} $$
(a) 证明 $\eta = \xi$ 对 $m = 0$ 成立。
(b) 证明
$$ E = E_T \cosh \xi, $$
$$ p_z = E_T \sinh \xi. \tag{6.424} $$
(c) 证明
$$ \xi = \ln \frac{E + p_z}{E_T}, \tag{6.425} $$
且
$$ \xi = \frac{1}{2} \ln \frac{E + p_z}{E - p_z}. \tag{6.426} $$
(d) 假设这个粒子衰变为粒子 1 和粒子 2 ,证明
$$ m = \sqrt{m_1^2 + m_2^2 + 2[E_{1T}E_{2T} \cosh(\xi_1 - \xi_2) - \mathbf{p}_{1T} \cdot \mathbf{p}_{2T}]}, \tag{6.427} $$
其中 $m_i$、$E_{iT}$、$\mathbf{p}_{iT}$ 和 $\xi_i$ 分别是粒子 $i$ 的质量、横向能量、横向动量和快度。


\section{Feynman 图}


%%%%%%%%%%%%%%%%%%%%%%%%%%%%%%%%%%%%%%%%%%%%%%%%%%%%%%%%%%%%%%
\subsection{Yukawa 理论}

















补充推导:

(7.21)
\begin{equation}
    \begin{aligned}
     \langle 0|\mathrm{i}T_{1}^{(1)}|\mathbf{p}^+,\lambda ;\mathbf{q}^-,\lambda ^{\prime};\mathbf{k}\rangle &=-\mathrm{i}\kappa \int{\mathrm{d}^4x}\langle 0|\mathrm{N} [ \phi (x)\bar{\psi}(x)\psi (x) ] |\mathbf{p}^+,\lambda ;\mathbf{q}^-,\lambda ^{\prime};\mathbf{k}\rangle 
\\
\text{缩并}&=\wick{-\mathrm{i}\kappa \int{\mathrm{d}^4x}\langle 0|\mathrm{N} [ \c3{\phi}(x)\c2{\bar{\psi}}(x)\c1{\psi}(x) ] |\c1{\mathbf{p}}^+,\lambda ;\c2{\mathbf{q}}^-,\lambda ^{\prime};\c3{\mathbf{k}}\rangle } 
\\
\text{调整}&=\wick{-\mathrm{i}\kappa \int{\mathrm{d}^4x}\langle 0|\mathrm{N} [ \c3{\phi}(x)\c2{\bar{\psi}}_a(x)\c1{\psi}_a(x) ] |\c1{\mathbf{p}}^+,\lambda ;\c2{\mathbf{q}}^-,\lambda ^{\prime};\c3{\mathbf{k}}\rangle }
\\
&=-\mathrm{i}\kappa \int{\mathrm{d}^4x}\langle 0|\phi ^{(+)}(x)\bar{\psi}_{a}^{(+)}(x)\psi _{a}^{(+)}(x)|\mathbf{p}^+,\lambda ;\mathbf{q}^-,\lambda ^{\prime};\mathbf{k}\rangle 
\\
&=-\mathrm{i}\kappa \int{\mathrm{d}^4x}\langle 0|\mathrm{e}^{-\mathrm{i}k\cdot x}\bar{v}_a(\mathbf{q},\lambda ^{\prime})\mathrm{e}^{-\mathrm{i}q\cdot x}u_a(\mathbf{p},\lambda )\mathrm{e}^{-\mathrm{i}p\cdot x}|0\rangle 
\\
\text{忽略下标}a&=-\mathrm{i}\kappa \int{\mathrm{d}^4x}\bar{v}_a(\mathbf{q},\lambda ^{\prime})u_a(\mathbf{p},\lambda )\mathrm{e}^{-\mathrm{i}\left( p+q+k \right) \cdot x}\langle 0|0\rangle 
\\
&=-\mathrm{i}\kappa \int{\mathrm{d}^4x}\bar{v}(\mathbf{q},\lambda ^{\prime})u(\mathbf{p},\lambda )\mathrm{e}^{-\mathrm{i}\left( p+q+k \right) \cdot x}
\\
\text{带一横在前}&=-\mathrm{i}\kappa \bar{v}(\mathbf{q},\lambda ^{\prime})u(\mathbf{p},\lambda )\left( 2\pi \right) ^4\delta ^{(4)}(p+q+k)
    \end{aligned}
\end{equation}





(7.28)
\begin{equation}
    \begin{aligned}
       \langle \mathbf{p}^+,\lambda ;\mathbf{q}^-,\lambda ^{\prime};\mathbf{k}|\mathrm{i}T_{1}^{(1)}|0\rangle &=-\mathrm{i}\kappa \int{\mathrm{d}^4}x\langle \mathbf{p}^+,\lambda ;\mathbf{q}^-,\lambda ^{\prime};\mathbf{k}|\mathrm{N}[ \phi (x)\bar{\psi}(x)\psi (x) ] |0\rangle 
\\
\text{缩并}&=\wick{-\mathrm{i}\kappa \int{\mathrm{d}^4}x\langle \c3{\mathbf{p}}^+,\lambda ;\c2{\mathbf{q}}^-,\lambda ^{\prime};\c1{\mathbf{k}}|\mathrm{N}[ \c1{\phi}(x)\c3{\bar{\psi}}(x)\c2{\psi}(x) ] |0\rangle }
\\
\text{调整}&=\wick{{\color[RGB]{240, 0, 0} +}\mathrm{i}\kappa \int{\mathrm{d}^4}x\langle \c3{\mathbf{p}}^+,\lambda ;\c2{\mathbf{q}}^-,\lambda ^{\prime};\c1{\mathbf{k}}|\mathrm{N}[ \c1{\phi}(x)\c2{\psi}_a(x)\c3{\bar{\psi}}_a(x) ] |0\rangle }
\\
&=+\mathrm{i}\kappa \int{\mathrm{d}^4}x\langle \mathbf{p}^+,\lambda ;\mathbf{q}^-,\lambda ^{\prime};\mathbf{k}|\phi ^{(-)}(x)\psi _{a}^{(-)}(x)\bar{\psi}_{a}^{(-)}(x)|0\rangle 
\\
&=+\mathrm{i}\kappa \int{\mathrm{d}^4}x\langle 0|\mathrm{e}^{\mathrm{i}k\cdot x}v_a(\mathbf{q},\lambda ^{\prime})\mathrm{e}^{\mathrm{i}q\cdot x}\bar{u}_a(\mathbf{p},\lambda )\mathrm{e}^{\mathrm{i}p\cdot x}|0\rangle 
\\
&=+\mathrm{i}\kappa \int{\mathrm{d}^4}xv_a(\mathbf{q},\lambda ^{\prime})\bar{u}_a(\mathbf{p},\lambda )\mathrm{e}^{\mathrm{i}\left( p+q+k \right) \cdot x}\langle 0|0\rangle 
\\
\text{忽略下标,调整横在前}&=+\mathrm{i}\kappa \int{\mathrm{d}^4}x\bar{u}(\mathbf{p},\lambda )v(\mathbf{q},\lambda ^{\prime})\mathrm{e}^{\mathrm{i}\left( p+q+k \right) \cdot x}
\\
\delta \text{函数性质}&=+\mathrm{i}\kappa \bar{u}(\mathbf{p},\lambda )v(\mathbf{q},\lambda ^{\prime})\left( 2\pi \right) ^4\delta ^{(4)}(p+q+k)
    \end{aligned}
\end{equation}

(7.32)b
\begin{equation}
    \begin{aligned}
        \langle \mathbf{p}^+,\lambda ;\mathbf{q}^-,\lambda ^{\prime}|\mathrm{i}T_{1}^{(1)}|\mathbf{k}\rangle &=-\mathrm{i}\kappa \int{\mathrm{d}^4}x\langle \mathbf{p}^+,\lambda ;\mathbf{q}^-,\lambda ^{\prime}|\mathrm{N} [ \phi (x)\bar{\psi}(x)\psi (x) ] |\mathbf{k}\rangle 
\\
\text{缩并}&=\wick{-\mathrm{i}\kappa \int{\mathrm{d}^4}x\langle \c1{\mathbf{p}}^+,\lambda ;\c2{\mathbf{q}}^-,\lambda ^{\prime}|\mathrm{N} [\c3{\phi}(x)\c1{\bar{\psi}}(x)\c2{\psi}(x) ] |\c3{\mathbf{k}}\rangle }
\\
\text{调整}&=\wick{+\mathrm{i}\kappa \int{\mathrm{d}^4}x\langle \c2{\mathbf{p}}^+,\lambda ;\c1{\mathbf{q}}^-,\lambda ^{\prime}|\mathrm{N} [ \c1{\psi}_a(x)\c2{\bar{\psi}}_a(x)\c3{\phi}(x) ] |\c3{\mathbf{k}}\rangle }
\\
\text{去正规乘积}&=+\mathrm{i}\kappa \int{\mathrm{d}^4}x\langle \mathbf{p}^+,\lambda ;\mathbf{q}^-,\lambda ^{\prime}|\psi _{a}^{(-)}(x)\bar{\psi}_{a}^{(-)}(x)\phi ^{(+)}(x)|\mathbf{k}\rangle 
\\
&=+\mathrm{i}\kappa \int{\mathrm{d}^4}x\langle 0|v_a(\mathbf{q},\lambda ^{\prime})\mathrm{e}^{\mathrm{i}q\cdot x}\bar{u}_a(\mathbf{p},\lambda )\mathrm{e}^{\mathrm{i}p\cdot x}\mathrm{e}^{-\mathrm{i}k\cdot x}|0\rangle 
\\
&=+\mathrm{i}\kappa \int{\mathrm{d}^4}xv_a(\mathbf{q},\lambda ^{\prime})\bar{u}_a(\mathbf{p},\lambda )\mathrm{e}^{-\mathrm{i}\left( k-p-q \right) \cdot x}\langle 0|0\rangle 
\\
&=+\mathrm{i}\kappa \int{\mathrm{d}^4}x\bar{u}(\mathbf{p},\lambda )v(\mathbf{q},\lambda ^{\prime})\mathrm{e}^{-\mathrm{i}\left( k-p-q \right) \cdot x}
\\
\text{有一横在前面}&=+\mathrm{i}\kappa \bar{u}(\mathbf{p},\lambda )v(\mathbf{q},\lambda ^{\prime})\left( 2\pi \right) ^4\delta ^{(4)}(k-p-q)
    \end{aligned}
\end{equation}

(7.33)c
\begin{equation}
    \begin{aligned}
        \langle \mathbf{q}^+,\lambda ^{\prime};\mathbf{k}|\mathrm{i}T_{1}^{(1)}|\mathbf{p}^+,\lambda \rangle &=-\mathrm{i}\kappa \int{\mathrm{d}^4}x\langle \mathbf{q}^+,\lambda ^{\prime};\mathbf{k}|\mathrm{N} [ \phi (x)\bar{\psi}(x)\psi (x) ] |\mathbf{p}^+,\lambda \rangle 
\\
\text{缩并}&=\wick{-\mathrm{i}\kappa \int{\mathrm{d}^4}x\langle \c2{\mathbf{q}}^+,\lambda ^{\prime};\c1{\mathbf{k}}|\mathrm{N} [ \c1{\phi}(x)\c2{\bar{\psi}}(x)\c3{\psi}(x) ] |\c3{\mathbf{p}}^+,\lambda \rangle }
\\
\text{整理}&=\wick{-\mathrm{i}\kappa \int{\mathrm{d}^4}x\langle \c2{\mathbf{q}}^+,\lambda ^{\prime};\c1{\mathbf{k}}|\mathrm{N} [\c1{\phi}(x)\c2{\bar{\psi}}_a(x)\c3{\psi}_a(x) ] |\c3{\mathbf{p}}^+,\lambda \rangle }
\\
\text{去正规乘积}&=-\mathrm{i}\kappa \int{\mathrm{d}^4}x\langle \mathbf{q}^+,\lambda ^{\prime};\mathbf{k}|\phi ^{(-)}(x)\bar{\psi}_{a}^{(-)}(x)\psi _{a}^{(+)}(x)|\mathbf{p}^+,\lambda \rangle 
\\
&=-\mathrm{i}\kappa \int{\mathrm{d}^4}x\langle 0|\mathrm{e}^{\mathrm{i}k\cdot x}\bar{u}_a(\mathbf{q},\lambda ^{\prime})\mathrm{e}^{\mathrm{i}q\cdot x}u_a(\mathbf{p},\lambda )\mathrm{e}^{-\mathrm{i}p\cdot x}|0\rangle 
\\
&=-\mathrm{i}\kappa \int{\mathrm{d}^4}x\bar{u}_a(\mathbf{q},\lambda ^{\prime})u_a(\mathbf{p},\lambda )\mathrm{e}^{-\mathrm{i}\left( p-q-k \right) \cdot x}\langle 0|0\rangle 
\\
&=-\mathrm{i}\kappa \int{\mathrm{d}^4}x\bar{u}(\mathbf{q},\lambda ^{\prime})u(\mathbf{p},\lambda )\mathrm{e}^{-\mathrm{i}\left( p-q-k \right) \cdot x}
\\
&=-\mathrm{i}\kappa \bar{u}(\mathbf{q},\lambda ^{\prime})u(\mathbf{p},\lambda )\left( 2\pi \right) ^4\delta ^{(4)}(p-q-k)
    \end{aligned}
\end{equation}



(7.34)d
\begin{equation}
    \begin{aligned}
        \langle \mathbf{q}^-,\lambda ^{\prime};\mathbf{k}|\mathrm{i}T_{1}^{(1)}|\mathbf{p}^-,\lambda \rangle &=-\mathrm{i}\kappa \int{\mathrm{d}^4x}\langle \mathbf{q}^-,\lambda ^{\prime};\mathbf{k}|\mathrm{N} [ \phi (x)\bar{\psi}_a(x)\psi _a(x) ] |\mathbf{p}^-,\lambda \rangle 
\\
\text{缩并}&=\wick{-\mathrm{i}\kappa \int{\mathrm{d}^4}x\langle \c3{\mathbf{q}}^-,\lambda ^{\prime};\c1{\mathbf{k}}|\mathrm{N} [ \c1{\phi}(x)\c2{\bar{\psi}}(x)\c3{\psi}(x) ] |\c2{\mathbf{p}}^-,\lambda \rangle }
\\
\text{调整}&=\wick{+\mathrm{i}\kappa \int{\mathrm{d}^4}x\langle \c2{\mathbf{q}}^-,\lambda ^{\prime};\c1{\mathbf{k}}|\mathrm{N} [ \c1{\phi}(x)\c2{\bar{\psi}}_a(x)\c3{\psi}_a(x) ] |\c3{\mathbf{p}}^-,\lambda \rangle }
\\
&=+\mathrm{i}\kappa \int{\mathrm{d}^4}x\langle \mathbf{q}^-,\lambda ^{\prime};\mathbf{k}|\phi ^{(-)}(x)\psi _{a}^{(-)}(x)\bar{\psi}_{a}^{(+)}(x)|\mathbf{p}^-,\lambda \rangle 
\\
&=+\mathrm{i}\kappa \int{\mathrm{d}^4}x\langle 0|\mathrm{e}^{\mathrm{i}k\cdot x}v_a(\mathbf{q},\lambda ^{\prime})\mathrm{e}^{\mathrm{i}q\cdot x}\bar{v}_a(\mathbf{p},\lambda )\mathrm{e}^{-\mathrm{i}p\cdot x}|0\rangle 
\\
&=+\mathrm{i}\kappa \int{\mathrm{d}^4}xv_a(\mathbf{q},\lambda ^{\prime})\bar{v}_a(\mathbf{p},\lambda )\mathrm{e}^{-\mathrm{i}\left( p-q-k \right) \cdot x}\langle 0|0\rangle 
\\
&=+\mathrm{i}\kappa \int{\mathrm{d}^4}x\bar{v}(\mathbf{p},\lambda )v(\mathbf{q},\lambda ^{\prime})\mathrm{e}^{-\mathrm{i}\left( p-q-k \right) \cdot x}
\\
&=+\mathrm{i}\kappa \bar{v}(\mathbf{p},\lambda )v(\mathbf{q},\lambda ^{\prime})\left( 2\pi \right) ^4\delta ^{(4)}(p-q-k)
    \end{aligned}
\end{equation}

(7.35)f
\begin{equation}
    \begin{aligned}
        \langle \mathbf{k}|\mathrm{i}T_{1}^{(1)}|\mathbf{p}^+,\lambda ;\mathbf{q}^-,\lambda ^{\prime}\rangle &=-\mathrm{i}\kappa \int{\mathrm{d}^4}x\langle \mathbf{k}|\mathrm{N}[ \phi (x)\bar{\psi}(x)\psi (x) ] |\mathbf{p}^+,\lambda ;\mathbf{q}^-,\lambda ^{\prime}\rangle 
\\
\text{缩并}&=\wick{-\mathrm{i}\kappa \int{\mathrm{d}^4}x\langle \c1{\mathbf{k}}|\mathrm{N} [ \c1{\phi}(x)\c3{\bar{\psi}}(x)\c2{\psi}(x) ] |\c2{\mathbf{p}}^+,\lambda ;\c3{\mathbf{q}}^-,\lambda ^{\prime}\rangle }
\\
\text{调整}&=\wick{-\mathrm{i}\kappa \int{\mathrm{d}^4}x\langle \c1{\mathbf{k}}|\mathrm{N} [ \c1{\phi}(x)\c3{\bar{\psi}}_a(x)\c2{\psi}_a(x) ] |\c2{\mathbf{p}}^+,\lambda ;\c3{\mathbf{q}}^-,\lambda ^{\prime}\rangle }
\\
\text{去正规乘积}&=-\mathrm{i}\kappa \int{\mathrm{d}^4}x\langle \mathbf{k}|\phi ^{(-)}(x)\bar{\psi}_{a}^{(+)}(x)\psi _{a}^{(+)}(x)|\mathbf{p}^+,\lambda ;\mathbf{q}^-,\lambda ^{\prime}\rangle 
\\
&=-\mathrm{i}\kappa \int{\mathrm{d}^4}x\langle 0|\mathrm{e}^{\mathrm{i}k\cdot x}\bar{v}_a(\mathbf{q},\lambda ^{\prime})\mathrm{e}^{-\mathrm{i}q\cdot x}u_a(\mathbf{p},\lambda )\mathrm{e}^{-\mathrm{i}p\cdot x}|0\rangle 
\\
&=-\mathrm{i}\kappa \int{\mathrm{d}^4}x\bar{v}_a(\mathbf{q},\lambda ^{\prime})u_a(\mathbf{p},\lambda )\mathrm{e}^{-\mathrm{i}\left( p+q-k \right) \cdot x}\langle 0|0\rangle 
\\
&=-\mathrm{i}\kappa \int{\mathrm{d}^4}x\bar{v}(\mathbf{q},\lambda ^{\prime})u(\mathbf{p},\lambda )\mathrm{e}^{-\mathrm{i}\left( p+q-k \right) \cdot x}
\\
\delta \text{函数性质}&=-\mathrm{i}\kappa \bar{v}(\mathbf{q},\lambda ^{\prime})u(\mathbf{p},\lambda )\left( 2\pi \right) ^4\delta ^{(4)}(p+q-k)
    \end{aligned}
\end{equation}


(7.36)g
\begin{equation}
    \begin{aligned}
        \langle \mathbf{q}^+,\lambda ^{\prime}|\mathrm{i}T_{1}^{(1)}|\mathbf{p}^+,\lambda ;\mathbf{k}\rangle &=-\mathrm{i}\kappa \int{\mathrm{d}^4}x\langle \mathbf{q}^+,\lambda ^{\prime}|\mathrm{N}[ \phi (x)\bar{\psi}(x)\psi (x) ] |\mathbf{p}^+,\lambda ;\mathbf{k}\rangle 
\\
\text{缩并}&=\wick{-\mathrm{i}\kappa \int{\mathrm{d}^4}x\langle \c1{\mathbf{q}}^+,\lambda ^{\prime}|\mathrm{N}[ \c3{\phi}(x)\c1{\bar{\psi}}(x)\c2{\psi}(x) ] |\c2{\mathbf{p}}^+,\lambda ;\c3{\mathbf{k}}\rangle }
\\
\text{调整}&=\wick{-\mathrm{i}\kappa \int{\mathrm{d}^4}x\langle \c1{\mathbf{q}}^+,\lambda ^{\prime}|\mathrm{N}[ \c1{\bar{\psi}}_a(x)\c3{\phi}(x)\c2{\psi}_a(x) ] |\c2{\mathbf{p}}^+,\lambda ;\c3{\mathbf{k}}\rangle }
\\
\text{去正规乘积}&=-\mathrm{i}\kappa \int{\mathrm{d}^4}x\langle \mathbf{q}^+,\lambda ^{\prime}|\bar{\psi}_{a}^{(-)}(x)\phi ^{(+)}(x)\psi _{a}^{(+)}(x)|\mathbf{p}^+,\lambda ;\mathbf{k}\rangle 
\\
&=-\mathrm{i}\kappa \int{\mathrm{d}^4}x\langle 0|\bar{u}_a(\mathbf{q},\lambda ^{\prime})\mathrm{e}^{\mathrm{i}q\cdot x}\mathrm{e}^{-\mathrm{i}k\cdot x}u_a(\mathbf{p},\lambda )\mathrm{e}^{-\mathrm{i}p\cdot x}|0\rangle 
\\
&=-\mathrm{i}\kappa \int{\mathrm{d}^4}x\bar{u}_a(\mathbf{q},\lambda ^{\prime})u_a(\mathbf{p},\lambda )\mathrm{e}^{-\mathrm{i}\left( p+k-q \right) \cdot x}\langle 0|0\rangle 
\\
&=-\mathrm{i}\kappa \int{\mathrm{d}^4}x\bar{u}(\mathbf{q},\lambda ^{\prime})u(\mathbf{p},\lambda )\mathrm{e}^{-\mathrm{i}\left( p+k-q \right) \cdot x}
\\
\text{函数性质}&=-\mathrm{i}\kappa \bar{u}(\mathbf{q},\lambda ^{\prime})u(\mathbf{p},\lambda )\left( 2\pi \right) ^4\delta ^{(4)}(p+k-q)
    \end{aligned}
\end{equation}


(7.37)h
\begin{equation}
    \begin{aligned}
        \langle \mathbf{q}^-,\lambda ^{\prime}|\mathrm{i}T_{1}^{(1)}|\mathbf{p}^-,\lambda ;\mathbf{k}\rangle &=-\mathrm{i}\kappa \int{\mathrm{d}^4}x\langle \mathbf{q}^-,\lambda ^{\prime}|\mathrm{N}[ \phi (x)\bar{\psi}(x)\psi (x) ] |\mathbf{p}^-,\lambda ;\mathbf{k}\rangle 
\\
\text{缩并}&=\wick{-\mathrm{i}\kappa \int{\mathrm{d}^4}x\langle \c1{\mathbf{q}}^-,\lambda ^{\prime}|\mathrm{N}[ \c2{\phi}(x)\c3{\bar{\psi}}(x)\c1{\psi}(x) ] |\c3{\mathbf{p}}^-,\lambda ;\c2{\mathbf{k}}\rangle }
\\
\text{调整}&=\wick{+\mathrm{i}\kappa \int{\mathrm{d}^4}x\langle \c1{\mathbf{q}}^-,\lambda ^{\prime}|\mathrm{N} [ \c1{\psi}_a(x)\c3{\phi}(x)\c2{\bar{\psi}}_a(x) ] |\c2{\mathbf{p}}^-,\lambda ;\c3{\mathbf{k}}\rangle }
\\
\text{正负能解}&=+\mathrm{i}\kappa \int{\mathrm{d}^4}x\langle \mathbf{q}^-,\lambda ^{\prime}|\psi _{a}^{(-)}(x)\phi (x)\bar{\psi}_{a}^{(+)}(x)|\mathbf{p}^-,\lambda ;\mathbf{k}\rangle 
\\
&=+\mathrm{i}\kappa \int{\mathrm{d}^4}x\langle 0|v_a(\mathbf{q},\lambda ^{\prime})\mathrm{e}^{\mathrm{i}q\cdot x}\mathrm{e}^{-\mathrm{i}k\cdot x}\bar{v}_a(\mathbf{p},\lambda )\mathrm{e}^{-\mathrm{i}p\cdot x}|0\rangle 
\\
&=+\mathrm{i}\kappa \int{\mathrm{d}^4}x\bar{v}_a(\mathbf{p},\lambda )v_a(\mathbf{q},\lambda ^{\prime})\mathrm{e}^{-\mathrm{i}\left( p+k-q \right) \cdot x}\langle 0|0\rangle 
\\
\text{横在前}&=+\mathrm{i}\kappa \int{\mathrm{d}^4}x\bar{v}(\mathbf{p},\lambda )v(\mathbf{q},\lambda ^{\prime})\mathrm{e}^{-\mathrm{i}\left( p+k-q \right) \cdot x}
\\
\text{函数性质}&=+\mathrm{i}\kappa \bar{v}(\mathbf{p},\lambda )v(\mathbf{q},\lambda ^{\prime})\left( 2\pi \right) ^4\delta ^{(4)}(p+k-q)
    \end{aligned}
\end{equation}


\newpage




\newpage
(7.58)无缩并
\begin{equation}
    \mathrm{i}T_{1}^{(2)}=\frac{\left( -\mathrm{i}\kappa \right) ^2}{2!}\int{\mathrm{d}^4x\mathrm{d}^4y\mathrm{N[}\phi (x)\bar{\psi}(x)\psi (x)\phi (y)\bar{\psi}(y)\psi (y)]}
\end{equation}

一次
\begin{equation}
    \begin{aligned}
        \mathrm{i}T_{2}^{(2)}&=\wick{ \frac{\left( -\mathrm{i}\kappa \right) ^2}{2!}\int{\mathrm{d}^4x\mathrm{d}^4y\mathrm{N} [ \c{\phi}(x)\bar{\psi}(x)\psi(x)\c{\phi}(y)\bar{\psi}(y)\psi (y)]} }
\\
\mathrm{i}T_{3}^{(2)}&=\frac{\left( -\mathrm{i}\kappa \right) ^2}{2!}\int{\mathrm{d}^4x\mathrm{d}^4y\mathrm{N} [ \phi (x)\bar{\psi}(x)\psi (x)\phi (y)\bar{\psi}(y)\psi (y)]}
\\
\mathrm{i}T_{4}^{(2)}&=\frac{\left( -\mathrm{i}\kappa \right) ^2}{2!}\int{\mathrm{d}^4x\mathrm{d}^4y\mathrm{N} [ \phi (x)\bar{\psi}(x)\psi (x)\phi (y)\bar{\psi}(y)\psi (y)]}
\\
\mathrm{i}T_{5}^{(2)}&=\frac{\left( -\mathrm{i}\kappa \right) ^2}{2!}\int{\mathrm{d}^4x\mathrm{d}^4y\mathrm{N} [ \phi (x)\bar{\psi}(x)\psi (x)\phi (y)\bar{\psi}(y)\psi (y)]}
\\
\mathrm{i}T_{6}^{(2)}&=\frac{\left( -\mathrm{i}\kappa \right) ^2}{2!}\int{\mathrm{d}^4x\mathrm{d}^4y\mathrm{N} [ \phi (x)\bar{\psi}(x)\psi (x)\phi (y)\bar{\psi}(y)\psi (y)]}
    \end{aligned}
\end{equation}

二次缩并





三次缩并




、




























\section{Feynman 图}


%%%%%%%%%%%%%%%%%%%%%%%%%%%%%%%%%%%%%%%%%%%%%%%%%%%%%%%%%%%%%%
\subsection{Yukawa 理论}

















补充推导:

(7.21)
\begin{equation}
    \begin{aligned}
     \langle 0|\mathrm{i}T_{1}^{(1)}|\mathbf{p}^+,\lambda ;\mathbf{q}^-,\lambda ^{\prime};\mathbf{k}\rangle &=-\mathrm{i}\kappa \int{\mathrm{d}^4x}\langle 0|\mathrm{N} [ \phi (x)\bar{\psi}(x)\psi (x) ] |\mathbf{p}^+,\lambda ;\mathbf{q}^-,\lambda ^{\prime};\mathbf{k}\rangle 
\\
\text{缩并}&=\wick{-\mathrm{i}\kappa \int{\mathrm{d}^4x}\langle 0|\mathrm{N} [ \c3{\phi}(x)\c2{\bar{\psi}}(x)\c1{\psi}(x) ] |\c1{\mathbf{p}}^+,\lambda ;\c2{\mathbf{q}}^-,\lambda ^{\prime};\c3{\mathbf{k}}\rangle } 
\\
\text{调整}&=\wick{-\mathrm{i}\kappa \int{\mathrm{d}^4x}\langle 0|\mathrm{N} [ \c3{\phi}(x)\c2{\bar{\psi}}_a(x)\c1{\psi}_a(x) ] |\c1{\mathbf{p}}^+,\lambda ;\c2{\mathbf{q}}^-,\lambda ^{\prime};\c3{\mathbf{k}}\rangle }
\\
&=-\mathrm{i}\kappa \int{\mathrm{d}^4x}\langle 0|\phi ^{(+)}(x)\bar{\psi}_{a}^{(+)}(x)\psi _{a}^{(+)}(x)|\mathbf{p}^+,\lambda ;\mathbf{q}^-,\lambda ^{\prime};\mathbf{k}\rangle 
\\
&=-\mathrm{i}\kappa \int{\mathrm{d}^4x}\langle 0|\mathrm{e}^{-\mathrm{i}k\cdot x}\bar{v}_a(\mathbf{q},\lambda ^{\prime})\mathrm{e}^{-\mathrm{i}q\cdot x}u_a(\mathbf{p},\lambda )\mathrm{e}^{-\mathrm{i}p\cdot x}|0\rangle 
\\
\text{忽略下标}a&=-\mathrm{i}\kappa \int{\mathrm{d}^4x}\bar{v}_a(\mathbf{q},\lambda ^{\prime})u_a(\mathbf{p},\lambda )\mathrm{e}^{-\mathrm{i}\left( p+q+k \right) \cdot x}\langle 0|0\rangle 
\\
&=-\mathrm{i}\kappa \int{\mathrm{d}^4x}\bar{v}(\mathbf{q},\lambda ^{\prime})u(\mathbf{p},\lambda )\mathrm{e}^{-\mathrm{i}\left( p+q+k \right) \cdot x}
\\
\text{带一横在前}&=-\mathrm{i}\kappa \bar{v}(\mathbf{q},\lambda ^{\prime})u(\mathbf{p},\lambda )\left( 2\pi \right) ^4\delta ^{(4)}(p+q+k)
    \end{aligned}
\end{equation}





(7.28)
\begin{equation}
    \begin{aligned}
       \langle \mathbf{p}^+,\lambda ;\mathbf{q}^-,\lambda ^{\prime};\mathbf{k}|\mathrm{i}T_{1}^{(1)}|0\rangle &=-\mathrm{i}\kappa \int{\mathrm{d}^4}x\langle \mathbf{p}^+,\lambda ;\mathbf{q}^-,\lambda ^{\prime};\mathbf{k}|\mathrm{N}[ \phi (x)\bar{\psi}(x)\psi (x) ] |0\rangle 
\\
\text{缩并}&=\wick{-\mathrm{i}\kappa \int{\mathrm{d}^4}x\langle \c3{\mathbf{p}}^+,\lambda ;\c2{\mathbf{q}}^-,\lambda ^{\prime};\c1{\mathbf{k}}|\mathrm{N}[ \c1{\phi}(x)\c3{\bar{\psi}}(x)\c2{\psi}(x) ] |0\rangle }
\\
\text{调整}&=\wick{{\color[RGB]{240, 0, 0} +}\mathrm{i}\kappa \int{\mathrm{d}^4}x\langle \c3{\mathbf{p}}^+,\lambda ;\c2{\mathbf{q}}^-,\lambda ^{\prime};\c1{\mathbf{k}}|\mathrm{N}[ \c1{\phi}(x)\c2{\psi}_a(x)\c3{\bar{\psi}}_a(x) ] |0\rangle }
\\
&=+\mathrm{i}\kappa \int{\mathrm{d}^4}x\langle \mathbf{p}^+,\lambda ;\mathbf{q}^-,\lambda ^{\prime};\mathbf{k}|\phi ^{(-)}(x)\psi _{a}^{(-)}(x)\bar{\psi}_{a}^{(-)}(x)|0\rangle 
\\
&=+\mathrm{i}\kappa \int{\mathrm{d}^4}x\langle 0|\mathrm{e}^{\mathrm{i}k\cdot x}v_a(\mathbf{q},\lambda ^{\prime})\mathrm{e}^{\mathrm{i}q\cdot x}\bar{u}_a(\mathbf{p},\lambda )\mathrm{e}^{\mathrm{i}p\cdot x}|0\rangle 
\\
&=+\mathrm{i}\kappa \int{\mathrm{d}^4}xv_a(\mathbf{q},\lambda ^{\prime})\bar{u}_a(\mathbf{p},\lambda )\mathrm{e}^{\mathrm{i}\left( p+q+k \right) \cdot x}\langle 0|0\rangle 
\\
\text{忽略下标,调整横在前}&=+\mathrm{i}\kappa \int{\mathrm{d}^4}x\bar{u}(\mathbf{p},\lambda )v(\mathbf{q},\lambda ^{\prime})\mathrm{e}^{\mathrm{i}\left( p+q+k \right) \cdot x}
\\
\delta \text{函数性质}&=+\mathrm{i}\kappa \bar{u}(\mathbf{p},\lambda )v(\mathbf{q},\lambda ^{\prime})\left( 2\pi \right) ^4\delta ^{(4)}(p+q+k)
    \end{aligned}
\end{equation}

(7.32)b
\begin{equation}
    \begin{aligned}
        \langle \mathbf{p}^+,\lambda ;\mathbf{q}^-,\lambda ^{\prime}|\mathrm{i}T_{1}^{(1)}|\mathbf{k}\rangle &=-\mathrm{i}\kappa \int{\mathrm{d}^4}x\langle \mathbf{p}^+,\lambda ;\mathbf{q}^-,\lambda ^{\prime}|\mathrm{N} [ \phi (x)\bar{\psi}(x)\psi (x) ] |\mathbf{k}\rangle 
\\
\text{缩并}&=\wick{-\mathrm{i}\kappa \int{\mathrm{d}^4}x\langle \c1{\mathbf{p}}^+,\lambda ;\c2{\mathbf{q}}^-,\lambda ^{\prime}|\mathrm{N} [\c3{\phi}(x)\c1{\bar{\psi}}(x)\c2{\psi}(x) ] |\c3{\mathbf{k}}\rangle }
\\
\text{调整}&=\wick{+\mathrm{i}\kappa \int{\mathrm{d}^4}x\langle \c2{\mathbf{p}}^+,\lambda ;\c1{\mathbf{q}}^-,\lambda ^{\prime}|\mathrm{N} [ \c1{\psi}_a(x)\c2{\bar{\psi}}_a(x)\c3{\phi}(x) ] |\c3{\mathbf{k}}\rangle }
\\
\text{去正规乘积}&=+\mathrm{i}\kappa \int{\mathrm{d}^4}x\langle \mathbf{p}^+,\lambda ;\mathbf{q}^-,\lambda ^{\prime}|\psi _{a}^{(-)}(x)\bar{\psi}_{a}^{(-)}(x)\phi ^{(+)}(x)|\mathbf{k}\rangle 
\\
&=+\mathrm{i}\kappa \int{\mathrm{d}^4}x\langle 0|v_a(\mathbf{q},\lambda ^{\prime})\mathrm{e}^{\mathrm{i}q\cdot x}\bar{u}_a(\mathbf{p},\lambda )\mathrm{e}^{\mathrm{i}p\cdot x}\mathrm{e}^{-\mathrm{i}k\cdot x}|0\rangle 
\\
&=+\mathrm{i}\kappa \int{\mathrm{d}^4}xv_a(\mathbf{q},\lambda ^{\prime})\bar{u}_a(\mathbf{p},\lambda )\mathrm{e}^{-\mathrm{i}\left( k-p-q \right) \cdot x}\langle 0|0\rangle 
\\
&=+\mathrm{i}\kappa \int{\mathrm{d}^4}x\bar{u}(\mathbf{p},\lambda )v(\mathbf{q},\lambda ^{\prime})\mathrm{e}^{-\mathrm{i}\left( k-p-q \right) \cdot x}
\\
\text{有一横在前面}&=+\mathrm{i}\kappa \bar{u}(\mathbf{p},\lambda )v(\mathbf{q},\lambda ^{\prime})\left( 2\pi \right) ^4\delta ^{(4)}(k-p-q)
    \end{aligned}
\end{equation}

(7.33)c
\begin{equation}
    \begin{aligned}
        \langle \mathbf{q}^+,\lambda ^{\prime};\mathbf{k}|\mathrm{i}T_{1}^{(1)}|\mathbf{p}^+,\lambda \rangle &=-\mathrm{i}\kappa \int{\mathrm{d}^4}x\langle \mathbf{q}^+,\lambda ^{\prime};\mathbf{k}|\mathrm{N} [ \phi (x)\bar{\psi}(x)\psi (x) ] |\mathbf{p}^+,\lambda \rangle 
\\
\text{缩并}&=\wick{-\mathrm{i}\kappa \int{\mathrm{d}^4}x\langle \c2{\mathbf{q}}^+,\lambda ^{\prime};\c1{\mathbf{k}}|\mathrm{N} [ \c1{\phi}(x)\c2{\bar{\psi}}(x)\c3{\psi}(x) ] |\c3{\mathbf{p}}^+,\lambda \rangle }
\\
\text{整理}&=\wick{-\mathrm{i}\kappa \int{\mathrm{d}^4}x\langle \c2{\mathbf{q}}^+,\lambda ^{\prime};\c1{\mathbf{k}}|\mathrm{N} [\c1{\phi}(x)\c2{\bar{\psi}}_a(x)\c3{\psi}_a(x) ] |\c3{\mathbf{p}}^+,\lambda \rangle }
\\
\text{去正规乘积}&=-\mathrm{i}\kappa \int{\mathrm{d}^4}x\langle \mathbf{q}^+,\lambda ^{\prime};\mathbf{k}|\phi ^{(-)}(x)\bar{\psi}_{a}^{(-)}(x)\psi _{a}^{(+)}(x)|\mathbf{p}^+,\lambda \rangle 
\\
&=-\mathrm{i}\kappa \int{\mathrm{d}^4}x\langle 0|\mathrm{e}^{\mathrm{i}k\cdot x}\bar{u}_a(\mathbf{q},\lambda ^{\prime})\mathrm{e}^{\mathrm{i}q\cdot x}u_a(\mathbf{p},\lambda )\mathrm{e}^{-\mathrm{i}p\cdot x}|0\rangle 
\\
&=-\mathrm{i}\kappa \int{\mathrm{d}^4}x\bar{u}_a(\mathbf{q},\lambda ^{\prime})u_a(\mathbf{p},\lambda )\mathrm{e}^{-\mathrm{i}\left( p-q-k \right) \cdot x}\langle 0|0\rangle 
\\
&=-\mathrm{i}\kappa \int{\mathrm{d}^4}x\bar{u}(\mathbf{q},\lambda ^{\prime})u(\mathbf{p},\lambda )\mathrm{e}^{-\mathrm{i}\left( p-q-k \right) \cdot x}
\\
&=-\mathrm{i}\kappa \bar{u}(\mathbf{q},\lambda ^{\prime})u(\mathbf{p},\lambda )\left( 2\pi \right) ^4\delta ^{(4)}(p-q-k)
    \end{aligned}
\end{equation}



(7.34)d
\begin{equation}
    \begin{aligned}
        \langle \mathbf{q}^-,\lambda ^{\prime};\mathbf{k}|\mathrm{i}T_{1}^{(1)}|\mathbf{p}^-,\lambda \rangle &=-\mathrm{i}\kappa \int{\mathrm{d}^4x}\langle \mathbf{q}^-,\lambda ^{\prime};\mathbf{k}|\mathrm{N} [ \phi (x)\bar{\psi}_a(x)\psi _a(x) ] |\mathbf{p}^-,\lambda \rangle 
\\
\text{缩并}&=\wick{-\mathrm{i}\kappa \int{\mathrm{d}^4}x\langle \c3{\mathbf{q}}^-,\lambda ^{\prime};\c1{\mathbf{k}}|\mathrm{N} [ \c1{\phi}(x)\c2{\bar{\psi}}(x)\c3{\psi}(x) ] |\c2{\mathbf{p}}^-,\lambda \rangle }
\\
\text{调整}&=\wick{+\mathrm{i}\kappa \int{\mathrm{d}^4}x\langle \c2{\mathbf{q}}^-,\lambda ^{\prime};\c1{\mathbf{k}}|\mathrm{N} [ \c1{\phi}(x)\c2{\bar{\psi}}_a(x)\c3{\psi}_a(x) ] |\c3{\mathbf{p}}^-,\lambda \rangle }
\\
&=+\mathrm{i}\kappa \int{\mathrm{d}^4}x\langle \mathbf{q}^-,\lambda ^{\prime};\mathbf{k}|\phi ^{(-)}(x)\psi _{a}^{(-)}(x)\bar{\psi}_{a}^{(+)}(x)|\mathbf{p}^-,\lambda \rangle 
\\
&=+\mathrm{i}\kappa \int{\mathrm{d}^4}x\langle 0|\mathrm{e}^{\mathrm{i}k\cdot x}v_a(\mathbf{q},\lambda ^{\prime})\mathrm{e}^{\mathrm{i}q\cdot x}\bar{v}_a(\mathbf{p},\lambda )\mathrm{e}^{-\mathrm{i}p\cdot x}|0\rangle 
\\
&=+\mathrm{i}\kappa \int{\mathrm{d}^4}xv_a(\mathbf{q},\lambda ^{\prime})\bar{v}_a(\mathbf{p},\lambda )\mathrm{e}^{-\mathrm{i}\left( p-q-k \right) \cdot x}\langle 0|0\rangle 
\\
&=+\mathrm{i}\kappa \int{\mathrm{d}^4}x\bar{v}(\mathbf{p},\lambda )v(\mathbf{q},\lambda ^{\prime})\mathrm{e}^{-\mathrm{i}\left( p-q-k \right) \cdot x}
\\
&=+\mathrm{i}\kappa \bar{v}(\mathbf{p},\lambda )v(\mathbf{q},\lambda ^{\prime})\left( 2\pi \right) ^4\delta ^{(4)}(p-q-k)
    \end{aligned}
\end{equation}

(7.35)f
\begin{equation}
    \begin{aligned}
        \langle \mathbf{k}|\mathrm{i}T_{1}^{(1)}|\mathbf{p}^+,\lambda ;\mathbf{q}^-,\lambda ^{\prime}\rangle &=-\mathrm{i}\kappa \int{\mathrm{d}^4}x\langle \mathbf{k}|\mathrm{N}[ \phi (x)\bar{\psi}(x)\psi (x) ] |\mathbf{p}^+,\lambda ;\mathbf{q}^-,\lambda ^{\prime}\rangle 
\\
\text{缩并}&=\wick{-\mathrm{i}\kappa \int{\mathrm{d}^4}x\langle \c1{\mathbf{k}}|\mathrm{N} [ \c1{\phi}(x)\c3{\bar{\psi}}(x)\c2{\psi}(x) ] |\c2{\mathbf{p}}^+,\lambda ;\c3{\mathbf{q}}^-,\lambda ^{\prime}\rangle }
\\
\text{调整}&=\wick{-\mathrm{i}\kappa \int{\mathrm{d}^4}x\langle \c1{\mathbf{k}}|\mathrm{N} [ \c1{\phi}(x)\c3{\bar{\psi}}_a(x)\c2{\psi}_a(x) ] |\c2{\mathbf{p}}^+,\lambda ;\c3{\mathbf{q}}^-,\lambda ^{\prime}\rangle }
\\
\text{去正规乘积}&=-\mathrm{i}\kappa \int{\mathrm{d}^4}x\langle \mathbf{k}|\phi ^{(-)}(x)\bar{\psi}_{a}^{(+)}(x)\psi _{a}^{(+)}(x)|\mathbf{p}^+,\lambda ;\mathbf{q}^-,\lambda ^{\prime}\rangle 
\\
&=-\mathrm{i}\kappa \int{\mathrm{d}^4}x\langle 0|\mathrm{e}^{\mathrm{i}k\cdot x}\bar{v}_a(\mathbf{q},\lambda ^{\prime})\mathrm{e}^{-\mathrm{i}q\cdot x}u_a(\mathbf{p},\lambda )\mathrm{e}^{-\mathrm{i}p\cdot x}|0\rangle 
\\
&=-\mathrm{i}\kappa \int{\mathrm{d}^4}x\bar{v}_a(\mathbf{q},\lambda ^{\prime})u_a(\mathbf{p},\lambda )\mathrm{e}^{-\mathrm{i}\left( p+q-k \right) \cdot x}\langle 0|0\rangle 
\\
&=-\mathrm{i}\kappa \int{\mathrm{d}^4}x\bar{v}(\mathbf{q},\lambda ^{\prime})u(\mathbf{p},\lambda )\mathrm{e}^{-\mathrm{i}\left( p+q-k \right) \cdot x}
\\
\delta \text{函数性质}&=-\mathrm{i}\kappa \bar{v}(\mathbf{q},\lambda ^{\prime})u(\mathbf{p},\lambda )\left( 2\pi \right) ^4\delta ^{(4)}(p+q-k)
    \end{aligned}
\end{equation}


(7.36)g
\begin{equation}
    \begin{aligned}
        \langle \mathbf{q}^+,\lambda ^{\prime}|\mathrm{i}T_{1}^{(1)}|\mathbf{p}^+,\lambda ;\mathbf{k}\rangle &=-\mathrm{i}\kappa \int{\mathrm{d}^4}x\langle \mathbf{q}^+,\lambda ^{\prime}|\mathrm{N}[ \phi (x)\bar{\psi}(x)\psi (x) ] |\mathbf{p}^+,\lambda ;\mathbf{k}\rangle 
\\
\text{缩并}&=\wick{-\mathrm{i}\kappa \int{\mathrm{d}^4}x\langle \c1{\mathbf{q}}^+,\lambda ^{\prime}|\mathrm{N}[ \c3{\phi}(x)\c1{\bar{\psi}}(x)\c2{\psi}(x) ] |\c2{\mathbf{p}}^+,\lambda ;\c3{\mathbf{k}}\rangle }
\\
\text{调整}&=\wick{-\mathrm{i}\kappa \int{\mathrm{d}^4}x\langle \c1{\mathbf{q}}^+,\lambda ^{\prime}|\mathrm{N}[ \c1{\bar{\psi}}_a(x)\c3{\phi}(x)\c2{\psi}_a(x) ] |\c2{\mathbf{p}}^+,\lambda ;\c3{\mathbf{k}}\rangle }
\\
\text{去正规乘积}&=-\mathrm{i}\kappa \int{\mathrm{d}^4}x\langle \mathbf{q}^+,\lambda ^{\prime}|\bar{\psi}_{a}^{(-)}(x)\phi ^{(+)}(x)\psi _{a}^{(+)}(x)|\mathbf{p}^+,\lambda ;\mathbf{k}\rangle 
\\
&=-\mathrm{i}\kappa \int{\mathrm{d}^4}x\langle 0|\bar{u}_a(\mathbf{q},\lambda ^{\prime})\mathrm{e}^{\mathrm{i}q\cdot x}\mathrm{e}^{-\mathrm{i}k\cdot x}u_a(\mathbf{p},\lambda )\mathrm{e}^{-\mathrm{i}p\cdot x}|0\rangle 
\\
&=-\mathrm{i}\kappa \int{\mathrm{d}^4}x\bar{u}_a(\mathbf{q},\lambda ^{\prime})u_a(\mathbf{p},\lambda )\mathrm{e}^{-\mathrm{i}\left( p+k-q \right) \cdot x}\langle 0|0\rangle 
\\
&=-\mathrm{i}\kappa \int{\mathrm{d}^4}x\bar{u}(\mathbf{q},\lambda ^{\prime})u(\mathbf{p},\lambda )\mathrm{e}^{-\mathrm{i}\left( p+k-q \right) \cdot x}
\\
\text{函数性质}&=-\mathrm{i}\kappa \bar{u}(\mathbf{q},\lambda ^{\prime})u(\mathbf{p},\lambda )\left( 2\pi \right) ^4\delta ^{(4)}(p+k-q)
    \end{aligned}
\end{equation}


(7.37)h
\begin{equation}
    \begin{aligned}
        \langle \mathbf{q}^-,\lambda ^{\prime}|\mathrm{i}T_{1}^{(1)}|\mathbf{p}^-,\lambda ;\mathbf{k}\rangle &=-\mathrm{i}\kappa \int{\mathrm{d}^4}x\langle \mathbf{q}^-,\lambda ^{\prime}|\mathrm{N}[ \phi (x)\bar{\psi}(x)\psi (x) ] |\mathbf{p}^-,\lambda ;\mathbf{k}\rangle 
\\
\text{缩并}&=\wick{-\mathrm{i}\kappa \int{\mathrm{d}^4}x\langle \c1{\mathbf{q}}^-,\lambda ^{\prime}|\mathrm{N}[ \c2{\phi}(x)\c3{\bar{\psi}}(x)\c1{\psi}(x) ] |\c3{\mathbf{p}}^-,\lambda ;\c2{\mathbf{k}}\rangle }
\\
\text{调整}&=\wick{+\mathrm{i}\kappa \int{\mathrm{d}^4}x\langle \c1{\mathbf{q}}^-,\lambda ^{\prime}|\mathrm{N} [ \c1{\psi}_a(x)\c3{\phi}(x)\c2{\bar{\psi}}_a(x) ] |\c2{\mathbf{p}}^-,\lambda ;\c3{\mathbf{k}}\rangle }
\\
\text{正负能解}&=+\mathrm{i}\kappa \int{\mathrm{d}^4}x\langle \mathbf{q}^-,\lambda ^{\prime}|\psi _{a}^{(-)}(x)\phi (x)\bar{\psi}_{a}^{(+)}(x)|\mathbf{p}^-,\lambda ;\mathbf{k}\rangle 
\\
&=+\mathrm{i}\kappa \int{\mathrm{d}^4}x\langle 0|v_a(\mathbf{q},\lambda ^{\prime})\mathrm{e}^{\mathrm{i}q\cdot x}\mathrm{e}^{-\mathrm{i}k\cdot x}\bar{v}_a(\mathbf{p},\lambda )\mathrm{e}^{-\mathrm{i}p\cdot x}|0\rangle 
\\
&=+\mathrm{i}\kappa \int{\mathrm{d}^4}x\bar{v}_a(\mathbf{p},\lambda )v_a(\mathbf{q},\lambda ^{\prime})\mathrm{e}^{-\mathrm{i}\left( p+k-q \right) \cdot x}\langle 0|0\rangle 
\\
\text{横在前}&=+\mathrm{i}\kappa \int{\mathrm{d}^4}x\bar{v}(\mathbf{p},\lambda )v(\mathbf{q},\lambda ^{\prime})\mathrm{e}^{-\mathrm{i}\left( p+k-q \right) \cdot x}
\\
\text{函数性质}&=+\mathrm{i}\kappa \bar{v}(\mathbf{p},\lambda )v(\mathbf{q},\lambda ^{\prime})\left( 2\pi \right) ^4\delta ^{(4)}(p+k-q)
    \end{aligned}
\end{equation}


\newpage




\newpage
(7.58)无缩并
\begin{equation}
    \mathrm{i}T_{1}^{(2)}=\frac{\left( -\mathrm{i}\kappa \right) ^2}{2!}\int{\mathrm{d}^4x\mathrm{d}^4y\mathrm{N[}\phi (x)\bar{\psi}(x)\psi (x)\phi (y)\bar{\psi}(y)\psi (y)]}
\end{equation}

一次
\begin{equation}
    \begin{aligned}
        \mathrm{i}T_{2}^{(2)}&=\wick{ \frac{\left( -\mathrm{i}\kappa \right) ^2}{2!}\int{\mathrm{d}^4x\mathrm{d}^4y\mathrm{N} [ \c{\phi}(x)\bar{\psi}(x)\psi(x)\c{\phi}(y)\bar{\psi}(y)\psi (y)]} }
\\
\mathrm{i}T_{3}^{(2)}&=\frac{\left( -\mathrm{i}\kappa \right) ^2}{2!}\int{\mathrm{d}^4x\mathrm{d}^4y\mathrm{N} [ \phi (x)\bar{\psi}(x)\psi (x)\phi (y)\bar{\psi}(y)\psi (y)]}
\\
\mathrm{i}T_{4}^{(2)}&=\frac{\left( -\mathrm{i}\kappa \right) ^2}{2!}\int{\mathrm{d}^4x\mathrm{d}^4y\mathrm{N} [ \phi (x)\bar{\psi}(x)\psi (x)\phi (y)\bar{\psi}(y)\psi (y)]}
\\
\mathrm{i}T_{5}^{(2)}&=\frac{\left( -\mathrm{i}\kappa \right) ^2}{2!}\int{\mathrm{d}^4x\mathrm{d}^4y\mathrm{N} [ \phi (x)\bar{\psi}(x)\psi (x)\phi (y)\bar{\psi}(y)\psi (y)]}
\\
\mathrm{i}T_{6}^{(2)}&=\frac{\left( -\mathrm{i}\kappa \right) ^2}{2!}\int{\mathrm{d}^4x\mathrm{d}^4y\mathrm{N} [ \phi (x)\bar{\psi}(x)\psi (x)\phi (y)\bar{\psi}(y)\psi (y)]}
    \end{aligned}
\end{equation}

二次缩并





三次缩并




、




























\section{Feynman 图}


%%%%%%%%%%%%%%%%%%%%%%%%%%%%%%%%%%%%%%%%%%%%%%%%%%%%%%%%%%%%%%
\subsection{Yukawa 理论}

















补充推导:

(7.21)
\begin{equation}
    \begin{aligned}
     \langle 0|\mathrm{i}T_{1}^{(1)}|\mathbf{p}^+,\lambda ;\mathbf{q}^-,\lambda ^{\prime};\mathbf{k}\rangle &=-\mathrm{i}\kappa \int{\mathrm{d}^4x}\langle 0|\mathrm{N} [ \phi (x)\bar{\psi}(x)\psi (x) ] |\mathbf{p}^+,\lambda ;\mathbf{q}^-,\lambda ^{\prime};\mathbf{k}\rangle 
\\
\text{缩并}&=\wick{-\mathrm{i}\kappa \int{\mathrm{d}^4x}\langle 0|\mathrm{N} [ \c3{\phi}(x)\c2{\bar{\psi}}(x)\c1{\psi}(x) ] |\c1{\mathbf{p}}^+,\lambda ;\c2{\mathbf{q}}^-,\lambda ^{\prime};\c3{\mathbf{k}}\rangle } 
\\
\text{调整}&=\wick{-\mathrm{i}\kappa \int{\mathrm{d}^4x}\langle 0|\mathrm{N} [ \c3{\phi}(x)\c2{\bar{\psi}}_a(x)\c1{\psi}_a(x) ] |\c1{\mathbf{p}}^+,\lambda ;\c2{\mathbf{q}}^-,\lambda ^{\prime};\c3{\mathbf{k}}\rangle }
\\
&=-\mathrm{i}\kappa \int{\mathrm{d}^4x}\langle 0|\phi ^{(+)}(x)\bar{\psi}_{a}^{(+)}(x)\psi _{a}^{(+)}(x)|\mathbf{p}^+,\lambda ;\mathbf{q}^-,\lambda ^{\prime};\mathbf{k}\rangle 
\\
&=-\mathrm{i}\kappa \int{\mathrm{d}^4x}\langle 0|\mathrm{e}^{-\mathrm{i}k\cdot x}\bar{v}_a(\mathbf{q},\lambda ^{\prime})\mathrm{e}^{-\mathrm{i}q\cdot x}u_a(\mathbf{p},\lambda )\mathrm{e}^{-\mathrm{i}p\cdot x}|0\rangle 
\\
\text{忽略下标}a&=-\mathrm{i}\kappa \int{\mathrm{d}^4x}\bar{v}_a(\mathbf{q},\lambda ^{\prime})u_a(\mathbf{p},\lambda )\mathrm{e}^{-\mathrm{i}\left( p+q+k \right) \cdot x}\langle 0|0\rangle 
\\
&=-\mathrm{i}\kappa \int{\mathrm{d}^4x}\bar{v}(\mathbf{q},\lambda ^{\prime})u(\mathbf{p},\lambda )\mathrm{e}^{-\mathrm{i}\left( p+q+k \right) \cdot x}
\\
\text{带一横在前}&=-\mathrm{i}\kappa \bar{v}(\mathbf{q},\lambda ^{\prime})u(\mathbf{p},\lambda )\left( 2\pi \right) ^4\delta ^{(4)}(p+q+k)
    \end{aligned}
\end{equation}





(7.28)
\begin{equation}
    \begin{aligned}
       \langle \mathbf{p}^+,\lambda ;\mathbf{q}^-,\lambda ^{\prime};\mathbf{k}|\mathrm{i}T_{1}^{(1)}|0\rangle &=-\mathrm{i}\kappa \int{\mathrm{d}^4}x\langle \mathbf{p}^+,\lambda ;\mathbf{q}^-,\lambda ^{\prime};\mathbf{k}|\mathrm{N}[ \phi (x)\bar{\psi}(x)\psi (x) ] |0\rangle 
\\
\text{缩并}&=\wick{-\mathrm{i}\kappa \int{\mathrm{d}^4}x\langle \c3{\mathbf{p}}^+,\lambda ;\c2{\mathbf{q}}^-,\lambda ^{\prime};\c1{\mathbf{k}}|\mathrm{N}[ \c1{\phi}(x)\c3{\bar{\psi}}(x)\c2{\psi}(x) ] |0\rangle }
\\
\text{调整}&=\wick{{\color[RGB]{240, 0, 0} +}\mathrm{i}\kappa \int{\mathrm{d}^4}x\langle \c3{\mathbf{p}}^+,\lambda ;\c2{\mathbf{q}}^-,\lambda ^{\prime};\c1{\mathbf{k}}|\mathrm{N}[ \c1{\phi}(x)\c2{\psi}_a(x)\c3{\bar{\psi}}_a(x) ] |0\rangle }
\\
&=+\mathrm{i}\kappa \int{\mathrm{d}^4}x\langle \mathbf{p}^+,\lambda ;\mathbf{q}^-,\lambda ^{\prime};\mathbf{k}|\phi ^{(-)}(x)\psi _{a}^{(-)}(x)\bar{\psi}_{a}^{(-)}(x)|0\rangle 
\\
&=+\mathrm{i}\kappa \int{\mathrm{d}^4}x\langle 0|\mathrm{e}^{\mathrm{i}k\cdot x}v_a(\mathbf{q},\lambda ^{\prime})\mathrm{e}^{\mathrm{i}q\cdot x}\bar{u}_a(\mathbf{p},\lambda )\mathrm{e}^{\mathrm{i}p\cdot x}|0\rangle 
\\
&=+\mathrm{i}\kappa \int{\mathrm{d}^4}xv_a(\mathbf{q},\lambda ^{\prime})\bar{u}_a(\mathbf{p},\lambda )\mathrm{e}^{\mathrm{i}\left( p+q+k \right) \cdot x}\langle 0|0\rangle 
\\
\text{忽略下标,调整横在前}&=+\mathrm{i}\kappa \int{\mathrm{d}^4}x\bar{u}(\mathbf{p},\lambda )v(\mathbf{q},\lambda ^{\prime})\mathrm{e}^{\mathrm{i}\left( p+q+k \right) \cdot x}
\\
\delta \text{函数性质}&=+\mathrm{i}\kappa \bar{u}(\mathbf{p},\lambda )v(\mathbf{q},\lambda ^{\prime})\left( 2\pi \right) ^4\delta ^{(4)}(p+q+k)
    \end{aligned}
\end{equation}

(7.32)b
\begin{equation}
    \begin{aligned}
        \langle \mathbf{p}^+,\lambda ;\mathbf{q}^-,\lambda ^{\prime}|\mathrm{i}T_{1}^{(1)}|\mathbf{k}\rangle &=-\mathrm{i}\kappa \int{\mathrm{d}^4}x\langle \mathbf{p}^+,\lambda ;\mathbf{q}^-,\lambda ^{\prime}|\mathrm{N} [ \phi (x)\bar{\psi}(x)\psi (x) ] |\mathbf{k}\rangle 
\\
\text{缩并}&=\wick{-\mathrm{i}\kappa \int{\mathrm{d}^4}x\langle \c1{\mathbf{p}}^+,\lambda ;\c2{\mathbf{q}}^-,\lambda ^{\prime}|\mathrm{N} [\c3{\phi}(x)\c1{\bar{\psi}}(x)\c2{\psi}(x) ] |\c3{\mathbf{k}}\rangle }
\\
\text{调整}&=\wick{+\mathrm{i}\kappa \int{\mathrm{d}^4}x\langle \c2{\mathbf{p}}^+,\lambda ;\c1{\mathbf{q}}^-,\lambda ^{\prime}|\mathrm{N} [ \c1{\psi}_a(x)\c2{\bar{\psi}}_a(x)\c3{\phi}(x) ] |\c3{\mathbf{k}}\rangle }
\\
\text{去正规乘积}&=+\mathrm{i}\kappa \int{\mathrm{d}^4}x\langle \mathbf{p}^+,\lambda ;\mathbf{q}^-,\lambda ^{\prime}|\psi _{a}^{(-)}(x)\bar{\psi}_{a}^{(-)}(x)\phi ^{(+)}(x)|\mathbf{k}\rangle 
\\
&=+\mathrm{i}\kappa \int{\mathrm{d}^4}x\langle 0|v_a(\mathbf{q},\lambda ^{\prime})\mathrm{e}^{\mathrm{i}q\cdot x}\bar{u}_a(\mathbf{p},\lambda )\mathrm{e}^{\mathrm{i}p\cdot x}\mathrm{e}^{-\mathrm{i}k\cdot x}|0\rangle 
\\
&=+\mathrm{i}\kappa \int{\mathrm{d}^4}xv_a(\mathbf{q},\lambda ^{\prime})\bar{u}_a(\mathbf{p},\lambda )\mathrm{e}^{-\mathrm{i}\left( k-p-q \right) \cdot x}\langle 0|0\rangle 
\\
&=+\mathrm{i}\kappa \int{\mathrm{d}^4}x\bar{u}(\mathbf{p},\lambda )v(\mathbf{q},\lambda ^{\prime})\mathrm{e}^{-\mathrm{i}\left( k-p-q \right) \cdot x}
\\
\text{有一横在前面}&=+\mathrm{i}\kappa \bar{u}(\mathbf{p},\lambda )v(\mathbf{q},\lambda ^{\prime})\left( 2\pi \right) ^4\delta ^{(4)}(k-p-q)
    \end{aligned}
\end{equation}

(7.33)c
\begin{equation}
    \begin{aligned}
        \langle \mathbf{q}^+,\lambda ^{\prime};\mathbf{k}|\mathrm{i}T_{1}^{(1)}|\mathbf{p}^+,\lambda \rangle &=-\mathrm{i}\kappa \int{\mathrm{d}^4}x\langle \mathbf{q}^+,\lambda ^{\prime};\mathbf{k}|\mathrm{N} [ \phi (x)\bar{\psi}(x)\psi (x) ] |\mathbf{p}^+,\lambda \rangle 
\\
\text{缩并}&=\wick{-\mathrm{i}\kappa \int{\mathrm{d}^4}x\langle \c2{\mathbf{q}}^+,\lambda ^{\prime};\c1{\mathbf{k}}|\mathrm{N} [ \c1{\phi}(x)\c2{\bar{\psi}}(x)\c3{\psi}(x) ] |\c3{\mathbf{p}}^+,\lambda \rangle }
\\
\text{整理}&=\wick{-\mathrm{i}\kappa \int{\mathrm{d}^4}x\langle \c2{\mathbf{q}}^+,\lambda ^{\prime};\c1{\mathbf{k}}|\mathrm{N} [\c1{\phi}(x)\c2{\bar{\psi}}_a(x)\c3{\psi}_a(x) ] |\c3{\mathbf{p}}^+,\lambda \rangle }
\\
\text{去正规乘积}&=-\mathrm{i}\kappa \int{\mathrm{d}^4}x\langle \mathbf{q}^+,\lambda ^{\prime};\mathbf{k}|\phi ^{(-)}(x)\bar{\psi}_{a}^{(-)}(x)\psi _{a}^{(+)}(x)|\mathbf{p}^+,\lambda \rangle 
\\
&=-\mathrm{i}\kappa \int{\mathrm{d}^4}x\langle 0|\mathrm{e}^{\mathrm{i}k\cdot x}\bar{u}_a(\mathbf{q},\lambda ^{\prime})\mathrm{e}^{\mathrm{i}q\cdot x}u_a(\mathbf{p},\lambda )\mathrm{e}^{-\mathrm{i}p\cdot x}|0\rangle 
\\
&=-\mathrm{i}\kappa \int{\mathrm{d}^4}x\bar{u}_a(\mathbf{q},\lambda ^{\prime})u_a(\mathbf{p},\lambda )\mathrm{e}^{-\mathrm{i}\left( p-q-k \right) \cdot x}\langle 0|0\rangle 
\\
&=-\mathrm{i}\kappa \int{\mathrm{d}^4}x\bar{u}(\mathbf{q},\lambda ^{\prime})u(\mathbf{p},\lambda )\mathrm{e}^{-\mathrm{i}\left( p-q-k \right) \cdot x}
\\
&=-\mathrm{i}\kappa \bar{u}(\mathbf{q},\lambda ^{\prime})u(\mathbf{p},\lambda )\left( 2\pi \right) ^4\delta ^{(4)}(p-q-k)
    \end{aligned}
\end{equation}



(7.34)d
\begin{equation}
    \begin{aligned}
        \langle \mathbf{q}^-,\lambda ^{\prime};\mathbf{k}|\mathrm{i}T_{1}^{(1)}|\mathbf{p}^-,\lambda \rangle &=-\mathrm{i}\kappa \int{\mathrm{d}^4x}\langle \mathbf{q}^-,\lambda ^{\prime};\mathbf{k}|\mathrm{N} [ \phi (x)\bar{\psi}_a(x)\psi _a(x) ] |\mathbf{p}^-,\lambda \rangle 
\\
\text{缩并}&=\wick{-\mathrm{i}\kappa \int{\mathrm{d}^4}x\langle \c3{\mathbf{q}}^-,\lambda ^{\prime};\c1{\mathbf{k}}|\mathrm{N} [ \c1{\phi}(x)\c2{\bar{\psi}}(x)\c3{\psi}(x) ] |\c2{\mathbf{p}}^-,\lambda \rangle }
\\
\text{调整}&=\wick{+\mathrm{i}\kappa \int{\mathrm{d}^4}x\langle \c2{\mathbf{q}}^-,\lambda ^{\prime};\c1{\mathbf{k}}|\mathrm{N} [ \c1{\phi}(x)\c2{\bar{\psi}}_a(x)\c3{\psi}_a(x) ] |\c3{\mathbf{p}}^-,\lambda \rangle }
\\
&=+\mathrm{i}\kappa \int{\mathrm{d}^4}x\langle \mathbf{q}^-,\lambda ^{\prime};\mathbf{k}|\phi ^{(-)}(x)\psi _{a}^{(-)}(x)\bar{\psi}_{a}^{(+)}(x)|\mathbf{p}^-,\lambda \rangle 
\\
&=+\mathrm{i}\kappa \int{\mathrm{d}^4}x\langle 0|\mathrm{e}^{\mathrm{i}k\cdot x}v_a(\mathbf{q},\lambda ^{\prime})\mathrm{e}^{\mathrm{i}q\cdot x}\bar{v}_a(\mathbf{p},\lambda )\mathrm{e}^{-\mathrm{i}p\cdot x}|0\rangle 
\\
&=+\mathrm{i}\kappa \int{\mathrm{d}^4}xv_a(\mathbf{q},\lambda ^{\prime})\bar{v}_a(\mathbf{p},\lambda )\mathrm{e}^{-\mathrm{i}\left( p-q-k \right) \cdot x}\langle 0|0\rangle 
\\
&=+\mathrm{i}\kappa \int{\mathrm{d}^4}x\bar{v}(\mathbf{p},\lambda )v(\mathbf{q},\lambda ^{\prime})\mathrm{e}^{-\mathrm{i}\left( p-q-k \right) \cdot x}
\\
&=+\mathrm{i}\kappa \bar{v}(\mathbf{p},\lambda )v(\mathbf{q},\lambda ^{\prime})\left( 2\pi \right) ^4\delta ^{(4)}(p-q-k)
    \end{aligned}
\end{equation}

(7.35)f
\begin{equation}
    \begin{aligned}
        \langle \mathbf{k}|\mathrm{i}T_{1}^{(1)}|\mathbf{p}^+,\lambda ;\mathbf{q}^-,\lambda ^{\prime}\rangle &=-\mathrm{i}\kappa \int{\mathrm{d}^4}x\langle \mathbf{k}|\mathrm{N}[ \phi (x)\bar{\psi}(x)\psi (x) ] |\mathbf{p}^+,\lambda ;\mathbf{q}^-,\lambda ^{\prime}\rangle 
\\
\text{缩并}&=\wick{-\mathrm{i}\kappa \int{\mathrm{d}^4}x\langle \c1{\mathbf{k}}|\mathrm{N} [ \c1{\phi}(x)\c3{\bar{\psi}}(x)\c2{\psi}(x) ] |\c2{\mathbf{p}}^+,\lambda ;\c3{\mathbf{q}}^-,\lambda ^{\prime}\rangle }
\\
\text{调整}&=\wick{-\mathrm{i}\kappa \int{\mathrm{d}^4}x\langle \c1{\mathbf{k}}|\mathrm{N} [ \c1{\phi}(x)\c3{\bar{\psi}}_a(x)\c2{\psi}_a(x) ] |\c2{\mathbf{p}}^+,\lambda ;\c3{\mathbf{q}}^-,\lambda ^{\prime}\rangle }
\\
\text{去正规乘积}&=-\mathrm{i}\kappa \int{\mathrm{d}^4}x\langle \mathbf{k}|\phi ^{(-)}(x)\bar{\psi}_{a}^{(+)}(x)\psi _{a}^{(+)}(x)|\mathbf{p}^+,\lambda ;\mathbf{q}^-,\lambda ^{\prime}\rangle 
\\
&=-\mathrm{i}\kappa \int{\mathrm{d}^4}x\langle 0|\mathrm{e}^{\mathrm{i}k\cdot x}\bar{v}_a(\mathbf{q},\lambda ^{\prime})\mathrm{e}^{-\mathrm{i}q\cdot x}u_a(\mathbf{p},\lambda )\mathrm{e}^{-\mathrm{i}p\cdot x}|0\rangle 
\\
&=-\mathrm{i}\kappa \int{\mathrm{d}^4}x\bar{v}_a(\mathbf{q},\lambda ^{\prime})u_a(\mathbf{p},\lambda )\mathrm{e}^{-\mathrm{i}\left( p+q-k \right) \cdot x}\langle 0|0\rangle 
\\
&=-\mathrm{i}\kappa \int{\mathrm{d}^4}x\bar{v}(\mathbf{q},\lambda ^{\prime})u(\mathbf{p},\lambda )\mathrm{e}^{-\mathrm{i}\left( p+q-k \right) \cdot x}
\\
\delta \text{函数性质}&=-\mathrm{i}\kappa \bar{v}(\mathbf{q},\lambda ^{\prime})u(\mathbf{p},\lambda )\left( 2\pi \right) ^4\delta ^{(4)}(p+q-k)
    \end{aligned}
\end{equation}


(7.36)g
\begin{equation}
    \begin{aligned}
        \langle \mathbf{q}^+,\lambda ^{\prime}|\mathrm{i}T_{1}^{(1)}|\mathbf{p}^+,\lambda ;\mathbf{k}\rangle &=-\mathrm{i}\kappa \int{\mathrm{d}^4}x\langle \mathbf{q}^+,\lambda ^{\prime}|\mathrm{N}[ \phi (x)\bar{\psi}(x)\psi (x) ] |\mathbf{p}^+,\lambda ;\mathbf{k}\rangle 
\\
\text{缩并}&=\wick{-\mathrm{i}\kappa \int{\mathrm{d}^4}x\langle \c1{\mathbf{q}}^+,\lambda ^{\prime}|\mathrm{N}[ \c3{\phi}(x)\c1{\bar{\psi}}(x)\c2{\psi}(x) ] |\c2{\mathbf{p}}^+,\lambda ;\c3{\mathbf{k}}\rangle }
\\
\text{调整}&=\wick{-\mathrm{i}\kappa \int{\mathrm{d}^4}x\langle \c1{\mathbf{q}}^+,\lambda ^{\prime}|\mathrm{N}[ \c1{\bar{\psi}}_a(x)\c3{\phi}(x)\c2{\psi}_a(x) ] |\c2{\mathbf{p}}^+,\lambda ;\c3{\mathbf{k}}\rangle }
\\
\text{去正规乘积}&=-\mathrm{i}\kappa \int{\mathrm{d}^4}x\langle \mathbf{q}^+,\lambda ^{\prime}|\bar{\psi}_{a}^{(-)}(x)\phi ^{(+)}(x)\psi _{a}^{(+)}(x)|\mathbf{p}^+,\lambda ;\mathbf{k}\rangle 
\\
&=-\mathrm{i}\kappa \int{\mathrm{d}^4}x\langle 0|\bar{u}_a(\mathbf{q},\lambda ^{\prime})\mathrm{e}^{\mathrm{i}q\cdot x}\mathrm{e}^{-\mathrm{i}k\cdot x}u_a(\mathbf{p},\lambda )\mathrm{e}^{-\mathrm{i}p\cdot x}|0\rangle 
\\
&=-\mathrm{i}\kappa \int{\mathrm{d}^4}x\bar{u}_a(\mathbf{q},\lambda ^{\prime})u_a(\mathbf{p},\lambda )\mathrm{e}^{-\mathrm{i}\left( p+k-q \right) \cdot x}\langle 0|0\rangle 
\\
&=-\mathrm{i}\kappa \int{\mathrm{d}^4}x\bar{u}(\mathbf{q},\lambda ^{\prime})u(\mathbf{p},\lambda )\mathrm{e}^{-\mathrm{i}\left( p+k-q \right) \cdot x}
\\
\text{函数性质}&=-\mathrm{i}\kappa \bar{u}(\mathbf{q},\lambda ^{\prime})u(\mathbf{p},\lambda )\left( 2\pi \right) ^4\delta ^{(4)}(p+k-q)
    \end{aligned}
\end{equation}


(7.37)h
\begin{equation}
    \begin{aligned}
        \langle \mathbf{q}^-,\lambda ^{\prime}|\mathrm{i}T_{1}^{(1)}|\mathbf{p}^-,\lambda ;\mathbf{k}\rangle &=-\mathrm{i}\kappa \int{\mathrm{d}^4}x\langle \mathbf{q}^-,\lambda ^{\prime}|\mathrm{N}[ \phi (x)\bar{\psi}(x)\psi (x) ] |\mathbf{p}^-,\lambda ;\mathbf{k}\rangle 
\\
\text{缩并}&=\wick{-\mathrm{i}\kappa \int{\mathrm{d}^4}x\langle \c1{\mathbf{q}}^-,\lambda ^{\prime}|\mathrm{N}[ \c2{\phi}(x)\c3{\bar{\psi}}(x)\c1{\psi}(x) ] |\c3{\mathbf{p}}^-,\lambda ;\c2{\mathbf{k}}\rangle }
\\
\text{调整}&=\wick{+\mathrm{i}\kappa \int{\mathrm{d}^4}x\langle \c1{\mathbf{q}}^-,\lambda ^{\prime}|\mathrm{N} [ \c1{\psi}_a(x)\c3{\phi}(x)\c2{\bar{\psi}}_a(x) ] |\c2{\mathbf{p}}^-,\lambda ;\c3{\mathbf{k}}\rangle }
\\
\text{正负能解}&=+\mathrm{i}\kappa \int{\mathrm{d}^4}x\langle \mathbf{q}^-,\lambda ^{\prime}|\psi _{a}^{(-)}(x)\phi (x)\bar{\psi}_{a}^{(+)}(x)|\mathbf{p}^-,\lambda ;\mathbf{k}\rangle 
\\
&=+\mathrm{i}\kappa \int{\mathrm{d}^4}x\langle 0|v_a(\mathbf{q},\lambda ^{\prime})\mathrm{e}^{\mathrm{i}q\cdot x}\mathrm{e}^{-\mathrm{i}k\cdot x}\bar{v}_a(\mathbf{p},\lambda )\mathrm{e}^{-\mathrm{i}p\cdot x}|0\rangle 
\\
&=+\mathrm{i}\kappa \int{\mathrm{d}^4}x\bar{v}_a(\mathbf{p},\lambda )v_a(\mathbf{q},\lambda ^{\prime})\mathrm{e}^{-\mathrm{i}\left( p+k-q \right) \cdot x}\langle 0|0\rangle 
\\
\text{横在前}&=+\mathrm{i}\kappa \int{\mathrm{d}^4}x\bar{v}(\mathbf{p},\lambda )v(\mathbf{q},\lambda ^{\prime})\mathrm{e}^{-\mathrm{i}\left( p+k-q \right) \cdot x}
\\
\text{函数性质}&=+\mathrm{i}\kappa \bar{v}(\mathbf{p},\lambda )v(\mathbf{q},\lambda ^{\prime})\left( 2\pi \right) ^4\delta ^{(4)}(p+k-q)
    \end{aligned}
\end{equation}


\newpage




\newpage
(7.58)无缩并
\begin{equation}
    \mathrm{i}T_{1}^{(2)}=\frac{\left( -\mathrm{i}\kappa \right) ^2}{2!}\int{\mathrm{d}^4x\mathrm{d}^4y\mathrm{N[}\phi (x)\bar{\psi}(x)\psi (x)\phi (y)\bar{\psi}(y)\psi (y)]}
\end{equation}

一次
\begin{equation}
    \begin{aligned}
        \mathrm{i}T_{2}^{(2)}&=\wick{ \frac{\left( -\mathrm{i}\kappa \right) ^2}{2!}\int{\mathrm{d}^4x\mathrm{d}^4y\mathrm{N} [ \c{\phi}(x)\bar{\psi}(x)\psi(x)\c{\phi}(y)\bar{\psi}(y)\psi (y)]} }
\\
\mathrm{i}T_{3}^{(2)}&=\frac{\left( -\mathrm{i}\kappa \right) ^2}{2!}\int{\mathrm{d}^4x\mathrm{d}^4y\mathrm{N} [ \phi (x)\bar{\psi}(x)\psi (x)\phi (y)\bar{\psi}(y)\psi (y)]}
\\
\mathrm{i}T_{4}^{(2)}&=\frac{\left( -\mathrm{i}\kappa \right) ^2}{2!}\int{\mathrm{d}^4x\mathrm{d}^4y\mathrm{N} [ \phi (x)\bar{\psi}(x)\psi (x)\phi (y)\bar{\psi}(y)\psi (y)]}
\\
\mathrm{i}T_{5}^{(2)}&=\frac{\left( -\mathrm{i}\kappa \right) ^2}{2!}\int{\mathrm{d}^4x\mathrm{d}^4y\mathrm{N} [ \phi (x)\bar{\psi}(x)\psi (x)\phi (y)\bar{\psi}(y)\psi (y)]}
\\
\mathrm{i}T_{6}^{(2)}&=\frac{\left( -\mathrm{i}\kappa \right) ^2}{2!}\int{\mathrm{d}^4x\mathrm{d}^4y\mathrm{N} [ \phi (x)\bar{\psi}(x)\psi (x)\phi (y)\bar{\psi}(y)\psi (y)]}
    \end{aligned}
\end{equation}

二次缩并





三次缩并




、




























\section{Feynman 图}


%%%%%%%%%%%%%%%%%%%%%%%%%%%%%%%%%%%%%%%%%%%%%%%%%%%%%%%%%%%%%%
\subsection{Yukawa 理论}

















补充推导:

(7.21)
\begin{equation}
    \begin{aligned}
     \langle 0|\mathrm{i}T_{1}^{(1)}|\mathbf{p}^+,\lambda ;\mathbf{q}^-,\lambda ^{\prime};\mathbf{k}\rangle &=-\mathrm{i}\kappa \int{\mathrm{d}^4x}\langle 0|\mathrm{N} [ \phi (x)\bar{\psi}(x)\psi (x) ] |\mathbf{p}^+,\lambda ;\mathbf{q}^-,\lambda ^{\prime};\mathbf{k}\rangle 
\\
\text{缩并}&=\wick{-\mathrm{i}\kappa \int{\mathrm{d}^4x}\langle 0|\mathrm{N} [ \c3{\phi}(x)\c2{\bar{\psi}}(x)\c1{\psi}(x) ] |\c1{\mathbf{p}}^+,\lambda ;\c2{\mathbf{q}}^-,\lambda ^{\prime};\c3{\mathbf{k}}\rangle } 
\\
\text{调整}&=\wick{-\mathrm{i}\kappa \int{\mathrm{d}^4x}\langle 0|\mathrm{N} [ \c3{\phi}(x)\c2{\bar{\psi}}_a(x)\c1{\psi}_a(x) ] |\c1{\mathbf{p}}^+,\lambda ;\c2{\mathbf{q}}^-,\lambda ^{\prime};\c3{\mathbf{k}}\rangle }
\\
&=-\mathrm{i}\kappa \int{\mathrm{d}^4x}\langle 0|\phi ^{(+)}(x)\bar{\psi}_{a}^{(+)}(x)\psi _{a}^{(+)}(x)|\mathbf{p}^+,\lambda ;\mathbf{q}^-,\lambda ^{\prime};\mathbf{k}\rangle 
\\
&=-\mathrm{i}\kappa \int{\mathrm{d}^4x}\langle 0|\mathrm{e}^{-\mathrm{i}k\cdot x}\bar{v}_a(\mathbf{q},\lambda ^{\prime})\mathrm{e}^{-\mathrm{i}q\cdot x}u_a(\mathbf{p},\lambda )\mathrm{e}^{-\mathrm{i}p\cdot x}|0\rangle 
\\
\text{忽略下标}a&=-\mathrm{i}\kappa \int{\mathrm{d}^4x}\bar{v}_a(\mathbf{q},\lambda ^{\prime})u_a(\mathbf{p},\lambda )\mathrm{e}^{-\mathrm{i}\left( p+q+k \right) \cdot x}\langle 0|0\rangle 
\\
&=-\mathrm{i}\kappa \int{\mathrm{d}^4x}\bar{v}(\mathbf{q},\lambda ^{\prime})u(\mathbf{p},\lambda )\mathrm{e}^{-\mathrm{i}\left( p+q+k \right) \cdot x}
\\
\text{带一横在前}&=-\mathrm{i}\kappa \bar{v}(\mathbf{q},\lambda ^{\prime})u(\mathbf{p},\lambda )\left( 2\pi \right) ^4\delta ^{(4)}(p+q+k)
    \end{aligned}
\end{equation}





(7.28)
\begin{equation}
    \begin{aligned}
       \langle \mathbf{p}^+,\lambda ;\mathbf{q}^-,\lambda ^{\prime};\mathbf{k}|\mathrm{i}T_{1}^{(1)}|0\rangle &=-\mathrm{i}\kappa \int{\mathrm{d}^4}x\langle \mathbf{p}^+,\lambda ;\mathbf{q}^-,\lambda ^{\prime};\mathbf{k}|\mathrm{N}[ \phi (x)\bar{\psi}(x)\psi (x) ] |0\rangle 
\\
\text{缩并}&=\wick{-\mathrm{i}\kappa \int{\mathrm{d}^4}x\langle \c3{\mathbf{p}}^+,\lambda ;\c2{\mathbf{q}}^-,\lambda ^{\prime};\c1{\mathbf{k}}|\mathrm{N}[ \c1{\phi}(x)\c3{\bar{\psi}}(x)\c2{\psi}(x) ] |0\rangle }
\\
\text{调整}&=\wick{{\color[RGB]{240, 0, 0} +}\mathrm{i}\kappa \int{\mathrm{d}^4}x\langle \c3{\mathbf{p}}^+,\lambda ;\c2{\mathbf{q}}^-,\lambda ^{\prime};\c1{\mathbf{k}}|\mathrm{N}[ \c1{\phi}(x)\c2{\psi}_a(x)\c3{\bar{\psi}}_a(x) ] |0\rangle }
\\
&=+\mathrm{i}\kappa \int{\mathrm{d}^4}x\langle \mathbf{p}^+,\lambda ;\mathbf{q}^-,\lambda ^{\prime};\mathbf{k}|\phi ^{(-)}(x)\psi _{a}^{(-)}(x)\bar{\psi}_{a}^{(-)}(x)|0\rangle 
\\
&=+\mathrm{i}\kappa \int{\mathrm{d}^4}x\langle 0|\mathrm{e}^{\mathrm{i}k\cdot x}v_a(\mathbf{q},\lambda ^{\prime})\mathrm{e}^{\mathrm{i}q\cdot x}\bar{u}_a(\mathbf{p},\lambda )\mathrm{e}^{\mathrm{i}p\cdot x}|0\rangle 
\\
&=+\mathrm{i}\kappa \int{\mathrm{d}^4}xv_a(\mathbf{q},\lambda ^{\prime})\bar{u}_a(\mathbf{p},\lambda )\mathrm{e}^{\mathrm{i}\left( p+q+k \right) \cdot x}\langle 0|0\rangle 
\\
\text{忽略下标,调整横在前}&=+\mathrm{i}\kappa \int{\mathrm{d}^4}x\bar{u}(\mathbf{p},\lambda )v(\mathbf{q},\lambda ^{\prime})\mathrm{e}^{\mathrm{i}\left( p+q+k \right) \cdot x}
\\
\delta \text{函数性质}&=+\mathrm{i}\kappa \bar{u}(\mathbf{p},\lambda )v(\mathbf{q},\lambda ^{\prime})\left( 2\pi \right) ^4\delta ^{(4)}(p+q+k)
    \end{aligned}
\end{equation}

(7.32)b
\begin{equation}
    \begin{aligned}
        \langle \mathbf{p}^+,\lambda ;\mathbf{q}^-,\lambda ^{\prime}|\mathrm{i}T_{1}^{(1)}|\mathbf{k}\rangle &=-\mathrm{i}\kappa \int{\mathrm{d}^4}x\langle \mathbf{p}^+,\lambda ;\mathbf{q}^-,\lambda ^{\prime}|\mathrm{N} [ \phi (x)\bar{\psi}(x)\psi (x) ] |\mathbf{k}\rangle 
\\
\text{缩并}&=\wick{-\mathrm{i}\kappa \int{\mathrm{d}^4}x\langle \c1{\mathbf{p}}^+,\lambda ;\c2{\mathbf{q}}^-,\lambda ^{\prime}|\mathrm{N} [\c3{\phi}(x)\c1{\bar{\psi}}(x)\c2{\psi}(x) ] |\c3{\mathbf{k}}\rangle }
\\
\text{调整}&=\wick{+\mathrm{i}\kappa \int{\mathrm{d}^4}x\langle \c2{\mathbf{p}}^+,\lambda ;\c1{\mathbf{q}}^-,\lambda ^{\prime}|\mathrm{N} [ \c1{\psi}_a(x)\c2{\bar{\psi}}_a(x)\c3{\phi}(x) ] |\c3{\mathbf{k}}\rangle }
\\
\text{去正规乘积}&=+\mathrm{i}\kappa \int{\mathrm{d}^4}x\langle \mathbf{p}^+,\lambda ;\mathbf{q}^-,\lambda ^{\prime}|\psi _{a}^{(-)}(x)\bar{\psi}_{a}^{(-)}(x)\phi ^{(+)}(x)|\mathbf{k}\rangle 
\\
&=+\mathrm{i}\kappa \int{\mathrm{d}^4}x\langle 0|v_a(\mathbf{q},\lambda ^{\prime})\mathrm{e}^{\mathrm{i}q\cdot x}\bar{u}_a(\mathbf{p},\lambda )\mathrm{e}^{\mathrm{i}p\cdot x}\mathrm{e}^{-\mathrm{i}k\cdot x}|0\rangle 
\\
&=+\mathrm{i}\kappa \int{\mathrm{d}^4}xv_a(\mathbf{q},\lambda ^{\prime})\bar{u}_a(\mathbf{p},\lambda )\mathrm{e}^{-\mathrm{i}\left( k-p-q \right) \cdot x}\langle 0|0\rangle 
\\
&=+\mathrm{i}\kappa \int{\mathrm{d}^4}x\bar{u}(\mathbf{p},\lambda )v(\mathbf{q},\lambda ^{\prime})\mathrm{e}^{-\mathrm{i}\left( k-p-q \right) \cdot x}
\\
\text{有一横在前面}&=+\mathrm{i}\kappa \bar{u}(\mathbf{p},\lambda )v(\mathbf{q},\lambda ^{\prime})\left( 2\pi \right) ^4\delta ^{(4)}(k-p-q)
    \end{aligned}
\end{equation}

(7.33)c
\begin{equation}
    \begin{aligned}
        \langle \mathbf{q}^+,\lambda ^{\prime};\mathbf{k}|\mathrm{i}T_{1}^{(1)}|\mathbf{p}^+,\lambda \rangle &=-\mathrm{i}\kappa \int{\mathrm{d}^4}x\langle \mathbf{q}^+,\lambda ^{\prime};\mathbf{k}|\mathrm{N} [ \phi (x)\bar{\psi}(x)\psi (x) ] |\mathbf{p}^+,\lambda \rangle 
\\
\text{缩并}&=\wick{-\mathrm{i}\kappa \int{\mathrm{d}^4}x\langle \c2{\mathbf{q}}^+,\lambda ^{\prime};\c1{\mathbf{k}}|\mathrm{N} [ \c1{\phi}(x)\c2{\bar{\psi}}(x)\c3{\psi}(x) ] |\c3{\mathbf{p}}^+,\lambda \rangle }
\\
\text{整理}&=\wick{-\mathrm{i}\kappa \int{\mathrm{d}^4}x\langle \c2{\mathbf{q}}^+,\lambda ^{\prime};\c1{\mathbf{k}}|\mathrm{N} [\c1{\phi}(x)\c2{\bar{\psi}}_a(x)\c3{\psi}_a(x) ] |\c3{\mathbf{p}}^+,\lambda \rangle }
\\
\text{去正规乘积}&=-\mathrm{i}\kappa \int{\mathrm{d}^4}x\langle \mathbf{q}^+,\lambda ^{\prime};\mathbf{k}|\phi ^{(-)}(x)\bar{\psi}_{a}^{(-)}(x)\psi _{a}^{(+)}(x)|\mathbf{p}^+,\lambda \rangle 
\\
&=-\mathrm{i}\kappa \int{\mathrm{d}^4}x\langle 0|\mathrm{e}^{\mathrm{i}k\cdot x}\bar{u}_a(\mathbf{q},\lambda ^{\prime})\mathrm{e}^{\mathrm{i}q\cdot x}u_a(\mathbf{p},\lambda )\mathrm{e}^{-\mathrm{i}p\cdot x}|0\rangle 
\\
&=-\mathrm{i}\kappa \int{\mathrm{d}^4}x\bar{u}_a(\mathbf{q},\lambda ^{\prime})u_a(\mathbf{p},\lambda )\mathrm{e}^{-\mathrm{i}\left( p-q-k \right) \cdot x}\langle 0|0\rangle 
\\
&=-\mathrm{i}\kappa \int{\mathrm{d}^4}x\bar{u}(\mathbf{q},\lambda ^{\prime})u(\mathbf{p},\lambda )\mathrm{e}^{-\mathrm{i}\left( p-q-k \right) \cdot x}
\\
&=-\mathrm{i}\kappa \bar{u}(\mathbf{q},\lambda ^{\prime})u(\mathbf{p},\lambda )\left( 2\pi \right) ^4\delta ^{(4)}(p-q-k)
    \end{aligned}
\end{equation}



(7.34)d
\begin{equation}
    \begin{aligned}
        \langle \mathbf{q}^-,\lambda ^{\prime};\mathbf{k}|\mathrm{i}T_{1}^{(1)}|\mathbf{p}^-,\lambda \rangle &=-\mathrm{i}\kappa \int{\mathrm{d}^4x}\langle \mathbf{q}^-,\lambda ^{\prime};\mathbf{k}|\mathrm{N} [ \phi (x)\bar{\psi}_a(x)\psi _a(x) ] |\mathbf{p}^-,\lambda \rangle 
\\
\text{缩并}&=\wick{-\mathrm{i}\kappa \int{\mathrm{d}^4}x\langle \c3{\mathbf{q}}^-,\lambda ^{\prime};\c1{\mathbf{k}}|\mathrm{N} [ \c1{\phi}(x)\c2{\bar{\psi}}(x)\c3{\psi}(x) ] |\c2{\mathbf{p}}^-,\lambda \rangle }
\\
\text{调整}&=\wick{+\mathrm{i}\kappa \int{\mathrm{d}^4}x\langle \c2{\mathbf{q}}^-,\lambda ^{\prime};\c1{\mathbf{k}}|\mathrm{N} [ \c1{\phi}(x)\c2{\bar{\psi}}_a(x)\c3{\psi}_a(x) ] |\c3{\mathbf{p}}^-,\lambda \rangle }
\\
&=+\mathrm{i}\kappa \int{\mathrm{d}^4}x\langle \mathbf{q}^-,\lambda ^{\prime};\mathbf{k}|\phi ^{(-)}(x)\psi _{a}^{(-)}(x)\bar{\psi}_{a}^{(+)}(x)|\mathbf{p}^-,\lambda \rangle 
\\
&=+\mathrm{i}\kappa \int{\mathrm{d}^4}x\langle 0|\mathrm{e}^{\mathrm{i}k\cdot x}v_a(\mathbf{q},\lambda ^{\prime})\mathrm{e}^{\mathrm{i}q\cdot x}\bar{v}_a(\mathbf{p},\lambda )\mathrm{e}^{-\mathrm{i}p\cdot x}|0\rangle 
\\
&=+\mathrm{i}\kappa \int{\mathrm{d}^4}xv_a(\mathbf{q},\lambda ^{\prime})\bar{v}_a(\mathbf{p},\lambda )\mathrm{e}^{-\mathrm{i}\left( p-q-k \right) \cdot x}\langle 0|0\rangle 
\\
&=+\mathrm{i}\kappa \int{\mathrm{d}^4}x\bar{v}(\mathbf{p},\lambda )v(\mathbf{q},\lambda ^{\prime})\mathrm{e}^{-\mathrm{i}\left( p-q-k \right) \cdot x}
\\
&=+\mathrm{i}\kappa \bar{v}(\mathbf{p},\lambda )v(\mathbf{q},\lambda ^{\prime})\left( 2\pi \right) ^4\delta ^{(4)}(p-q-k)
    \end{aligned}
\end{equation}

(7.35)f
\begin{equation}
    \begin{aligned}
        \langle \mathbf{k}|\mathrm{i}T_{1}^{(1)}|\mathbf{p}^+,\lambda ;\mathbf{q}^-,\lambda ^{\prime}\rangle &=-\mathrm{i}\kappa \int{\mathrm{d}^4}x\langle \mathbf{k}|\mathrm{N}[ \phi (x)\bar{\psi}(x)\psi (x) ] |\mathbf{p}^+,\lambda ;\mathbf{q}^-,\lambda ^{\prime}\rangle 
\\
\text{缩并}&=\wick{-\mathrm{i}\kappa \int{\mathrm{d}^4}x\langle \c1{\mathbf{k}}|\mathrm{N} [ \c1{\phi}(x)\c3{\bar{\psi}}(x)\c2{\psi}(x) ] |\c2{\mathbf{p}}^+,\lambda ;\c3{\mathbf{q}}^-,\lambda ^{\prime}\rangle }
\\
\text{调整}&=\wick{-\mathrm{i}\kappa \int{\mathrm{d}^4}x\langle \c1{\mathbf{k}}|\mathrm{N} [ \c1{\phi}(x)\c3{\bar{\psi}}_a(x)\c2{\psi}_a(x) ] |\c2{\mathbf{p}}^+,\lambda ;\c3{\mathbf{q}}^-,\lambda ^{\prime}\rangle }
\\
\text{去正规乘积}&=-\mathrm{i}\kappa \int{\mathrm{d}^4}x\langle \mathbf{k}|\phi ^{(-)}(x)\bar{\psi}_{a}^{(+)}(x)\psi _{a}^{(+)}(x)|\mathbf{p}^+,\lambda ;\mathbf{q}^-,\lambda ^{\prime}\rangle 
\\
&=-\mathrm{i}\kappa \int{\mathrm{d}^4}x\langle 0|\mathrm{e}^{\mathrm{i}k\cdot x}\bar{v}_a(\mathbf{q},\lambda ^{\prime})\mathrm{e}^{-\mathrm{i}q\cdot x}u_a(\mathbf{p},\lambda )\mathrm{e}^{-\mathrm{i}p\cdot x}|0\rangle 
\\
&=-\mathrm{i}\kappa \int{\mathrm{d}^4}x\bar{v}_a(\mathbf{q},\lambda ^{\prime})u_a(\mathbf{p},\lambda )\mathrm{e}^{-\mathrm{i}\left( p+q-k \right) \cdot x}\langle 0|0\rangle 
\\
&=-\mathrm{i}\kappa \int{\mathrm{d}^4}x\bar{v}(\mathbf{q},\lambda ^{\prime})u(\mathbf{p},\lambda )\mathrm{e}^{-\mathrm{i}\left( p+q-k \right) \cdot x}
\\
\delta \text{函数性质}&=-\mathrm{i}\kappa \bar{v}(\mathbf{q},\lambda ^{\prime})u(\mathbf{p},\lambda )\left( 2\pi \right) ^4\delta ^{(4)}(p+q-k)
    \end{aligned}
\end{equation}


(7.36)g
\begin{equation}
    \begin{aligned}
        \langle \mathbf{q}^+,\lambda ^{\prime}|\mathrm{i}T_{1}^{(1)}|\mathbf{p}^+,\lambda ;\mathbf{k}\rangle &=-\mathrm{i}\kappa \int{\mathrm{d}^4}x\langle \mathbf{q}^+,\lambda ^{\prime}|\mathrm{N}[ \phi (x)\bar{\psi}(x)\psi (x) ] |\mathbf{p}^+,\lambda ;\mathbf{k}\rangle 
\\
\text{缩并}&=\wick{-\mathrm{i}\kappa \int{\mathrm{d}^4}x\langle \c1{\mathbf{q}}^+,\lambda ^{\prime}|\mathrm{N}[ \c3{\phi}(x)\c1{\bar{\psi}}(x)\c2{\psi}(x) ] |\c2{\mathbf{p}}^+,\lambda ;\c3{\mathbf{k}}\rangle }
\\
\text{调整}&=\wick{-\mathrm{i}\kappa \int{\mathrm{d}^4}x\langle \c1{\mathbf{q}}^+,\lambda ^{\prime}|\mathrm{N}[ \c1{\bar{\psi}}_a(x)\c3{\phi}(x)\c2{\psi}_a(x) ] |\c2{\mathbf{p}}^+,\lambda ;\c3{\mathbf{k}}\rangle }
\\
\text{去正规乘积}&=-\mathrm{i}\kappa \int{\mathrm{d}^4}x\langle \mathbf{q}^+,\lambda ^{\prime}|\bar{\psi}_{a}^{(-)}(x)\phi ^{(+)}(x)\psi _{a}^{(+)}(x)|\mathbf{p}^+,\lambda ;\mathbf{k}\rangle 
\\
&=-\mathrm{i}\kappa \int{\mathrm{d}^4}x\langle 0|\bar{u}_a(\mathbf{q},\lambda ^{\prime})\mathrm{e}^{\mathrm{i}q\cdot x}\mathrm{e}^{-\mathrm{i}k\cdot x}u_a(\mathbf{p},\lambda )\mathrm{e}^{-\mathrm{i}p\cdot x}|0\rangle 
\\
&=-\mathrm{i}\kappa \int{\mathrm{d}^4}x\bar{u}_a(\mathbf{q},\lambda ^{\prime})u_a(\mathbf{p},\lambda )\mathrm{e}^{-\mathrm{i}\left( p+k-q \right) \cdot x}\langle 0|0\rangle 
\\
&=-\mathrm{i}\kappa \int{\mathrm{d}^4}x\bar{u}(\mathbf{q},\lambda ^{\prime})u(\mathbf{p},\lambda )\mathrm{e}^{-\mathrm{i}\left( p+k-q \right) \cdot x}
\\
\text{函数性质}&=-\mathrm{i}\kappa \bar{u}(\mathbf{q},\lambda ^{\prime})u(\mathbf{p},\lambda )\left( 2\pi \right) ^4\delta ^{(4)}(p+k-q)
    \end{aligned}
\end{equation}


(7.37)h
\begin{equation}
    \begin{aligned}
        \langle \mathbf{q}^-,\lambda ^{\prime}|\mathrm{i}T_{1}^{(1)}|\mathbf{p}^-,\lambda ;\mathbf{k}\rangle &=-\mathrm{i}\kappa \int{\mathrm{d}^4}x\langle \mathbf{q}^-,\lambda ^{\prime}|\mathrm{N}[ \phi (x)\bar{\psi}(x)\psi (x) ] |\mathbf{p}^-,\lambda ;\mathbf{k}\rangle 
\\
\text{缩并}&=\wick{-\mathrm{i}\kappa \int{\mathrm{d}^4}x\langle \c1{\mathbf{q}}^-,\lambda ^{\prime}|\mathrm{N}[ \c2{\phi}(x)\c3{\bar{\psi}}(x)\c1{\psi}(x) ] |\c3{\mathbf{p}}^-,\lambda ;\c2{\mathbf{k}}\rangle }
\\
\text{调整}&=\wick{+\mathrm{i}\kappa \int{\mathrm{d}^4}x\langle \c1{\mathbf{q}}^-,\lambda ^{\prime}|\mathrm{N} [ \c1{\psi}_a(x)\c3{\phi}(x)\c2{\bar{\psi}}_a(x) ] |\c2{\mathbf{p}}^-,\lambda ;\c3{\mathbf{k}}\rangle }
\\
\text{正负能解}&=+\mathrm{i}\kappa \int{\mathrm{d}^4}x\langle \mathbf{q}^-,\lambda ^{\prime}|\psi _{a}^{(-)}(x)\phi (x)\bar{\psi}_{a}^{(+)}(x)|\mathbf{p}^-,\lambda ;\mathbf{k}\rangle 
\\
&=+\mathrm{i}\kappa \int{\mathrm{d}^4}x\langle 0|v_a(\mathbf{q},\lambda ^{\prime})\mathrm{e}^{\mathrm{i}q\cdot x}\mathrm{e}^{-\mathrm{i}k\cdot x}\bar{v}_a(\mathbf{p},\lambda )\mathrm{e}^{-\mathrm{i}p\cdot x}|0\rangle 
\\
&=+\mathrm{i}\kappa \int{\mathrm{d}^4}x\bar{v}_a(\mathbf{p},\lambda )v_a(\mathbf{q},\lambda ^{\prime})\mathrm{e}^{-\mathrm{i}\left( p+k-q \right) \cdot x}\langle 0|0\rangle 
\\
\text{横在前}&=+\mathrm{i}\kappa \int{\mathrm{d}^4}x\bar{v}(\mathbf{p},\lambda )v(\mathbf{q},\lambda ^{\prime})\mathrm{e}^{-\mathrm{i}\left( p+k-q \right) \cdot x}
\\
\text{函数性质}&=+\mathrm{i}\kappa \bar{v}(\mathbf{p},\lambda )v(\mathbf{q},\lambda ^{\prime})\left( 2\pi \right) ^4\delta ^{(4)}(p+k-q)
    \end{aligned}
\end{equation}


\newpage




\newpage
(7.58)无缩并
\begin{equation}
    \mathrm{i}T_{1}^{(2)}=\frac{\left( -\mathrm{i}\kappa \right) ^2}{2!}\int{\mathrm{d}^4x\mathrm{d}^4y\mathrm{N[}\phi (x)\bar{\psi}(x)\psi (x)\phi (y)\bar{\psi}(y)\psi (y)]}
\end{equation}

一次
\begin{equation}
    \begin{aligned}
        \mathrm{i}T_{2}^{(2)}&=\wick{ \frac{\left( -\mathrm{i}\kappa \right) ^2}{2!}\int{\mathrm{d}^4x\mathrm{d}^4y\mathrm{N} [ \c{\phi}(x)\bar{\psi}(x)\psi(x)\c{\phi}(y)\bar{\psi}(y)\psi (y)]} }
\\
\mathrm{i}T_{3}^{(2)}&=\frac{\left( -\mathrm{i}\kappa \right) ^2}{2!}\int{\mathrm{d}^4x\mathrm{d}^4y\mathrm{N} [ \phi (x)\bar{\psi}(x)\psi (x)\phi (y)\bar{\psi}(y)\psi (y)]}
\\
\mathrm{i}T_{4}^{(2)}&=\frac{\left( -\mathrm{i}\kappa \right) ^2}{2!}\int{\mathrm{d}^4x\mathrm{d}^4y\mathrm{N} [ \phi (x)\bar{\psi}(x)\psi (x)\phi (y)\bar{\psi}(y)\psi (y)]}
\\
\mathrm{i}T_{5}^{(2)}&=\frac{\left( -\mathrm{i}\kappa \right) ^2}{2!}\int{\mathrm{d}^4x\mathrm{d}^4y\mathrm{N} [ \phi (x)\bar{\psi}(x)\psi (x)\phi (y)\bar{\psi}(y)\psi (y)]}
\\
\mathrm{i}T_{6}^{(2)}&=\frac{\left( -\mathrm{i}\kappa \right) ^2}{2!}\int{\mathrm{d}^4x\mathrm{d}^4y\mathrm{N} [ \phi (x)\bar{\psi}(x)\psi (x)\phi (y)\bar{\psi}(y)\psi (y)]}
    \end{aligned}
\end{equation}

二次缩并





三次缩并




、




























\section{习题7}

\newpage
\subsection{7.1}
对于7.1节讨论的Yukawa理论,画出 $\psi \phi \rightarrow \psi \phi$散射过程的所有领头阶Feynman图,确定子图之间的相对符号,根据动量空间Feynman规则写出不变振幅$iM$的表达式。

\newpage
\subsection{7.2}
在7.3节讨论的实际场$\phi^4$理论中,对于$\phi \phi \rightarrow \phi \phi$散射过程,画出所有包含2个顶点的单圈Feynman图,分析它们的对称性因子。

\newpage
\subsection{7.3}
考虑拉氏量
$$ \mathcal{L} = \frac{1}{2}(\partial^\mu \chi)\partial_\mu \chi - \frac{1}{2}m_x^2 \chi^2 + (\partial^\mu \phi^i)\partial_\mu \phi - m_{\phi}^2 \phi^i \phi + \lambda \chi \phi^i \phi. \tag{7.159} $$
其中$\chi(x)$是实标量场,相应的玻色子记作$\chi_0$. $\phi(x)$是复标量场,相应的正反玻色子记作$\phi$和$\phi_0$. $\lambda$是实耦合常数。

(a) $\lambda$的量纲是什么?

(b) 写出动量空间中的顶点Feynman规则。

(c) 当$m_x > 2m_{\phi}$时,画出$\chi \rightarrow \phi \phi$衰变过程的领头阶Feynman图,并计算相应的衰变宽度。

(d) 画出下列散射过程的所有领头阶Feynman图。
i. $\phi \phi \rightarrow \phi \phi$.
ii. $\phi \phi \rightarrow \phi \phi$.
iii. $\phi \chi \rightarrow \phi \chi$.
iv. $\phi \phi \rightarrow \chi \chi$.
v. $\chi \chi \rightarrow \chi \chi$.

\newpage
\subsection{7.4}
考虑拉氏量
$$ \mathcal{L} = \frac{1}{2}(\partial^\mu \phi)\partial_\mu \phi - \frac{1}{2}m_{\phi}^2 \phi^2 - \frac{1}{4}F_{\mu \nu}F^{\mu \nu} + \frac{1}{2}m_A^2A_\mu A^\mu + \frac{\kappa}{2}g_{\mu \nu}A^\mu A^\nu \phi, \tag{7.160} $$
其中 $\phi(x)$ 是实标量场,$A^\mu(x)$ 是实矢量场,相应玻色子分别记作 $\phi$ 和 $A$,$F_{\mu\nu} = \partial_\mu A_\nu - \partial_\nu A_\mu$ 是 $A^\mu$ 的场强张量,$\kappa$ 是实耦合常数。

(a) 写出动量空间中的顶点 Feynman 规则。
(b) 当 $m_\phi > 2m_A$ 时,画出 $\phi \rightarrow AA$ 衰变过程的领头阶 Feynman 图,计算衰变宽度。
(c) 画出 $AA \rightarrow AA$ 散射过程的所有领头阶 Feynman 图。

\newpage
\subsection{7.5}
对于有质量复矢量场 $A^\mu(x)$,在拉氏量 (4.290) 中加入相互作用项
$$ \mathcal{L}_1 = -g(J_\mu A^\mu + J_\mu^t A^{\mu t}), \tag{7.161} $$
其中 $g$ 是实耦合常数,$J_\mu(x)$ 是由其它场组成的复流。

(a) 论证 Feynman 传播子 (6.419) 中的非协变项在微扰论的 $g^2$ 阶计算中没有贡献,因而位置空间中有质量复矢量场的内线规则为
$$ x; \quad \underbrace{p}_{y}; \quad \underbrace{\mu}_{x} = \overbrace{A^\mu(y)A^{\nu t}(x)}_{} \text{的 Lorentz 协变项} $$
$$ = \int \frac{d^4p}{(2\pi)^4} \frac{-i(g^{\mu\nu} - p^\mu p^\nu / m_A^2)}{p^2 - m_A^2 + i\epsilon} e^{-ip\cdot(y-x)}. \tag{7.162} $$

这里的 $m_A$ 就是拉氏量 (4.290) 中的 $m_0$。

(b) 根据平面波展开式 (4.293),推出位置空间中有质量复矢量场的外线规则为
$$ A, \lambda; \quad \underbrace{p}_{x} = \langle 0 | \overbrace{A^\mu(x)}^{\sigma} | p^+, \lambda \rangle = \langle 0 | A^{\mu(t+)} (x) | p^+, \lambda \rangle $$
$$ = \varepsilon^\mu(p, \lambda)e^{-ip\cdot x}, \tag{7.163} $$
$$ \bar{A}, \lambda; \quad \underbrace{p}_{x} = \langle 0 | \overbrace{A^\mu(x)}^{\sigma} | p^- , \lambda \rangle = \langle 0 | A^{\mu(t+)} (x) | p^- , \lambda \rangle $$
$$ = \varepsilon^\mu(p, \lambda)e^{-ip\cdot x}, \tag{7.164} $$
$$ x \quad \underbrace{p}_{x} \sim \overbrace{A, \lambda; \mu}_{} = \langle p^+, \lambda | A^{\mu t}(x) | 0 \rangle = \langle p^+, \lambda | A^{\mu(t-)} (x) | 0 \rangle $$
$$ = \varepsilon^\mu(p, \lambda)e^{ip\cdot x}, \tag{7.165} $$
$$ x \quad \underbrace{p}_{x} \sim \overbrace{A, \lambda; \mu}_{} = \langle p^-, \lambda | A^{\mu t}(x) | 0 \rangle = \langle p^-, \lambda | A^{\mu(t-)} (x) | 0 \rangle $$
$$ = \varepsilon^\mu(p, \lambda)e^{ip\cdot x}. \tag{7.166} $$

因此,动量空间中有质量复矢量场的一般内外线规则为
• 有质量正矢量玻色子入射外线:$A, \lambda; \mu \sim \underbrace{p}_{} = \varepsilon^\mu(p, \lambda)$。

• 有质量反矢量玻色子入射外线: $\bar{A}, \lambda; \mu \rightarrow \overset{P}{\longleftrightarrow} = \varepsilon^{\mu}(p, \lambda)$。

• 有质量正矢量玻色子出射外线: $\overset{P}{\longleftrightarrow} A, \lambda; \mu = \varepsilon^{\mu*}(p, \lambda)$。

• 有质量反矢量玻色子出射外线: $\overset{P}{\longleftrightarrow} \bar{A}, \lambda; \mu = \varepsilon^{\mu*}(p, \lambda)$。

• 有质量复矢量玻色子传播子: $\nu \rightarrow \overset{P}{\longleftrightarrow} \overset{\mu}{=} \frac{-i(g^{\mu\nu} - p^{\mu}p^{\nu}/m_A^2)}{p^2 - m_A^2 + i\epsilon}$。

\section{8}



\subsection{8.1}





\subsection{8.2}


1.对于Feynman图,根据Feynman 规则,写出散射过程的不变振幅

根据,


求解双线性型的复共轭为

得到,iM 的复共轭为


根据上面的结论

不变振幅的模方为


Casimir 技巧


计算非极化不变振幅模方


\subsection{8.4}


在高能极限下,忽略质量,
\\左手Dirac旋量场$\psi_\mathrm{L}$/左手Weyl旋量场$\eta_\mathrm{L}$ :描述 左旋极化的正费米子 和 右旋极化的反费米子,
\\右手Dirac旋量场$\psi_\mathrm{R}$/右手Weyl旋量场$\eta_\mathrm{R}$ :描述 右旋极化的正费米子 和 左旋极化的反费米子,
\\$\psi_\mathrm{L}$ 和$\psi_\mathrm{R}$成为两个相互独立的场。
\\左手Dirac旋量场$\psi_\mathrm{L}$ 等价于左手Weyl旋量场$\eta_\mathrm{L}$ 
\\右手Dirac旋量场$\psi_\mathrm{R}$ 等价于右手Weyl旋量场$\eta_\mathrm{R}$

左旋极化 $\lambda=-$
右旋极化 $\lambda=+$


\subsection{8.5}

交叉对称性
一个过程包含一个四维动量为$p^\mathrm{\mu}$的粒子$\Phi$的初态,
一个过程包含一个四维动量为$k^\mathrm{\mu}$的反粒子$\bar{\Phi}$的末态,
则这两个过程的不变振幅可以通过动量替换$k^\mu=-p^\mu$联系起来。

一个粒子沿着时间方向运动等价于它的反粒子逆着时间方向运动,这样的反粒子具有负能量和相反动量


\subsection{8.6}

\subsection{$e^{-}\gamma\to e^{-}\gamma$}
Compton 散射:电子与光子的散射过程

s通道的

u通道的

得到总的



\subsection{$e^+e^-\to\gamma\gamma$}






\section{8.2}

\subsection{推导:散射振幅}


1.对于Feynman图,根据Feynman 规则,写出散射过程的不变振幅

根据,
入射粒子(外线):

出射粒子(外线):

传播子(内线):

写出不变振幅
\begin{equation}
    \begin{aligned}
        \mathrm{i}\mathcal{M} &=\bar{v}(\mathbf{k}_2,\lambda _2)\left( \mathrm{i}e\gamma ^{\mu} \right) u(\mathbf{k}_1,\lambda _1)\frac{-\mathrm{i}g_{\mu \nu}}{q^2}\bar{u}(\mathbf{p}_1,\lambda _{1}^{\prime})\left( \mathrm{i}e\gamma ^{\nu} \right) v(\mathbf{p}_2,\lambda _{2}^{\prime})
\\
&=\frac{\mathrm{i}e^2}{q^2}\bar{v}(\mathbf{k}_2,\lambda _2)\gamma ^{\mu}u(\mathbf{k}_1,\lambda _1)\bar{u}(\mathbf{p}_1,\lambda _{1}^{\prime})\gamma _{\mu}v(\mathbf{p}_2,\lambda _{2}^{\prime})
    \end{aligned}
\end{equation}


1.2求解双线性型的复共轭为
利用
\begin{equation}
    \begin{aligned}
        \left( ABC \right) ^{\dagger}&=C^{\dagger}B^{\dagger}A^{\dagger}
\\
\gamma ^{0\dagger}&=\gamma ^0
\\
\gamma ^{\mu \dagger}\gamma ^0&=\gamma ^0\gamma ^{\mu}
    \end{aligned}
\end{equation}
计算
\begin{equation}
    \begin{aligned}
       \left( \bar{v}\gamma ^{\mu}u \right) ^*&=\left( \bar{v}\gamma ^{\mu}u \right) ^{\dagger}
\\
&=\left( v^{\dagger}\gamma ^0\gamma ^{\mu}u \right) ^{\dagger}
\\
&=u^{\dagger}\gamma ^{\mu \dagger}\gamma ^{0\dagger}v^{\dagger \dagger}
\\
&=u^{\dagger}\gamma ^{\mu \dagger}\gamma ^0v
\\
&=u^{\dagger}\gamma ^0\gamma ^{\mu}v
\\
&=\bar{u}\gamma ^{\mu}v
    \end{aligned}
\end{equation}
以及,下指标同理
\begin{equation}
    \begin{aligned}
        \left( \bar{u}\gamma _{\mu}v \right) ^*&=\left( \bar{u}\gamma _{\mu}v \right) ^{\dagger}
\\
&=\left( u^{\dagger}\gamma ^0\gamma _{\mu}v \right) ^{\dagger}
\\
&=v^{\dagger}\gamma _{\mu}^{\dagger}\gamma ^{0\dagger}u^{\dagger \dagger}
\\
&=v^{\dagger}\gamma _{\mu}^{\dagger}\gamma ^0u
\\
&=v^{\dagger}\gamma ^0\gamma _{\mu}u
\\
&=\bar{v}\gamma _{\mu}u
    \end{aligned}
\end{equation}
得到,iM 的复共轭为
\begin{equation}
    \begin{aligned}
        \left( \mathrm{i}\mathcal{M} \right) ^*&=-\frac{\mathrm{i}e^2}{q^2}\left( \bar{v}(\mathbf{k}_2,\lambda _2)\gamma ^{\mu}u(\mathbf{k}_1,\lambda _1) \right) ^*\left( \bar{u}(\mathbf{p}_1,\lambda _{1}^{\prime})\gamma _{\mu}v(\mathbf{p}_2,\lambda _{2}^{\prime}) \right) ^*
\\
&=-\frac{\mathrm{i}e^2}{q^2}\bar{u}(\mathbf{k}_1,\lambda _1)\gamma ^{\nu}v(\mathbf{k}_2,\lambda _2)\bar{v}(\mathbf{p}_2,\lambda _{2}^{\prime})\gamma _{\nu}u(\mathbf{p}_1,\lambda _{1}^{\prime})
    \end{aligned}
\end{equation}

1.3根据上面的结论
\begin{equation}
    \begin{aligned}
        \mathrm{i}\mathcal{M} &=\frac{\mathrm{i}e^2}{E_{\mathrm{CM}}^{2}}\bar{v}(\mathbf{k}_2,\lambda _2)\gamma ^{\mu}u(\mathbf{k}_1,\lambda _1)\bar{u}(\mathbf{p}_1,\lambda _{1}^{\prime})\gamma _{\mu}v(\mathbf{p}_2,\lambda _{2}^{\prime})
\\
\left( \mathrm{i}\mathcal{M} \right) ^*&=-\frac{\mathrm{i}e^2}{E_{\mathrm{CM}}^{2}}\bar{u}(\mathbf{k}_1,\lambda _1)\gamma ^{\nu}v(\mathbf{k}_2,\lambda _2)\bar{v}(\mathbf{p}_2,\lambda _{2}^{\prime})\gamma _{\nu}u(\mathbf{p}_1,\lambda _{1}^{\prime})
    \end{aligned}
\end{equation}
不变振幅的模方为
\begin{equation}
    \begin{aligned}
        \left| \mathcal{M} \right|^2&=\mathrm{i}\mathcal{M} \left( \mathrm{i}\mathcal{M} \right) ^*
\\
&=\frac{e^4}{E_{\mathrm{CM}}^{4}}{\color[RGB]{240, 0, 0} \bar{v}(\mathbf{k}_2,\lambda _2)\gamma ^{\mu}u(\mathbf{k}_1,\lambda _1)\bar{u}(\mathbf{p}_1,\lambda _{1}^{\prime})\gamma _{\mu}v(\mathbf{p}_2,\lambda _{2}^{\prime})}{\color[RGB]{0, 0, 240} \bar{u}(\mathbf{k}_1,\lambda _1)\gamma ^{\nu}v(\mathbf{k}_2,\lambda _2)\bar{v}(\mathbf{p}_2,\lambda _{2}^{\prime})\gamma _{\nu}u(\mathbf{p}_1,\lambda _{1}^{\prime})}
\\
&=\frac{e^4}{E_{\mathrm{CM}}^{4}}{\color[RGB]{240, 0, 0} \bar{v}(\mathbf{k}_2,\lambda _2)\gamma ^{\mu}u(\mathbf{k}_1,\lambda _1)}{\color[RGB]{0, 0, 240} \bar{u}(\mathbf{k}_1,\lambda _1)\gamma ^{\nu}v(\mathbf{k}_2,\lambda _2)}{\color[RGB]{240, 0, 0} \bar{u}(\mathbf{p}_1,\lambda _{1}^{\prime})\gamma _{\mu}v(\mathbf{p}_2,\lambda _{2}^{\prime})}{\color[RGB]{0, 0, 240} \bar{v}(\mathbf{p}_2,\lambda _{2}^{\prime})\gamma _{\nu}u(\mathbf{p}_1,\lambda _{1}^{\prime})}
    \end{aligned}
\end{equation}

1.4利用Casimir 技巧得到
\begin{equation}
    \left| \mathcal{M} \right|^2=\frac{e^4}{E_{\mathrm{CM}}^{4}}\mathrm{tr}\left[ v(\mathbf{k}_2,\lambda _2)\bar{v}(\mathbf{k}_2,\lambda _2)\gamma ^{\mu}u(\mathbf{k}_1,\lambda _1)\bar{u}(\mathbf{k}_1,\lambda _1)\gamma ^{\nu} \right] \mathrm{tr}\left[ u(\mathbf{p}_1,\lambda _{1}^{\prime})\bar{u}(\mathbf{p}_1,\lambda _{1}^{\prime})\gamma _{\mu}v(\mathbf{p}_2,\lambda _{2}^{\prime})\bar{v}(\mathbf{p}_2,\lambda _{2}^{\prime})\gamma _{\nu} \right] 
\end{equation}


2计算平均
\begin{equation}
    \begin{aligned}
        \overline{\left| \mathcal{M} \right|^2} &= {\color[RGB]{0, 0, 240} \frac{1}{2}\sum_{\lambda _1 =\pm}{{\color[RGB]{240, 0, 0} \frac{1}{2}\sum_{\lambda _2=\pm}{\sum_{\lambda _{1}^{\prime}=\pm}{\sum_{\lambda _{2}^{\prime}=\pm}{\left| \mathcal{M} \right|^2}}}}}}
\\
&=\frac{1}{4}\sum_{\lambda _1\lambda _2\lambda _{1}^{\prime}\lambda _{2}^{\prime}}{\left| \mathcal{M} \right|^2}
    \end{aligned}
\end{equation}
得到
\begin{equation}
    \begin{aligned}
        \overline{\left| \mathcal{M} \right|^2}=\frac{e^4}{4E_{\mathrm{CM}}^{4}}\sum_{\lambda _1\lambda _2\lambda _{1}^{\prime}\lambda _{2}^{\prime}}{\mathrm{tr}\left[ v(\mathbf{k}_2,\lambda _2)\bar{v}(\mathbf{k}_2,\lambda _2)\gamma ^{\mu}u(\mathbf{k}_1,\lambda _1)\bar{u}(\mathbf{k}_1,\lambda _1)\gamma ^{\nu} \right] \mathrm{tr}\left[ u(\mathbf{p}_1,\lambda _{1}^{\prime})\bar{u}(\mathbf{p}_1,\lambda _{1}^{\prime})\gamma _{\mu}v(\mathbf{p}_2,\lambda _{2}^{\prime})\bar{v}(\mathbf{p}_2,\lambda _{2}^{\prime})\gamma _{\nu} \right]}
    \end{aligned}
\end{equation}
3
对于
\begin{equation}
    \overline{\left| \mathcal{M} \right|^2}=\frac{e^4}{4E_{\mathrm{CM}}^{4}}\mathrm{tr}\left[ \sum_{\lambda _1\lambda _2}{v(\mathbf{k}_2,\lambda _2)\bar{v}(\mathbf{k}_2,\lambda _2)}\gamma ^{\mu}\sum_{\lambda _1\lambda _2}{u(\mathbf{k}_1,\lambda _1)\bar{u}(\mathbf{k}_1,\lambda _1)}\gamma ^{\nu} \right] \mathrm{tr}\left[ \sum_{\lambda _{1}^{\prime}\lambda _{2}^{\prime}}{u(\mathbf{p}_1,\lambda _{1}^{\prime})\bar{u}(\mathbf{p}_1,\lambda _{1}^{\prime})}\gamma _{\mu}\sum_{\lambda _{1}^{\prime}\lambda _{2}^{\prime}}{v(\mathbf{p}_2,\lambda _{2}^{\prime})\bar{v}(\mathbf{p}_2,\lambda _{2}^{\prime})}\gamma _{\nu} \right] 
\end{equation}
利用求和关系

写出
\begin{equation}
    \begin{aligned}
        \sum_{\lambda _1\lambda _2}{v(\mathbf{k}_2,\lambda _2)\bar{v}(\mathbf{k}_2,\lambda _2)}&=\slashed{k_2}-m_e
\\
\sum_{\lambda _1\lambda _2}{u(\mathbf{k}_1,\lambda _1)\bar{u}(\mathbf{k}_1,\lambda _1)}&=\slashed{k_1}+m_e
\\
\sum_{\lambda _{1}^{\prime}\lambda _{2}^{\prime}}{u(\mathbf{p}_1,\lambda _{1}^{\prime})\bar{u}(\mathbf{p}_1,\lambda _{1}^{\prime})}&=\slashed{p_1}+m_{\mu}
\\
\sum_{\lambda _{1}^{\prime}\lambda _{2}^{\prime}}{v(\mathbf{p}_2,\lambda _{2}^{\prime})\bar{v}(\mathbf{p}_2,\lambda _{2}^{\prime})}&=\slashed{p_2}-m_{\mu}
    \end{aligned}
\end{equation}
得到
\begin{equation}
    \overline{\left| \mathcal{M} \right|^2}=\frac{e^4}{4E_{\mathrm{CM}}^{4}}\mathrm{tr}\left[ \left( \slashed{k_2}-m_e \right) \gamma ^{\mu}\left( \slashed{k_1}+m_e \right) \gamma ^{\nu} \right] \mathrm{tr}\left[ \left( \slashed{p_1}+m_{\mu} \right) \gamma _{\mu}\left( \slashed{p_2}-m_{\mu} \right) \gamma _{\nu} \right] 
\end{equation}


4
利用

计算
\begin{equation}
    \begin{aligned}
        \mathrm{tr}\left[ \left( \slashed{k_2}-m_e \right) \gamma ^{\mu}\left( \slashed{k_1}+m_e \right) \gamma ^{\nu} \right] &=\mathrm{tr}\left[ \left( \slashed{k_2}\gamma ^{\mu}-m_e\gamma ^{\mu} \right) \left( \slashed{k_1}\gamma ^{\nu}+m_e\gamma ^{\nu} \right) \right] 
\\
&=\mathrm{tr}\left[ \slashed{k_2}\gamma ^{\mu}\slashed{k_1}\gamma ^{\nu}-m_e\gamma ^{\mu}\slashed{k_1}\gamma ^{\nu}+\slashed{k_2}\gamma ^{\mu}m_e\gamma ^{\nu}-m_e\gamma ^{\mu}m_e\gamma ^{\nu} \right] 
\\
&=\mathrm{tr}\left( \slashed{k_2}\gamma ^{\mu}\slashed{k_1}\gamma ^{\nu} \right) -m_e\mathrm{tr}\left( \gamma ^{\mu}\slashed{k_1}\gamma ^{\nu} \right) +m_e\mathrm{tr}\left( \slashed{k_2}\gamma ^{\mu}\gamma ^{\nu} \right) +m_{e}^{2}\mathrm{tr}\left( \gamma ^{\mu}\gamma ^{\nu} \right) 
\\
&=\mathrm{tr}\left( k_{2\rho}\gamma ^{\rho}\gamma ^{\mu}k_{1\sigma}\gamma ^{\sigma}\gamma ^{\nu} \right) -m_e\mathrm{tr}\left( \gamma ^{\mu}k_{1\rho}\gamma ^{\rho}\gamma ^{\nu} \right) +m_e\mathrm{tr}\left( k_{2\rho}\gamma ^{\rho}\gamma ^{\mu}\gamma ^{\nu} \right) -m_{e}^{2}\mathrm{tr}\left( \gamma ^{\mu}\gamma ^{\nu} \right) 
\\
&=k_{1\sigma}k_{2\rho}\mathrm{tr}\left( \gamma ^{\rho}\gamma ^{\mu}\gamma ^{\sigma}\gamma ^{\nu} \right) -m_ek_{1\rho}\mathrm{tr}\left( \gamma ^{\mu}\gamma ^{\rho}\gamma ^{\nu} \right) +m_ek_{2\rho}\mathrm{tr}\left( \gamma ^{\rho}\gamma ^{\mu}\gamma ^{\nu} \right) -m_{e}^{2}\mathrm{tr}\left( \gamma ^{\mu}\gamma ^{\nu} \right) 
\\
&=4k_{1\sigma}k_{2\rho}\left( g^{\rho \mu}g^{\nu \sigma}-g^{\rho \sigma}g^{\mu \nu}+g^{\rho \nu}g^{\mu \sigma} \right) -4m_{e}^{2}g^{\mu \nu}
\\
&=4\left( k_{2\rho}g^{\rho \mu}k_{1\sigma}g^{\nu \sigma}-k_{1\sigma}k_{2\rho}g^{\rho \sigma}g^{\mu \nu}+k_{2\rho}g^{\rho \nu}k_{1\sigma}g^{\mu \sigma} \right) -4m_{e}^{2}g^{\mu \nu}
\\
&=4\left( k_{2}^{\mu}k_{1}^{\nu}-k_{2\rho}k_{1}^{\rho}g^{\mu \nu}+k_{2}^{\nu}k_{1}^{\mu} \right) -4m_{e}^{2}g^{\mu \nu}
\\
&=4\left( k_{2}^{\mu}k_{1}^{\nu}+k_{2}^{\nu}k_{1}^{\mu}-k_{2\rho}k_{1}^{\rho}g^{\mu \nu}-m_{e}^{2}g^{\mu \nu} \right) 
\\
&=4\left[ k_{2}^{\mu}k_{1}^{\nu}+k_{2}^{\nu}k_{1}^{\mu}-g^{\mu \nu}\left( k_1\cdot k_2+m_{e}^{2} \right) \right] 
    \end{aligned}
\end{equation}
以及
\begin{equation}
    \begin{aligned}
        \mathrm{tr}\left[ \left( \slashed{p}_1+m_{\mu} \right) \gamma _{\mu}\left( \slashed{p}_2-m_{\mu} \right) \gamma _{\nu} \right] &=\mathrm{tr}\left[ \left( \slashed{p}_1\gamma _{\mu}+m_{\mu}\gamma _{\mu} \right) \left( \slashed{p}_2\gamma _{\nu}-m_{\mu}\gamma _{\nu} \right) \right] 
\\
&=\mathrm{tr}\left[ \slashed{p}_1\gamma _{\mu}\slashed{p}_2\gamma _{\nu}+m_{\mu}\gamma _{\mu}\slashed{p}_2\gamma _{\nu}-\slashed{p}_1\gamma _{\mu}m_{\mu}\gamma _{\nu}-m_{\mu}\gamma _{\mu}m_{\mu}\gamma _{\nu} \right] 
\\
&=\mathrm{tr}\left( \slashed{p}_1\gamma _{\mu}\slashed{p}_2\gamma _{\nu} \right) +m_{\mu}\mathrm{tr}\left( \gamma _{\mu}\slashed{p}_2\gamma _{\nu} \right) -m_{\mu}\mathrm{tr}\left( \slashed{p}_1\gamma _{\mu}\gamma _{\nu} \right) -m_{\mu}^{2}\mathrm{tr}\left( \gamma _{\mu}\gamma _{\nu} \right) 
\\
&=\mathrm{tr}\left( p_{1}^{\rho}\gamma _{\rho}\gamma _{\mu}p_{2}^{\sigma}\gamma _{\sigma}\gamma _{\nu} \right) +m_{\mu}\mathrm{tr}\left( \gamma _{\mu}p_{2}^{\rho}\gamma _{\rho}\gamma _{\nu} \right) -m_{\mu}\mathrm{tr}\left( p_{1}^{\rho}\gamma _{\rho}\gamma _{\mu}\gamma _{\nu} \right) -m_{\mu}^{2}\mathrm{tr}\left( \gamma _{\mu}\gamma _{\nu} \right) 
\\
&=p_{1}^{\rho}p_{2}^{\sigma}\mathrm{tr}\left( \gamma _{\rho}\gamma _{\mu}\gamma _{\sigma}\gamma _{\nu} \right) +m_{\mu}p_{2}^{\rho}\mathrm{tr}\left( \gamma _{\mu}\gamma _{\rho}\gamma _{\nu} \right) -m_{\mu}p_{1}^{\rho}\mathrm{tr}\left( \gamma _{\rho}\gamma _{\mu}\gamma _{\nu} \right) -m_{\mu}^{2}\mathrm{tr}\left( \gamma _{\mu}\gamma _{\nu} \right) 
\\
&=4p_{1}^{\rho}p_{2}^{\sigma}\left( g_{\rho \mu}g_{\nu \sigma}-g_{\rho \sigma}g_{\mu \nu}+g_{\rho \nu}g_{\mu \sigma} \right) -m_{\mu}^{2}\mathrm{tr}\left( \gamma _{\mu}\gamma _{\nu} \right) 
\\
&=4\left( p_{1}^{\rho}g_{\rho \mu}p_{2}^{\sigma}g_{\nu \sigma}-p_{1}^{\rho}p_{2}^{\sigma}g_{\rho \sigma}g_{\mu \nu}+p_{1}^{\rho}g_{\rho \nu}p_{2}^{\sigma}g_{\mu \sigma} \right) -4m_{\mu}^{2}g_{\mu \nu}
\\
&=4\left( p_{1\mu}p_{2\nu}-p_{1}^{\rho}p_{2\rho}g_{\mu \nu}+p_{1\nu}p_{2\mu} \right) -4m_{\mu}^{2}g_{\mu \nu}
\\
&=4\left( p_{1\mu}p_{2\nu}+p_{1\nu}p_{2\mu}-p_{1}^{\rho}p_{2\rho}g_{\mu \nu}+m_{\mu}^{2}g_{\mu \nu} \right) 
\\
&=4\left[ p_{1\mu}p_{2\nu}+p_{1\nu}p_{2\mu}-g_{\mu \nu}\left( p_1\cdot p_2+m_{\mu}^{2} \right) \right] 
    \end{aligned}
\end{equation}
得到
\begin{equation}
    \overline{\left| \mathcal{M} \right|^2}=\frac{4e^4}{E_{\mathrm{CM}}^{4}}\left[ k_{2}^{\mu}k_{1}^{\nu}+k_{2}^{\nu}k_{1}^{\mu}-g^{\mu \nu}\left( k_1\cdot k_2+m_{e}^{2} \right) \right] \left[ p_{1\mu}p_{2\nu}+p_{1\nu}p_{2\mu}-g_{\mu \nu}\left( p_1\cdot p_2+m_{\mu}^{2} \right) \right] 
\end{equation}
5展开化简
\begin{equation}
    \begin{aligned}
        \overline{\left| \mathcal{M} \right|^2}&=\frac{4e^4}{E_{\mathrm{CM}}^{4}}\left[ k_{2}^{\mu}k_{1}^{\nu}+k_{2}^{\nu}k_{1}^{\mu}-g^{\mu \nu}\left( k_1\cdot k_2+m_{e}^{2} \right) \right] \left[ p_{1\mu}p_{2\nu}+p_{1\nu}p_{2\mu}-g_{\mu \nu}\left( p_1\cdot p_2+m_{\mu}^{2} \right) \right] 
\\
&=\frac{4e^4}{E_{\mathrm{CM}}^{4}}\left[ \begin{array}{c}
	k_{2}^{\mu}k_{1}^{\nu}p_{1\mu}p_{2\nu}+k_{2}^{\nu}k_{1}^{\mu}p_{1\mu}p_{2\nu}-g^{\mu \nu}p_{1\mu}p_{2\nu}\left( k_1\cdot k_2+m_{e}^{2} \right)\\
	k_{2}^{\mu}k_{1}^{\nu}p_{1\nu}p_{2\mu}+k_{2}^{\nu}k_{1}^{\mu}p_{1\nu}p_{2\mu}-g^{\mu \nu}p_{1\nu}p_{2\mu}\left( k_1\cdot k_2+m_{e}^{2} \right)\\
	-k_{2}^{\mu}k_{1}^{\nu}g_{\mu \nu}\left( p_1\cdot p_2+m_{\mu}^{2} \right) -k_{2}^{\nu}k_{1}^{\mu}g_{\mu \nu}\left( p_1\cdot p_2+m_{\mu}^{2} \right) +g^{\mu \nu}\left( k_1\cdot k_2+m_{e}^{2} \right) g_{\mu \nu}\left( p_1\cdot p_2+m_{\mu}^{2} \right)\\
\end{array} \right] 
\\
&=\frac{4e^4}{E_{\mathrm{CM}}^{4}}\left[ \begin{array}{c}
	k_{2}^{\mu}p_{1\mu}\cdot k_{1}^{\nu}p_{2\nu}+k_{1}^{\mu}p_{1\mu}\cdot k_{2}^{\nu}p_{2\nu}-p_{1\mu}p_{2}^{\mu}\left( k_1\cdot k_2+m_{e}^{2} \right)\\
	k_{1}^{\nu}p_{1\nu}\cdot k_{2}^{\mu}p_{2\mu}+k_{2}^{\nu}p_{1\nu}\cdot k_{1}^{\mu}p_{2\mu}-p_{1}^{\mu}p_{2\mu}\left( k_1\cdot k_2+m_{e}^{2} \right)\\
	-k_{2}^{\mu}k_{1\mu}\left( p_1\cdot p_2+m_{\mu}^{2} \right) -k_{2\mu}k_{1}^{\mu}\left( p_1\cdot p_2+m_{\mu}^{2} \right) +g^{\mu \nu}g_{\mu \nu}\left( k_1\cdot k_2+m_{e}^{2} \right) \left( p_1\cdot p_2+m_{\mu}^{2} \right)\\
\end{array} \right] 
\\
&=\frac{4e^4}{E_{\mathrm{CM}}^{4}}\left[ \begin{array}{c}
	{\color[RGB]{240, 0, 0} \left( k_2\cdot p_1 \right) \left( k_1\cdot p_2 \right) +\left( k_1\cdot p_1 \right) \left( k_2\cdot p_2 \right) }-\left( p_1\cdot p_2 \right) \left( k_1\cdot k_2+m_{e}^{2} \right)\\
	{\color[RGB]{240, 0, 0} \left( k_1\cdot p_1 \right) \left( k_2\cdot p_2 \right) +\left( k_2\cdot p_1 \right) \left( k_1\cdot p_2 \right) }-\left( p_1\cdot p_2 \right) \left( k_1\cdot k_2+m_{e}^{2} \right)\\
	-\left( k_1\cdot k_2 \right) \left( p_1\cdot p_2+m_{\mu}^{2} \right) -\left( k_2\cdot k_1 \right) \left( p_1\cdot p_2+m_{\mu}^{2} \right) {\color[RGB]{240, 0, 0} +4\left( k_1\cdot k_2+m_{e}^{2} \right) \left( p_1\cdot p_2+m_{\mu}^{2} \right) }\\
\end{array} \right] 
\\
&=\frac{4e^4}{E_{\mathrm{CM}}^{4}}\left[ 2\left( k_1\cdot p_1 \right) \left( k_2\cdot p_2 \right) +2\left( k_1\cdot p_2 \right) \left( k_2\cdot p_1 \right) -2\left( p_1\cdot p_2 \right) \left( k_1\cdot k_2+m_{e}^{2} \right) -2\left( k_1\cdot k_2 \right) \left( p_1\cdot p_2+m_{\mu}^{2} \right) +4\left( k_1\cdot k_2+m_{e}^{2} \right) \left( p_1\cdot p_2+m_{\mu}^{2} \right) \right] 
\\
&=\frac{8e^4}{E_{\mathrm{CM}}^{4}}\left[ \begin{array}{c}
	{\color[RGB]{240, 0, 0} \left( k_1\cdot p_1 \right) \left( k_2\cdot p_2 \right) +\left( k_1\cdot p_2 \right) \left( k_2\cdot p_1 \right) }\\
	-\left( p_1\cdot p_2 \right) \left( k_1\cdot k_2 \right) -m_{e}^{2}\left( p_1\cdot p_2 \right)\\
	-\left( k_1\cdot k_2 \right) \left( p_1\cdot p_2 \right) -m_{\mu}^{2}\left( k_1\cdot k_2 \right)\\
	+2\left( k_1\cdot k_2 \right) \left( p_1\cdot p_2 \right) +2{\color[RGB]{240, 0, 0} m_{\mu}^{2}\left( k_1\cdot k_2 \right) }+2{\color[RGB]{240, 0, 0} m_{e}^{2}\left( p_1\cdot p_2 \right) +2m_{e}^{2}m_{\mu}^{2}}\\
\end{array} \right] 
\\
&=\frac{8e^4}{E_{\mathrm{CM}}^{4}}\left[ \left( k_1\cdot p_1 \right) \left( k_2\cdot p_2 \right) +\left( k_1\cdot p_2 \right) \left( k_2\cdot p_1 \right) +m_{e}^{2}\left( p_1\cdot p_2 \right) +m_{\mu}^{2}\left( k_1\cdot k_2 \right) +2m_{e}^{2}m_{\mu}^{2} \right] 
    \end{aligned}
\end{equation}









\subsection{8.4}


在高能极限下,忽略质量,
\\左手Dirac旋量场$\psi_\mathrm{L}$/左手Weyl旋量场$\eta_\mathrm{L}$ :描述 左旋极化的正费米子 和 右旋极化的反费米子,
\\右手Dirac旋量场$\psi_\mathrm{R}$/右手Weyl旋量场$\eta_\mathrm{R}$ :描述 右旋极化的正费米子 和 左旋极化的反费米子,
\\$\psi_\mathrm{L}$ 和$\psi_\mathrm{R}$成为两个相互独立的场。
\\左手Dirac旋量场$\psi_\mathrm{L}$ 等价于左手Weyl旋量场$\eta_\mathrm{L}$ 
\\右手Dirac旋量场$\psi_\mathrm{R}$ 等价于右手Weyl旋量场$\eta_\mathrm{R}$

左旋极化 $\lambda=-$
右旋极化 $\lambda=+$


\subsection{8.5}

交叉对称性
一个过程包含一个四维动量为$p^\mathrm{\mu}$的粒子$\Phi$的初态,
一个过程包含一个四维动量为$k^\mathrm{\mu}$的反粒子$\bar{\Phi}$的末态,
则这两个过程的不变振幅可以通过动量替换$k^\mu=-p^\mu$联系起来。

一个粒子沿着时间方向运动等价于它的反粒子逆着时间方向运动,这样的反粒子具有负能量和相反动量


\subsection{8.6}

\subsection{$e^{-}\gamma\to e^{-}\gamma$}
Compton 散射:电子与光子的散射过程

s通道的

u通道的

得到总的



\subsection{$e^+e^-\to\gamma\gamma$}






\section{8}



\subsection{8.1}





\subsection{8.2}


1.对于Feynman图,根据Feynman 规则,写出散射过程的不变振幅

根据,


求解双线性型的复共轭为

得到,iM 的复共轭为


根据上面的结论

不变振幅的模方为


Casimir 技巧


计算非极化不变振幅模方


\subsection{8.4}


在高能极限下,忽略质量,
\\左手Dirac旋量场$\psi_\mathrm{L}$/左手Weyl旋量场$\eta_\mathrm{L}$ :描述 左旋极化的正费米子 和 右旋极化的反费米子,
\\右手Dirac旋量场$\psi_\mathrm{R}$/右手Weyl旋量场$\eta_\mathrm{R}$ :描述 右旋极化的正费米子 和 左旋极化的反费米子,
\\$\psi_\mathrm{L}$ 和$\psi_\mathrm{R}$成为两个相互独立的场。
\\左手Dirac旋量场$\psi_\mathrm{L}$ 等价于左手Weyl旋量场$\eta_\mathrm{L}$ 
\\右手Dirac旋量场$\psi_\mathrm{R}$ 等价于右手Weyl旋量场$\eta_\mathrm{R}$

左旋极化 $\lambda=-$
右旋极化 $\lambda=+$


\subsection{8.5}

交叉对称性
一个过程包含一个四维动量为$p^\mathrm{\mu}$的粒子$\Phi$的初态,
一个过程包含一个四维动量为$k^\mathrm{\mu}$的反粒子$\bar{\Phi}$的末态,
则这两个过程的不变振幅可以通过动量替换$k^\mu=-p^\mu$联系起来。

一个粒子沿着时间方向运动等价于它的反粒子逆着时间方向运动,这样的反粒子具有负能量和相反动量


\subsection{8.6}

\subsection{$e^{-}\gamma\to e^{-}\gamma$}
Compton 散射:电子与光子的散射过程

s通道的

u通道的

得到总的



\subsection{$e^+e^-\to\gamma\gamma$}






\section{8.4}

\subsection{笔记}
在高能极限下,忽略质量,
\\左手Dirac旋量场$\psi_\mathrm{L}$/左手Weyl旋量场$\eta_\mathrm{L}$ :描述 左旋极化的正费米子 和 右旋极化的反费米子,
\\右手Dirac旋量场$\psi_\mathrm{R}$/右手Weyl旋量场$\eta_\mathrm{R}$ :描述 右旋极化的正费米子 和 左旋极化的反费米子,
\\$\psi_\mathrm{L}$ 和$\psi_\mathrm{R}$成为两个相互独立的场。
\\左手Dirac旋量场$\psi_\mathrm{L}$ 等价于左手Weyl旋量场$\eta_\mathrm{L}$ 
\\右手Dirac旋量场$\psi_\mathrm{R}$ 等价于右手Weyl旋量场$\eta_\mathrm{R}$

左旋极化 $\lambda=-$
右旋极化 $\lambda=+$
\section{8.5}


\subsection{笔记}

交叉对称性
一个过程包含一个四维动量为$p^\mathrm{\mu}$的粒子$\Phi$的初态,
一个过程包含一个四维动量为$k^\mathrm{\mu}$的反粒子$\bar{\Phi}$的末态,
则这两个过程的不变振幅可以通过动量替换$k^\mu=-p^\mu$联系起来。

一个粒子沿着时间方向运动等价于它的反粒子逆着时间方向运动,这样的反粒子具有负能量和相反动量




\section{8.6}

\subsection{推导}

\subsection{$e^{-}\gamma\to e^{-}\gamma$}
Compton 散射:电子与光子的散射过程

s通道的

u通道的

得到总的


\subsection{$e^+e^-\to\gamma\gamma$}



\section{习题8}

\newpage
\subsection{8.1}
推出以下公式。

(a) $\text{tr}(\slash{p}\gamma^\mu) = 4p^\mu$。

(b) $\text{tr}(\slash{p}\slash{k}\slash{q}\gamma^\mu) = 4[q^\mu(p \cdot k) - k^\mu(p \cdot q) + p^\mu(k \cdot q)]$。

(c)
$$
\text{tr}(\gamma^\mu\gamma^\nu\gamma^\rho\gamma^\sigma\gamma^\tau\gamma^\phi) = 4g^{\mu\nu}(g^{\rho\sigma}g^{\tau\phi} - g^{\rho\tau}g^{\sigma\phi} + g^{\rho\phi}g^{\sigma\tau}) - 4g^{\mu\rho}(g^{\nu\sigma}g^{\tau\phi} - g^{\nu\tau}g^{\sigma\phi} + g^{\nu\phi}g^{\sigma\tau})
$$
$$
+ 4g^{\mu\sigma}(g^{\nu\rho}g^{\tau\phi} - g^{\nu\tau}g^{\rho\phi} + g^{\nu\phi}g^{\rho\tau}) - 4g^{\mu\tau}(g^{\nu\rho}g^{\sigma\phi} - g^{\nu\sigma}g^{\rho\phi} + g^{\nu\phi}g^{\rho\sigma})
$$
$$
+ 4g^{\mu\phi}(g^{\nu\rho}g^{\sigma\tau} - g^{\nu\sigma}g^{\rho\tau} + g^{\nu\tau}g^{\rho\sigma})。
$$

(d)
$$
\varepsilon^{\alpha\mu\nu\rho}\varepsilon_{\alpha\beta\gamma\delta} = -\delta^\mu_\beta\delta^\nu_\gamma\delta^\rho_\delta - \delta^\mu_\gamma\delta^\nu_\delta\delta^\rho_\beta - \delta^\mu_\delta\delta^\nu_\beta\delta^\rho_\gamma + \delta^\mu_\gamma\delta^\nu_\beta\delta^\rho_\delta + \delta^\mu_\beta\delta^\nu_\delta\delta^\rho_\gamma + \delta^\mu_\delta\delta^\nu_\gamma\delta^\rho_\beta。
$$

\newpage
\subsection{8.2}
在QED 领头阶,Bhabha 散射$e^+e^- \to e^+e^-$和Møller 散射$e^-e^- \to e^-e^-$的Feynman 图分别如(8.312) 和(8.313) 所示。

(a) 忽略电子质量,证明Bhabha 散射的非极化振幅模方为
$$
|\mathcal{M}_B|^2 = 32\pi^2\alpha^2 \left[ u^2 \left( \frac{1}{s} + \frac{1}{t} \right)^2 + \frac{t^2}{s^2} + \frac{s^2}{t^2} \right]。
$$

(b) 利用交叉对称性,求出Møller 散射的非极化振幅模方$|\mathcal{M}_M|^2$。

\newpage
\subsection{8.3}
验证用(8.334) 式表达的$M^{\mu\nu}$ 满足Ward 恒等式$k_{2\nu}M^{\mu\nu}$。

\newpage
\subsection{8.4}
考虑另一种形式的Yukawa 理论,拉氏量为
$$
\mathcal{L} = \frac{1}{2}(\partial_\mu\phi)\partial^\mu\phi - \frac{1}{2} m_\phi^2 \phi^2 + i \bar{\psi}\gamma^\mu\partial_\mu\psi - m_\psi \bar{\psi}\psi - \kappa \phi \bar{\psi} i\gamma^5 \psi,
$$
其中$\phi$ 是实标量场,$\psi$ 是Dirac 旋量场,$\kappa$ 是实耦合常数。

(a) 写出动量空间中的顶点 Feynman 规则。

(b) 设 $m_\phi > 2m_\psi$,画出衰变过程 $\phi \to \psi\bar{\psi}$ 的领头阶 Feynman 图,计算非极化振幅模方 $|\mathcal{M}|^2$ 和衰变宽度 $\Gamma$。

(c) 画出湮灭过程 $\psi\bar{\psi} \to \phi\phi$ 的领头阶 Feynman 图。设 $m_\psi = m_\phi = 0$,计算非极化振幅模方 $|\mathcal{M}|^2$,并在质心系中求出微分散射截面 $d\sigma/d\Omega$。

\newpage
\subsection{8.5}
考虑四费米子相互作用理论,拉氏量为
$$
\mathcal{L} = \sum_{f=\mu, e,\nu_{\mu}, \nu_e} (i\bar{\psi}_f \gamma^\mu \partial_\mu \psi_f - m_f \bar{\psi}_f \psi_f)
$$
$$
- \frac{G_F}{\sqrt{2}} [\bar{\psi}_{\nu_{\mu}} \gamma^\rho (1-\gamma^5) \psi_{\mu} \bar{\psi}_e \gamma_\rho (1-\gamma^5) \psi_{\nu_e} + \bar{\psi}_{\mu} \gamma^\rho (1-\gamma^5) \psi_{\nu_{\mu}} \bar{\psi}_{\nu_e} \gamma_\rho (1-\gamma^5) \psi_e],
$$
其中 Dirac 旋量场 $\psi_{\mu}$、$\psi_e$、$\psi_{\nu_{\mu}}$ 和 $\psi_{\nu_e}$ 分别描述 $\mu$ 子、电子、$\mu$ 子型中微子和电子型中微子,Fermi 常数 $G_F = 1.166 \times 10^{-5} \, \text{GeV}^{-2}$,方括号中两项互为厄米共轭。

(a) 写出动量空间中所有顶点的 Feynman 规则。

(b) 画出三体衰变过程 $\mu^- \to e^- \bar{\nu}_e \nu_{\mu}$ 的领头阶 Feynman 图。设 $m_e = m_{\nu_e} = m_{\nu_{\mu}} = 0$,计算非极化振幅模方 $|\mathcal{M}|^2$ 和衰变宽度 $\Gamma$。

(c) 计算 $\mu$ 子寿命 $\tau = 1/\Gamma$ 的数值,以秒为单位。

\newpage
\subsection{8.6}
考虑标准模型里面电中性矢量玻色子 $Z$ 和带电矢量玻色子 $W^{\pm}$ 与第一代轻子的相互作用,拉氏量为
$$
\mathcal{L}_{ZW} = -\frac{1}{4} Z_{\mu \nu} Z^{\mu \nu} + \frac{1}{2} m_Z^2 Z_{\mu} Z^{\mu} - \frac{1}{2} W_{\mu \nu} W^{+ \mu \nu} + m_W^2 W_{\mu}^{-} W^{+ \mu} + i \bar{\psi}_e \gamma^\mu \partial_\mu \psi_e - m_e \bar{\psi}_e \psi_e + i \bar{\psi}_{\nu_e} \gamma^\mu \partial_\mu \psi_{\nu_e}
$$
$$
- \frac{g}{2 \cos \theta_W} Z_{\mu} [\bar{\psi}_e \gamma^\mu (g_V^e - g_A^e \gamma^5) \psi_e + \bar{\psi}_{\nu_e} \gamma^\mu (g_V^{\nu_e} - g_A^{\nu_e} \gamma^5) \psi_{\nu_e}] - \frac{g}{\sqrt{2}} (W_{\mu}^+ \bar{\psi}_{\nu_e} \gamma^\mu P_L \psi_e + \text{H.c.})
$$
其中,实矢量场 $Z^{\mu}$ 描述质量为 $m_Z = 91.19 \, \text{GeV}$ 的 $Z$ 玻色子;复矢量场 $W^{+ \mu}$ 描述质量为 $m_W = 80.38 \, \text{GeV}$ 的 $W^{\pm}$ 玻色子,且 $W^{- \mu} = (W^{+ \mu})^*$,相应的内外线 Feynman 规则见习题 7.5。场强张量 $Z_{\mu \nu} = \partial_\mu Z_{\nu} - \partial_\nu Z_{\mu}$,$W_{\mu\nu}^{\pm} = \partial_\mu W_{\nu}^{\pm} - \partial_\nu W_{\mu}^{\pm}$。Dirac 旋量场 $\psi_e$ 和 $\psi_{\nu_e}$ 分别描述电子和电子型中微子。$g$ 是实的弱耦合常数,$\theta_w$ 是弱混合角。$g_V^e$、$g_A^e$、$g_V^{\nu_e}$、$g_A^{\nu_e}$ 都是实的无量纲常数。H.c. 代表厄米共轭。

(a) 写出动量空间中所有顶点的 Feynman 规则。

(b) 画出衰变过程 $Z \to e^+ e^-$ 和 $Z \to \nu_e \bar{\nu}_e$ 的领头阶 Feynman 图。计算 $Z \to e^+ e^-$ 的非极化振幅模方 $|\mathcal{M}|^2$ 和衰变分宽度 $\Gamma (Z \to e^+ e^-)$。类比给出衰变分宽度 $\Gamma (Z \to \nu_e \bar{\nu}_e)$。

(c) 画出衰变过程 $W^+ \to e^+ \nu_e$ 的领头阶 Feynman 图,计算非极化振幅模方 $|\mathcal{M}|^2$ 和衰变分宽度 $\Gamma (W^+ \to e^+ \nu_e)$。

(d) 已知
$$
\cos^2 \theta_W = \frac{m_W^2}{m_Z^2}, \quad g = \frac{e}{\sin \theta_W}, \quad g_V^e = -\frac{1}{2} + 2 \sin^2 \theta_W, \quad g_A^e = -\frac{1}{2}, \quad g_V^{\nu_e} = g_A^{\nu_e} = \frac{1}{2}。
$$
计算上述三个衰变分宽度的数值,以 GeV 为单位。

\newpage
\subsection{8.7}
以实标量场 $H(x)$ 描述质量为 $m_H$ 的标准模型 Higgs 玻色子 $H$,考虑拉氏量
$$
\mathcal{L} = \mathcal{L}_{ZW} + \frac{1}{2} (\partial^\mu H) \partial_\mu H - \frac{1}{2} m_H^2 H^2 + \frac{m_Z^2}{v} H Z_\mu Z^\mu + \frac{2m_W^2}{v} H W_\mu^{-} W^{+,\mu},
$$
其中 $\mathcal{L}_{ZW}$ 由 (8.469) 式给出,而 $v = (\sqrt{2} G_F)^{-1/2} = 246.2 \, \text{GeV}$ 是 Higgs 场的真空期望值。

(a) 写出动量空间中 $HZZ$ 和 $HWW$ 顶点的 Feynman 规则。

(b) 设 $m_H > 2m_Z$,画出衰变过程 $H \to ZZ$ 的领头阶 Feynman 图,计算相应的非极化振幅模方 $|\mathcal{M}|^2$ 和衰变分宽度 $\Gamma (H \to ZZ)$。

(c) 设 $m_H > 2m_W$,画出衰变过程 $H \to W^+ W^-$ 的领头阶 Feynman 图,计算相应的非极化振幅模方 $|\mathcal{M}|^2$ 和衰变分宽度 $\Gamma (H \to W^+ W^-)$。

(d) 画出散射过程 $e^+ e^- \to ZH$ 的领头阶 Feynman 图,忽略电子质量 $m_e$,计算相应的非极化振幅模方 $|\mathcal{M}|^2$ 和散射截面 $\sigma$,将 $\sigma$ 表达为质心能 $\sqrt{s}$ 的函数。取 $m_H = 125 \, \text{GeV}$ 和 $\sqrt{s} = 240 \, \text{GeV}$,利用 (8.470) 式计算 $\sigma$ 的数值,以 pb 为单位。


\section{9}
%%%%%%%%%%%%%%%%%%%%%%%%%%%%%%%%%%%%%%%%%%%%%%%%%%%%%%%
\subsection{9.1}

宇称变换
\begin{equation}
    {\mathcal{P} ^{\mu}}_{\nu}=(\mathcal{P} ^{-1}{)^{\mu}}_{\nu}=\left( \begin{matrix}
	+1&		&		&		\\
	&		-1&		&		\\
	&		&		-1&		\\
	&		&		&		-1\\
\end{matrix} \right) 
\end{equation}
时空坐标变换
\begin{equation}
    x^{\mu}=\left( t,\mathbf{x} \right) \Rightarrow x^{\prime \mu}={\mathcal{P} ^{\mu}}_{\nu}x^{\nu}=(\mathcal{P} x)^{\mu}=\left( t,-\mathbf{x} \right) 
\end{equation}
四维动量变换
\begin{equation}
    p^{\mu}=\left( E,\mathbf{p} \right) \Rightarrow p^{\prime \mu}={\mathcal{P} ^{\mu}}_{\nu}p^{\nu}=(\mathcal{P} p)^{\mu}=\left( E,-\mathbf{p} \right) 
\end{equation}
时空导数变换
\begin{equation}
    \partial _{\mu}^{\prime}=(\mathcal{P} ^{-1}{)^{\nu}}_{\mu}\partial _{\nu}
\end{equation}
保持时空体积元不变
\begin{equation}
    \mathrm{d}^4x^{\prime}=\left| \det \left( \mathcal{P} \right) \right|\mathrm{d}^4x=\mathrm{d}^4x
\end{equation}

如果场论系统的作用量S在宇称变换下不变,则运动方程的形式也在宇称变换下不变,此时称系统是宇称守恒的,即具有空间反射对称性。


在宇称守恒的量子理论中,宇称变换在 Hilbert 空间中诱导出态矢$|\Psi \rangle$的线性幺正变换
\begin{equation}
    |\Psi ^{\prime}\rangle =U(\mathcal{P} )|\Psi \rangle =P|\Psi \rangle 
\end{equation}





推导:
1.由标量场
\begin{equation}
    \phi (x)=\int{\frac{\mathrm{d}^3p}{\left( 2\pi \right) ^3}}\frac{1}{\sqrt{2E_{\mathbf{p}}}}\left( a_{\mathbf{p}}\mathrm{e}^{-\mathrm{i}p\cdot x}+b_{\mathbf{p}}^{\dagger}\mathrm{e}^{\mathrm{i}p\cdot x} \right) 
\end{equation}
得到宇称变换后的标量场
\begin{equation}
    \phi (\mathcal{P} x)=\int{\frac{\mathrm{d}^3p}{\left( 2\pi \right) ^3}}\frac{1}{\sqrt{2E_{\mathbf{p}}}}\left( a_{\mathbf{p}}\mathrm{e}^{-\mathrm{i}p\cdot \left( \mathcal{P} x \right)}+b_{\mathbf{p}}^{\dagger}\mathrm{e}^{\mathrm{i}p\cdot \left( \mathcal{P} x \right)} \right) 
\end{equation}
2.计算
\begin{equation}
    \begin{aligned}
        P^{-1}\phi (x)P&=\int{\frac{\mathrm{d}^3p}{\left( 2\pi \right) ^3}}\frac{1}{\sqrt{2E_{\mathbf{p}}}}\left( P^{-1}a_{\mathbf{p}}P\mathrm{e}^{-\mathrm{i}p\cdot x}+P^{-1}b_{\mathbf{p}}^{\dagger}P\mathrm{e}^{\mathrm{i}p\cdot x} \right) 
\\
&=\int{\frac{\mathrm{d}^3p}{\left( 2\pi \right) ^3}}\frac{1}{\sqrt{2E_{\mathbf{p}}}}\left( \eta _{P}^{*}a_{-\mathbf{p}}\mathrm{e}^{-\mathrm{i}p\cdot x}+\tilde{\eta}_Pb_{-\mathbf{p}}^{\dagger}\mathrm{e}^{\mathrm{i}p\cdot x} \right) 
\\
&=\int{\frac{\mathrm{d}^3p}{\left( 2\pi \right) ^3}}\frac{1}{\sqrt{2E_{\mathbf{p}}}}\left( \eta _{P}^{*}a_{\mathbf{p}}\mathrm{e}^{-\mathrm{i}\left( \mathcal{P} p \right) \cdot x}+\tilde{\eta}_Pb_{\mathbf{p}}^{\dagger}\mathrm{e}^{\mathrm{i}\left( \mathcal{P} p \right) \cdot x} \right) 
\\
&=\int{\frac{\mathrm{d}^3p}{\left( 2\pi \right) ^3}}\frac{1}{\sqrt{2E_{\mathbf{p}}}}\left( \eta _{P}^{*}a_{\mathbf{p}}\mathrm{e}^{-\mathrm{i}p\cdot \left( \mathcal{P} x \right)}+\tilde{\eta}_Pb_{\mathbf{p}}^{\dagger}\mathrm{e}^{\mathrm{i}p\cdot \left( \mathcal{P} x \right)} \right) 
\\
&=\int{\frac{\mathrm{d}^3p}{\left( 2\pi \right) ^3}}\frac{1}{\sqrt{2E_{\mathbf{p}}}}\left( \eta _{P}^{*}a_{\mathbf{p}}\mathrm{e}^{-\mathrm{i}p\cdot \left( \mathcal{P} x \right)}+\eta _{P}^{*}b_{\mathbf{p}}^{\dagger}\mathrm{e}^{\mathrm{i}p\cdot \left( \mathcal{P} x \right)} \right) 
\\
&=\eta _{P}^{*}\int{\frac{\mathrm{d}^3p}{\left( 2\pi \right) ^3}}\frac{1}{\sqrt{2E_{\mathbf{p}}}}\left( a_{\mathbf{p}}\mathrm{e}^{-\mathrm{i}p\cdot \left( \mathcal{P} x \right)}+b_{\mathbf{p}}^{\dagger}\mathrm{e}^{\mathrm{i}p\cdot \left( \mathcal{P} x \right)} \right) 
    \end{aligned}
\end{equation}
得到
\begin{equation}
    P^{-1}\phi (x)P=\eta _{P}^{*}\phi (\mathcal{P} x)
\end{equation}


推导:
1.由标量场的复共轭
\begin{equation}
    \phi ^{\dagger}(x)=\int{\frac{\mathrm{d}^3p}{\left( 2\pi \right) ^3}}\frac{1}{\sqrt{2E_{\mathbf{p}}}}\left( b_{\mathbf{p}}\mathrm{e}^{-\mathrm{i}p\cdot x}+a_{\mathbf{p}}^{\dagger}\mathrm{e}^{\mathrm{i}p\cdot x} \right) 
\end{equation}
得到
\begin{equation}
    \phi ^{\dagger}(\mathcal{P} x)=\int{\frac{\mathrm{d}^3p}{\left( 2\pi \right) ^3}}\frac{1}{\sqrt{2E_{\mathbf{p}}}}\left( b_{\mathbf{p}}\mathrm{e}^{-\mathrm{i}p\cdot \left( \mathcal{P} x \right)}+a_{\mathbf{p}}^{\dagger}\mathrm{e}^{\mathrm{i}p\cdot \left( \mathcal{P} x \right)} \right) 
\end{equation}
2.计算
\begin{equation}
    \begin{aligned}
        P^{-1}\phi ^{\dagger}(x)P&=\int{\frac{\mathrm{d}^3p}{\left( 2\pi \right) ^3}}\frac{1}{\sqrt{2E_{\mathbf{p}}}}\left( P^{-1}b_{\mathbf{p}}P\mathrm{e}^{-\mathrm{i}p\cdot x}+P^{-1}a_{\mathbf{p}}^{\dagger}P\mathrm{e}^{\mathrm{i}p\cdot x} \right) 
\\
&=\int{\frac{\mathrm{d}^3p}{\left( 2\pi \right) ^3}}\frac{1}{\sqrt{2E_{\mathbf{p}}}}\left( \tilde{\eta}_{P}^{*}b_{-\mathbf{p}}\mathrm{e}^{-\mathrm{i}p\cdot x}+\eta _Pa_{-\mathbf{p}}^{\dagger}\mathrm{e}^{\mathrm{i}p\cdot x} \right) 
\\
&=\int{\frac{\mathrm{d}^3p}{\left( 2\pi \right) ^3}}\frac{1}{\sqrt{2E_{\mathbf{p}}}}\left( \tilde{\eta}_{P}^{*}b_{\mathbf{p}}\mathrm{e}^{-\mathrm{i}\left( \mathcal{P} p \right) \cdot x}+\eta _Pa_{\mathbf{p}}^{\dagger}\mathrm{e}^{\mathrm{i}\left( \mathcal{P} p \right) \cdot x} \right) 
\\
&=\int{\frac{\mathrm{d}^3p}{\left( 2\pi \right) ^3}}\frac{1}{\sqrt{2E_{\mathbf{p}}}}\left( \tilde{\eta}_{P}^{*}b_{\mathbf{p}}\mathrm{e}^{-\mathrm{i}p\cdot \left( \mathcal{P} x \right)}+\eta _Pa_{\mathbf{p}}^{\dagger}\mathrm{e}^{\mathrm{i}p\cdot \left( \mathcal{P} x \right)} \right) 
\\
&=\int{\frac{\mathrm{d}^3p}{\left( 2\pi \right) ^3}}\frac{1}{\sqrt{2E_{\mathbf{p}}}}\left( \eta _Pb_{\mathbf{p}}\mathrm{e}^{-\mathrm{i}p\cdot \left( \mathcal{P} x \right)}+\eta _Pa_{\mathbf{p}}^{\dagger}\mathrm{e}^{\mathrm{i}p\cdot \left( \mathcal{P} x \right)} \right) 
\\
&=\eta _P\int{\frac{\mathrm{d}^3p}{\left( 2\pi \right) ^3}}\frac{1}{\sqrt{2E_{\mathbf{p}}}}\left( b_{\mathbf{p}}\mathrm{e}^{-\mathrm{i}p\cdot \left( \mathcal{P} x \right)}+a_{\mathbf{p}}^{\dagger}\mathrm{e}^{\mathrm{i}p\cdot \left( \mathcal{P} x \right)} \right) 
    \end{aligned}
\end{equation}
得到
\begin{equation}
    P^{-1}\phi ^{\dagger}(x)P=\eta _P\phi ^{\dagger}(\mathcal{P} x)
\end{equation}



\subsubsection{标量场的T变换}

时间反演变换
\begin{equation}
    {\mathcal{T} ^{\mu}}_{\nu}=(\mathcal{T} ^{-1}{)^{\mu}}_{\nu}=\left( \begin{matrix}
	-1&		&		&		\\
	&		+1&		&		\\
	&		&		+1&		\\
	&		&		&		+1\\
\end{matrix} \right) 
\end{equation}
时空坐标变换
\begin{equation}
    x^{\mu}=\left( t,\mathbf{x} \right) \Rightarrow \,\,x^{\prime \mu}={\mathcal{T} ^{\mu}}_{\nu}x^{\nu}=(\mathcal{T} x)^{\mu}=\left( -t,\mathbf{x} \right) 
\end{equation}
四维动量变换
\begin{equation}
    p^{\mu}=\left( E,\mathbf{p} \right) \Rightarrow \,\,p^{\prime \mu}={\mathcal{T} ^{\mu}}_{\nu}p^{\nu}=(\mathcal{T} p)^{\mu}=\left( -E,\mathbf{p} \right) 
\end{equation}
时空导数变换
\begin{equation}
    \partial _{\mu}^{\prime}=(\mathcal{T} ^{-1}{)^{\nu}}_{\mu}\partial _{\nu}
\end{equation}
时间反演变换保持时空体积元不变
\begin{equation}
    \mathrm{d}^4x^{\prime}=\left| \det\mathrm{(}\mathcal{T} ) \right|\mathrm{d}^4x=\mathrm{d}^4x
\end{equation}


推导:
1.由标量场
\begin{equation}
    \phi (x)=\int{\frac{\mathrm{d}^3p}{\left( 2\pi \right) ^3}\frac{1}{\sqrt{2E_{\mathbf{p}}}}\left( a_{\mathbf{p}}\mathrm{e}^{-\mathrm{i}p\cdot x}+b_{\mathbf{p}}^{\dagger}\mathrm{e}^{\mathrm{i}p\cdot x} \right)}
\end{equation}
得到时间反演变换后的标量场
\begin{equation}
    \phi (\mathcal{T} x)=\int{\frac{\mathrm{d}^3p}{\left( 2\pi \right) ^3}\frac{1}{\sqrt{2E_{\mathbf{p}}}}\left( a_{\mathbf{p}}\mathrm{e}^{-\mathrm{i}p\cdot \left( \mathcal{T} x \right)}+b_{\mathbf{p}}^{\dagger}\mathrm{e}^{\mathrm{i}p\cdot \left( \mathcal{T} x \right)} \right)}
\end{equation}
2.计算
\begin{equation}
    \begin{aligned}
        T^{-1}\phi \left( x \right) T&=\int{\frac{\mathrm{d}^3p}{\left( 2\pi \right) ^3}\frac{1}{\sqrt{2E_{\mathbf{p}}}}T^{-1}\left( a_{\mathbf{p}}\mathrm{e}^{-\mathrm{i}p\cdot x}+b_{\mathbf{p}}^{\dagger}\mathrm{e}^{\mathrm{i}p\cdot x} \right)}T
\\
&=\int{\frac{\mathrm{d}^3p}{\left( 2\pi \right) ^3}}\frac{1}{\sqrt{2E_{\mathbf{p}}}}\left( T^{-1}a_{\mathbf{p}}TT^{-1}\mathrm{e}^{-\mathrm{i}p\cdot x}T+T^{-1}b_{\mathbf{p}}^{\dagger}TT^{-1}\mathrm{e}^{\mathrm{i}p\cdot x}T \right) 
\\
&=\int{\frac{\mathrm{d}^3p}{\left( 2\pi \right) ^3}}\frac{1}{\sqrt{2E_{\mathbf{p}}}}\left( \eta _{T}^{*}a_{-\mathbf{p}}\mathrm{e}^{\mathrm{i}p\cdot x}+\tilde{\eta}_Tb_{-\mathbf{p}}^{\dagger}\mathrm{e}^{-\mathrm{i}p\cdot x} \right) 
\\
&=\int{\frac{\mathrm{d}^3p}{\left( 2\pi \right) ^3}}\frac{1}{\sqrt{2E_{\mathbf{p}}}}\left( \eta _{T}^{*}a_{\mathbf{p}}\mathrm{e}^{\mathrm{i}\left( \mathcal{P} p \right) \cdot x}+\tilde{\eta}_Tb_{\mathbf{p}}^{\dagger}\mathrm{e}^{-\mathrm{i}\left( \mathcal{P} p \right) \cdot x} \right) 
\\
&=\int{\frac{\mathrm{d}^3p}{\left( 2\pi \right) ^3}}\frac{1}{\sqrt{2E_{\mathbf{p}}}}\left( \eta _{T}^{*}a_{\mathbf{p}}\mathrm{e}^{\mathrm{i}p\cdot \left( \mathcal{P} x \right)}+\tilde{\eta}_Tb_{\mathbf{p}}^{\dagger}\mathrm{e}^{-\mathrm{i}p\cdot \left( \mathcal{P} x \right)} \right) 
\\
&=\int{\frac{\mathrm{d}^3p}{\left( 2\pi \right) ^3}}\frac{1}{\sqrt{2E_{\mathbf{p}}}}\left( \eta _{T}^{*}a_{\mathbf{p}}\mathrm{e}^{-\mathrm{i}p\cdot (\mathcal{T} x)}+\tilde{\eta}_Tb_{\mathbf{p}}^{\dagger}\mathrm{e}^{\mathrm{i}p\cdot (\mathcal{T} x)} \right) 
\\
&=\int{\frac{\mathrm{d}^3p}{\left( 2\pi \right) ^3}}\frac{1}{\sqrt{2E_{\mathbf{p}}}}\left( \eta _{T}^{*}a_{\mathbf{p}}\mathrm{e}^{-\mathrm{i}p\cdot (\mathcal{T} x)}+\eta _{T}^{*}b_{\mathbf{p}}^{\dagger}\mathrm{e}^{\mathrm{i}p\cdot (\mathcal{T} x)} \right) 
\\
&=\eta _{T}^{*}\int{\frac{\mathrm{d}^3p}{\left( 2\pi \right) ^3}}\frac{1}{\sqrt{2E_{\mathbf{p}}}}\left( a_{\mathbf{p}}\mathrm{e}^{-\mathrm{i}p\cdot (\mathcal{T} x)}+b_{\mathbf{p}}^{\dagger}\mathrm{e}^{\mathrm{i}p\cdot (\mathcal{T} x)} \right) 
    \end{aligned}
\end{equation}
对比得到


1.由
\begin{equation}
    \phi ^{\dagger}(x)=\int{\frac{\mathrm{d}^3p}{\left( 2\pi \right) ^3}}\frac{1}{\sqrt{2E_{\mathbf{p}}}}\left( b_{\mathbf{p}}\mathrm{e}^{-\mathrm{i}p\cdot x}+a_{\mathbf{p}}^{\dagger}\mathrm{e}^{\mathrm{i}p\cdot x} \right) 
\end{equation}
得到
\begin{equation}
    \phi ^{\dagger}(\mathcal{T} x)=\int{\frac{\mathrm{d}^3p}{\left( 2\pi \right) ^3}}\frac{1}{\sqrt{2E_{\mathbf{p}}}}\left( b_{\mathbf{p}}\mathrm{e}^{-\mathrm{i}p\cdot \left( \mathcal{T} x \right)}+a_{\mathbf{p}}^{\dagger}\mathrm{e}^{\mathrm{i}p\cdot \left( \mathcal{T} x \right)} \right) 
\end{equation}
2.计算
\begin{equation}
    \begin{aligned}
        T^{-1}\phi ^{\dagger}\left( x \right) T&=\int{\frac{\mathrm{d}^3p}{\left( 2\pi \right) ^3}}\frac{1}{\sqrt{2E_{\mathbf{p}}}}T^{-1}\left( b_{\mathbf{p}}\mathrm{e}^{-\mathrm{i}p\cdot x}+a_{\mathbf{p}}^{\dagger}\mathrm{e}^{\mathrm{i}p\cdot x} \right) T
\\
&=\int{\frac{\mathrm{d}^3p}{\left( 2\pi \right) ^3}}\frac{1}{\sqrt{2E_{\mathbf{p}}}}\left( T^{-1}b_{\mathbf{p}}TT^{-1}\mathrm{e}^{-\mathrm{i}p\cdot x}T+T^{-1}a_{\mathbf{p}}^{\dagger}TT^{-1}\mathrm{e}^{\mathrm{i}p\cdot x}T \right) 
\\
&=\int{\frac{\mathrm{d}^3p}{\left( 2\pi \right) ^3}}\frac{1}{\sqrt{2E_{\mathbf{p}}}}\left( \tilde{\eta}_{T}^{*}b_{-\mathbf{p}}\mathrm{e}^{\mathrm{i}p\cdot x}+\eta _Ta_{-\mathbf{p}}^{\dagger}\mathrm{e}^{-\mathrm{i}p\cdot x} \right) 
\\
&=\int{\frac{\mathrm{d}^3p}{\left( 2\pi \right) ^3}}\frac{1}{\sqrt{2E_{\mathbf{p}}}}\left( \tilde{\eta}_{T}^{*}b_{\mathbf{p}}\mathrm{e}^{\mathrm{i}\left( \mathcal{P} p \right) \cdot x}+\eta _Ta_{\mathbf{p}}^{\dagger}\mathrm{e}^{-\mathrm{i}\left( \mathcal{P} p \right) \cdot x} \right) 
\\
&=\int{\frac{\mathrm{d}^3p}{\left( 2\pi \right) ^3}}\frac{1}{\sqrt{2E_{\mathbf{p}}}}\left( \tilde{\eta}_{T}^{*}b_{\mathbf{p}}\mathrm{e}^{\mathrm{i}p\cdot \left( \mathcal{P} x \right)}+\eta _Ta_{\mathbf{p}}^{\dagger}\mathrm{e}^{-\mathrm{i}p\cdot \left( \mathcal{P} x \right)} \right) 
\\
&=\int{\frac{\mathrm{d}^3p}{\left( 2\pi \right) ^3}}\frac{1}{\sqrt{2E_{\mathbf{p}}}}\left( \tilde{\eta}_{T}^{*}b_{\mathbf{p}}\mathrm{e}^{-\mathrm{i}p\cdot (\mathcal{T} x)}+\eta _Ta_{\mathbf{p}}^{\dagger}\mathrm{e}^{\mathrm{i}p\cdot (\mathcal{T} x)} \right) 
\\
&=\int{\frac{\mathrm{d}^3p}{\left( 2\pi \right) ^3}}\frac{1}{\sqrt{2E_{\mathbf{p}}}}\left( \eta _Tb_{\mathbf{p}}\mathrm{e}^{-\mathrm{i}p\cdot (\mathcal{T} x)}+\eta _Ta_{\mathbf{p}}^{\dagger}\mathrm{e}^{\mathrm{i}p\cdot (\mathcal{T} x)} \right) 
\\
&=\eta _T\int{\frac{\mathrm{d}^3p}{\left( 2\pi \right) ^3}}\frac{1}{\sqrt{2E_{\mathbf{p}}}}\left( b_{\mathbf{p}}\mathrm{e}^{-\mathrm{i}p\cdot (\mathcal{T} x)}+a_{\mathbf{p}}^{\dagger}\mathrm{e}^{\mathrm{i}p\cdot (\mathcal{T} x)} \right) 
    \end{aligned}
\end{equation}
对比得到
\begin{equation}
    T^{-1}\phi ^{\dagger}\left( x \right) T=\eta _T\phi ^{\dagger}(\mathcal{T} x)
\end{equation}


由标量场
\begin{equation}
    \phi \left( x \right) =\int{\frac{\mathrm{d}^3p}{\left( 2\pi \right) ^3}}\frac{1}{\sqrt{2E_{\mathbf{p}}}}\left( a_{\mathbf{p}}\mathrm{e}^{-\mathrm{i}p\cdot x}+b_{\mathbf{p}}^{\dagger}\mathrm{e}^{\mathrm{i}p\cdot x} \right) 
\end{equation}
和复共轭
\begin{equation}
    \phi ^{\dagger}(x)=\int{\frac{\mathrm{d}^3p}{\left( 2\pi \right) ^3}}\frac{1}{\sqrt{2E_{\mathbf{p}}}}\left( b_{\mathbf{p}}\mathrm{e}^{-\mathrm{i}p\cdot x}+a_{\mathbf{p}}^{\dagger}\mathrm{e}^{\mathrm{i}p\cdot x} \right) 
\end{equation}

2.计算
\begin{equation}
    \begin{aligned}
        C^{-1}\phi C&=\int{\frac{\mathrm{d}^3p}{\left( 2\pi \right) ^3}}\frac{1}{\sqrt{2E_{\mathbf{p}}}}C^{-1}\left( a_{\mathbf{p}}\mathrm{e}^{-\mathrm{i}p\cdot x}+b_{\mathbf{p}}^{\dagger}\mathrm{e}^{\mathrm{i}p\cdot x} \right) C
\\
&=\int{\frac{\mathrm{d}^3p}{\left( 2\pi \right) ^3}}\frac{1}{\sqrt{2E_{\mathbf{p}}}}\left( C^{-1}a_{\mathbf{p}}C\mathrm{e}^{-\mathrm{i}p\cdot x}+C^{-1}b_{\mathbf{p}}^{\dagger}C\mathrm{e}^{\mathrm{i}p\cdot x} \right) 
\\
&=\int{\frac{\mathrm{d}^3p}{\left( 2\pi \right) ^3}}\frac{1}{\sqrt{2E_{\mathbf{p}}}}\left( \eta _{C}^{*}b_{\mathbf{p}}\mathrm{e}^{-\mathrm{i}p\cdot x}+\eta _{C}^{*}a_{\mathbf{p}}^{\dagger}\mathrm{e}^{\mathrm{i}p\cdot x} \right) 
\\
&=\eta _{C}^{*}\int{\frac{\mathrm{d}^3p}{\left( 2\pi \right) ^3}}\frac{1}{\sqrt{2E_{\mathbf{p}}}}\left( b_{\mathbf{p}}\mathrm{e}^{-\mathrm{i}p\cdot x}+a_{\mathbf{p}}^{\dagger}\mathrm{e}^{\mathrm{i}p\cdot x} \right) 
    \end{aligned}
\end{equation}
对比得到
\begin{equation}
    C^{-1}\phi (x)C=\eta _{C}^{*}\phi ^{\dagger}(x)
\end{equation}
同样地
\begin{equation}
    \begin{aligned}
        C^{-1}\phi ^{\dagger}(x)C&=\int{\frac{\mathrm{d}^3p}{\left( 2\pi \right) ^3}}\frac{1}{\sqrt{2E_{\mathbf{p}}}}C^{-1}\left( b_{\mathbf{p}}\mathrm{e}^{-\mathrm{i}p\cdot x}+a_{\mathbf{p}}^{\dagger}\mathrm{e}^{\mathrm{i}p\cdot x} \right) C
\\
&=\int{\frac{\mathrm{d}^3p}{\left( 2\pi \right) ^3}}\frac{1}{\sqrt{2E_{\mathbf{p}}}}\left( C^{-1}b_{\mathbf{p}}C\mathrm{e}^{-\mathrm{i}p\cdot x}+C^{-1}a_{\mathbf{p}}^{\dagger}C\mathrm{e}^{\mathrm{i}p\cdot x} \right) 
\\
&=\int{\frac{\mathrm{d}^3p}{\left( 2\pi \right) ^3}}\frac{1}{\sqrt{2E_{\mathbf{p}}}}\left( \eta _Ca_{\mathbf{p}}\mathrm{e}^{-\mathrm{i}p\cdot x}+\eta _Cb_{\mathbf{p}}^{\dagger}\mathrm{e}^{\mathrm{i}p\cdot x} \right) 
\\
&=\eta _C\int{\frac{\mathrm{d}^3p}{\left( 2\pi \right) ^3}}\frac{1}{\sqrt{2E_{\mathbf{p}}}}\left( a_{\mathbf{p}}\mathrm{e}^{-\mathrm{i}p\cdot x}+b_{\mathbf{p}}^{\dagger}\mathrm{e}^{\mathrm{i}p\cdot x} \right) 
    \end{aligned}
\end{equation}
对比得到
\begin{equation}
    C^{-1}\phi ^{\dagger}(x)C=\eta _C\phi (x)
\end{equation}

\section{9}
%%%%%%%%%%%%%%%%%%%%%%%%%%%%%%%%%%%%%%%%%%%%%%%%%%%%%%%
\subsection{9.1}

宇称变换
\begin{equation}
    {\mathcal{P} ^{\mu}}_{\nu}=(\mathcal{P} ^{-1}{)^{\mu}}_{\nu}=\left( \begin{matrix}
	+1&		&		&		\\
	&		-1&		&		\\
	&		&		-1&		\\
	&		&		&		-1\\
\end{matrix} \right) 
\end{equation}
时空坐标变换
\begin{equation}
    x^{\mu}=\left( t,\mathbf{x} \right) \Rightarrow x^{\prime \mu}={\mathcal{P} ^{\mu}}_{\nu}x^{\nu}=(\mathcal{P} x)^{\mu}=\left( t,-\mathbf{x} \right) 
\end{equation}
四维动量变换
\begin{equation}
    p^{\mu}=\left( E,\mathbf{p} \right) \Rightarrow p^{\prime \mu}={\mathcal{P} ^{\mu}}_{\nu}p^{\nu}=(\mathcal{P} p)^{\mu}=\left( E,-\mathbf{p} \right) 
\end{equation}
时空导数变换
\begin{equation}
    \partial _{\mu}^{\prime}=(\mathcal{P} ^{-1}{)^{\nu}}_{\mu}\partial _{\nu}
\end{equation}
保持时空体积元不变
\begin{equation}
    \mathrm{d}^4x^{\prime}=\left| \det \left( \mathcal{P} \right) \right|\mathrm{d}^4x=\mathrm{d}^4x
\end{equation}

如果场论系统的作用量S在宇称变换下不变,则运动方程的形式也在宇称变换下不变,此时称系统是宇称守恒的,即具有空间反射对称性。


在宇称守恒的量子理论中,宇称变换在 Hilbert 空间中诱导出态矢$|\Psi \rangle$的线性幺正变换
\begin{equation}
    |\Psi ^{\prime}\rangle =U(\mathcal{P} )|\Psi \rangle =P|\Psi \rangle 
\end{equation}





推导:
1.由标量场
\begin{equation}
    \phi (x)=\int{\frac{\mathrm{d}^3p}{\left( 2\pi \right) ^3}}\frac{1}{\sqrt{2E_{\mathbf{p}}}}\left( a_{\mathbf{p}}\mathrm{e}^{-\mathrm{i}p\cdot x}+b_{\mathbf{p}}^{\dagger}\mathrm{e}^{\mathrm{i}p\cdot x} \right) 
\end{equation}
得到宇称变换后的标量场
\begin{equation}
    \phi (\mathcal{P} x)=\int{\frac{\mathrm{d}^3p}{\left( 2\pi \right) ^3}}\frac{1}{\sqrt{2E_{\mathbf{p}}}}\left( a_{\mathbf{p}}\mathrm{e}^{-\mathrm{i}p\cdot \left( \mathcal{P} x \right)}+b_{\mathbf{p}}^{\dagger}\mathrm{e}^{\mathrm{i}p\cdot \left( \mathcal{P} x \right)} \right) 
\end{equation}
2.计算
\begin{equation}
    \begin{aligned}
        P^{-1}\phi (x)P&=\int{\frac{\mathrm{d}^3p}{\left( 2\pi \right) ^3}}\frac{1}{\sqrt{2E_{\mathbf{p}}}}\left( P^{-1}a_{\mathbf{p}}P\mathrm{e}^{-\mathrm{i}p\cdot x}+P^{-1}b_{\mathbf{p}}^{\dagger}P\mathrm{e}^{\mathrm{i}p\cdot x} \right) 
\\
&=\int{\frac{\mathrm{d}^3p}{\left( 2\pi \right) ^3}}\frac{1}{\sqrt{2E_{\mathbf{p}}}}\left( \eta _{P}^{*}a_{-\mathbf{p}}\mathrm{e}^{-\mathrm{i}p\cdot x}+\tilde{\eta}_Pb_{-\mathbf{p}}^{\dagger}\mathrm{e}^{\mathrm{i}p\cdot x} \right) 
\\
&=\int{\frac{\mathrm{d}^3p}{\left( 2\pi \right) ^3}}\frac{1}{\sqrt{2E_{\mathbf{p}}}}\left( \eta _{P}^{*}a_{\mathbf{p}}\mathrm{e}^{-\mathrm{i}\left( \mathcal{P} p \right) \cdot x}+\tilde{\eta}_Pb_{\mathbf{p}}^{\dagger}\mathrm{e}^{\mathrm{i}\left( \mathcal{P} p \right) \cdot x} \right) 
\\
&=\int{\frac{\mathrm{d}^3p}{\left( 2\pi \right) ^3}}\frac{1}{\sqrt{2E_{\mathbf{p}}}}\left( \eta _{P}^{*}a_{\mathbf{p}}\mathrm{e}^{-\mathrm{i}p\cdot \left( \mathcal{P} x \right)}+\tilde{\eta}_Pb_{\mathbf{p}}^{\dagger}\mathrm{e}^{\mathrm{i}p\cdot \left( \mathcal{P} x \right)} \right) 
\\
&=\int{\frac{\mathrm{d}^3p}{\left( 2\pi \right) ^3}}\frac{1}{\sqrt{2E_{\mathbf{p}}}}\left( \eta _{P}^{*}a_{\mathbf{p}}\mathrm{e}^{-\mathrm{i}p\cdot \left( \mathcal{P} x \right)}+\eta _{P}^{*}b_{\mathbf{p}}^{\dagger}\mathrm{e}^{\mathrm{i}p\cdot \left( \mathcal{P} x \right)} \right) 
\\
&=\eta _{P}^{*}\int{\frac{\mathrm{d}^3p}{\left( 2\pi \right) ^3}}\frac{1}{\sqrt{2E_{\mathbf{p}}}}\left( a_{\mathbf{p}}\mathrm{e}^{-\mathrm{i}p\cdot \left( \mathcal{P} x \right)}+b_{\mathbf{p}}^{\dagger}\mathrm{e}^{\mathrm{i}p\cdot \left( \mathcal{P} x \right)} \right) 
    \end{aligned}
\end{equation}
得到
\begin{equation}
    P^{-1}\phi (x)P=\eta _{P}^{*}\phi (\mathcal{P} x)
\end{equation}


推导:
1.由标量场的复共轭
\begin{equation}
    \phi ^{\dagger}(x)=\int{\frac{\mathrm{d}^3p}{\left( 2\pi \right) ^3}}\frac{1}{\sqrt{2E_{\mathbf{p}}}}\left( b_{\mathbf{p}}\mathrm{e}^{-\mathrm{i}p\cdot x}+a_{\mathbf{p}}^{\dagger}\mathrm{e}^{\mathrm{i}p\cdot x} \right) 
\end{equation}
得到
\begin{equation}
    \phi ^{\dagger}(\mathcal{P} x)=\int{\frac{\mathrm{d}^3p}{\left( 2\pi \right) ^3}}\frac{1}{\sqrt{2E_{\mathbf{p}}}}\left( b_{\mathbf{p}}\mathrm{e}^{-\mathrm{i}p\cdot \left( \mathcal{P} x \right)}+a_{\mathbf{p}}^{\dagger}\mathrm{e}^{\mathrm{i}p\cdot \left( \mathcal{P} x \right)} \right) 
\end{equation}
2.计算
\begin{equation}
    \begin{aligned}
        P^{-1}\phi ^{\dagger}(x)P&=\int{\frac{\mathrm{d}^3p}{\left( 2\pi \right) ^3}}\frac{1}{\sqrt{2E_{\mathbf{p}}}}\left( P^{-1}b_{\mathbf{p}}P\mathrm{e}^{-\mathrm{i}p\cdot x}+P^{-1}a_{\mathbf{p}}^{\dagger}P\mathrm{e}^{\mathrm{i}p\cdot x} \right) 
\\
&=\int{\frac{\mathrm{d}^3p}{\left( 2\pi \right) ^3}}\frac{1}{\sqrt{2E_{\mathbf{p}}}}\left( \tilde{\eta}_{P}^{*}b_{-\mathbf{p}}\mathrm{e}^{-\mathrm{i}p\cdot x}+\eta _Pa_{-\mathbf{p}}^{\dagger}\mathrm{e}^{\mathrm{i}p\cdot x} \right) 
\\
&=\int{\frac{\mathrm{d}^3p}{\left( 2\pi \right) ^3}}\frac{1}{\sqrt{2E_{\mathbf{p}}}}\left( \tilde{\eta}_{P}^{*}b_{\mathbf{p}}\mathrm{e}^{-\mathrm{i}\left( \mathcal{P} p \right) \cdot x}+\eta _Pa_{\mathbf{p}}^{\dagger}\mathrm{e}^{\mathrm{i}\left( \mathcal{P} p \right) \cdot x} \right) 
\\
&=\int{\frac{\mathrm{d}^3p}{\left( 2\pi \right) ^3}}\frac{1}{\sqrt{2E_{\mathbf{p}}}}\left( \tilde{\eta}_{P}^{*}b_{\mathbf{p}}\mathrm{e}^{-\mathrm{i}p\cdot \left( \mathcal{P} x \right)}+\eta _Pa_{\mathbf{p}}^{\dagger}\mathrm{e}^{\mathrm{i}p\cdot \left( \mathcal{P} x \right)} \right) 
\\
&=\int{\frac{\mathrm{d}^3p}{\left( 2\pi \right) ^3}}\frac{1}{\sqrt{2E_{\mathbf{p}}}}\left( \eta _Pb_{\mathbf{p}}\mathrm{e}^{-\mathrm{i}p\cdot \left( \mathcal{P} x \right)}+\eta _Pa_{\mathbf{p}}^{\dagger}\mathrm{e}^{\mathrm{i}p\cdot \left( \mathcal{P} x \right)} \right) 
\\
&=\eta _P\int{\frac{\mathrm{d}^3p}{\left( 2\pi \right) ^3}}\frac{1}{\sqrt{2E_{\mathbf{p}}}}\left( b_{\mathbf{p}}\mathrm{e}^{-\mathrm{i}p\cdot \left( \mathcal{P} x \right)}+a_{\mathbf{p}}^{\dagger}\mathrm{e}^{\mathrm{i}p\cdot \left( \mathcal{P} x \right)} \right) 
    \end{aligned}
\end{equation}
得到
\begin{equation}
    P^{-1}\phi ^{\dagger}(x)P=\eta _P\phi ^{\dagger}(\mathcal{P} x)
\end{equation}



\subsubsection{标量场的T变换}

时间反演变换
\begin{equation}
    {\mathcal{T} ^{\mu}}_{\nu}=(\mathcal{T} ^{-1}{)^{\mu}}_{\nu}=\left( \begin{matrix}
	-1&		&		&		\\
	&		+1&		&		\\
	&		&		+1&		\\
	&		&		&		+1\\
\end{matrix} \right) 
\end{equation}
时空坐标变换
\begin{equation}
    x^{\mu}=\left( t,\mathbf{x} \right) \Rightarrow \,\,x^{\prime \mu}={\mathcal{T} ^{\mu}}_{\nu}x^{\nu}=(\mathcal{T} x)^{\mu}=\left( -t,\mathbf{x} \right) 
\end{equation}
四维动量变换
\begin{equation}
    p^{\mu}=\left( E,\mathbf{p} \right) \Rightarrow \,\,p^{\prime \mu}={\mathcal{T} ^{\mu}}_{\nu}p^{\nu}=(\mathcal{T} p)^{\mu}=\left( -E,\mathbf{p} \right) 
\end{equation}
时空导数变换
\begin{equation}
    \partial _{\mu}^{\prime}=(\mathcal{T} ^{-1}{)^{\nu}}_{\mu}\partial _{\nu}
\end{equation}
时间反演变换保持时空体积元不变
\begin{equation}
    \mathrm{d}^4x^{\prime}=\left| \det\mathrm{(}\mathcal{T} ) \right|\mathrm{d}^4x=\mathrm{d}^4x
\end{equation}


推导:
1.由标量场
\begin{equation}
    \phi (x)=\int{\frac{\mathrm{d}^3p}{\left( 2\pi \right) ^3}\frac{1}{\sqrt{2E_{\mathbf{p}}}}\left( a_{\mathbf{p}}\mathrm{e}^{-\mathrm{i}p\cdot x}+b_{\mathbf{p}}^{\dagger}\mathrm{e}^{\mathrm{i}p\cdot x} \right)}
\end{equation}
得到时间反演变换后的标量场
\begin{equation}
    \phi (\mathcal{T} x)=\int{\frac{\mathrm{d}^3p}{\left( 2\pi \right) ^3}\frac{1}{\sqrt{2E_{\mathbf{p}}}}\left( a_{\mathbf{p}}\mathrm{e}^{-\mathrm{i}p\cdot \left( \mathcal{T} x \right)}+b_{\mathbf{p}}^{\dagger}\mathrm{e}^{\mathrm{i}p\cdot \left( \mathcal{T} x \right)} \right)}
\end{equation}
2.计算
\begin{equation}
    \begin{aligned}
        T^{-1}\phi \left( x \right) T&=\int{\frac{\mathrm{d}^3p}{\left( 2\pi \right) ^3}\frac{1}{\sqrt{2E_{\mathbf{p}}}}T^{-1}\left( a_{\mathbf{p}}\mathrm{e}^{-\mathrm{i}p\cdot x}+b_{\mathbf{p}}^{\dagger}\mathrm{e}^{\mathrm{i}p\cdot x} \right)}T
\\
&=\int{\frac{\mathrm{d}^3p}{\left( 2\pi \right) ^3}}\frac{1}{\sqrt{2E_{\mathbf{p}}}}\left( T^{-1}a_{\mathbf{p}}TT^{-1}\mathrm{e}^{-\mathrm{i}p\cdot x}T+T^{-1}b_{\mathbf{p}}^{\dagger}TT^{-1}\mathrm{e}^{\mathrm{i}p\cdot x}T \right) 
\\
&=\int{\frac{\mathrm{d}^3p}{\left( 2\pi \right) ^3}}\frac{1}{\sqrt{2E_{\mathbf{p}}}}\left( \eta _{T}^{*}a_{-\mathbf{p}}\mathrm{e}^{\mathrm{i}p\cdot x}+\tilde{\eta}_Tb_{-\mathbf{p}}^{\dagger}\mathrm{e}^{-\mathrm{i}p\cdot x} \right) 
\\
&=\int{\frac{\mathrm{d}^3p}{\left( 2\pi \right) ^3}}\frac{1}{\sqrt{2E_{\mathbf{p}}}}\left( \eta _{T}^{*}a_{\mathbf{p}}\mathrm{e}^{\mathrm{i}\left( \mathcal{P} p \right) \cdot x}+\tilde{\eta}_Tb_{\mathbf{p}}^{\dagger}\mathrm{e}^{-\mathrm{i}\left( \mathcal{P} p \right) \cdot x} \right) 
\\
&=\int{\frac{\mathrm{d}^3p}{\left( 2\pi \right) ^3}}\frac{1}{\sqrt{2E_{\mathbf{p}}}}\left( \eta _{T}^{*}a_{\mathbf{p}}\mathrm{e}^{\mathrm{i}p\cdot \left( \mathcal{P} x \right)}+\tilde{\eta}_Tb_{\mathbf{p}}^{\dagger}\mathrm{e}^{-\mathrm{i}p\cdot \left( \mathcal{P} x \right)} \right) 
\\
&=\int{\frac{\mathrm{d}^3p}{\left( 2\pi \right) ^3}}\frac{1}{\sqrt{2E_{\mathbf{p}}}}\left( \eta _{T}^{*}a_{\mathbf{p}}\mathrm{e}^{-\mathrm{i}p\cdot (\mathcal{T} x)}+\tilde{\eta}_Tb_{\mathbf{p}}^{\dagger}\mathrm{e}^{\mathrm{i}p\cdot (\mathcal{T} x)} \right) 
\\
&=\int{\frac{\mathrm{d}^3p}{\left( 2\pi \right) ^3}}\frac{1}{\sqrt{2E_{\mathbf{p}}}}\left( \eta _{T}^{*}a_{\mathbf{p}}\mathrm{e}^{-\mathrm{i}p\cdot (\mathcal{T} x)}+\eta _{T}^{*}b_{\mathbf{p}}^{\dagger}\mathrm{e}^{\mathrm{i}p\cdot (\mathcal{T} x)} \right) 
\\
&=\eta _{T}^{*}\int{\frac{\mathrm{d}^3p}{\left( 2\pi \right) ^3}}\frac{1}{\sqrt{2E_{\mathbf{p}}}}\left( a_{\mathbf{p}}\mathrm{e}^{-\mathrm{i}p\cdot (\mathcal{T} x)}+b_{\mathbf{p}}^{\dagger}\mathrm{e}^{\mathrm{i}p\cdot (\mathcal{T} x)} \right) 
    \end{aligned}
\end{equation}
对比得到


1.由
\begin{equation}
    \phi ^{\dagger}(x)=\int{\frac{\mathrm{d}^3p}{\left( 2\pi \right) ^3}}\frac{1}{\sqrt{2E_{\mathbf{p}}}}\left( b_{\mathbf{p}}\mathrm{e}^{-\mathrm{i}p\cdot x}+a_{\mathbf{p}}^{\dagger}\mathrm{e}^{\mathrm{i}p\cdot x} \right) 
\end{equation}
得到
\begin{equation}
    \phi ^{\dagger}(\mathcal{T} x)=\int{\frac{\mathrm{d}^3p}{\left( 2\pi \right) ^3}}\frac{1}{\sqrt{2E_{\mathbf{p}}}}\left( b_{\mathbf{p}}\mathrm{e}^{-\mathrm{i}p\cdot \left( \mathcal{T} x \right)}+a_{\mathbf{p}}^{\dagger}\mathrm{e}^{\mathrm{i}p\cdot \left( \mathcal{T} x \right)} \right) 
\end{equation}
2.计算
\begin{equation}
    \begin{aligned}
        T^{-1}\phi ^{\dagger}\left( x \right) T&=\int{\frac{\mathrm{d}^3p}{\left( 2\pi \right) ^3}}\frac{1}{\sqrt{2E_{\mathbf{p}}}}T^{-1}\left( b_{\mathbf{p}}\mathrm{e}^{-\mathrm{i}p\cdot x}+a_{\mathbf{p}}^{\dagger}\mathrm{e}^{\mathrm{i}p\cdot x} \right) T
\\
&=\int{\frac{\mathrm{d}^3p}{\left( 2\pi \right) ^3}}\frac{1}{\sqrt{2E_{\mathbf{p}}}}\left( T^{-1}b_{\mathbf{p}}TT^{-1}\mathrm{e}^{-\mathrm{i}p\cdot x}T+T^{-1}a_{\mathbf{p}}^{\dagger}TT^{-1}\mathrm{e}^{\mathrm{i}p\cdot x}T \right) 
\\
&=\int{\frac{\mathrm{d}^3p}{\left( 2\pi \right) ^3}}\frac{1}{\sqrt{2E_{\mathbf{p}}}}\left( \tilde{\eta}_{T}^{*}b_{-\mathbf{p}}\mathrm{e}^{\mathrm{i}p\cdot x}+\eta _Ta_{-\mathbf{p}}^{\dagger}\mathrm{e}^{-\mathrm{i}p\cdot x} \right) 
\\
&=\int{\frac{\mathrm{d}^3p}{\left( 2\pi \right) ^3}}\frac{1}{\sqrt{2E_{\mathbf{p}}}}\left( \tilde{\eta}_{T}^{*}b_{\mathbf{p}}\mathrm{e}^{\mathrm{i}\left( \mathcal{P} p \right) \cdot x}+\eta _Ta_{\mathbf{p}}^{\dagger}\mathrm{e}^{-\mathrm{i}\left( \mathcal{P} p \right) \cdot x} \right) 
\\
&=\int{\frac{\mathrm{d}^3p}{\left( 2\pi \right) ^3}}\frac{1}{\sqrt{2E_{\mathbf{p}}}}\left( \tilde{\eta}_{T}^{*}b_{\mathbf{p}}\mathrm{e}^{\mathrm{i}p\cdot \left( \mathcal{P} x \right)}+\eta _Ta_{\mathbf{p}}^{\dagger}\mathrm{e}^{-\mathrm{i}p\cdot \left( \mathcal{P} x \right)} \right) 
\\
&=\int{\frac{\mathrm{d}^3p}{\left( 2\pi \right) ^3}}\frac{1}{\sqrt{2E_{\mathbf{p}}}}\left( \tilde{\eta}_{T}^{*}b_{\mathbf{p}}\mathrm{e}^{-\mathrm{i}p\cdot (\mathcal{T} x)}+\eta _Ta_{\mathbf{p}}^{\dagger}\mathrm{e}^{\mathrm{i}p\cdot (\mathcal{T} x)} \right) 
\\
&=\int{\frac{\mathrm{d}^3p}{\left( 2\pi \right) ^3}}\frac{1}{\sqrt{2E_{\mathbf{p}}}}\left( \eta _Tb_{\mathbf{p}}\mathrm{e}^{-\mathrm{i}p\cdot (\mathcal{T} x)}+\eta _Ta_{\mathbf{p}}^{\dagger}\mathrm{e}^{\mathrm{i}p\cdot (\mathcal{T} x)} \right) 
\\
&=\eta _T\int{\frac{\mathrm{d}^3p}{\left( 2\pi \right) ^3}}\frac{1}{\sqrt{2E_{\mathbf{p}}}}\left( b_{\mathbf{p}}\mathrm{e}^{-\mathrm{i}p\cdot (\mathcal{T} x)}+a_{\mathbf{p}}^{\dagger}\mathrm{e}^{\mathrm{i}p\cdot (\mathcal{T} x)} \right) 
    \end{aligned}
\end{equation}
对比得到
\begin{equation}
    T^{-1}\phi ^{\dagger}\left( x \right) T=\eta _T\phi ^{\dagger}(\mathcal{T} x)
\end{equation}


由标量场
\begin{equation}
    \phi \left( x \right) =\int{\frac{\mathrm{d}^3p}{\left( 2\pi \right) ^3}}\frac{1}{\sqrt{2E_{\mathbf{p}}}}\left( a_{\mathbf{p}}\mathrm{e}^{-\mathrm{i}p\cdot x}+b_{\mathbf{p}}^{\dagger}\mathrm{e}^{\mathrm{i}p\cdot x} \right) 
\end{equation}
和复共轭
\begin{equation}
    \phi ^{\dagger}(x)=\int{\frac{\mathrm{d}^3p}{\left( 2\pi \right) ^3}}\frac{1}{\sqrt{2E_{\mathbf{p}}}}\left( b_{\mathbf{p}}\mathrm{e}^{-\mathrm{i}p\cdot x}+a_{\mathbf{p}}^{\dagger}\mathrm{e}^{\mathrm{i}p\cdot x} \right) 
\end{equation}

2.计算
\begin{equation}
    \begin{aligned}
        C^{-1}\phi C&=\int{\frac{\mathrm{d}^3p}{\left( 2\pi \right) ^3}}\frac{1}{\sqrt{2E_{\mathbf{p}}}}C^{-1}\left( a_{\mathbf{p}}\mathrm{e}^{-\mathrm{i}p\cdot x}+b_{\mathbf{p}}^{\dagger}\mathrm{e}^{\mathrm{i}p\cdot x} \right) C
\\
&=\int{\frac{\mathrm{d}^3p}{\left( 2\pi \right) ^3}}\frac{1}{\sqrt{2E_{\mathbf{p}}}}\left( C^{-1}a_{\mathbf{p}}C\mathrm{e}^{-\mathrm{i}p\cdot x}+C^{-1}b_{\mathbf{p}}^{\dagger}C\mathrm{e}^{\mathrm{i}p\cdot x} \right) 
\\
&=\int{\frac{\mathrm{d}^3p}{\left( 2\pi \right) ^3}}\frac{1}{\sqrt{2E_{\mathbf{p}}}}\left( \eta _{C}^{*}b_{\mathbf{p}}\mathrm{e}^{-\mathrm{i}p\cdot x}+\eta _{C}^{*}a_{\mathbf{p}}^{\dagger}\mathrm{e}^{\mathrm{i}p\cdot x} \right) 
\\
&=\eta _{C}^{*}\int{\frac{\mathrm{d}^3p}{\left( 2\pi \right) ^3}}\frac{1}{\sqrt{2E_{\mathbf{p}}}}\left( b_{\mathbf{p}}\mathrm{e}^{-\mathrm{i}p\cdot x}+a_{\mathbf{p}}^{\dagger}\mathrm{e}^{\mathrm{i}p\cdot x} \right) 
    \end{aligned}
\end{equation}
对比得到
\begin{equation}
    C^{-1}\phi (x)C=\eta _{C}^{*}\phi ^{\dagger}(x)
\end{equation}
同样地
\begin{equation}
    \begin{aligned}
        C^{-1}\phi ^{\dagger}(x)C&=\int{\frac{\mathrm{d}^3p}{\left( 2\pi \right) ^3}}\frac{1}{\sqrt{2E_{\mathbf{p}}}}C^{-1}\left( b_{\mathbf{p}}\mathrm{e}^{-\mathrm{i}p\cdot x}+a_{\mathbf{p}}^{\dagger}\mathrm{e}^{\mathrm{i}p\cdot x} \right) C
\\
&=\int{\frac{\mathrm{d}^3p}{\left( 2\pi \right) ^3}}\frac{1}{\sqrt{2E_{\mathbf{p}}}}\left( C^{-1}b_{\mathbf{p}}C\mathrm{e}^{-\mathrm{i}p\cdot x}+C^{-1}a_{\mathbf{p}}^{\dagger}C\mathrm{e}^{\mathrm{i}p\cdot x} \right) 
\\
&=\int{\frac{\mathrm{d}^3p}{\left( 2\pi \right) ^3}}\frac{1}{\sqrt{2E_{\mathbf{p}}}}\left( \eta _Ca_{\mathbf{p}}\mathrm{e}^{-\mathrm{i}p\cdot x}+\eta _Cb_{\mathbf{p}}^{\dagger}\mathrm{e}^{\mathrm{i}p\cdot x} \right) 
\\
&=\eta _C\int{\frac{\mathrm{d}^3p}{\left( 2\pi \right) ^3}}\frac{1}{\sqrt{2E_{\mathbf{p}}}}\left( a_{\mathbf{p}}\mathrm{e}^{-\mathrm{i}p\cdot x}+b_{\mathbf{p}}^{\dagger}\mathrm{e}^{\mathrm{i}p\cdot x} \right) 
    \end{aligned}
\end{equation}
对比得到
\begin{equation}
    C^{-1}\phi ^{\dagger}(x)C=\eta _C\phi (x)
\end{equation}

\section{9}
%%%%%%%%%%%%%%%%%%%%%%%%%%%%%%%%%%%%%%%%%%%%%%%%%%%%%%%
\subsection{9.1}

宇称变换
\begin{equation}
    {\mathcal{P} ^{\mu}}_{\nu}=(\mathcal{P} ^{-1}{)^{\mu}}_{\nu}=\left( \begin{matrix}
	+1&		&		&		\\
	&		-1&		&		\\
	&		&		-1&		\\
	&		&		&		-1\\
\end{matrix} \right) 
\end{equation}
时空坐标变换
\begin{equation}
    x^{\mu}=\left( t,\mathbf{x} \right) \Rightarrow x^{\prime \mu}={\mathcal{P} ^{\mu}}_{\nu}x^{\nu}=(\mathcal{P} x)^{\mu}=\left( t,-\mathbf{x} \right) 
\end{equation}
四维动量变换
\begin{equation}
    p^{\mu}=\left( E,\mathbf{p} \right) \Rightarrow p^{\prime \mu}={\mathcal{P} ^{\mu}}_{\nu}p^{\nu}=(\mathcal{P} p)^{\mu}=\left( E,-\mathbf{p} \right) 
\end{equation}
时空导数变换
\begin{equation}
    \partial _{\mu}^{\prime}=(\mathcal{P} ^{-1}{)^{\nu}}_{\mu}\partial _{\nu}
\end{equation}
保持时空体积元不变
\begin{equation}
    \mathrm{d}^4x^{\prime}=\left| \det \left( \mathcal{P} \right) \right|\mathrm{d}^4x=\mathrm{d}^4x
\end{equation}

如果场论系统的作用量S在宇称变换下不变,则运动方程的形式也在宇称变换下不变,此时称系统是宇称守恒的,即具有空间反射对称性。


在宇称守恒的量子理论中,宇称变换在 Hilbert 空间中诱导出态矢$|\Psi \rangle$的线性幺正变换
\begin{equation}
    |\Psi ^{\prime}\rangle =U(\mathcal{P} )|\Psi \rangle =P|\Psi \rangle 
\end{equation}





推导:
1.由标量场
\begin{equation}
    \phi (x)=\int{\frac{\mathrm{d}^3p}{\left( 2\pi \right) ^3}}\frac{1}{\sqrt{2E_{\mathbf{p}}}}\left( a_{\mathbf{p}}\mathrm{e}^{-\mathrm{i}p\cdot x}+b_{\mathbf{p}}^{\dagger}\mathrm{e}^{\mathrm{i}p\cdot x} \right) 
\end{equation}
得到宇称变换后的标量场
\begin{equation}
    \phi (\mathcal{P} x)=\int{\frac{\mathrm{d}^3p}{\left( 2\pi \right) ^3}}\frac{1}{\sqrt{2E_{\mathbf{p}}}}\left( a_{\mathbf{p}}\mathrm{e}^{-\mathrm{i}p\cdot \left( \mathcal{P} x \right)}+b_{\mathbf{p}}^{\dagger}\mathrm{e}^{\mathrm{i}p\cdot \left( \mathcal{P} x \right)} \right) 
\end{equation}
2.计算
\begin{equation}
    \begin{aligned}
        P^{-1}\phi (x)P&=\int{\frac{\mathrm{d}^3p}{\left( 2\pi \right) ^3}}\frac{1}{\sqrt{2E_{\mathbf{p}}}}\left( P^{-1}a_{\mathbf{p}}P\mathrm{e}^{-\mathrm{i}p\cdot x}+P^{-1}b_{\mathbf{p}}^{\dagger}P\mathrm{e}^{\mathrm{i}p\cdot x} \right) 
\\
&=\int{\frac{\mathrm{d}^3p}{\left( 2\pi \right) ^3}}\frac{1}{\sqrt{2E_{\mathbf{p}}}}\left( \eta _{P}^{*}a_{-\mathbf{p}}\mathrm{e}^{-\mathrm{i}p\cdot x}+\tilde{\eta}_Pb_{-\mathbf{p}}^{\dagger}\mathrm{e}^{\mathrm{i}p\cdot x} \right) 
\\
&=\int{\frac{\mathrm{d}^3p}{\left( 2\pi \right) ^3}}\frac{1}{\sqrt{2E_{\mathbf{p}}}}\left( \eta _{P}^{*}a_{\mathbf{p}}\mathrm{e}^{-\mathrm{i}\left( \mathcal{P} p \right) \cdot x}+\tilde{\eta}_Pb_{\mathbf{p}}^{\dagger}\mathrm{e}^{\mathrm{i}\left( \mathcal{P} p \right) \cdot x} \right) 
\\
&=\int{\frac{\mathrm{d}^3p}{\left( 2\pi \right) ^3}}\frac{1}{\sqrt{2E_{\mathbf{p}}}}\left( \eta _{P}^{*}a_{\mathbf{p}}\mathrm{e}^{-\mathrm{i}p\cdot \left( \mathcal{P} x \right)}+\tilde{\eta}_Pb_{\mathbf{p}}^{\dagger}\mathrm{e}^{\mathrm{i}p\cdot \left( \mathcal{P} x \right)} \right) 
\\
&=\int{\frac{\mathrm{d}^3p}{\left( 2\pi \right) ^3}}\frac{1}{\sqrt{2E_{\mathbf{p}}}}\left( \eta _{P}^{*}a_{\mathbf{p}}\mathrm{e}^{-\mathrm{i}p\cdot \left( \mathcal{P} x \right)}+\eta _{P}^{*}b_{\mathbf{p}}^{\dagger}\mathrm{e}^{\mathrm{i}p\cdot \left( \mathcal{P} x \right)} \right) 
\\
&=\eta _{P}^{*}\int{\frac{\mathrm{d}^3p}{\left( 2\pi \right) ^3}}\frac{1}{\sqrt{2E_{\mathbf{p}}}}\left( a_{\mathbf{p}}\mathrm{e}^{-\mathrm{i}p\cdot \left( \mathcal{P} x \right)}+b_{\mathbf{p}}^{\dagger}\mathrm{e}^{\mathrm{i}p\cdot \left( \mathcal{P} x \right)} \right) 
    \end{aligned}
\end{equation}
得到
\begin{equation}
    P^{-1}\phi (x)P=\eta _{P}^{*}\phi (\mathcal{P} x)
\end{equation}


推导:
1.由标量场的复共轭
\begin{equation}
    \phi ^{\dagger}(x)=\int{\frac{\mathrm{d}^3p}{\left( 2\pi \right) ^3}}\frac{1}{\sqrt{2E_{\mathbf{p}}}}\left( b_{\mathbf{p}}\mathrm{e}^{-\mathrm{i}p\cdot x}+a_{\mathbf{p}}^{\dagger}\mathrm{e}^{\mathrm{i}p\cdot x} \right) 
\end{equation}
得到
\begin{equation}
    \phi ^{\dagger}(\mathcal{P} x)=\int{\frac{\mathrm{d}^3p}{\left( 2\pi \right) ^3}}\frac{1}{\sqrt{2E_{\mathbf{p}}}}\left( b_{\mathbf{p}}\mathrm{e}^{-\mathrm{i}p\cdot \left( \mathcal{P} x \right)}+a_{\mathbf{p}}^{\dagger}\mathrm{e}^{\mathrm{i}p\cdot \left( \mathcal{P} x \right)} \right) 
\end{equation}
2.计算
\begin{equation}
    \begin{aligned}
        P^{-1}\phi ^{\dagger}(x)P&=\int{\frac{\mathrm{d}^3p}{\left( 2\pi \right) ^3}}\frac{1}{\sqrt{2E_{\mathbf{p}}}}\left( P^{-1}b_{\mathbf{p}}P\mathrm{e}^{-\mathrm{i}p\cdot x}+P^{-1}a_{\mathbf{p}}^{\dagger}P\mathrm{e}^{\mathrm{i}p\cdot x} \right) 
\\
&=\int{\frac{\mathrm{d}^3p}{\left( 2\pi \right) ^3}}\frac{1}{\sqrt{2E_{\mathbf{p}}}}\left( \tilde{\eta}_{P}^{*}b_{-\mathbf{p}}\mathrm{e}^{-\mathrm{i}p\cdot x}+\eta _Pa_{-\mathbf{p}}^{\dagger}\mathrm{e}^{\mathrm{i}p\cdot x} \right) 
\\
&=\int{\frac{\mathrm{d}^3p}{\left( 2\pi \right) ^3}}\frac{1}{\sqrt{2E_{\mathbf{p}}}}\left( \tilde{\eta}_{P}^{*}b_{\mathbf{p}}\mathrm{e}^{-\mathrm{i}\left( \mathcal{P} p \right) \cdot x}+\eta _Pa_{\mathbf{p}}^{\dagger}\mathrm{e}^{\mathrm{i}\left( \mathcal{P} p \right) \cdot x} \right) 
\\
&=\int{\frac{\mathrm{d}^3p}{\left( 2\pi \right) ^3}}\frac{1}{\sqrt{2E_{\mathbf{p}}}}\left( \tilde{\eta}_{P}^{*}b_{\mathbf{p}}\mathrm{e}^{-\mathrm{i}p\cdot \left( \mathcal{P} x \right)}+\eta _Pa_{\mathbf{p}}^{\dagger}\mathrm{e}^{\mathrm{i}p\cdot \left( \mathcal{P} x \right)} \right) 
\\
&=\int{\frac{\mathrm{d}^3p}{\left( 2\pi \right) ^3}}\frac{1}{\sqrt{2E_{\mathbf{p}}}}\left( \eta _Pb_{\mathbf{p}}\mathrm{e}^{-\mathrm{i}p\cdot \left( \mathcal{P} x \right)}+\eta _Pa_{\mathbf{p}}^{\dagger}\mathrm{e}^{\mathrm{i}p\cdot \left( \mathcal{P} x \right)} \right) 
\\
&=\eta _P\int{\frac{\mathrm{d}^3p}{\left( 2\pi \right) ^3}}\frac{1}{\sqrt{2E_{\mathbf{p}}}}\left( b_{\mathbf{p}}\mathrm{e}^{-\mathrm{i}p\cdot \left( \mathcal{P} x \right)}+a_{\mathbf{p}}^{\dagger}\mathrm{e}^{\mathrm{i}p\cdot \left( \mathcal{P} x \right)} \right) 
    \end{aligned}
\end{equation}
得到
\begin{equation}
    P^{-1}\phi ^{\dagger}(x)P=\eta _P\phi ^{\dagger}(\mathcal{P} x)
\end{equation}



\subsubsection{标量场的T变换}

时间反演变换
\begin{equation}
    {\mathcal{T} ^{\mu}}_{\nu}=(\mathcal{T} ^{-1}{)^{\mu}}_{\nu}=\left( \begin{matrix}
	-1&		&		&		\\
	&		+1&		&		\\
	&		&		+1&		\\
	&		&		&		+1\\
\end{matrix} \right) 
\end{equation}
时空坐标变换
\begin{equation}
    x^{\mu}=\left( t,\mathbf{x} \right) \Rightarrow \,\,x^{\prime \mu}={\mathcal{T} ^{\mu}}_{\nu}x^{\nu}=(\mathcal{T} x)^{\mu}=\left( -t,\mathbf{x} \right) 
\end{equation}
四维动量变换
\begin{equation}
    p^{\mu}=\left( E,\mathbf{p} \right) \Rightarrow \,\,p^{\prime \mu}={\mathcal{T} ^{\mu}}_{\nu}p^{\nu}=(\mathcal{T} p)^{\mu}=\left( -E,\mathbf{p} \right) 
\end{equation}
时空导数变换
\begin{equation}
    \partial _{\mu}^{\prime}=(\mathcal{T} ^{-1}{)^{\nu}}_{\mu}\partial _{\nu}
\end{equation}
时间反演变换保持时空体积元不变
\begin{equation}
    \mathrm{d}^4x^{\prime}=\left| \det\mathrm{(}\mathcal{T} ) \right|\mathrm{d}^4x=\mathrm{d}^4x
\end{equation}


推导:
1.由标量场
\begin{equation}
    \phi (x)=\int{\frac{\mathrm{d}^3p}{\left( 2\pi \right) ^3}\frac{1}{\sqrt{2E_{\mathbf{p}}}}\left( a_{\mathbf{p}}\mathrm{e}^{-\mathrm{i}p\cdot x}+b_{\mathbf{p}}^{\dagger}\mathrm{e}^{\mathrm{i}p\cdot x} \right)}
\end{equation}
得到时间反演变换后的标量场
\begin{equation}
    \phi (\mathcal{T} x)=\int{\frac{\mathrm{d}^3p}{\left( 2\pi \right) ^3}\frac{1}{\sqrt{2E_{\mathbf{p}}}}\left( a_{\mathbf{p}}\mathrm{e}^{-\mathrm{i}p\cdot \left( \mathcal{T} x \right)}+b_{\mathbf{p}}^{\dagger}\mathrm{e}^{\mathrm{i}p\cdot \left( \mathcal{T} x \right)} \right)}
\end{equation}
2.计算
\begin{equation}
    \begin{aligned}
        T^{-1}\phi \left( x \right) T&=\int{\frac{\mathrm{d}^3p}{\left( 2\pi \right) ^3}\frac{1}{\sqrt{2E_{\mathbf{p}}}}T^{-1}\left( a_{\mathbf{p}}\mathrm{e}^{-\mathrm{i}p\cdot x}+b_{\mathbf{p}}^{\dagger}\mathrm{e}^{\mathrm{i}p\cdot x} \right)}T
\\
&=\int{\frac{\mathrm{d}^3p}{\left( 2\pi \right) ^3}}\frac{1}{\sqrt{2E_{\mathbf{p}}}}\left( T^{-1}a_{\mathbf{p}}TT^{-1}\mathrm{e}^{-\mathrm{i}p\cdot x}T+T^{-1}b_{\mathbf{p}}^{\dagger}TT^{-1}\mathrm{e}^{\mathrm{i}p\cdot x}T \right) 
\\
&=\int{\frac{\mathrm{d}^3p}{\left( 2\pi \right) ^3}}\frac{1}{\sqrt{2E_{\mathbf{p}}}}\left( \eta _{T}^{*}a_{-\mathbf{p}}\mathrm{e}^{\mathrm{i}p\cdot x}+\tilde{\eta}_Tb_{-\mathbf{p}}^{\dagger}\mathrm{e}^{-\mathrm{i}p\cdot x} \right) 
\\
&=\int{\frac{\mathrm{d}^3p}{\left( 2\pi \right) ^3}}\frac{1}{\sqrt{2E_{\mathbf{p}}}}\left( \eta _{T}^{*}a_{\mathbf{p}}\mathrm{e}^{\mathrm{i}\left( \mathcal{P} p \right) \cdot x}+\tilde{\eta}_Tb_{\mathbf{p}}^{\dagger}\mathrm{e}^{-\mathrm{i}\left( \mathcal{P} p \right) \cdot x} \right) 
\\
&=\int{\frac{\mathrm{d}^3p}{\left( 2\pi \right) ^3}}\frac{1}{\sqrt{2E_{\mathbf{p}}}}\left( \eta _{T}^{*}a_{\mathbf{p}}\mathrm{e}^{\mathrm{i}p\cdot \left( \mathcal{P} x \right)}+\tilde{\eta}_Tb_{\mathbf{p}}^{\dagger}\mathrm{e}^{-\mathrm{i}p\cdot \left( \mathcal{P} x \right)} \right) 
\\
&=\int{\frac{\mathrm{d}^3p}{\left( 2\pi \right) ^3}}\frac{1}{\sqrt{2E_{\mathbf{p}}}}\left( \eta _{T}^{*}a_{\mathbf{p}}\mathrm{e}^{-\mathrm{i}p\cdot (\mathcal{T} x)}+\tilde{\eta}_Tb_{\mathbf{p}}^{\dagger}\mathrm{e}^{\mathrm{i}p\cdot (\mathcal{T} x)} \right) 
\\
&=\int{\frac{\mathrm{d}^3p}{\left( 2\pi \right) ^3}}\frac{1}{\sqrt{2E_{\mathbf{p}}}}\left( \eta _{T}^{*}a_{\mathbf{p}}\mathrm{e}^{-\mathrm{i}p\cdot (\mathcal{T} x)}+\eta _{T}^{*}b_{\mathbf{p}}^{\dagger}\mathrm{e}^{\mathrm{i}p\cdot (\mathcal{T} x)} \right) 
\\
&=\eta _{T}^{*}\int{\frac{\mathrm{d}^3p}{\left( 2\pi \right) ^3}}\frac{1}{\sqrt{2E_{\mathbf{p}}}}\left( a_{\mathbf{p}}\mathrm{e}^{-\mathrm{i}p\cdot (\mathcal{T} x)}+b_{\mathbf{p}}^{\dagger}\mathrm{e}^{\mathrm{i}p\cdot (\mathcal{T} x)} \right) 
    \end{aligned}
\end{equation}
对比得到


1.由
\begin{equation}
    \phi ^{\dagger}(x)=\int{\frac{\mathrm{d}^3p}{\left( 2\pi \right) ^3}}\frac{1}{\sqrt{2E_{\mathbf{p}}}}\left( b_{\mathbf{p}}\mathrm{e}^{-\mathrm{i}p\cdot x}+a_{\mathbf{p}}^{\dagger}\mathrm{e}^{\mathrm{i}p\cdot x} \right) 
\end{equation}
得到
\begin{equation}
    \phi ^{\dagger}(\mathcal{T} x)=\int{\frac{\mathrm{d}^3p}{\left( 2\pi \right) ^3}}\frac{1}{\sqrt{2E_{\mathbf{p}}}}\left( b_{\mathbf{p}}\mathrm{e}^{-\mathrm{i}p\cdot \left( \mathcal{T} x \right)}+a_{\mathbf{p}}^{\dagger}\mathrm{e}^{\mathrm{i}p\cdot \left( \mathcal{T} x \right)} \right) 
\end{equation}
2.计算
\begin{equation}
    \begin{aligned}
        T^{-1}\phi ^{\dagger}\left( x \right) T&=\int{\frac{\mathrm{d}^3p}{\left( 2\pi \right) ^3}}\frac{1}{\sqrt{2E_{\mathbf{p}}}}T^{-1}\left( b_{\mathbf{p}}\mathrm{e}^{-\mathrm{i}p\cdot x}+a_{\mathbf{p}}^{\dagger}\mathrm{e}^{\mathrm{i}p\cdot x} \right) T
\\
&=\int{\frac{\mathrm{d}^3p}{\left( 2\pi \right) ^3}}\frac{1}{\sqrt{2E_{\mathbf{p}}}}\left( T^{-1}b_{\mathbf{p}}TT^{-1}\mathrm{e}^{-\mathrm{i}p\cdot x}T+T^{-1}a_{\mathbf{p}}^{\dagger}TT^{-1}\mathrm{e}^{\mathrm{i}p\cdot x}T \right) 
\\
&=\int{\frac{\mathrm{d}^3p}{\left( 2\pi \right) ^3}}\frac{1}{\sqrt{2E_{\mathbf{p}}}}\left( \tilde{\eta}_{T}^{*}b_{-\mathbf{p}}\mathrm{e}^{\mathrm{i}p\cdot x}+\eta _Ta_{-\mathbf{p}}^{\dagger}\mathrm{e}^{-\mathrm{i}p\cdot x} \right) 
\\
&=\int{\frac{\mathrm{d}^3p}{\left( 2\pi \right) ^3}}\frac{1}{\sqrt{2E_{\mathbf{p}}}}\left( \tilde{\eta}_{T}^{*}b_{\mathbf{p}}\mathrm{e}^{\mathrm{i}\left( \mathcal{P} p \right) \cdot x}+\eta _Ta_{\mathbf{p}}^{\dagger}\mathrm{e}^{-\mathrm{i}\left( \mathcal{P} p \right) \cdot x} \right) 
\\
&=\int{\frac{\mathrm{d}^3p}{\left( 2\pi \right) ^3}}\frac{1}{\sqrt{2E_{\mathbf{p}}}}\left( \tilde{\eta}_{T}^{*}b_{\mathbf{p}}\mathrm{e}^{\mathrm{i}p\cdot \left( \mathcal{P} x \right)}+\eta _Ta_{\mathbf{p}}^{\dagger}\mathrm{e}^{-\mathrm{i}p\cdot \left( \mathcal{P} x \right)} \right) 
\\
&=\int{\frac{\mathrm{d}^3p}{\left( 2\pi \right) ^3}}\frac{1}{\sqrt{2E_{\mathbf{p}}}}\left( \tilde{\eta}_{T}^{*}b_{\mathbf{p}}\mathrm{e}^{-\mathrm{i}p\cdot (\mathcal{T} x)}+\eta _Ta_{\mathbf{p}}^{\dagger}\mathrm{e}^{\mathrm{i}p\cdot (\mathcal{T} x)} \right) 
\\
&=\int{\frac{\mathrm{d}^3p}{\left( 2\pi \right) ^3}}\frac{1}{\sqrt{2E_{\mathbf{p}}}}\left( \eta _Tb_{\mathbf{p}}\mathrm{e}^{-\mathrm{i}p\cdot (\mathcal{T} x)}+\eta _Ta_{\mathbf{p}}^{\dagger}\mathrm{e}^{\mathrm{i}p\cdot (\mathcal{T} x)} \right) 
\\
&=\eta _T\int{\frac{\mathrm{d}^3p}{\left( 2\pi \right) ^3}}\frac{1}{\sqrt{2E_{\mathbf{p}}}}\left( b_{\mathbf{p}}\mathrm{e}^{-\mathrm{i}p\cdot (\mathcal{T} x)}+a_{\mathbf{p}}^{\dagger}\mathrm{e}^{\mathrm{i}p\cdot (\mathcal{T} x)} \right) 
    \end{aligned}
\end{equation}
对比得到
\begin{equation}
    T^{-1}\phi ^{\dagger}\left( x \right) T=\eta _T\phi ^{\dagger}(\mathcal{T} x)
\end{equation}


由标量场
\begin{equation}
    \phi \left( x \right) =\int{\frac{\mathrm{d}^3p}{\left( 2\pi \right) ^3}}\frac{1}{\sqrt{2E_{\mathbf{p}}}}\left( a_{\mathbf{p}}\mathrm{e}^{-\mathrm{i}p\cdot x}+b_{\mathbf{p}}^{\dagger}\mathrm{e}^{\mathrm{i}p\cdot x} \right) 
\end{equation}
和复共轭
\begin{equation}
    \phi ^{\dagger}(x)=\int{\frac{\mathrm{d}^3p}{\left( 2\pi \right) ^3}}\frac{1}{\sqrt{2E_{\mathbf{p}}}}\left( b_{\mathbf{p}}\mathrm{e}^{-\mathrm{i}p\cdot x}+a_{\mathbf{p}}^{\dagger}\mathrm{e}^{\mathrm{i}p\cdot x} \right) 
\end{equation}

2.计算
\begin{equation}
    \begin{aligned}
        C^{-1}\phi C&=\int{\frac{\mathrm{d}^3p}{\left( 2\pi \right) ^3}}\frac{1}{\sqrt{2E_{\mathbf{p}}}}C^{-1}\left( a_{\mathbf{p}}\mathrm{e}^{-\mathrm{i}p\cdot x}+b_{\mathbf{p}}^{\dagger}\mathrm{e}^{\mathrm{i}p\cdot x} \right) C
\\
&=\int{\frac{\mathrm{d}^3p}{\left( 2\pi \right) ^3}}\frac{1}{\sqrt{2E_{\mathbf{p}}}}\left( C^{-1}a_{\mathbf{p}}C\mathrm{e}^{-\mathrm{i}p\cdot x}+C^{-1}b_{\mathbf{p}}^{\dagger}C\mathrm{e}^{\mathrm{i}p\cdot x} \right) 
\\
&=\int{\frac{\mathrm{d}^3p}{\left( 2\pi \right) ^3}}\frac{1}{\sqrt{2E_{\mathbf{p}}}}\left( \eta _{C}^{*}b_{\mathbf{p}}\mathrm{e}^{-\mathrm{i}p\cdot x}+\eta _{C}^{*}a_{\mathbf{p}}^{\dagger}\mathrm{e}^{\mathrm{i}p\cdot x} \right) 
\\
&=\eta _{C}^{*}\int{\frac{\mathrm{d}^3p}{\left( 2\pi \right) ^3}}\frac{1}{\sqrt{2E_{\mathbf{p}}}}\left( b_{\mathbf{p}}\mathrm{e}^{-\mathrm{i}p\cdot x}+a_{\mathbf{p}}^{\dagger}\mathrm{e}^{\mathrm{i}p\cdot x} \right) 
    \end{aligned}
\end{equation}
对比得到
\begin{equation}
    C^{-1}\phi (x)C=\eta _{C}^{*}\phi ^{\dagger}(x)
\end{equation}
同样地
\begin{equation}
    \begin{aligned}
        C^{-1}\phi ^{\dagger}(x)C&=\int{\frac{\mathrm{d}^3p}{\left( 2\pi \right) ^3}}\frac{1}{\sqrt{2E_{\mathbf{p}}}}C^{-1}\left( b_{\mathbf{p}}\mathrm{e}^{-\mathrm{i}p\cdot x}+a_{\mathbf{p}}^{\dagger}\mathrm{e}^{\mathrm{i}p\cdot x} \right) C
\\
&=\int{\frac{\mathrm{d}^3p}{\left( 2\pi \right) ^3}}\frac{1}{\sqrt{2E_{\mathbf{p}}}}\left( C^{-1}b_{\mathbf{p}}C\mathrm{e}^{-\mathrm{i}p\cdot x}+C^{-1}a_{\mathbf{p}}^{\dagger}C\mathrm{e}^{\mathrm{i}p\cdot x} \right) 
\\
&=\int{\frac{\mathrm{d}^3p}{\left( 2\pi \right) ^3}}\frac{1}{\sqrt{2E_{\mathbf{p}}}}\left( \eta _Ca_{\mathbf{p}}\mathrm{e}^{-\mathrm{i}p\cdot x}+\eta _Cb_{\mathbf{p}}^{\dagger}\mathrm{e}^{\mathrm{i}p\cdot x} \right) 
\\
&=\eta _C\int{\frac{\mathrm{d}^3p}{\left( 2\pi \right) ^3}}\frac{1}{\sqrt{2E_{\mathbf{p}}}}\left( a_{\mathbf{p}}\mathrm{e}^{-\mathrm{i}p\cdot x}+b_{\mathbf{p}}^{\dagger}\mathrm{e}^{\mathrm{i}p\cdot x} \right) 
    \end{aligned}
\end{equation}
对比得到
\begin{equation}
    C^{-1}\phi ^{\dagger}(x)C=\eta _C\phi (x)
\end{equation}

\section{9}
%%%%%%%%%%%%%%%%%%%%%%%%%%%%%%%%%%%%%%%%%%%%%%%%%%%%%%%
\subsection{9.1}

宇称变换
\begin{equation}
    {\mathcal{P} ^{\mu}}_{\nu}=(\mathcal{P} ^{-1}{)^{\mu}}_{\nu}=\left( \begin{matrix}
	+1&		&		&		\\
	&		-1&		&		\\
	&		&		-1&		\\
	&		&		&		-1\\
\end{matrix} \right) 
\end{equation}
时空坐标变换
\begin{equation}
    x^{\mu}=\left( t,\mathbf{x} \right) \Rightarrow x^{\prime \mu}={\mathcal{P} ^{\mu}}_{\nu}x^{\nu}=(\mathcal{P} x)^{\mu}=\left( t,-\mathbf{x} \right) 
\end{equation}
四维动量变换
\begin{equation}
    p^{\mu}=\left( E,\mathbf{p} \right) \Rightarrow p^{\prime \mu}={\mathcal{P} ^{\mu}}_{\nu}p^{\nu}=(\mathcal{P} p)^{\mu}=\left( E,-\mathbf{p} \right) 
\end{equation}
时空导数变换
\begin{equation}
    \partial _{\mu}^{\prime}=(\mathcal{P} ^{-1}{)^{\nu}}_{\mu}\partial _{\nu}
\end{equation}
保持时空体积元不变
\begin{equation}
    \mathrm{d}^4x^{\prime}=\left| \det \left( \mathcal{P} \right) \right|\mathrm{d}^4x=\mathrm{d}^4x
\end{equation}

如果场论系统的作用量S在宇称变换下不变,则运动方程的形式也在宇称变换下不变,此时称系统是宇称守恒的,即具有空间反射对称性。


在宇称守恒的量子理论中,宇称变换在 Hilbert 空间中诱导出态矢$|\Psi \rangle$的线性幺正变换
\begin{equation}
    |\Psi ^{\prime}\rangle =U(\mathcal{P} )|\Psi \rangle =P|\Psi \rangle 
\end{equation}





推导:
1.由标量场
\begin{equation}
    \phi (x)=\int{\frac{\mathrm{d}^3p}{\left( 2\pi \right) ^3}}\frac{1}{\sqrt{2E_{\mathbf{p}}}}\left( a_{\mathbf{p}}\mathrm{e}^{-\mathrm{i}p\cdot x}+b_{\mathbf{p}}^{\dagger}\mathrm{e}^{\mathrm{i}p\cdot x} \right) 
\end{equation}
得到宇称变换后的标量场
\begin{equation}
    \phi (\mathcal{P} x)=\int{\frac{\mathrm{d}^3p}{\left( 2\pi \right) ^3}}\frac{1}{\sqrt{2E_{\mathbf{p}}}}\left( a_{\mathbf{p}}\mathrm{e}^{-\mathrm{i}p\cdot \left( \mathcal{P} x \right)}+b_{\mathbf{p}}^{\dagger}\mathrm{e}^{\mathrm{i}p\cdot \left( \mathcal{P} x \right)} \right) 
\end{equation}
2.计算
\begin{equation}
    \begin{aligned}
        P^{-1}\phi (x)P&=\int{\frac{\mathrm{d}^3p}{\left( 2\pi \right) ^3}}\frac{1}{\sqrt{2E_{\mathbf{p}}}}\left( P^{-1}a_{\mathbf{p}}P\mathrm{e}^{-\mathrm{i}p\cdot x}+P^{-1}b_{\mathbf{p}}^{\dagger}P\mathrm{e}^{\mathrm{i}p\cdot x} \right) 
\\
&=\int{\frac{\mathrm{d}^3p}{\left( 2\pi \right) ^3}}\frac{1}{\sqrt{2E_{\mathbf{p}}}}\left( \eta _{P}^{*}a_{-\mathbf{p}}\mathrm{e}^{-\mathrm{i}p\cdot x}+\tilde{\eta}_Pb_{-\mathbf{p}}^{\dagger}\mathrm{e}^{\mathrm{i}p\cdot x} \right) 
\\
&=\int{\frac{\mathrm{d}^3p}{\left( 2\pi \right) ^3}}\frac{1}{\sqrt{2E_{\mathbf{p}}}}\left( \eta _{P}^{*}a_{\mathbf{p}}\mathrm{e}^{-\mathrm{i}\left( \mathcal{P} p \right) \cdot x}+\tilde{\eta}_Pb_{\mathbf{p}}^{\dagger}\mathrm{e}^{\mathrm{i}\left( \mathcal{P} p \right) \cdot x} \right) 
\\
&=\int{\frac{\mathrm{d}^3p}{\left( 2\pi \right) ^3}}\frac{1}{\sqrt{2E_{\mathbf{p}}}}\left( \eta _{P}^{*}a_{\mathbf{p}}\mathrm{e}^{-\mathrm{i}p\cdot \left( \mathcal{P} x \right)}+\tilde{\eta}_Pb_{\mathbf{p}}^{\dagger}\mathrm{e}^{\mathrm{i}p\cdot \left( \mathcal{P} x \right)} \right) 
\\
&=\int{\frac{\mathrm{d}^3p}{\left( 2\pi \right) ^3}}\frac{1}{\sqrt{2E_{\mathbf{p}}}}\left( \eta _{P}^{*}a_{\mathbf{p}}\mathrm{e}^{-\mathrm{i}p\cdot \left( \mathcal{P} x \right)}+\eta _{P}^{*}b_{\mathbf{p}}^{\dagger}\mathrm{e}^{\mathrm{i}p\cdot \left( \mathcal{P} x \right)} \right) 
\\
&=\eta _{P}^{*}\int{\frac{\mathrm{d}^3p}{\left( 2\pi \right) ^3}}\frac{1}{\sqrt{2E_{\mathbf{p}}}}\left( a_{\mathbf{p}}\mathrm{e}^{-\mathrm{i}p\cdot \left( \mathcal{P} x \right)}+b_{\mathbf{p}}^{\dagger}\mathrm{e}^{\mathrm{i}p\cdot \left( \mathcal{P} x \right)} \right) 
    \end{aligned}
\end{equation}
得到
\begin{equation}
    P^{-1}\phi (x)P=\eta _{P}^{*}\phi (\mathcal{P} x)
\end{equation}


推导:
1.由标量场的复共轭
\begin{equation}
    \phi ^{\dagger}(x)=\int{\frac{\mathrm{d}^3p}{\left( 2\pi \right) ^3}}\frac{1}{\sqrt{2E_{\mathbf{p}}}}\left( b_{\mathbf{p}}\mathrm{e}^{-\mathrm{i}p\cdot x}+a_{\mathbf{p}}^{\dagger}\mathrm{e}^{\mathrm{i}p\cdot x} \right) 
\end{equation}
得到
\begin{equation}
    \phi ^{\dagger}(\mathcal{P} x)=\int{\frac{\mathrm{d}^3p}{\left( 2\pi \right) ^3}}\frac{1}{\sqrt{2E_{\mathbf{p}}}}\left( b_{\mathbf{p}}\mathrm{e}^{-\mathrm{i}p\cdot \left( \mathcal{P} x \right)}+a_{\mathbf{p}}^{\dagger}\mathrm{e}^{\mathrm{i}p\cdot \left( \mathcal{P} x \right)} \right) 
\end{equation}
2.计算
\begin{equation}
    \begin{aligned}
        P^{-1}\phi ^{\dagger}(x)P&=\int{\frac{\mathrm{d}^3p}{\left( 2\pi \right) ^3}}\frac{1}{\sqrt{2E_{\mathbf{p}}}}\left( P^{-1}b_{\mathbf{p}}P\mathrm{e}^{-\mathrm{i}p\cdot x}+P^{-1}a_{\mathbf{p}}^{\dagger}P\mathrm{e}^{\mathrm{i}p\cdot x} \right) 
\\
&=\int{\frac{\mathrm{d}^3p}{\left( 2\pi \right) ^3}}\frac{1}{\sqrt{2E_{\mathbf{p}}}}\left( \tilde{\eta}_{P}^{*}b_{-\mathbf{p}}\mathrm{e}^{-\mathrm{i}p\cdot x}+\eta _Pa_{-\mathbf{p}}^{\dagger}\mathrm{e}^{\mathrm{i}p\cdot x} \right) 
\\
&=\int{\frac{\mathrm{d}^3p}{\left( 2\pi \right) ^3}}\frac{1}{\sqrt{2E_{\mathbf{p}}}}\left( \tilde{\eta}_{P}^{*}b_{\mathbf{p}}\mathrm{e}^{-\mathrm{i}\left( \mathcal{P} p \right) \cdot x}+\eta _Pa_{\mathbf{p}}^{\dagger}\mathrm{e}^{\mathrm{i}\left( \mathcal{P} p \right) \cdot x} \right) 
\\
&=\int{\frac{\mathrm{d}^3p}{\left( 2\pi \right) ^3}}\frac{1}{\sqrt{2E_{\mathbf{p}}}}\left( \tilde{\eta}_{P}^{*}b_{\mathbf{p}}\mathrm{e}^{-\mathrm{i}p\cdot \left( \mathcal{P} x \right)}+\eta _Pa_{\mathbf{p}}^{\dagger}\mathrm{e}^{\mathrm{i}p\cdot \left( \mathcal{P} x \right)} \right) 
\\
&=\int{\frac{\mathrm{d}^3p}{\left( 2\pi \right) ^3}}\frac{1}{\sqrt{2E_{\mathbf{p}}}}\left( \eta _Pb_{\mathbf{p}}\mathrm{e}^{-\mathrm{i}p\cdot \left( \mathcal{P} x \right)}+\eta _Pa_{\mathbf{p}}^{\dagger}\mathrm{e}^{\mathrm{i}p\cdot \left( \mathcal{P} x \right)} \right) 
\\
&=\eta _P\int{\frac{\mathrm{d}^3p}{\left( 2\pi \right) ^3}}\frac{1}{\sqrt{2E_{\mathbf{p}}}}\left( b_{\mathbf{p}}\mathrm{e}^{-\mathrm{i}p\cdot \left( \mathcal{P} x \right)}+a_{\mathbf{p}}^{\dagger}\mathrm{e}^{\mathrm{i}p\cdot \left( \mathcal{P} x \right)} \right) 
    \end{aligned}
\end{equation}
得到
\begin{equation}
    P^{-1}\phi ^{\dagger}(x)P=\eta _P\phi ^{\dagger}(\mathcal{P} x)
\end{equation}



\subsubsection{标量场的T变换}

时间反演变换
\begin{equation}
    {\mathcal{T} ^{\mu}}_{\nu}=(\mathcal{T} ^{-1}{)^{\mu}}_{\nu}=\left( \begin{matrix}
	-1&		&		&		\\
	&		+1&		&		\\
	&		&		+1&		\\
	&		&		&		+1\\
\end{matrix} \right) 
\end{equation}
时空坐标变换
\begin{equation}
    x^{\mu}=\left( t,\mathbf{x} \right) \Rightarrow \,\,x^{\prime \mu}={\mathcal{T} ^{\mu}}_{\nu}x^{\nu}=(\mathcal{T} x)^{\mu}=\left( -t,\mathbf{x} \right) 
\end{equation}
四维动量变换
\begin{equation}
    p^{\mu}=\left( E,\mathbf{p} \right) \Rightarrow \,\,p^{\prime \mu}={\mathcal{T} ^{\mu}}_{\nu}p^{\nu}=(\mathcal{T} p)^{\mu}=\left( -E,\mathbf{p} \right) 
\end{equation}
时空导数变换
\begin{equation}
    \partial _{\mu}^{\prime}=(\mathcal{T} ^{-1}{)^{\nu}}_{\mu}\partial _{\nu}
\end{equation}
时间反演变换保持时空体积元不变
\begin{equation}
    \mathrm{d}^4x^{\prime}=\left| \det\mathrm{(}\mathcal{T} ) \right|\mathrm{d}^4x=\mathrm{d}^4x
\end{equation}


推导:
1.由标量场
\begin{equation}
    \phi (x)=\int{\frac{\mathrm{d}^3p}{\left( 2\pi \right) ^3}\frac{1}{\sqrt{2E_{\mathbf{p}}}}\left( a_{\mathbf{p}}\mathrm{e}^{-\mathrm{i}p\cdot x}+b_{\mathbf{p}}^{\dagger}\mathrm{e}^{\mathrm{i}p\cdot x} \right)}
\end{equation}
得到时间反演变换后的标量场
\begin{equation}
    \phi (\mathcal{T} x)=\int{\frac{\mathrm{d}^3p}{\left( 2\pi \right) ^3}\frac{1}{\sqrt{2E_{\mathbf{p}}}}\left( a_{\mathbf{p}}\mathrm{e}^{-\mathrm{i}p\cdot \left( \mathcal{T} x \right)}+b_{\mathbf{p}}^{\dagger}\mathrm{e}^{\mathrm{i}p\cdot \left( \mathcal{T} x \right)} \right)}
\end{equation}
2.计算
\begin{equation}
    \begin{aligned}
        T^{-1}\phi \left( x \right) T&=\int{\frac{\mathrm{d}^3p}{\left( 2\pi \right) ^3}\frac{1}{\sqrt{2E_{\mathbf{p}}}}T^{-1}\left( a_{\mathbf{p}}\mathrm{e}^{-\mathrm{i}p\cdot x}+b_{\mathbf{p}}^{\dagger}\mathrm{e}^{\mathrm{i}p\cdot x} \right)}T
\\
&=\int{\frac{\mathrm{d}^3p}{\left( 2\pi \right) ^3}}\frac{1}{\sqrt{2E_{\mathbf{p}}}}\left( T^{-1}a_{\mathbf{p}}TT^{-1}\mathrm{e}^{-\mathrm{i}p\cdot x}T+T^{-1}b_{\mathbf{p}}^{\dagger}TT^{-1}\mathrm{e}^{\mathrm{i}p\cdot x}T \right) 
\\
&=\int{\frac{\mathrm{d}^3p}{\left( 2\pi \right) ^3}}\frac{1}{\sqrt{2E_{\mathbf{p}}}}\left( \eta _{T}^{*}a_{-\mathbf{p}}\mathrm{e}^{\mathrm{i}p\cdot x}+\tilde{\eta}_Tb_{-\mathbf{p}}^{\dagger}\mathrm{e}^{-\mathrm{i}p\cdot x} \right) 
\\
&=\int{\frac{\mathrm{d}^3p}{\left( 2\pi \right) ^3}}\frac{1}{\sqrt{2E_{\mathbf{p}}}}\left( \eta _{T}^{*}a_{\mathbf{p}}\mathrm{e}^{\mathrm{i}\left( \mathcal{P} p \right) \cdot x}+\tilde{\eta}_Tb_{\mathbf{p}}^{\dagger}\mathrm{e}^{-\mathrm{i}\left( \mathcal{P} p \right) \cdot x} \right) 
\\
&=\int{\frac{\mathrm{d}^3p}{\left( 2\pi \right) ^3}}\frac{1}{\sqrt{2E_{\mathbf{p}}}}\left( \eta _{T}^{*}a_{\mathbf{p}}\mathrm{e}^{\mathrm{i}p\cdot \left( \mathcal{P} x \right)}+\tilde{\eta}_Tb_{\mathbf{p}}^{\dagger}\mathrm{e}^{-\mathrm{i}p\cdot \left( \mathcal{P} x \right)} \right) 
\\
&=\int{\frac{\mathrm{d}^3p}{\left( 2\pi \right) ^3}}\frac{1}{\sqrt{2E_{\mathbf{p}}}}\left( \eta _{T}^{*}a_{\mathbf{p}}\mathrm{e}^{-\mathrm{i}p\cdot (\mathcal{T} x)}+\tilde{\eta}_Tb_{\mathbf{p}}^{\dagger}\mathrm{e}^{\mathrm{i}p\cdot (\mathcal{T} x)} \right) 
\\
&=\int{\frac{\mathrm{d}^3p}{\left( 2\pi \right) ^3}}\frac{1}{\sqrt{2E_{\mathbf{p}}}}\left( \eta _{T}^{*}a_{\mathbf{p}}\mathrm{e}^{-\mathrm{i}p\cdot (\mathcal{T} x)}+\eta _{T}^{*}b_{\mathbf{p}}^{\dagger}\mathrm{e}^{\mathrm{i}p\cdot (\mathcal{T} x)} \right) 
\\
&=\eta _{T}^{*}\int{\frac{\mathrm{d}^3p}{\left( 2\pi \right) ^3}}\frac{1}{\sqrt{2E_{\mathbf{p}}}}\left( a_{\mathbf{p}}\mathrm{e}^{-\mathrm{i}p\cdot (\mathcal{T} x)}+b_{\mathbf{p}}^{\dagger}\mathrm{e}^{\mathrm{i}p\cdot (\mathcal{T} x)} \right) 
    \end{aligned}
\end{equation}
对比得到


1.由
\begin{equation}
    \phi ^{\dagger}(x)=\int{\frac{\mathrm{d}^3p}{\left( 2\pi \right) ^3}}\frac{1}{\sqrt{2E_{\mathbf{p}}}}\left( b_{\mathbf{p}}\mathrm{e}^{-\mathrm{i}p\cdot x}+a_{\mathbf{p}}^{\dagger}\mathrm{e}^{\mathrm{i}p\cdot x} \right) 
\end{equation}
得到
\begin{equation}
    \phi ^{\dagger}(\mathcal{T} x)=\int{\frac{\mathrm{d}^3p}{\left( 2\pi \right) ^3}}\frac{1}{\sqrt{2E_{\mathbf{p}}}}\left( b_{\mathbf{p}}\mathrm{e}^{-\mathrm{i}p\cdot \left( \mathcal{T} x \right)}+a_{\mathbf{p}}^{\dagger}\mathrm{e}^{\mathrm{i}p\cdot \left( \mathcal{T} x \right)} \right) 
\end{equation}
2.计算
\begin{equation}
    \begin{aligned}
        T^{-1}\phi ^{\dagger}\left( x \right) T&=\int{\frac{\mathrm{d}^3p}{\left( 2\pi \right) ^3}}\frac{1}{\sqrt{2E_{\mathbf{p}}}}T^{-1}\left( b_{\mathbf{p}}\mathrm{e}^{-\mathrm{i}p\cdot x}+a_{\mathbf{p}}^{\dagger}\mathrm{e}^{\mathrm{i}p\cdot x} \right) T
\\
&=\int{\frac{\mathrm{d}^3p}{\left( 2\pi \right) ^3}}\frac{1}{\sqrt{2E_{\mathbf{p}}}}\left( T^{-1}b_{\mathbf{p}}TT^{-1}\mathrm{e}^{-\mathrm{i}p\cdot x}T+T^{-1}a_{\mathbf{p}}^{\dagger}TT^{-1}\mathrm{e}^{\mathrm{i}p\cdot x}T \right) 
\\
&=\int{\frac{\mathrm{d}^3p}{\left( 2\pi \right) ^3}}\frac{1}{\sqrt{2E_{\mathbf{p}}}}\left( \tilde{\eta}_{T}^{*}b_{-\mathbf{p}}\mathrm{e}^{\mathrm{i}p\cdot x}+\eta _Ta_{-\mathbf{p}}^{\dagger}\mathrm{e}^{-\mathrm{i}p\cdot x} \right) 
\\
&=\int{\frac{\mathrm{d}^3p}{\left( 2\pi \right) ^3}}\frac{1}{\sqrt{2E_{\mathbf{p}}}}\left( \tilde{\eta}_{T}^{*}b_{\mathbf{p}}\mathrm{e}^{\mathrm{i}\left( \mathcal{P} p \right) \cdot x}+\eta _Ta_{\mathbf{p}}^{\dagger}\mathrm{e}^{-\mathrm{i}\left( \mathcal{P} p \right) \cdot x} \right) 
\\
&=\int{\frac{\mathrm{d}^3p}{\left( 2\pi \right) ^3}}\frac{1}{\sqrt{2E_{\mathbf{p}}}}\left( \tilde{\eta}_{T}^{*}b_{\mathbf{p}}\mathrm{e}^{\mathrm{i}p\cdot \left( \mathcal{P} x \right)}+\eta _Ta_{\mathbf{p}}^{\dagger}\mathrm{e}^{-\mathrm{i}p\cdot \left( \mathcal{P} x \right)} \right) 
\\
&=\int{\frac{\mathrm{d}^3p}{\left( 2\pi \right) ^3}}\frac{1}{\sqrt{2E_{\mathbf{p}}}}\left( \tilde{\eta}_{T}^{*}b_{\mathbf{p}}\mathrm{e}^{-\mathrm{i}p\cdot (\mathcal{T} x)}+\eta _Ta_{\mathbf{p}}^{\dagger}\mathrm{e}^{\mathrm{i}p\cdot (\mathcal{T} x)} \right) 
\\
&=\int{\frac{\mathrm{d}^3p}{\left( 2\pi \right) ^3}}\frac{1}{\sqrt{2E_{\mathbf{p}}}}\left( \eta _Tb_{\mathbf{p}}\mathrm{e}^{-\mathrm{i}p\cdot (\mathcal{T} x)}+\eta _Ta_{\mathbf{p}}^{\dagger}\mathrm{e}^{\mathrm{i}p\cdot (\mathcal{T} x)} \right) 
\\
&=\eta _T\int{\frac{\mathrm{d}^3p}{\left( 2\pi \right) ^3}}\frac{1}{\sqrt{2E_{\mathbf{p}}}}\left( b_{\mathbf{p}}\mathrm{e}^{-\mathrm{i}p\cdot (\mathcal{T} x)}+a_{\mathbf{p}}^{\dagger}\mathrm{e}^{\mathrm{i}p\cdot (\mathcal{T} x)} \right) 
    \end{aligned}
\end{equation}
对比得到
\begin{equation}
    T^{-1}\phi ^{\dagger}\left( x \right) T=\eta _T\phi ^{\dagger}(\mathcal{T} x)
\end{equation}


由标量场
\begin{equation}
    \phi \left( x \right) =\int{\frac{\mathrm{d}^3p}{\left( 2\pi \right) ^3}}\frac{1}{\sqrt{2E_{\mathbf{p}}}}\left( a_{\mathbf{p}}\mathrm{e}^{-\mathrm{i}p\cdot x}+b_{\mathbf{p}}^{\dagger}\mathrm{e}^{\mathrm{i}p\cdot x} \right) 
\end{equation}
和复共轭
\begin{equation}
    \phi ^{\dagger}(x)=\int{\frac{\mathrm{d}^3p}{\left( 2\pi \right) ^3}}\frac{1}{\sqrt{2E_{\mathbf{p}}}}\left( b_{\mathbf{p}}\mathrm{e}^{-\mathrm{i}p\cdot x}+a_{\mathbf{p}}^{\dagger}\mathrm{e}^{\mathrm{i}p\cdot x} \right) 
\end{equation}

2.计算
\begin{equation}
    \begin{aligned}
        C^{-1}\phi C&=\int{\frac{\mathrm{d}^3p}{\left( 2\pi \right) ^3}}\frac{1}{\sqrt{2E_{\mathbf{p}}}}C^{-1}\left( a_{\mathbf{p}}\mathrm{e}^{-\mathrm{i}p\cdot x}+b_{\mathbf{p}}^{\dagger}\mathrm{e}^{\mathrm{i}p\cdot x} \right) C
\\
&=\int{\frac{\mathrm{d}^3p}{\left( 2\pi \right) ^3}}\frac{1}{\sqrt{2E_{\mathbf{p}}}}\left( C^{-1}a_{\mathbf{p}}C\mathrm{e}^{-\mathrm{i}p\cdot x}+C^{-1}b_{\mathbf{p}}^{\dagger}C\mathrm{e}^{\mathrm{i}p\cdot x} \right) 
\\
&=\int{\frac{\mathrm{d}^3p}{\left( 2\pi \right) ^3}}\frac{1}{\sqrt{2E_{\mathbf{p}}}}\left( \eta _{C}^{*}b_{\mathbf{p}}\mathrm{e}^{-\mathrm{i}p\cdot x}+\eta _{C}^{*}a_{\mathbf{p}}^{\dagger}\mathrm{e}^{\mathrm{i}p\cdot x} \right) 
\\
&=\eta _{C}^{*}\int{\frac{\mathrm{d}^3p}{\left( 2\pi \right) ^3}}\frac{1}{\sqrt{2E_{\mathbf{p}}}}\left( b_{\mathbf{p}}\mathrm{e}^{-\mathrm{i}p\cdot x}+a_{\mathbf{p}}^{\dagger}\mathrm{e}^{\mathrm{i}p\cdot x} \right) 
    \end{aligned}
\end{equation}
对比得到
\begin{equation}
    C^{-1}\phi (x)C=\eta _{C}^{*}\phi ^{\dagger}(x)
\end{equation}
同样地
\begin{equation}
    \begin{aligned}
        C^{-1}\phi ^{\dagger}(x)C&=\int{\frac{\mathrm{d}^3p}{\left( 2\pi \right) ^3}}\frac{1}{\sqrt{2E_{\mathbf{p}}}}C^{-1}\left( b_{\mathbf{p}}\mathrm{e}^{-\mathrm{i}p\cdot x}+a_{\mathbf{p}}^{\dagger}\mathrm{e}^{\mathrm{i}p\cdot x} \right) C
\\
&=\int{\frac{\mathrm{d}^3p}{\left( 2\pi \right) ^3}}\frac{1}{\sqrt{2E_{\mathbf{p}}}}\left( C^{-1}b_{\mathbf{p}}C\mathrm{e}^{-\mathrm{i}p\cdot x}+C^{-1}a_{\mathbf{p}}^{\dagger}C\mathrm{e}^{\mathrm{i}p\cdot x} \right) 
\\
&=\int{\frac{\mathrm{d}^3p}{\left( 2\pi \right) ^3}}\frac{1}{\sqrt{2E_{\mathbf{p}}}}\left( \eta _Ca_{\mathbf{p}}\mathrm{e}^{-\mathrm{i}p\cdot x}+\eta _Cb_{\mathbf{p}}^{\dagger}\mathrm{e}^{\mathrm{i}p\cdot x} \right) 
\\
&=\eta _C\int{\frac{\mathrm{d}^3p}{\left( 2\pi \right) ^3}}\frac{1}{\sqrt{2E_{\mathbf{p}}}}\left( a_{\mathbf{p}}\mathrm{e}^{-\mathrm{i}p\cdot x}+b_{\mathbf{p}}^{\dagger}\mathrm{e}^{\mathrm{i}p\cdot x} \right) 
    \end{aligned}
\end{equation}
对比得到
\begin{equation}
    C^{-1}\phi ^{\dagger}(x)C=\eta _C\phi (x)
\end{equation}

\section{9}
%%%%%%%%%%%%%%%%%%%%%%%%%%%%%%%%%%%%%%%%%%%%%%%%%%%%%%%
\subsection{9.1}

宇称变换
\begin{equation}
    {\mathcal{P} ^{\mu}}_{\nu}=(\mathcal{P} ^{-1}{)^{\mu}}_{\nu}=\left( \begin{matrix}
	+1&		&		&		\\
	&		-1&		&		\\
	&		&		-1&		\\
	&		&		&		-1\\
\end{matrix} \right) 
\end{equation}
时空坐标变换
\begin{equation}
    x^{\mu}=\left( t,\mathbf{x} \right) \Rightarrow x^{\prime \mu}={\mathcal{P} ^{\mu}}_{\nu}x^{\nu}=(\mathcal{P} x)^{\mu}=\left( t,-\mathbf{x} \right) 
\end{equation}
四维动量变换
\begin{equation}
    p^{\mu}=\left( E,\mathbf{p} \right) \Rightarrow p^{\prime \mu}={\mathcal{P} ^{\mu}}_{\nu}p^{\nu}=(\mathcal{P} p)^{\mu}=\left( E,-\mathbf{p} \right) 
\end{equation}
时空导数变换
\begin{equation}
    \partial _{\mu}^{\prime}=(\mathcal{P} ^{-1}{)^{\nu}}_{\mu}\partial _{\nu}
\end{equation}
保持时空体积元不变
\begin{equation}
    \mathrm{d}^4x^{\prime}=\left| \det \left( \mathcal{P} \right) \right|\mathrm{d}^4x=\mathrm{d}^4x
\end{equation}

如果场论系统的作用量S在宇称变换下不变,则运动方程的形式也在宇称变换下不变,此时称系统是宇称守恒的,即具有空间反射对称性。


在宇称守恒的量子理论中,宇称变换在 Hilbert 空间中诱导出态矢$|\Psi \rangle$的线性幺正变换
\begin{equation}
    |\Psi ^{\prime}\rangle =U(\mathcal{P} )|\Psi \rangle =P|\Psi \rangle 
\end{equation}





推导:
1.由标量场
\begin{equation}
    \phi (x)=\int{\frac{\mathrm{d}^3p}{\left( 2\pi \right) ^3}}\frac{1}{\sqrt{2E_{\mathbf{p}}}}\left( a_{\mathbf{p}}\mathrm{e}^{-\mathrm{i}p\cdot x}+b_{\mathbf{p}}^{\dagger}\mathrm{e}^{\mathrm{i}p\cdot x} \right) 
\end{equation}
得到宇称变换后的标量场
\begin{equation}
    \phi (\mathcal{P} x)=\int{\frac{\mathrm{d}^3p}{\left( 2\pi \right) ^3}}\frac{1}{\sqrt{2E_{\mathbf{p}}}}\left( a_{\mathbf{p}}\mathrm{e}^{-\mathrm{i}p\cdot \left( \mathcal{P} x \right)}+b_{\mathbf{p}}^{\dagger}\mathrm{e}^{\mathrm{i}p\cdot \left( \mathcal{P} x \right)} \right) 
\end{equation}
2.计算
\begin{equation}
    \begin{aligned}
        P^{-1}\phi (x)P&=\int{\frac{\mathrm{d}^3p}{\left( 2\pi \right) ^3}}\frac{1}{\sqrt{2E_{\mathbf{p}}}}\left( P^{-1}a_{\mathbf{p}}P\mathrm{e}^{-\mathrm{i}p\cdot x}+P^{-1}b_{\mathbf{p}}^{\dagger}P\mathrm{e}^{\mathrm{i}p\cdot x} \right) 
\\
&=\int{\frac{\mathrm{d}^3p}{\left( 2\pi \right) ^3}}\frac{1}{\sqrt{2E_{\mathbf{p}}}}\left( \eta _{P}^{*}a_{-\mathbf{p}}\mathrm{e}^{-\mathrm{i}p\cdot x}+\tilde{\eta}_Pb_{-\mathbf{p}}^{\dagger}\mathrm{e}^{\mathrm{i}p\cdot x} \right) 
\\
&=\int{\frac{\mathrm{d}^3p}{\left( 2\pi \right) ^3}}\frac{1}{\sqrt{2E_{\mathbf{p}}}}\left( \eta _{P}^{*}a_{\mathbf{p}}\mathrm{e}^{-\mathrm{i}\left( \mathcal{P} p \right) \cdot x}+\tilde{\eta}_Pb_{\mathbf{p}}^{\dagger}\mathrm{e}^{\mathrm{i}\left( \mathcal{P} p \right) \cdot x} \right) 
\\
&=\int{\frac{\mathrm{d}^3p}{\left( 2\pi \right) ^3}}\frac{1}{\sqrt{2E_{\mathbf{p}}}}\left( \eta _{P}^{*}a_{\mathbf{p}}\mathrm{e}^{-\mathrm{i}p\cdot \left( \mathcal{P} x \right)}+\tilde{\eta}_Pb_{\mathbf{p}}^{\dagger}\mathrm{e}^{\mathrm{i}p\cdot \left( \mathcal{P} x \right)} \right) 
\\
&=\int{\frac{\mathrm{d}^3p}{\left( 2\pi \right) ^3}}\frac{1}{\sqrt{2E_{\mathbf{p}}}}\left( \eta _{P}^{*}a_{\mathbf{p}}\mathrm{e}^{-\mathrm{i}p\cdot \left( \mathcal{P} x \right)}+\eta _{P}^{*}b_{\mathbf{p}}^{\dagger}\mathrm{e}^{\mathrm{i}p\cdot \left( \mathcal{P} x \right)} \right) 
\\
&=\eta _{P}^{*}\int{\frac{\mathrm{d}^3p}{\left( 2\pi \right) ^3}}\frac{1}{\sqrt{2E_{\mathbf{p}}}}\left( a_{\mathbf{p}}\mathrm{e}^{-\mathrm{i}p\cdot \left( \mathcal{P} x \right)}+b_{\mathbf{p}}^{\dagger}\mathrm{e}^{\mathrm{i}p\cdot \left( \mathcal{P} x \right)} \right) 
    \end{aligned}
\end{equation}
得到
\begin{equation}
    P^{-1}\phi (x)P=\eta _{P}^{*}\phi (\mathcal{P} x)
\end{equation}


推导:
1.由标量场的复共轭
\begin{equation}
    \phi ^{\dagger}(x)=\int{\frac{\mathrm{d}^3p}{\left( 2\pi \right) ^3}}\frac{1}{\sqrt{2E_{\mathbf{p}}}}\left( b_{\mathbf{p}}\mathrm{e}^{-\mathrm{i}p\cdot x}+a_{\mathbf{p}}^{\dagger}\mathrm{e}^{\mathrm{i}p\cdot x} \right) 
\end{equation}
得到
\begin{equation}
    \phi ^{\dagger}(\mathcal{P} x)=\int{\frac{\mathrm{d}^3p}{\left( 2\pi \right) ^3}}\frac{1}{\sqrt{2E_{\mathbf{p}}}}\left( b_{\mathbf{p}}\mathrm{e}^{-\mathrm{i}p\cdot \left( \mathcal{P} x \right)}+a_{\mathbf{p}}^{\dagger}\mathrm{e}^{\mathrm{i}p\cdot \left( \mathcal{P} x \right)} \right) 
\end{equation}
2.计算
\begin{equation}
    \begin{aligned}
        P^{-1}\phi ^{\dagger}(x)P&=\int{\frac{\mathrm{d}^3p}{\left( 2\pi \right) ^3}}\frac{1}{\sqrt{2E_{\mathbf{p}}}}\left( P^{-1}b_{\mathbf{p}}P\mathrm{e}^{-\mathrm{i}p\cdot x}+P^{-1}a_{\mathbf{p}}^{\dagger}P\mathrm{e}^{\mathrm{i}p\cdot x} \right) 
\\
&=\int{\frac{\mathrm{d}^3p}{\left( 2\pi \right) ^3}}\frac{1}{\sqrt{2E_{\mathbf{p}}}}\left( \tilde{\eta}_{P}^{*}b_{-\mathbf{p}}\mathrm{e}^{-\mathrm{i}p\cdot x}+\eta _Pa_{-\mathbf{p}}^{\dagger}\mathrm{e}^{\mathrm{i}p\cdot x} \right) 
\\
&=\int{\frac{\mathrm{d}^3p}{\left( 2\pi \right) ^3}}\frac{1}{\sqrt{2E_{\mathbf{p}}}}\left( \tilde{\eta}_{P}^{*}b_{\mathbf{p}}\mathrm{e}^{-\mathrm{i}\left( \mathcal{P} p \right) \cdot x}+\eta _Pa_{\mathbf{p}}^{\dagger}\mathrm{e}^{\mathrm{i}\left( \mathcal{P} p \right) \cdot x} \right) 
\\
&=\int{\frac{\mathrm{d}^3p}{\left( 2\pi \right) ^3}}\frac{1}{\sqrt{2E_{\mathbf{p}}}}\left( \tilde{\eta}_{P}^{*}b_{\mathbf{p}}\mathrm{e}^{-\mathrm{i}p\cdot \left( \mathcal{P} x \right)}+\eta _Pa_{\mathbf{p}}^{\dagger}\mathrm{e}^{\mathrm{i}p\cdot \left( \mathcal{P} x \right)} \right) 
\\
&=\int{\frac{\mathrm{d}^3p}{\left( 2\pi \right) ^3}}\frac{1}{\sqrt{2E_{\mathbf{p}}}}\left( \eta _Pb_{\mathbf{p}}\mathrm{e}^{-\mathrm{i}p\cdot \left( \mathcal{P} x \right)}+\eta _Pa_{\mathbf{p}}^{\dagger}\mathrm{e}^{\mathrm{i}p\cdot \left( \mathcal{P} x \right)} \right) 
\\
&=\eta _P\int{\frac{\mathrm{d}^3p}{\left( 2\pi \right) ^3}}\frac{1}{\sqrt{2E_{\mathbf{p}}}}\left( b_{\mathbf{p}}\mathrm{e}^{-\mathrm{i}p\cdot \left( \mathcal{P} x \right)}+a_{\mathbf{p}}^{\dagger}\mathrm{e}^{\mathrm{i}p\cdot \left( \mathcal{P} x \right)} \right) 
    \end{aligned}
\end{equation}
得到
\begin{equation}
    P^{-1}\phi ^{\dagger}(x)P=\eta _P\phi ^{\dagger}(\mathcal{P} x)
\end{equation}



\subsubsection{标量场的T变换}

时间反演变换
\begin{equation}
    {\mathcal{T} ^{\mu}}_{\nu}=(\mathcal{T} ^{-1}{)^{\mu}}_{\nu}=\left( \begin{matrix}
	-1&		&		&		\\
	&		+1&		&		\\
	&		&		+1&		\\
	&		&		&		+1\\
\end{matrix} \right) 
\end{equation}
时空坐标变换
\begin{equation}
    x^{\mu}=\left( t,\mathbf{x} \right) \Rightarrow \,\,x^{\prime \mu}={\mathcal{T} ^{\mu}}_{\nu}x^{\nu}=(\mathcal{T} x)^{\mu}=\left( -t,\mathbf{x} \right) 
\end{equation}
四维动量变换
\begin{equation}
    p^{\mu}=\left( E,\mathbf{p} \right) \Rightarrow \,\,p^{\prime \mu}={\mathcal{T} ^{\mu}}_{\nu}p^{\nu}=(\mathcal{T} p)^{\mu}=\left( -E,\mathbf{p} \right) 
\end{equation}
时空导数变换
\begin{equation}
    \partial _{\mu}^{\prime}=(\mathcal{T} ^{-1}{)^{\nu}}_{\mu}\partial _{\nu}
\end{equation}
时间反演变换保持时空体积元不变
\begin{equation}
    \mathrm{d}^4x^{\prime}=\left| \det\mathrm{(}\mathcal{T} ) \right|\mathrm{d}^4x=\mathrm{d}^4x
\end{equation}


推导:
1.由标量场
\begin{equation}
    \phi (x)=\int{\frac{\mathrm{d}^3p}{\left( 2\pi \right) ^3}\frac{1}{\sqrt{2E_{\mathbf{p}}}}\left( a_{\mathbf{p}}\mathrm{e}^{-\mathrm{i}p\cdot x}+b_{\mathbf{p}}^{\dagger}\mathrm{e}^{\mathrm{i}p\cdot x} \right)}
\end{equation}
得到时间反演变换后的标量场
\begin{equation}
    \phi (\mathcal{T} x)=\int{\frac{\mathrm{d}^3p}{\left( 2\pi \right) ^3}\frac{1}{\sqrt{2E_{\mathbf{p}}}}\left( a_{\mathbf{p}}\mathrm{e}^{-\mathrm{i}p\cdot \left( \mathcal{T} x \right)}+b_{\mathbf{p}}^{\dagger}\mathrm{e}^{\mathrm{i}p\cdot \left( \mathcal{T} x \right)} \right)}
\end{equation}
2.计算
\begin{equation}
    \begin{aligned}
        T^{-1}\phi \left( x \right) T&=\int{\frac{\mathrm{d}^3p}{\left( 2\pi \right) ^3}\frac{1}{\sqrt{2E_{\mathbf{p}}}}T^{-1}\left( a_{\mathbf{p}}\mathrm{e}^{-\mathrm{i}p\cdot x}+b_{\mathbf{p}}^{\dagger}\mathrm{e}^{\mathrm{i}p\cdot x} \right)}T
\\
&=\int{\frac{\mathrm{d}^3p}{\left( 2\pi \right) ^3}}\frac{1}{\sqrt{2E_{\mathbf{p}}}}\left( T^{-1}a_{\mathbf{p}}TT^{-1}\mathrm{e}^{-\mathrm{i}p\cdot x}T+T^{-1}b_{\mathbf{p}}^{\dagger}TT^{-1}\mathrm{e}^{\mathrm{i}p\cdot x}T \right) 
\\
&=\int{\frac{\mathrm{d}^3p}{\left( 2\pi \right) ^3}}\frac{1}{\sqrt{2E_{\mathbf{p}}}}\left( \eta _{T}^{*}a_{-\mathbf{p}}\mathrm{e}^{\mathrm{i}p\cdot x}+\tilde{\eta}_Tb_{-\mathbf{p}}^{\dagger}\mathrm{e}^{-\mathrm{i}p\cdot x} \right) 
\\
&=\int{\frac{\mathrm{d}^3p}{\left( 2\pi \right) ^3}}\frac{1}{\sqrt{2E_{\mathbf{p}}}}\left( \eta _{T}^{*}a_{\mathbf{p}}\mathrm{e}^{\mathrm{i}\left( \mathcal{P} p \right) \cdot x}+\tilde{\eta}_Tb_{\mathbf{p}}^{\dagger}\mathrm{e}^{-\mathrm{i}\left( \mathcal{P} p \right) \cdot x} \right) 
\\
&=\int{\frac{\mathrm{d}^3p}{\left( 2\pi \right) ^3}}\frac{1}{\sqrt{2E_{\mathbf{p}}}}\left( \eta _{T}^{*}a_{\mathbf{p}}\mathrm{e}^{\mathrm{i}p\cdot \left( \mathcal{P} x \right)}+\tilde{\eta}_Tb_{\mathbf{p}}^{\dagger}\mathrm{e}^{-\mathrm{i}p\cdot \left( \mathcal{P} x \right)} \right) 
\\
&=\int{\frac{\mathrm{d}^3p}{\left( 2\pi \right) ^3}}\frac{1}{\sqrt{2E_{\mathbf{p}}}}\left( \eta _{T}^{*}a_{\mathbf{p}}\mathrm{e}^{-\mathrm{i}p\cdot (\mathcal{T} x)}+\tilde{\eta}_Tb_{\mathbf{p}}^{\dagger}\mathrm{e}^{\mathrm{i}p\cdot (\mathcal{T} x)} \right) 
\\
&=\int{\frac{\mathrm{d}^3p}{\left( 2\pi \right) ^3}}\frac{1}{\sqrt{2E_{\mathbf{p}}}}\left( \eta _{T}^{*}a_{\mathbf{p}}\mathrm{e}^{-\mathrm{i}p\cdot (\mathcal{T} x)}+\eta _{T}^{*}b_{\mathbf{p}}^{\dagger}\mathrm{e}^{\mathrm{i}p\cdot (\mathcal{T} x)} \right) 
\\
&=\eta _{T}^{*}\int{\frac{\mathrm{d}^3p}{\left( 2\pi \right) ^3}}\frac{1}{\sqrt{2E_{\mathbf{p}}}}\left( a_{\mathbf{p}}\mathrm{e}^{-\mathrm{i}p\cdot (\mathcal{T} x)}+b_{\mathbf{p}}^{\dagger}\mathrm{e}^{\mathrm{i}p\cdot (\mathcal{T} x)} \right) 
    \end{aligned}
\end{equation}
对比得到


1.由
\begin{equation}
    \phi ^{\dagger}(x)=\int{\frac{\mathrm{d}^3p}{\left( 2\pi \right) ^3}}\frac{1}{\sqrt{2E_{\mathbf{p}}}}\left( b_{\mathbf{p}}\mathrm{e}^{-\mathrm{i}p\cdot x}+a_{\mathbf{p}}^{\dagger}\mathrm{e}^{\mathrm{i}p\cdot x} \right) 
\end{equation}
得到
\begin{equation}
    \phi ^{\dagger}(\mathcal{T} x)=\int{\frac{\mathrm{d}^3p}{\left( 2\pi \right) ^3}}\frac{1}{\sqrt{2E_{\mathbf{p}}}}\left( b_{\mathbf{p}}\mathrm{e}^{-\mathrm{i}p\cdot \left( \mathcal{T} x \right)}+a_{\mathbf{p}}^{\dagger}\mathrm{e}^{\mathrm{i}p\cdot \left( \mathcal{T} x \right)} \right) 
\end{equation}
2.计算
\begin{equation}
    \begin{aligned}
        T^{-1}\phi ^{\dagger}\left( x \right) T&=\int{\frac{\mathrm{d}^3p}{\left( 2\pi \right) ^3}}\frac{1}{\sqrt{2E_{\mathbf{p}}}}T^{-1}\left( b_{\mathbf{p}}\mathrm{e}^{-\mathrm{i}p\cdot x}+a_{\mathbf{p}}^{\dagger}\mathrm{e}^{\mathrm{i}p\cdot x} \right) T
\\
&=\int{\frac{\mathrm{d}^3p}{\left( 2\pi \right) ^3}}\frac{1}{\sqrt{2E_{\mathbf{p}}}}\left( T^{-1}b_{\mathbf{p}}TT^{-1}\mathrm{e}^{-\mathrm{i}p\cdot x}T+T^{-1}a_{\mathbf{p}}^{\dagger}TT^{-1}\mathrm{e}^{\mathrm{i}p\cdot x}T \right) 
\\
&=\int{\frac{\mathrm{d}^3p}{\left( 2\pi \right) ^3}}\frac{1}{\sqrt{2E_{\mathbf{p}}}}\left( \tilde{\eta}_{T}^{*}b_{-\mathbf{p}}\mathrm{e}^{\mathrm{i}p\cdot x}+\eta _Ta_{-\mathbf{p}}^{\dagger}\mathrm{e}^{-\mathrm{i}p\cdot x} \right) 
\\
&=\int{\frac{\mathrm{d}^3p}{\left( 2\pi \right) ^3}}\frac{1}{\sqrt{2E_{\mathbf{p}}}}\left( \tilde{\eta}_{T}^{*}b_{\mathbf{p}}\mathrm{e}^{\mathrm{i}\left( \mathcal{P} p \right) \cdot x}+\eta _Ta_{\mathbf{p}}^{\dagger}\mathrm{e}^{-\mathrm{i}\left( \mathcal{P} p \right) \cdot x} \right) 
\\
&=\int{\frac{\mathrm{d}^3p}{\left( 2\pi \right) ^3}}\frac{1}{\sqrt{2E_{\mathbf{p}}}}\left( \tilde{\eta}_{T}^{*}b_{\mathbf{p}}\mathrm{e}^{\mathrm{i}p\cdot \left( \mathcal{P} x \right)}+\eta _Ta_{\mathbf{p}}^{\dagger}\mathrm{e}^{-\mathrm{i}p\cdot \left( \mathcal{P} x \right)} \right) 
\\
&=\int{\frac{\mathrm{d}^3p}{\left( 2\pi \right) ^3}}\frac{1}{\sqrt{2E_{\mathbf{p}}}}\left( \tilde{\eta}_{T}^{*}b_{\mathbf{p}}\mathrm{e}^{-\mathrm{i}p\cdot (\mathcal{T} x)}+\eta _Ta_{\mathbf{p}}^{\dagger}\mathrm{e}^{\mathrm{i}p\cdot (\mathcal{T} x)} \right) 
\\
&=\int{\frac{\mathrm{d}^3p}{\left( 2\pi \right) ^3}}\frac{1}{\sqrt{2E_{\mathbf{p}}}}\left( \eta _Tb_{\mathbf{p}}\mathrm{e}^{-\mathrm{i}p\cdot (\mathcal{T} x)}+\eta _Ta_{\mathbf{p}}^{\dagger}\mathrm{e}^{\mathrm{i}p\cdot (\mathcal{T} x)} \right) 
\\
&=\eta _T\int{\frac{\mathrm{d}^3p}{\left( 2\pi \right) ^3}}\frac{1}{\sqrt{2E_{\mathbf{p}}}}\left( b_{\mathbf{p}}\mathrm{e}^{-\mathrm{i}p\cdot (\mathcal{T} x)}+a_{\mathbf{p}}^{\dagger}\mathrm{e}^{\mathrm{i}p\cdot (\mathcal{T} x)} \right) 
    \end{aligned}
\end{equation}
对比得到
\begin{equation}
    T^{-1}\phi ^{\dagger}\left( x \right) T=\eta _T\phi ^{\dagger}(\mathcal{T} x)
\end{equation}


由标量场
\begin{equation}
    \phi \left( x \right) =\int{\frac{\mathrm{d}^3p}{\left( 2\pi \right) ^3}}\frac{1}{\sqrt{2E_{\mathbf{p}}}}\left( a_{\mathbf{p}}\mathrm{e}^{-\mathrm{i}p\cdot x}+b_{\mathbf{p}}^{\dagger}\mathrm{e}^{\mathrm{i}p\cdot x} \right) 
\end{equation}
和复共轭
\begin{equation}
    \phi ^{\dagger}(x)=\int{\frac{\mathrm{d}^3p}{\left( 2\pi \right) ^3}}\frac{1}{\sqrt{2E_{\mathbf{p}}}}\left( b_{\mathbf{p}}\mathrm{e}^{-\mathrm{i}p\cdot x}+a_{\mathbf{p}}^{\dagger}\mathrm{e}^{\mathrm{i}p\cdot x} \right) 
\end{equation}

2.计算
\begin{equation}
    \begin{aligned}
        C^{-1}\phi C&=\int{\frac{\mathrm{d}^3p}{\left( 2\pi \right) ^3}}\frac{1}{\sqrt{2E_{\mathbf{p}}}}C^{-1}\left( a_{\mathbf{p}}\mathrm{e}^{-\mathrm{i}p\cdot x}+b_{\mathbf{p}}^{\dagger}\mathrm{e}^{\mathrm{i}p\cdot x} \right) C
\\
&=\int{\frac{\mathrm{d}^3p}{\left( 2\pi \right) ^3}}\frac{1}{\sqrt{2E_{\mathbf{p}}}}\left( C^{-1}a_{\mathbf{p}}C\mathrm{e}^{-\mathrm{i}p\cdot x}+C^{-1}b_{\mathbf{p}}^{\dagger}C\mathrm{e}^{\mathrm{i}p\cdot x} \right) 
\\
&=\int{\frac{\mathrm{d}^3p}{\left( 2\pi \right) ^3}}\frac{1}{\sqrt{2E_{\mathbf{p}}}}\left( \eta _{C}^{*}b_{\mathbf{p}}\mathrm{e}^{-\mathrm{i}p\cdot x}+\eta _{C}^{*}a_{\mathbf{p}}^{\dagger}\mathrm{e}^{\mathrm{i}p\cdot x} \right) 
\\
&=\eta _{C}^{*}\int{\frac{\mathrm{d}^3p}{\left( 2\pi \right) ^3}}\frac{1}{\sqrt{2E_{\mathbf{p}}}}\left( b_{\mathbf{p}}\mathrm{e}^{-\mathrm{i}p\cdot x}+a_{\mathbf{p}}^{\dagger}\mathrm{e}^{\mathrm{i}p\cdot x} \right) 
    \end{aligned}
\end{equation}
对比得到
\begin{equation}
    C^{-1}\phi (x)C=\eta _{C}^{*}\phi ^{\dagger}(x)
\end{equation}
同样地
\begin{equation}
    \begin{aligned}
        C^{-1}\phi ^{\dagger}(x)C&=\int{\frac{\mathrm{d}^3p}{\left( 2\pi \right) ^3}}\frac{1}{\sqrt{2E_{\mathbf{p}}}}C^{-1}\left( b_{\mathbf{p}}\mathrm{e}^{-\mathrm{i}p\cdot x}+a_{\mathbf{p}}^{\dagger}\mathrm{e}^{\mathrm{i}p\cdot x} \right) C
\\
&=\int{\frac{\mathrm{d}^3p}{\left( 2\pi \right) ^3}}\frac{1}{\sqrt{2E_{\mathbf{p}}}}\left( C^{-1}b_{\mathbf{p}}C\mathrm{e}^{-\mathrm{i}p\cdot x}+C^{-1}a_{\mathbf{p}}^{\dagger}C\mathrm{e}^{\mathrm{i}p\cdot x} \right) 
\\
&=\int{\frac{\mathrm{d}^3p}{\left( 2\pi \right) ^3}}\frac{1}{\sqrt{2E_{\mathbf{p}}}}\left( \eta _Ca_{\mathbf{p}}\mathrm{e}^{-\mathrm{i}p\cdot x}+\eta _Cb_{\mathbf{p}}^{\dagger}\mathrm{e}^{\mathrm{i}p\cdot x} \right) 
\\
&=\eta _C\int{\frac{\mathrm{d}^3p}{\left( 2\pi \right) ^3}}\frac{1}{\sqrt{2E_{\mathbf{p}}}}\left( a_{\mathbf{p}}\mathrm{e}^{-\mathrm{i}p\cdot x}+b_{\mathbf{p}}^{\dagger}\mathrm{e}^{\mathrm{i}p\cdot x} \right) 
    \end{aligned}
\end{equation}
对比得到
\begin{equation}
    C^{-1}\phi ^{\dagger}(x)C=\eta _C\phi (x)
\end{equation}

\section{9}
%%%%%%%%%%%%%%%%%%%%%%%%%%%%%%%%%%%%%%%%%%%%%%%%%%%%%%%
\subsection{9.1}

宇称变换
\begin{equation}
    {\mathcal{P} ^{\mu}}_{\nu}=(\mathcal{P} ^{-1}{)^{\mu}}_{\nu}=\left( \begin{matrix}
	+1&		&		&		\\
	&		-1&		&		\\
	&		&		-1&		\\
	&		&		&		-1\\
\end{matrix} \right) 
\end{equation}
时空坐标变换
\begin{equation}
    x^{\mu}=\left( t,\mathbf{x} \right) \Rightarrow x^{\prime \mu}={\mathcal{P} ^{\mu}}_{\nu}x^{\nu}=(\mathcal{P} x)^{\mu}=\left( t,-\mathbf{x} \right) 
\end{equation}
四维动量变换
\begin{equation}
    p^{\mu}=\left( E,\mathbf{p} \right) \Rightarrow p^{\prime \mu}={\mathcal{P} ^{\mu}}_{\nu}p^{\nu}=(\mathcal{P} p)^{\mu}=\left( E,-\mathbf{p} \right) 
\end{equation}
时空导数变换
\begin{equation}
    \partial _{\mu}^{\prime}=(\mathcal{P} ^{-1}{)^{\nu}}_{\mu}\partial _{\nu}
\end{equation}
保持时空体积元不变
\begin{equation}
    \mathrm{d}^4x^{\prime}=\left| \det \left( \mathcal{P} \right) \right|\mathrm{d}^4x=\mathrm{d}^4x
\end{equation}

如果场论系统的作用量S在宇称变换下不变,则运动方程的形式也在宇称变换下不变,此时称系统是宇称守恒的,即具有空间反射对称性。


在宇称守恒的量子理论中,宇称变换在 Hilbert 空间中诱导出态矢$|\Psi \rangle$的线性幺正变换
\begin{equation}
    |\Psi ^{\prime}\rangle =U(\mathcal{P} )|\Psi \rangle =P|\Psi \rangle 
\end{equation}





推导:
1.由标量场
\begin{equation}
    \phi (x)=\int{\frac{\mathrm{d}^3p}{\left( 2\pi \right) ^3}}\frac{1}{\sqrt{2E_{\mathbf{p}}}}\left( a_{\mathbf{p}}\mathrm{e}^{-\mathrm{i}p\cdot x}+b_{\mathbf{p}}^{\dagger}\mathrm{e}^{\mathrm{i}p\cdot x} \right) 
\end{equation}
得到宇称变换后的标量场
\begin{equation}
    \phi (\mathcal{P} x)=\int{\frac{\mathrm{d}^3p}{\left( 2\pi \right) ^3}}\frac{1}{\sqrt{2E_{\mathbf{p}}}}\left( a_{\mathbf{p}}\mathrm{e}^{-\mathrm{i}p\cdot \left( \mathcal{P} x \right)}+b_{\mathbf{p}}^{\dagger}\mathrm{e}^{\mathrm{i}p\cdot \left( \mathcal{P} x \right)} \right) 
\end{equation}
2.计算
\begin{equation}
    \begin{aligned}
        P^{-1}\phi (x)P&=\int{\frac{\mathrm{d}^3p}{\left( 2\pi \right) ^3}}\frac{1}{\sqrt{2E_{\mathbf{p}}}}\left( P^{-1}a_{\mathbf{p}}P\mathrm{e}^{-\mathrm{i}p\cdot x}+P^{-1}b_{\mathbf{p}}^{\dagger}P\mathrm{e}^{\mathrm{i}p\cdot x} \right) 
\\
&=\int{\frac{\mathrm{d}^3p}{\left( 2\pi \right) ^3}}\frac{1}{\sqrt{2E_{\mathbf{p}}}}\left( \eta _{P}^{*}a_{-\mathbf{p}}\mathrm{e}^{-\mathrm{i}p\cdot x}+\tilde{\eta}_Pb_{-\mathbf{p}}^{\dagger}\mathrm{e}^{\mathrm{i}p\cdot x} \right) 
\\
&=\int{\frac{\mathrm{d}^3p}{\left( 2\pi \right) ^3}}\frac{1}{\sqrt{2E_{\mathbf{p}}}}\left( \eta _{P}^{*}a_{\mathbf{p}}\mathrm{e}^{-\mathrm{i}\left( \mathcal{P} p \right) \cdot x}+\tilde{\eta}_Pb_{\mathbf{p}}^{\dagger}\mathrm{e}^{\mathrm{i}\left( \mathcal{P} p \right) \cdot x} \right) 
\\
&=\int{\frac{\mathrm{d}^3p}{\left( 2\pi \right) ^3}}\frac{1}{\sqrt{2E_{\mathbf{p}}}}\left( \eta _{P}^{*}a_{\mathbf{p}}\mathrm{e}^{-\mathrm{i}p\cdot \left( \mathcal{P} x \right)}+\tilde{\eta}_Pb_{\mathbf{p}}^{\dagger}\mathrm{e}^{\mathrm{i}p\cdot \left( \mathcal{P} x \right)} \right) 
\\
&=\int{\frac{\mathrm{d}^3p}{\left( 2\pi \right) ^3}}\frac{1}{\sqrt{2E_{\mathbf{p}}}}\left( \eta _{P}^{*}a_{\mathbf{p}}\mathrm{e}^{-\mathrm{i}p\cdot \left( \mathcal{P} x \right)}+\eta _{P}^{*}b_{\mathbf{p}}^{\dagger}\mathrm{e}^{\mathrm{i}p\cdot \left( \mathcal{P} x \right)} \right) 
\\
&=\eta _{P}^{*}\int{\frac{\mathrm{d}^3p}{\left( 2\pi \right) ^3}}\frac{1}{\sqrt{2E_{\mathbf{p}}}}\left( a_{\mathbf{p}}\mathrm{e}^{-\mathrm{i}p\cdot \left( \mathcal{P} x \right)}+b_{\mathbf{p}}^{\dagger}\mathrm{e}^{\mathrm{i}p\cdot \left( \mathcal{P} x \right)} \right) 
    \end{aligned}
\end{equation}
得到
\begin{equation}
    P^{-1}\phi (x)P=\eta _{P}^{*}\phi (\mathcal{P} x)
\end{equation}


推导:
1.由标量场的复共轭
\begin{equation}
    \phi ^{\dagger}(x)=\int{\frac{\mathrm{d}^3p}{\left( 2\pi \right) ^3}}\frac{1}{\sqrt{2E_{\mathbf{p}}}}\left( b_{\mathbf{p}}\mathrm{e}^{-\mathrm{i}p\cdot x}+a_{\mathbf{p}}^{\dagger}\mathrm{e}^{\mathrm{i}p\cdot x} \right) 
\end{equation}
得到
\begin{equation}
    \phi ^{\dagger}(\mathcal{P} x)=\int{\frac{\mathrm{d}^3p}{\left( 2\pi \right) ^3}}\frac{1}{\sqrt{2E_{\mathbf{p}}}}\left( b_{\mathbf{p}}\mathrm{e}^{-\mathrm{i}p\cdot \left( \mathcal{P} x \right)}+a_{\mathbf{p}}^{\dagger}\mathrm{e}^{\mathrm{i}p\cdot \left( \mathcal{P} x \right)} \right) 
\end{equation}
2.计算
\begin{equation}
    \begin{aligned}
        P^{-1}\phi ^{\dagger}(x)P&=\int{\frac{\mathrm{d}^3p}{\left( 2\pi \right) ^3}}\frac{1}{\sqrt{2E_{\mathbf{p}}}}\left( P^{-1}b_{\mathbf{p}}P\mathrm{e}^{-\mathrm{i}p\cdot x}+P^{-1}a_{\mathbf{p}}^{\dagger}P\mathrm{e}^{\mathrm{i}p\cdot x} \right) 
\\
&=\int{\frac{\mathrm{d}^3p}{\left( 2\pi \right) ^3}}\frac{1}{\sqrt{2E_{\mathbf{p}}}}\left( \tilde{\eta}_{P}^{*}b_{-\mathbf{p}}\mathrm{e}^{-\mathrm{i}p\cdot x}+\eta _Pa_{-\mathbf{p}}^{\dagger}\mathrm{e}^{\mathrm{i}p\cdot x} \right) 
\\
&=\int{\frac{\mathrm{d}^3p}{\left( 2\pi \right) ^3}}\frac{1}{\sqrt{2E_{\mathbf{p}}}}\left( \tilde{\eta}_{P}^{*}b_{\mathbf{p}}\mathrm{e}^{-\mathrm{i}\left( \mathcal{P} p \right) \cdot x}+\eta _Pa_{\mathbf{p}}^{\dagger}\mathrm{e}^{\mathrm{i}\left( \mathcal{P} p \right) \cdot x} \right) 
\\
&=\int{\frac{\mathrm{d}^3p}{\left( 2\pi \right) ^3}}\frac{1}{\sqrt{2E_{\mathbf{p}}}}\left( \tilde{\eta}_{P}^{*}b_{\mathbf{p}}\mathrm{e}^{-\mathrm{i}p\cdot \left( \mathcal{P} x \right)}+\eta _Pa_{\mathbf{p}}^{\dagger}\mathrm{e}^{\mathrm{i}p\cdot \left( \mathcal{P} x \right)} \right) 
\\
&=\int{\frac{\mathrm{d}^3p}{\left( 2\pi \right) ^3}}\frac{1}{\sqrt{2E_{\mathbf{p}}}}\left( \eta _Pb_{\mathbf{p}}\mathrm{e}^{-\mathrm{i}p\cdot \left( \mathcal{P} x \right)}+\eta _Pa_{\mathbf{p}}^{\dagger}\mathrm{e}^{\mathrm{i}p\cdot \left( \mathcal{P} x \right)} \right) 
\\
&=\eta _P\int{\frac{\mathrm{d}^3p}{\left( 2\pi \right) ^3}}\frac{1}{\sqrt{2E_{\mathbf{p}}}}\left( b_{\mathbf{p}}\mathrm{e}^{-\mathrm{i}p\cdot \left( \mathcal{P} x \right)}+a_{\mathbf{p}}^{\dagger}\mathrm{e}^{\mathrm{i}p\cdot \left( \mathcal{P} x \right)} \right) 
    \end{aligned}
\end{equation}
得到
\begin{equation}
    P^{-1}\phi ^{\dagger}(x)P=\eta _P\phi ^{\dagger}(\mathcal{P} x)
\end{equation}



\subsubsection{标量场的T变换}

时间反演变换
\begin{equation}
    {\mathcal{T} ^{\mu}}_{\nu}=(\mathcal{T} ^{-1}{)^{\mu}}_{\nu}=\left( \begin{matrix}
	-1&		&		&		\\
	&		+1&		&		\\
	&		&		+1&		\\
	&		&		&		+1\\
\end{matrix} \right) 
\end{equation}
时空坐标变换
\begin{equation}
    x^{\mu}=\left( t,\mathbf{x} \right) \Rightarrow \,\,x^{\prime \mu}={\mathcal{T} ^{\mu}}_{\nu}x^{\nu}=(\mathcal{T} x)^{\mu}=\left( -t,\mathbf{x} \right) 
\end{equation}
四维动量变换
\begin{equation}
    p^{\mu}=\left( E,\mathbf{p} \right) \Rightarrow \,\,p^{\prime \mu}={\mathcal{T} ^{\mu}}_{\nu}p^{\nu}=(\mathcal{T} p)^{\mu}=\left( -E,\mathbf{p} \right) 
\end{equation}
时空导数变换
\begin{equation}
    \partial _{\mu}^{\prime}=(\mathcal{T} ^{-1}{)^{\nu}}_{\mu}\partial _{\nu}
\end{equation}
时间反演变换保持时空体积元不变
\begin{equation}
    \mathrm{d}^4x^{\prime}=\left| \det\mathrm{(}\mathcal{T} ) \right|\mathrm{d}^4x=\mathrm{d}^4x
\end{equation}


推导:
1.由标量场
\begin{equation}
    \phi (x)=\int{\frac{\mathrm{d}^3p}{\left( 2\pi \right) ^3}\frac{1}{\sqrt{2E_{\mathbf{p}}}}\left( a_{\mathbf{p}}\mathrm{e}^{-\mathrm{i}p\cdot x}+b_{\mathbf{p}}^{\dagger}\mathrm{e}^{\mathrm{i}p\cdot x} \right)}
\end{equation}
得到时间反演变换后的标量场
\begin{equation}
    \phi (\mathcal{T} x)=\int{\frac{\mathrm{d}^3p}{\left( 2\pi \right) ^3}\frac{1}{\sqrt{2E_{\mathbf{p}}}}\left( a_{\mathbf{p}}\mathrm{e}^{-\mathrm{i}p\cdot \left( \mathcal{T} x \right)}+b_{\mathbf{p}}^{\dagger}\mathrm{e}^{\mathrm{i}p\cdot \left( \mathcal{T} x \right)} \right)}
\end{equation}
2.计算
\begin{equation}
    \begin{aligned}
        T^{-1}\phi \left( x \right) T&=\int{\frac{\mathrm{d}^3p}{\left( 2\pi \right) ^3}\frac{1}{\sqrt{2E_{\mathbf{p}}}}T^{-1}\left( a_{\mathbf{p}}\mathrm{e}^{-\mathrm{i}p\cdot x}+b_{\mathbf{p}}^{\dagger}\mathrm{e}^{\mathrm{i}p\cdot x} \right)}T
\\
&=\int{\frac{\mathrm{d}^3p}{\left( 2\pi \right) ^3}}\frac{1}{\sqrt{2E_{\mathbf{p}}}}\left( T^{-1}a_{\mathbf{p}}TT^{-1}\mathrm{e}^{-\mathrm{i}p\cdot x}T+T^{-1}b_{\mathbf{p}}^{\dagger}TT^{-1}\mathrm{e}^{\mathrm{i}p\cdot x}T \right) 
\\
&=\int{\frac{\mathrm{d}^3p}{\left( 2\pi \right) ^3}}\frac{1}{\sqrt{2E_{\mathbf{p}}}}\left( \eta _{T}^{*}a_{-\mathbf{p}}\mathrm{e}^{\mathrm{i}p\cdot x}+\tilde{\eta}_Tb_{-\mathbf{p}}^{\dagger}\mathrm{e}^{-\mathrm{i}p\cdot x} \right) 
\\
&=\int{\frac{\mathrm{d}^3p}{\left( 2\pi \right) ^3}}\frac{1}{\sqrt{2E_{\mathbf{p}}}}\left( \eta _{T}^{*}a_{\mathbf{p}}\mathrm{e}^{\mathrm{i}\left( \mathcal{P} p \right) \cdot x}+\tilde{\eta}_Tb_{\mathbf{p}}^{\dagger}\mathrm{e}^{-\mathrm{i}\left( \mathcal{P} p \right) \cdot x} \right) 
\\
&=\int{\frac{\mathrm{d}^3p}{\left( 2\pi \right) ^3}}\frac{1}{\sqrt{2E_{\mathbf{p}}}}\left( \eta _{T}^{*}a_{\mathbf{p}}\mathrm{e}^{\mathrm{i}p\cdot \left( \mathcal{P} x \right)}+\tilde{\eta}_Tb_{\mathbf{p}}^{\dagger}\mathrm{e}^{-\mathrm{i}p\cdot \left( \mathcal{P} x \right)} \right) 
\\
&=\int{\frac{\mathrm{d}^3p}{\left( 2\pi \right) ^3}}\frac{1}{\sqrt{2E_{\mathbf{p}}}}\left( \eta _{T}^{*}a_{\mathbf{p}}\mathrm{e}^{-\mathrm{i}p\cdot (\mathcal{T} x)}+\tilde{\eta}_Tb_{\mathbf{p}}^{\dagger}\mathrm{e}^{\mathrm{i}p\cdot (\mathcal{T} x)} \right) 
\\
&=\int{\frac{\mathrm{d}^3p}{\left( 2\pi \right) ^3}}\frac{1}{\sqrt{2E_{\mathbf{p}}}}\left( \eta _{T}^{*}a_{\mathbf{p}}\mathrm{e}^{-\mathrm{i}p\cdot (\mathcal{T} x)}+\eta _{T}^{*}b_{\mathbf{p}}^{\dagger}\mathrm{e}^{\mathrm{i}p\cdot (\mathcal{T} x)} \right) 
\\
&=\eta _{T}^{*}\int{\frac{\mathrm{d}^3p}{\left( 2\pi \right) ^3}}\frac{1}{\sqrt{2E_{\mathbf{p}}}}\left( a_{\mathbf{p}}\mathrm{e}^{-\mathrm{i}p\cdot (\mathcal{T} x)}+b_{\mathbf{p}}^{\dagger}\mathrm{e}^{\mathrm{i}p\cdot (\mathcal{T} x)} \right) 
    \end{aligned}
\end{equation}
对比得到


1.由
\begin{equation}
    \phi ^{\dagger}(x)=\int{\frac{\mathrm{d}^3p}{\left( 2\pi \right) ^3}}\frac{1}{\sqrt{2E_{\mathbf{p}}}}\left( b_{\mathbf{p}}\mathrm{e}^{-\mathrm{i}p\cdot x}+a_{\mathbf{p}}^{\dagger}\mathrm{e}^{\mathrm{i}p\cdot x} \right) 
\end{equation}
得到
\begin{equation}
    \phi ^{\dagger}(\mathcal{T} x)=\int{\frac{\mathrm{d}^3p}{\left( 2\pi \right) ^3}}\frac{1}{\sqrt{2E_{\mathbf{p}}}}\left( b_{\mathbf{p}}\mathrm{e}^{-\mathrm{i}p\cdot \left( \mathcal{T} x \right)}+a_{\mathbf{p}}^{\dagger}\mathrm{e}^{\mathrm{i}p\cdot \left( \mathcal{T} x \right)} \right) 
\end{equation}
2.计算
\begin{equation}
    \begin{aligned}
        T^{-1}\phi ^{\dagger}\left( x \right) T&=\int{\frac{\mathrm{d}^3p}{\left( 2\pi \right) ^3}}\frac{1}{\sqrt{2E_{\mathbf{p}}}}T^{-1}\left( b_{\mathbf{p}}\mathrm{e}^{-\mathrm{i}p\cdot x}+a_{\mathbf{p}}^{\dagger}\mathrm{e}^{\mathrm{i}p\cdot x} \right) T
\\
&=\int{\frac{\mathrm{d}^3p}{\left( 2\pi \right) ^3}}\frac{1}{\sqrt{2E_{\mathbf{p}}}}\left( T^{-1}b_{\mathbf{p}}TT^{-1}\mathrm{e}^{-\mathrm{i}p\cdot x}T+T^{-1}a_{\mathbf{p}}^{\dagger}TT^{-1}\mathrm{e}^{\mathrm{i}p\cdot x}T \right) 
\\
&=\int{\frac{\mathrm{d}^3p}{\left( 2\pi \right) ^3}}\frac{1}{\sqrt{2E_{\mathbf{p}}}}\left( \tilde{\eta}_{T}^{*}b_{-\mathbf{p}}\mathrm{e}^{\mathrm{i}p\cdot x}+\eta _Ta_{-\mathbf{p}}^{\dagger}\mathrm{e}^{-\mathrm{i}p\cdot x} \right) 
\\
&=\int{\frac{\mathrm{d}^3p}{\left( 2\pi \right) ^3}}\frac{1}{\sqrt{2E_{\mathbf{p}}}}\left( \tilde{\eta}_{T}^{*}b_{\mathbf{p}}\mathrm{e}^{\mathrm{i}\left( \mathcal{P} p \right) \cdot x}+\eta _Ta_{\mathbf{p}}^{\dagger}\mathrm{e}^{-\mathrm{i}\left( \mathcal{P} p \right) \cdot x} \right) 
\\
&=\int{\frac{\mathrm{d}^3p}{\left( 2\pi \right) ^3}}\frac{1}{\sqrt{2E_{\mathbf{p}}}}\left( \tilde{\eta}_{T}^{*}b_{\mathbf{p}}\mathrm{e}^{\mathrm{i}p\cdot \left( \mathcal{P} x \right)}+\eta _Ta_{\mathbf{p}}^{\dagger}\mathrm{e}^{-\mathrm{i}p\cdot \left( \mathcal{P} x \right)} \right) 
\\
&=\int{\frac{\mathrm{d}^3p}{\left( 2\pi \right) ^3}}\frac{1}{\sqrt{2E_{\mathbf{p}}}}\left( \tilde{\eta}_{T}^{*}b_{\mathbf{p}}\mathrm{e}^{-\mathrm{i}p\cdot (\mathcal{T} x)}+\eta _Ta_{\mathbf{p}}^{\dagger}\mathrm{e}^{\mathrm{i}p\cdot (\mathcal{T} x)} \right) 
\\
&=\int{\frac{\mathrm{d}^3p}{\left( 2\pi \right) ^3}}\frac{1}{\sqrt{2E_{\mathbf{p}}}}\left( \eta _Tb_{\mathbf{p}}\mathrm{e}^{-\mathrm{i}p\cdot (\mathcal{T} x)}+\eta _Ta_{\mathbf{p}}^{\dagger}\mathrm{e}^{\mathrm{i}p\cdot (\mathcal{T} x)} \right) 
\\
&=\eta _T\int{\frac{\mathrm{d}^3p}{\left( 2\pi \right) ^3}}\frac{1}{\sqrt{2E_{\mathbf{p}}}}\left( b_{\mathbf{p}}\mathrm{e}^{-\mathrm{i}p\cdot (\mathcal{T} x)}+a_{\mathbf{p}}^{\dagger}\mathrm{e}^{\mathrm{i}p\cdot (\mathcal{T} x)} \right) 
    \end{aligned}
\end{equation}
对比得到
\begin{equation}
    T^{-1}\phi ^{\dagger}\left( x \right) T=\eta _T\phi ^{\dagger}(\mathcal{T} x)
\end{equation}


由标量场
\begin{equation}
    \phi \left( x \right) =\int{\frac{\mathrm{d}^3p}{\left( 2\pi \right) ^3}}\frac{1}{\sqrt{2E_{\mathbf{p}}}}\left( a_{\mathbf{p}}\mathrm{e}^{-\mathrm{i}p\cdot x}+b_{\mathbf{p}}^{\dagger}\mathrm{e}^{\mathrm{i}p\cdot x} \right) 
\end{equation}
和复共轭
\begin{equation}
    \phi ^{\dagger}(x)=\int{\frac{\mathrm{d}^3p}{\left( 2\pi \right) ^3}}\frac{1}{\sqrt{2E_{\mathbf{p}}}}\left( b_{\mathbf{p}}\mathrm{e}^{-\mathrm{i}p\cdot x}+a_{\mathbf{p}}^{\dagger}\mathrm{e}^{\mathrm{i}p\cdot x} \right) 
\end{equation}

2.计算
\begin{equation}
    \begin{aligned}
        C^{-1}\phi C&=\int{\frac{\mathrm{d}^3p}{\left( 2\pi \right) ^3}}\frac{1}{\sqrt{2E_{\mathbf{p}}}}C^{-1}\left( a_{\mathbf{p}}\mathrm{e}^{-\mathrm{i}p\cdot x}+b_{\mathbf{p}}^{\dagger}\mathrm{e}^{\mathrm{i}p\cdot x} \right) C
\\
&=\int{\frac{\mathrm{d}^3p}{\left( 2\pi \right) ^3}}\frac{1}{\sqrt{2E_{\mathbf{p}}}}\left( C^{-1}a_{\mathbf{p}}C\mathrm{e}^{-\mathrm{i}p\cdot x}+C^{-1}b_{\mathbf{p}}^{\dagger}C\mathrm{e}^{\mathrm{i}p\cdot x} \right) 
\\
&=\int{\frac{\mathrm{d}^3p}{\left( 2\pi \right) ^3}}\frac{1}{\sqrt{2E_{\mathbf{p}}}}\left( \eta _{C}^{*}b_{\mathbf{p}}\mathrm{e}^{-\mathrm{i}p\cdot x}+\eta _{C}^{*}a_{\mathbf{p}}^{\dagger}\mathrm{e}^{\mathrm{i}p\cdot x} \right) 
\\
&=\eta _{C}^{*}\int{\frac{\mathrm{d}^3p}{\left( 2\pi \right) ^3}}\frac{1}{\sqrt{2E_{\mathbf{p}}}}\left( b_{\mathbf{p}}\mathrm{e}^{-\mathrm{i}p\cdot x}+a_{\mathbf{p}}^{\dagger}\mathrm{e}^{\mathrm{i}p\cdot x} \right) 
    \end{aligned}
\end{equation}
对比得到
\begin{equation}
    C^{-1}\phi (x)C=\eta _{C}^{*}\phi ^{\dagger}(x)
\end{equation}
同样地
\begin{equation}
    \begin{aligned}
        C^{-1}\phi ^{\dagger}(x)C&=\int{\frac{\mathrm{d}^3p}{\left( 2\pi \right) ^3}}\frac{1}{\sqrt{2E_{\mathbf{p}}}}C^{-1}\left( b_{\mathbf{p}}\mathrm{e}^{-\mathrm{i}p\cdot x}+a_{\mathbf{p}}^{\dagger}\mathrm{e}^{\mathrm{i}p\cdot x} \right) C
\\
&=\int{\frac{\mathrm{d}^3p}{\left( 2\pi \right) ^3}}\frac{1}{\sqrt{2E_{\mathbf{p}}}}\left( C^{-1}b_{\mathbf{p}}C\mathrm{e}^{-\mathrm{i}p\cdot x}+C^{-1}a_{\mathbf{p}}^{\dagger}C\mathrm{e}^{\mathrm{i}p\cdot x} \right) 
\\
&=\int{\frac{\mathrm{d}^3p}{\left( 2\pi \right) ^3}}\frac{1}{\sqrt{2E_{\mathbf{p}}}}\left( \eta _Ca_{\mathbf{p}}\mathrm{e}^{-\mathrm{i}p\cdot x}+\eta _Cb_{\mathbf{p}}^{\dagger}\mathrm{e}^{\mathrm{i}p\cdot x} \right) 
\\
&=\eta _C\int{\frac{\mathrm{d}^3p}{\left( 2\pi \right) ^3}}\frac{1}{\sqrt{2E_{\mathbf{p}}}}\left( a_{\mathbf{p}}\mathrm{e}^{-\mathrm{i}p\cdot x}+b_{\mathbf{p}}^{\dagger}\mathrm{e}^{\mathrm{i}p\cdot x} \right) 
    \end{aligned}
\end{equation}
对比得到
\begin{equation}
    C^{-1}\phi ^{\dagger}(x)C=\eta _C\phi (x)
\end{equation}

\section{9}
%%%%%%%%%%%%%%%%%%%%%%%%%%%%%%%%%%%%%%%%%%%%%%%%%%%%%%%
\subsection{9.1}

宇称变换
\begin{equation}
    {\mathcal{P} ^{\mu}}_{\nu}=(\mathcal{P} ^{-1}{)^{\mu}}_{\nu}=\left( \begin{matrix}
	+1&		&		&		\\
	&		-1&		&		\\
	&		&		-1&		\\
	&		&		&		-1\\
\end{matrix} \right) 
\end{equation}
时空坐标变换
\begin{equation}
    x^{\mu}=\left( t,\mathbf{x} \right) \Rightarrow x^{\prime \mu}={\mathcal{P} ^{\mu}}_{\nu}x^{\nu}=(\mathcal{P} x)^{\mu}=\left( t,-\mathbf{x} \right) 
\end{equation}
四维动量变换
\begin{equation}
    p^{\mu}=\left( E,\mathbf{p} \right) \Rightarrow p^{\prime \mu}={\mathcal{P} ^{\mu}}_{\nu}p^{\nu}=(\mathcal{P} p)^{\mu}=\left( E,-\mathbf{p} \right) 
\end{equation}
时空导数变换
\begin{equation}
    \partial _{\mu}^{\prime}=(\mathcal{P} ^{-1}{)^{\nu}}_{\mu}\partial _{\nu}
\end{equation}
保持时空体积元不变
\begin{equation}
    \mathrm{d}^4x^{\prime}=\left| \det \left( \mathcal{P} \right) \right|\mathrm{d}^4x=\mathrm{d}^4x
\end{equation}

如果场论系统的作用量S在宇称变换下不变,则运动方程的形式也在宇称变换下不变,此时称系统是宇称守恒的,即具有空间反射对称性。


在宇称守恒的量子理论中,宇称变换在 Hilbert 空间中诱导出态矢$|\Psi \rangle$的线性幺正变换
\begin{equation}
    |\Psi ^{\prime}\rangle =U(\mathcal{P} )|\Psi \rangle =P|\Psi \rangle 
\end{equation}





推导:
1.由标量场
\begin{equation}
    \phi (x)=\int{\frac{\mathrm{d}^3p}{\left( 2\pi \right) ^3}}\frac{1}{\sqrt{2E_{\mathbf{p}}}}\left( a_{\mathbf{p}}\mathrm{e}^{-\mathrm{i}p\cdot x}+b_{\mathbf{p}}^{\dagger}\mathrm{e}^{\mathrm{i}p\cdot x} \right) 
\end{equation}
得到宇称变换后的标量场
\begin{equation}
    \phi (\mathcal{P} x)=\int{\frac{\mathrm{d}^3p}{\left( 2\pi \right) ^3}}\frac{1}{\sqrt{2E_{\mathbf{p}}}}\left( a_{\mathbf{p}}\mathrm{e}^{-\mathrm{i}p\cdot \left( \mathcal{P} x \right)}+b_{\mathbf{p}}^{\dagger}\mathrm{e}^{\mathrm{i}p\cdot \left( \mathcal{P} x \right)} \right) 
\end{equation}
2.计算
\begin{equation}
    \begin{aligned}
        P^{-1}\phi (x)P&=\int{\frac{\mathrm{d}^3p}{\left( 2\pi \right) ^3}}\frac{1}{\sqrt{2E_{\mathbf{p}}}}\left( P^{-1}a_{\mathbf{p}}P\mathrm{e}^{-\mathrm{i}p\cdot x}+P^{-1}b_{\mathbf{p}}^{\dagger}P\mathrm{e}^{\mathrm{i}p\cdot x} \right) 
\\
&=\int{\frac{\mathrm{d}^3p}{\left( 2\pi \right) ^3}}\frac{1}{\sqrt{2E_{\mathbf{p}}}}\left( \eta _{P}^{*}a_{-\mathbf{p}}\mathrm{e}^{-\mathrm{i}p\cdot x}+\tilde{\eta}_Pb_{-\mathbf{p}}^{\dagger}\mathrm{e}^{\mathrm{i}p\cdot x} \right) 
\\
&=\int{\frac{\mathrm{d}^3p}{\left( 2\pi \right) ^3}}\frac{1}{\sqrt{2E_{\mathbf{p}}}}\left( \eta _{P}^{*}a_{\mathbf{p}}\mathrm{e}^{-\mathrm{i}\left( \mathcal{P} p \right) \cdot x}+\tilde{\eta}_Pb_{\mathbf{p}}^{\dagger}\mathrm{e}^{\mathrm{i}\left( \mathcal{P} p \right) \cdot x} \right) 
\\
&=\int{\frac{\mathrm{d}^3p}{\left( 2\pi \right) ^3}}\frac{1}{\sqrt{2E_{\mathbf{p}}}}\left( \eta _{P}^{*}a_{\mathbf{p}}\mathrm{e}^{-\mathrm{i}p\cdot \left( \mathcal{P} x \right)}+\tilde{\eta}_Pb_{\mathbf{p}}^{\dagger}\mathrm{e}^{\mathrm{i}p\cdot \left( \mathcal{P} x \right)} \right) 
\\
&=\int{\frac{\mathrm{d}^3p}{\left( 2\pi \right) ^3}}\frac{1}{\sqrt{2E_{\mathbf{p}}}}\left( \eta _{P}^{*}a_{\mathbf{p}}\mathrm{e}^{-\mathrm{i}p\cdot \left( \mathcal{P} x \right)}+\eta _{P}^{*}b_{\mathbf{p}}^{\dagger}\mathrm{e}^{\mathrm{i}p\cdot \left( \mathcal{P} x \right)} \right) 
\\
&=\eta _{P}^{*}\int{\frac{\mathrm{d}^3p}{\left( 2\pi \right) ^3}}\frac{1}{\sqrt{2E_{\mathbf{p}}}}\left( a_{\mathbf{p}}\mathrm{e}^{-\mathrm{i}p\cdot \left( \mathcal{P} x \right)}+b_{\mathbf{p}}^{\dagger}\mathrm{e}^{\mathrm{i}p\cdot \left( \mathcal{P} x \right)} \right) 
    \end{aligned}
\end{equation}
得到
\begin{equation}
    P^{-1}\phi (x)P=\eta _{P}^{*}\phi (\mathcal{P} x)
\end{equation}


推导:
1.由标量场的复共轭
\begin{equation}
    \phi ^{\dagger}(x)=\int{\frac{\mathrm{d}^3p}{\left( 2\pi \right) ^3}}\frac{1}{\sqrt{2E_{\mathbf{p}}}}\left( b_{\mathbf{p}}\mathrm{e}^{-\mathrm{i}p\cdot x}+a_{\mathbf{p}}^{\dagger}\mathrm{e}^{\mathrm{i}p\cdot x} \right) 
\end{equation}
得到
\begin{equation}
    \phi ^{\dagger}(\mathcal{P} x)=\int{\frac{\mathrm{d}^3p}{\left( 2\pi \right) ^3}}\frac{1}{\sqrt{2E_{\mathbf{p}}}}\left( b_{\mathbf{p}}\mathrm{e}^{-\mathrm{i}p\cdot \left( \mathcal{P} x \right)}+a_{\mathbf{p}}^{\dagger}\mathrm{e}^{\mathrm{i}p\cdot \left( \mathcal{P} x \right)} \right) 
\end{equation}
2.计算
\begin{equation}
    \begin{aligned}
        P^{-1}\phi ^{\dagger}(x)P&=\int{\frac{\mathrm{d}^3p}{\left( 2\pi \right) ^3}}\frac{1}{\sqrt{2E_{\mathbf{p}}}}\left( P^{-1}b_{\mathbf{p}}P\mathrm{e}^{-\mathrm{i}p\cdot x}+P^{-1}a_{\mathbf{p}}^{\dagger}P\mathrm{e}^{\mathrm{i}p\cdot x} \right) 
\\
&=\int{\frac{\mathrm{d}^3p}{\left( 2\pi \right) ^3}}\frac{1}{\sqrt{2E_{\mathbf{p}}}}\left( \tilde{\eta}_{P}^{*}b_{-\mathbf{p}}\mathrm{e}^{-\mathrm{i}p\cdot x}+\eta _Pa_{-\mathbf{p}}^{\dagger}\mathrm{e}^{\mathrm{i}p\cdot x} \right) 
\\
&=\int{\frac{\mathrm{d}^3p}{\left( 2\pi \right) ^3}}\frac{1}{\sqrt{2E_{\mathbf{p}}}}\left( \tilde{\eta}_{P}^{*}b_{\mathbf{p}}\mathrm{e}^{-\mathrm{i}\left( \mathcal{P} p \right) \cdot x}+\eta _Pa_{\mathbf{p}}^{\dagger}\mathrm{e}^{\mathrm{i}\left( \mathcal{P} p \right) \cdot x} \right) 
\\
&=\int{\frac{\mathrm{d}^3p}{\left( 2\pi \right) ^3}}\frac{1}{\sqrt{2E_{\mathbf{p}}}}\left( \tilde{\eta}_{P}^{*}b_{\mathbf{p}}\mathrm{e}^{-\mathrm{i}p\cdot \left( \mathcal{P} x \right)}+\eta _Pa_{\mathbf{p}}^{\dagger}\mathrm{e}^{\mathrm{i}p\cdot \left( \mathcal{P} x \right)} \right) 
\\
&=\int{\frac{\mathrm{d}^3p}{\left( 2\pi \right) ^3}}\frac{1}{\sqrt{2E_{\mathbf{p}}}}\left( \eta _Pb_{\mathbf{p}}\mathrm{e}^{-\mathrm{i}p\cdot \left( \mathcal{P} x \right)}+\eta _Pa_{\mathbf{p}}^{\dagger}\mathrm{e}^{\mathrm{i}p\cdot \left( \mathcal{P} x \right)} \right) 
\\
&=\eta _P\int{\frac{\mathrm{d}^3p}{\left( 2\pi \right) ^3}}\frac{1}{\sqrt{2E_{\mathbf{p}}}}\left( b_{\mathbf{p}}\mathrm{e}^{-\mathrm{i}p\cdot \left( \mathcal{P} x \right)}+a_{\mathbf{p}}^{\dagger}\mathrm{e}^{\mathrm{i}p\cdot \left( \mathcal{P} x \right)} \right) 
    \end{aligned}
\end{equation}
得到
\begin{equation}
    P^{-1}\phi ^{\dagger}(x)P=\eta _P\phi ^{\dagger}(\mathcal{P} x)
\end{equation}



\subsubsection{标量场的T变换}

时间反演变换
\begin{equation}
    {\mathcal{T} ^{\mu}}_{\nu}=(\mathcal{T} ^{-1}{)^{\mu}}_{\nu}=\left( \begin{matrix}
	-1&		&		&		\\
	&		+1&		&		\\
	&		&		+1&		\\
	&		&		&		+1\\
\end{matrix} \right) 
\end{equation}
时空坐标变换
\begin{equation}
    x^{\mu}=\left( t,\mathbf{x} \right) \Rightarrow \,\,x^{\prime \mu}={\mathcal{T} ^{\mu}}_{\nu}x^{\nu}=(\mathcal{T} x)^{\mu}=\left( -t,\mathbf{x} \right) 
\end{equation}
四维动量变换
\begin{equation}
    p^{\mu}=\left( E,\mathbf{p} \right) \Rightarrow \,\,p^{\prime \mu}={\mathcal{T} ^{\mu}}_{\nu}p^{\nu}=(\mathcal{T} p)^{\mu}=\left( -E,\mathbf{p} \right) 
\end{equation}
时空导数变换
\begin{equation}
    \partial _{\mu}^{\prime}=(\mathcal{T} ^{-1}{)^{\nu}}_{\mu}\partial _{\nu}
\end{equation}
时间反演变换保持时空体积元不变
\begin{equation}
    \mathrm{d}^4x^{\prime}=\left| \det\mathrm{(}\mathcal{T} ) \right|\mathrm{d}^4x=\mathrm{d}^4x
\end{equation}


推导:
1.由标量场
\begin{equation}
    \phi (x)=\int{\frac{\mathrm{d}^3p}{\left( 2\pi \right) ^3}\frac{1}{\sqrt{2E_{\mathbf{p}}}}\left( a_{\mathbf{p}}\mathrm{e}^{-\mathrm{i}p\cdot x}+b_{\mathbf{p}}^{\dagger}\mathrm{e}^{\mathrm{i}p\cdot x} \right)}
\end{equation}
得到时间反演变换后的标量场
\begin{equation}
    \phi (\mathcal{T} x)=\int{\frac{\mathrm{d}^3p}{\left( 2\pi \right) ^3}\frac{1}{\sqrt{2E_{\mathbf{p}}}}\left( a_{\mathbf{p}}\mathrm{e}^{-\mathrm{i}p\cdot \left( \mathcal{T} x \right)}+b_{\mathbf{p}}^{\dagger}\mathrm{e}^{\mathrm{i}p\cdot \left( \mathcal{T} x \right)} \right)}
\end{equation}
2.计算
\begin{equation}
    \begin{aligned}
        T^{-1}\phi \left( x \right) T&=\int{\frac{\mathrm{d}^3p}{\left( 2\pi \right) ^3}\frac{1}{\sqrt{2E_{\mathbf{p}}}}T^{-1}\left( a_{\mathbf{p}}\mathrm{e}^{-\mathrm{i}p\cdot x}+b_{\mathbf{p}}^{\dagger}\mathrm{e}^{\mathrm{i}p\cdot x} \right)}T
\\
&=\int{\frac{\mathrm{d}^3p}{\left( 2\pi \right) ^3}}\frac{1}{\sqrt{2E_{\mathbf{p}}}}\left( T^{-1}a_{\mathbf{p}}TT^{-1}\mathrm{e}^{-\mathrm{i}p\cdot x}T+T^{-1}b_{\mathbf{p}}^{\dagger}TT^{-1}\mathrm{e}^{\mathrm{i}p\cdot x}T \right) 
\\
&=\int{\frac{\mathrm{d}^3p}{\left( 2\pi \right) ^3}}\frac{1}{\sqrt{2E_{\mathbf{p}}}}\left( \eta _{T}^{*}a_{-\mathbf{p}}\mathrm{e}^{\mathrm{i}p\cdot x}+\tilde{\eta}_Tb_{-\mathbf{p}}^{\dagger}\mathrm{e}^{-\mathrm{i}p\cdot x} \right) 
\\
&=\int{\frac{\mathrm{d}^3p}{\left( 2\pi \right) ^3}}\frac{1}{\sqrt{2E_{\mathbf{p}}}}\left( \eta _{T}^{*}a_{\mathbf{p}}\mathrm{e}^{\mathrm{i}\left( \mathcal{P} p \right) \cdot x}+\tilde{\eta}_Tb_{\mathbf{p}}^{\dagger}\mathrm{e}^{-\mathrm{i}\left( \mathcal{P} p \right) \cdot x} \right) 
\\
&=\int{\frac{\mathrm{d}^3p}{\left( 2\pi \right) ^3}}\frac{1}{\sqrt{2E_{\mathbf{p}}}}\left( \eta _{T}^{*}a_{\mathbf{p}}\mathrm{e}^{\mathrm{i}p\cdot \left( \mathcal{P} x \right)}+\tilde{\eta}_Tb_{\mathbf{p}}^{\dagger}\mathrm{e}^{-\mathrm{i}p\cdot \left( \mathcal{P} x \right)} \right) 
\\
&=\int{\frac{\mathrm{d}^3p}{\left( 2\pi \right) ^3}}\frac{1}{\sqrt{2E_{\mathbf{p}}}}\left( \eta _{T}^{*}a_{\mathbf{p}}\mathrm{e}^{-\mathrm{i}p\cdot (\mathcal{T} x)}+\tilde{\eta}_Tb_{\mathbf{p}}^{\dagger}\mathrm{e}^{\mathrm{i}p\cdot (\mathcal{T} x)} \right) 
\\
&=\int{\frac{\mathrm{d}^3p}{\left( 2\pi \right) ^3}}\frac{1}{\sqrt{2E_{\mathbf{p}}}}\left( \eta _{T}^{*}a_{\mathbf{p}}\mathrm{e}^{-\mathrm{i}p\cdot (\mathcal{T} x)}+\eta _{T}^{*}b_{\mathbf{p}}^{\dagger}\mathrm{e}^{\mathrm{i}p\cdot (\mathcal{T} x)} \right) 
\\
&=\eta _{T}^{*}\int{\frac{\mathrm{d}^3p}{\left( 2\pi \right) ^3}}\frac{1}{\sqrt{2E_{\mathbf{p}}}}\left( a_{\mathbf{p}}\mathrm{e}^{-\mathrm{i}p\cdot (\mathcal{T} x)}+b_{\mathbf{p}}^{\dagger}\mathrm{e}^{\mathrm{i}p\cdot (\mathcal{T} x)} \right) 
    \end{aligned}
\end{equation}
对比得到


1.由
\begin{equation}
    \phi ^{\dagger}(x)=\int{\frac{\mathrm{d}^3p}{\left( 2\pi \right) ^3}}\frac{1}{\sqrt{2E_{\mathbf{p}}}}\left( b_{\mathbf{p}}\mathrm{e}^{-\mathrm{i}p\cdot x}+a_{\mathbf{p}}^{\dagger}\mathrm{e}^{\mathrm{i}p\cdot x} \right) 
\end{equation}
得到
\begin{equation}
    \phi ^{\dagger}(\mathcal{T} x)=\int{\frac{\mathrm{d}^3p}{\left( 2\pi \right) ^3}}\frac{1}{\sqrt{2E_{\mathbf{p}}}}\left( b_{\mathbf{p}}\mathrm{e}^{-\mathrm{i}p\cdot \left( \mathcal{T} x \right)}+a_{\mathbf{p}}^{\dagger}\mathrm{e}^{\mathrm{i}p\cdot \left( \mathcal{T} x \right)} \right) 
\end{equation}
2.计算
\begin{equation}
    \begin{aligned}
        T^{-1}\phi ^{\dagger}\left( x \right) T&=\int{\frac{\mathrm{d}^3p}{\left( 2\pi \right) ^3}}\frac{1}{\sqrt{2E_{\mathbf{p}}}}T^{-1}\left( b_{\mathbf{p}}\mathrm{e}^{-\mathrm{i}p\cdot x}+a_{\mathbf{p}}^{\dagger}\mathrm{e}^{\mathrm{i}p\cdot x} \right) T
\\
&=\int{\frac{\mathrm{d}^3p}{\left( 2\pi \right) ^3}}\frac{1}{\sqrt{2E_{\mathbf{p}}}}\left( T^{-1}b_{\mathbf{p}}TT^{-1}\mathrm{e}^{-\mathrm{i}p\cdot x}T+T^{-1}a_{\mathbf{p}}^{\dagger}TT^{-1}\mathrm{e}^{\mathrm{i}p\cdot x}T \right) 
\\
&=\int{\frac{\mathrm{d}^3p}{\left( 2\pi \right) ^3}}\frac{1}{\sqrt{2E_{\mathbf{p}}}}\left( \tilde{\eta}_{T}^{*}b_{-\mathbf{p}}\mathrm{e}^{\mathrm{i}p\cdot x}+\eta _Ta_{-\mathbf{p}}^{\dagger}\mathrm{e}^{-\mathrm{i}p\cdot x} \right) 
\\
&=\int{\frac{\mathrm{d}^3p}{\left( 2\pi \right) ^3}}\frac{1}{\sqrt{2E_{\mathbf{p}}}}\left( \tilde{\eta}_{T}^{*}b_{\mathbf{p}}\mathrm{e}^{\mathrm{i}\left( \mathcal{P} p \right) \cdot x}+\eta _Ta_{\mathbf{p}}^{\dagger}\mathrm{e}^{-\mathrm{i}\left( \mathcal{P} p \right) \cdot x} \right) 
\\
&=\int{\frac{\mathrm{d}^3p}{\left( 2\pi \right) ^3}}\frac{1}{\sqrt{2E_{\mathbf{p}}}}\left( \tilde{\eta}_{T}^{*}b_{\mathbf{p}}\mathrm{e}^{\mathrm{i}p\cdot \left( \mathcal{P} x \right)}+\eta _Ta_{\mathbf{p}}^{\dagger}\mathrm{e}^{-\mathrm{i}p\cdot \left( \mathcal{P} x \right)} \right) 
\\
&=\int{\frac{\mathrm{d}^3p}{\left( 2\pi \right) ^3}}\frac{1}{\sqrt{2E_{\mathbf{p}}}}\left( \tilde{\eta}_{T}^{*}b_{\mathbf{p}}\mathrm{e}^{-\mathrm{i}p\cdot (\mathcal{T} x)}+\eta _Ta_{\mathbf{p}}^{\dagger}\mathrm{e}^{\mathrm{i}p\cdot (\mathcal{T} x)} \right) 
\\
&=\int{\frac{\mathrm{d}^3p}{\left( 2\pi \right) ^3}}\frac{1}{\sqrt{2E_{\mathbf{p}}}}\left( \eta _Tb_{\mathbf{p}}\mathrm{e}^{-\mathrm{i}p\cdot (\mathcal{T} x)}+\eta _Ta_{\mathbf{p}}^{\dagger}\mathrm{e}^{\mathrm{i}p\cdot (\mathcal{T} x)} \right) 
\\
&=\eta _T\int{\frac{\mathrm{d}^3p}{\left( 2\pi \right) ^3}}\frac{1}{\sqrt{2E_{\mathbf{p}}}}\left( b_{\mathbf{p}}\mathrm{e}^{-\mathrm{i}p\cdot (\mathcal{T} x)}+a_{\mathbf{p}}^{\dagger}\mathrm{e}^{\mathrm{i}p\cdot (\mathcal{T} x)} \right) 
    \end{aligned}
\end{equation}
对比得到
\begin{equation}
    T^{-1}\phi ^{\dagger}\left( x \right) T=\eta _T\phi ^{\dagger}(\mathcal{T} x)
\end{equation}


由标量场
\begin{equation}
    \phi \left( x \right) =\int{\frac{\mathrm{d}^3p}{\left( 2\pi \right) ^3}}\frac{1}{\sqrt{2E_{\mathbf{p}}}}\left( a_{\mathbf{p}}\mathrm{e}^{-\mathrm{i}p\cdot x}+b_{\mathbf{p}}^{\dagger}\mathrm{e}^{\mathrm{i}p\cdot x} \right) 
\end{equation}
和复共轭
\begin{equation}
    \phi ^{\dagger}(x)=\int{\frac{\mathrm{d}^3p}{\left( 2\pi \right) ^3}}\frac{1}{\sqrt{2E_{\mathbf{p}}}}\left( b_{\mathbf{p}}\mathrm{e}^{-\mathrm{i}p\cdot x}+a_{\mathbf{p}}^{\dagger}\mathrm{e}^{\mathrm{i}p\cdot x} \right) 
\end{equation}

2.计算
\begin{equation}
    \begin{aligned}
        C^{-1}\phi C&=\int{\frac{\mathrm{d}^3p}{\left( 2\pi \right) ^3}}\frac{1}{\sqrt{2E_{\mathbf{p}}}}C^{-1}\left( a_{\mathbf{p}}\mathrm{e}^{-\mathrm{i}p\cdot x}+b_{\mathbf{p}}^{\dagger}\mathrm{e}^{\mathrm{i}p\cdot x} \right) C
\\
&=\int{\frac{\mathrm{d}^3p}{\left( 2\pi \right) ^3}}\frac{1}{\sqrt{2E_{\mathbf{p}}}}\left( C^{-1}a_{\mathbf{p}}C\mathrm{e}^{-\mathrm{i}p\cdot x}+C^{-1}b_{\mathbf{p}}^{\dagger}C\mathrm{e}^{\mathrm{i}p\cdot x} \right) 
\\
&=\int{\frac{\mathrm{d}^3p}{\left( 2\pi \right) ^3}}\frac{1}{\sqrt{2E_{\mathbf{p}}}}\left( \eta _{C}^{*}b_{\mathbf{p}}\mathrm{e}^{-\mathrm{i}p\cdot x}+\eta _{C}^{*}a_{\mathbf{p}}^{\dagger}\mathrm{e}^{\mathrm{i}p\cdot x} \right) 
\\
&=\eta _{C}^{*}\int{\frac{\mathrm{d}^3p}{\left( 2\pi \right) ^3}}\frac{1}{\sqrt{2E_{\mathbf{p}}}}\left( b_{\mathbf{p}}\mathrm{e}^{-\mathrm{i}p\cdot x}+a_{\mathbf{p}}^{\dagger}\mathrm{e}^{\mathrm{i}p\cdot x} \right) 
    \end{aligned}
\end{equation}
对比得到
\begin{equation}
    C^{-1}\phi (x)C=\eta _{C}^{*}\phi ^{\dagger}(x)
\end{equation}
同样地
\begin{equation}
    \begin{aligned}
        C^{-1}\phi ^{\dagger}(x)C&=\int{\frac{\mathrm{d}^3p}{\left( 2\pi \right) ^3}}\frac{1}{\sqrt{2E_{\mathbf{p}}}}C^{-1}\left( b_{\mathbf{p}}\mathrm{e}^{-\mathrm{i}p\cdot x}+a_{\mathbf{p}}^{\dagger}\mathrm{e}^{\mathrm{i}p\cdot x} \right) C
\\
&=\int{\frac{\mathrm{d}^3p}{\left( 2\pi \right) ^3}}\frac{1}{\sqrt{2E_{\mathbf{p}}}}\left( C^{-1}b_{\mathbf{p}}C\mathrm{e}^{-\mathrm{i}p\cdot x}+C^{-1}a_{\mathbf{p}}^{\dagger}C\mathrm{e}^{\mathrm{i}p\cdot x} \right) 
\\
&=\int{\frac{\mathrm{d}^3p}{\left( 2\pi \right) ^3}}\frac{1}{\sqrt{2E_{\mathbf{p}}}}\left( \eta _Ca_{\mathbf{p}}\mathrm{e}^{-\mathrm{i}p\cdot x}+\eta _Cb_{\mathbf{p}}^{\dagger}\mathrm{e}^{\mathrm{i}p\cdot x} \right) 
\\
&=\eta _C\int{\frac{\mathrm{d}^3p}{\left( 2\pi \right) ^3}}\frac{1}{\sqrt{2E_{\mathbf{p}}}}\left( a_{\mathbf{p}}\mathrm{e}^{-\mathrm{i}p\cdot x}+b_{\mathbf{p}}^{\dagger}\mathrm{e}^{\mathrm{i}p\cdot x} \right) 
    \end{aligned}
\end{equation}
对比得到
\begin{equation}
    C^{-1}\phi ^{\dagger}(x)C=\eta _C\phi (x)
\end{equation}

\section{习题9}

\newpage
\subsection{9.1}
利用(9.11)和(9.12)式,证明算符 $J \cdot P \equiv J^i P^i$ 满足
$$P^{-1}(J \cdot P)P = -J \cdot P. \tag{9.545}$$
设 $|E, p, \sigma\rangle$ 是哈密顿量算符 $H$、动量算符 $P$ 和算符 $J \cdot P$ 的共同本征态,满足
$$H | E, p, \sigma \rangle = E | E, p, \sigma \rangle, \quad P | E, p, \sigma \rangle = p | E, p, \sigma \rangle, \quad J \cdot P | E, p, \sigma \rangle = \sigma | E, p, \sigma \rangle. \tag{9.546}$$
用 $P$ 变换定义 $|E, p, \sigma \rangle' \equiv P | E, p, \sigma \rangle$,证明
$$H | E, p, \sigma \rangle' = E | E, p, \sigma \rangle', \quad P | E, p, \sigma \rangle' = -p | E, p, \sigma \rangle', \quad J \cdot P | E, p, \sigma \rangle' = -\sigma | E, p, \sigma \rangle'. \tag{9.547}$$

\newpage
\subsection{9.2}
根据(9.42)式,证明
$$P^{r-1} | \phi \bar{\phi} \rangle = (-)^L | \phi \bar{\phi} \rangle, \tag{9.548}$$
其中 $|\phi \bar{\phi}\rangle$ 是一对正反标量玻色子 $\phi \bar{\phi}$ 组成的态(9.28),$L$ 是它的轨道角动量量子数。

\newpage
\subsection{9.3}
根据(2.182)式把复标量场 $\phi(x)$ 分解为两个实标量场 $\phi_1(x)$ 和 $\phi_2(x)$,并将 $C$ 变换相位因子改写为 $\eta_C = e^{i\theta}$,其中 $\theta$ 是实数。

(a) 利用 $ C $ 变换性质 (9.79) 证明
$$C^{-1} \begin{pmatrix}
\phi_1(x) \\
-\phi_2(x)
\end{pmatrix} C = \begin{pmatrix}
c_\theta & -s_\theta \\
s_\theta & c_\theta
\end{pmatrix} \begin{pmatrix}
\phi_1(x) \\
\phi_2(x)
\end{pmatrix}, \tag{9.549}$$
其中 $ c_\theta \equiv \cos \theta, \, s_\theta \equiv \sin \theta $。

(b) 通过 O(2) 整体变换
$$\begin{pmatrix}
\hat{\phi}_1 \\
\hat{\phi}_2
\end{pmatrix} \equiv \begin{pmatrix}
c_{\theta/2} & s_{\theta/2} \\
-s_{\theta/2} & c_{\theta/2}
\end{pmatrix} \begin{pmatrix}
1 \\
-1
\end{pmatrix} \begin{pmatrix}
\phi_1 \\
\phi_2
\end{pmatrix} = \begin{pmatrix}
c_{\theta/2} & s_{\theta/2} \\
-s_{\theta/2} & c_{\theta/2}
\end{pmatrix} \begin{pmatrix}
\phi_1 \\
-\phi_2
\end{pmatrix} \tag{9.550}$$
引入实标量场 $\hat{\phi}_1(x)$ 和 $\hat{\phi}_2(x)$, 证明它们具有 $ C $ 变换性质
$$C^{-1} \hat{\phi}_1(x) C = + \hat{\phi}_1(x), \quad C^{-1} \hat{\phi}_2(x) C = -\hat{\phi}_2(x). \tag{9.551}$$
可见, $\hat{\phi}_1(x)$ 和 $\hat{\phi}_2(x)$ 是 $ C $ 字称分别为偶和奇的本征态。

\newpage
\subsection{9.4}
在 Dirac 表象 (5.300) 中,考虑 Dirac 旋量场 $\psi(x)$ 的平面波展开式 (5.314),产生湮灭算符 $(c_{p,\sigma}, c_{p,\sigma}^\dagger)$ 和 $(d_{p,\sigma}, d_{p,\sigma}^\dagger)$ 满足反对易关系 (5.315),平面波旋量系数 $u(p, \sigma)$ 和 $v(p, \sigma)$ 由 (5.301) 式给出。

(a) 推出 $C = i \gamma^0 \gamma^2$ 和 $C \gamma^5$ 的具体形式。

(b) 证明
$$Cu^T(p, \sigma) = v(p, \sigma), \quad Cu^T(p, \sigma) = u(p, \sigma), \tag{9.552}$$
$$\gamma^0 u(p, \sigma) = u(-p, \sigma), \quad \gamma^0 v(p, \sigma) = -v(-p, \sigma), \tag{9.553}$$
$$C \gamma^5 u(p, \sigma) = \tau_{-\sigma}^* u^*(-p, -\sigma), \quad C \gamma^5 v(p, \sigma) = -\tau_{\sigma} v^*(-p, -\sigma), \tag{9.554}$$
其中 $\tau_{\sigma}$ 是 (5.308) 式中的相位因子。

(c) 设产生湮灭算符的 $C, P, T$ 变换为
$$C^{-1} c_{p,\sigma} C = \zeta_C^* d_{p,\sigma}, \quad C^{-1} d_{p,\sigma}^\dagger C = \zeta_C^* c_{p,\sigma}^\dagger, \tag{9.555}$$
$$P^{-1} c_{p,\sigma} P = \zeta_P^* c_{-p,\sigma}, \quad P^{-1} d_{p,\sigma}^\dagger P = -\zeta_P^* d_{-p,\sigma}^\dagger, \tag{9.556}$$
$$T^{-1} c_{p,\sigma} T = \zeta_T^* \tau_{-\sigma}^* c_{-p,-\sigma}, \quad T^{-1} d_{p,\sigma}^\dagger T = -\zeta_T^* \tau_{-\sigma}^* d_{-p,-\sigma}^\dagger, \tag{9.557}$$
推出
$$C^{-1} \psi(x) C = \zeta_C^* \psi^C(x), \quad P^{-1} \psi(x) P = \zeta_P^* \gamma^0 \psi(Px), \quad T^{-1} \psi(x) T = \zeta_T^* C \gamma^5 \psi(Tx). \tag{9.558}$$

在质心系中考虑一对正反费米子 $\psi \bar{\psi}$ 组成的系统,当轨道角动量和总自旋角动量的量子数分别为 $L$ 和 $S$ 时,态矢表达为
$$|\psi \bar{\psi}\rangle_{L,S} = \sum_{\sigma, \sigma' = \pm 1/2} \int d^3p \Phi(p, \sigma, \sigma') c_p^i d_{-p,\sigma'}^i |0\rangle ,$$
其中波函数分解为
$$\Phi(p, \sigma, \sigma') = R(|p|) Y_{LM}(\theta, \phi) \chi^S_{\sigma S}(\sigma, \sigma'), \quad S = 0, 1, \quad \sigma_S = 0, \cdots, \pm S.$$
这里 $\theta$ 和 $\phi$ 分别是球坐标系中动量 p 的极角和方位角。$\chi^S_{\sigma S}(\sigma, \sigma')$ 是自旋本征波函数,可以用正费米子 $\psi$ 和反费米子 $\bar{\psi}$ 各自的自旋本征态 $\zeta_\sigma$ 和 $\eta_{\sigma'}$ 的张量积表达为
$$\chi_0^0(\sigma, \sigma') = \frac{1}{\sqrt{2}} (\zeta_{+1/2} \otimes \eta_{-1/2} - \zeta_{-1/2} \otimes \eta_{+1/2}),$$
$$\chi_{+1}^1(\sigma, \sigma') = \zeta_{+1/2} \otimes \eta_{+1/2},$$
$$\chi_0^1(\sigma, \sigma') = \frac{1}{\sqrt{2}} (\zeta_{+1/2} \otimes \eta_{-1/2} + \zeta_{-1/2} \otimes \eta_{+1/2}),$$
$$\chi_{-1}^1(\sigma, \sigma') = \zeta_{-1/2} \otimes \eta_{-1/2}.$$
(d) 证明
$$\Phi(-p, \sigma', \sigma) = (-)^{L+S+1} \Phi(p, \sigma, \sigma')$$
和
$$C |\psi \bar{\psi}\rangle_{L,S} = (-)^{L+S} |\psi \bar{\psi}\rangle_{L,S}, \quad P |\psi \bar{\psi}\rangle_{L,S} = (-)^{L+1} |\psi \bar{\psi}\rangle_{L,S}.$$
可见,扣除波函数 $\Phi(p, \sigma, \sigma')$ 对 $C$ 宇称的贡献 $(-)^{L+S+1}$ 之后,一对正反费米子的内禀 $C$ 宇称为奇。

\newpage
\subsection{9.5}
类似于 (4.276) 式,无质量复矢场 $A^\mu(x)$ 的平面波展开式为
$$A^\mu(x) = \int \frac{d^3p}{(2\pi)^3} \frac{1}{\sqrt{2E_p}} \sum_{\lambda = \pm} \left[ \epsilon^\mu(p, \lambda) a_{p,\lambda} e^{-ipx} + \epsilon^{\mu*}(p, \lambda) c_p^i \right] e^{ipx} + \int \frac{d^3p}{(2\pi)^3} \frac{1}{\sqrt{2E_p}} \sum_{\sigma = 0,3} \epsilon^\mu(p, \sigma) \left( b_{p,\sigma} e^{-ipx} + d_{p,\sigma}^i e^{ipx} \right).$$
其中极化矢量 $\epsilon^\mu(p, \lambda) (\lambda = \pm)$ 的表达式由 (4.105) 式给出,极化矢量 $\epsilon^\mu(p, \sigma) (\sigma = 0,3)$ 的表达式是 (4.186) 和 (4.189) 式。
(a) 证明
$$\epsilon^\mu(-p, \sigma) = P^\mu_{\nu} \epsilon^\nu(p, \sigma), \quad \sigma = 0, 3.$$
(b) 设产生湮灭算符的 $C, P, T$ 变换为
$$C^{-1} a_{p,\lambda} C = \xi_c^* c_{p,\lambda}, \quad C^{-1} c_{p,\lambda}^i C = \xi_c^* c_{p,\lambda}^i,$$
$$C^{-1}b_{p,\sigma}C = \xi^*_c d_{p,\sigma}, \quad C^{-1}d_{p,\sigma}^\dagger C = \xi^*_c b_{p,\sigma}^\dagger, \tag{9.570}$$
$$P^{-1}a_{p,\lambda}P = -\xi^*_p a_{-p,- \lambda}, \quad P^{-1}c_{p,\lambda}^\dagger P = -\xi^*_p c_{-p,- \lambda}^\dagger, \tag{9.571}$$
$$P^{-1}b_{p,\sigma}P = \xi^*_p b_{-p,\sigma}, \quad P^{-1}d_{p,\sigma}^\dagger P = \xi^*_p d_{-p,\sigma}^\dagger, \tag{9.572}$$
$$T^{-1}a_{p,\lambda}T = \xi^*_T a_{-p,\lambda}, \quad T^{-1}c_{p,\lambda}^\dagger T = \xi^*_T c_{-p,\lambda}^\dagger, \tag{9.573}$$
$$T^{-1}b_{p,\sigma}T = -\xi^*_T b_{-p,\sigma}, \quad T^{-1}d_{p,\sigma}^\dagger T = -\xi^*_T d_{-p,\sigma}^\dagger, \tag{9.574}$$
推出 $A^\mu(x)$ 的 $C, P, T$ 变换 (9.254), (9.255), (9.256)。

\newpage
\subsection{9.6}
验证无源 Maxwell 方程 $\partial_\mu F^{\mu\nu}(x) = 0$ 在 $C, P, T$ 变换下都保持不变。

\newpage
\subsection{9.7}
对于参与相互作用的复标量场,Heisenberg 绘景与相互作用绘景中场算符的变换关系为
$$\phi^H(x) = V^\dagger(t)\phi^I(x)V(t), \quad 其中 \quad V(t) = e^{iH_0^S t}e^{-iHt}. $$
类似于自由场,$\phi^I(x)$ 的 $P, T, C$ 变换为
$$P^{-1}\phi^I(x)P = \eta^*_P \phi^I(Px), \quad T^{-1}\phi^I(x)T = \eta^*_T \phi^I(Tx), \quad C^{-1}\phi^I(x)C = \eta^*_C \phi^H(x). \tag{9.575}$$
假设 $[H, P] = [H_0^S, P] = [H, T] = [H_0^S, T] = [H, C] = [H_0^S, C] = 0$, 证明 $\phi^H(x)$ 的 $P, T, C$ 变换为
$$P^{-1}\phi^H(x)P = \eta^*_P \phi^H(Px), \quad T^{-1}\phi^H(x)T = \eta^*_T \phi^H(Tx), \quad C^{-1}\phi^H(x)C = \eta^*_C \phi^H(x). \tag{9.576}$$

\newpage
\subsection{9.8}
设 $\eta^a$ 和 $\zeta_a$ 是左手 Weyl 旋量,证明 $\eta\sigma^\mu\sigma^\nu\zeta = \eta^a(\sigma^\mu)_{ab}(\sigma^\nu)^{bc}\zeta_c$ 是 Lorentz 张量。





\include{余钊焕-10}
\include{余钊焕-10习题}

\end{document}