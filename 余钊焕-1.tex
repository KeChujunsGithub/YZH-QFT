\section{预备知识}

量子场论与粒子物理密切相关。

粒子物理研究物质的基本结构和基本相互作用,组成物质的基本单元是粒子。

自然界中存在 4 种基本相互作用,
即引力相互作用 (gravitational interaction)、
电磁相互作用 (electromagnetic interaction) 、 
强相互作用 (strong interaction) 、
弱相互作用 (weak interaction),
支配着基本粒子的运动和转化。

基本粒子指尚未发现内部结构的粒子。
目前已发现 3 代基本费米子 ,
每一代包含带电轻子、中微子(中性轻子)、下型夸克、上型夸克各一种。
\\第1代基本费米子包括电子$(e)$、电子型中微子$(\nu_e)$、下夸克$(d)$和上夸克$(u)$ ;
\\第2代包括$\mu$子$(\mu)$、$\mu$子型中微子$(\nu_\mu)$、奇夸克$(s)$和粲夸克$(c)$;
\\第3代包括$\tau$子$(\tau)$、$\tau$子型中微子$(\nu_\tau)$、 底夸克$(b)$和顶夸克$(t)$。
\\某代某种费米子与它在另一代中相对应的费米子具有相同的量子数,但质量不同。(表格中费米子部分同一列)

夸克的种类也称为味道 (flavor), 6 种味道的夸克$d$、$u$、$s$、$c$、$b$、$t$具有不同的质量。
\\每一味夸克都具有 3 种颜色 (color),
\\同味异色的夸克具有相同质量,严格构成颜色三重态,与描述强相互作用的量子色动力学有关。

多个夸克可以通过强相互作用组成复合粒子、称为强子(hadron),比如,
\\1个介子(meson)由1个正夸克和1个反夸克组成,
\\一个重子(baryon)由3个正夸克或3个反夸克组成。
\\除 3 代中微子以外,其它基本费米子都具有电荷 (electric charge) , 参与电磁相互作用,相应的理论称为量子电动力学。
\\所有基本费米子都参与弱相互作用,它与电磁相互作用统一由电弱规范理论描述,这个理论包含了量子电动力学。电弱规范理论和量子色动力学统称为标准模型 (standard model)。

在标准模型中,费米子的相互作用由一些基本玻色子 (boson) 传递。
\\传递夸克间强相互作用的规范玻色子称为胶子 (gluon),
\\传递电磁相互作用的规范玻色子是光子 (photon),
\\传递弱相互作用的规范玻色子是$W^\pm$和$Z^0$玻色子。
\\此外,还存在一种 Higgs 玻色子,它与电弱规范对称性的自发破缺及基本粒子的质量起源有关。

标准模型是在研究基本粒子如何参与电磁、强、弱相互作用的过程中建立起来的,它的理论
基础就是量子场论。在粒子物理理论和实验发展的过程中,量子场论起着极为关键的作用。反
过来,粒子物理研究也极大地促进了量子场论的发展。

\begin{table}[ht]
\centering
\caption{标准模型粒子分类表}
\begin{tabular}{|l|c|c|c|c|c|}
\hline
 & \textbf{(第一代)费米子} & \textbf{(第二代)费米子} & \textbf{(第三代)费米子} & \textbf{规范玻色子} & \textbf{标量玻色子} \\
\hline
\multirow{4}{*}{\textbf{上型夸克}} 
& \makecell{u夸克 \\ 2.2\,MeV \\ +2/3 \\ 1/2} 
& \makecell{c夸克 \\ 1.27\,GeV \\ +2/3 \\ 1/2} 
& \makecell{t夸克 \\ 173\,GeV \\ +2/3 \\ 1/2} 
& \makecell{光子 ($\gamma$) \\ 0 \\ 0 \\ 1} 
& \multirow{-4}{*}{\makecell{希格斯玻色子 ($H^0$) \\ 125\,GeV \\ 0 \\ 0}} \\
\cline{2-5}

\multirow{4}{*}{\textbf{下型夸克}} 
& \makecell{d夸克 \\ 4.7\,MeV \\ -1/3 \\ 1/2} 
& \makecell{s夸克 \\ 95\,MeV \\ -1/3 \\ 1/2} 
& \makecell{b夸克 \\ 4.18\,GeV \\ -1/3 \\ 1/2} 
& \makecell{$W^\pm$玻色子 \\ 80.4\,GeV \\ $\pm$1 \\ 1} 
& \\ \cline{2-5}

\multirow{4}{*}{\textbf{带电轻子}} 
& \makecell{电子 ($e^-$) \\ 0.511\,MeV \\ -1 \\ 1/2} 
& \makecell{$\mu$子 ($\mu^-$) \\ 105.7\,MeV \\ -1 \\ 1/2} 
& \makecell{$\tau$子 ($\tau^-$) \\ 1.777\,GeV \\ -1 \\ 1/2} 
& \makecell{$Z$玻色子 \\ 91.2\,GeV \\ 0 \\ 1} 
& \\ \cline{2-5}

\multirow{4}{*}{\textbf{中微子}\\ \textbf{中性轻子}} 
& \makecell{$\nu_e$电子中微子 \\ <1\,eV \\ 0 \\ 1/2} 
& \makecell{$\nu_\mu$$\mu$子中微子 \\ <0.1\,eV \\ 0 \\ 1/2} 
& \makecell{$\nu_\tau$$\tau$子中微子 \\ <15.5\,MeV \\ 0 \\ 1/2} 
& \makecell{胶子 (8种) \\ 0 \\ 0 \\ 1} 
& \\ \hline
\end{tabular}
\end{table}

%%%%%%%%%%%%%%%%%%%%%%%%%%%%%%%%%%%%%%%%%%%%%%%%%%%%%5
\subsection{}












%%%%%%%%%%%%%%%%%%%%%%%%%%%%%%%%%%%%%%%%%%%%%%%%%%%%%5
\subsection{}















%%%%%%%%%%%%%%%%%%%%%%%%%%%%%%%%%%%%%%%%%%%%%%%%%%%%%5
\subsection{}












%%%%%%%%%%%%%%%%%%%%%%%%%%%%%%%%%%%%%%%%%%%%%%%%%%%%%5
\subsection{}














%%%%%%%%%%%%%%%%%%%%%%%%%%%%%%%%%%%%%%%%%%%%%%%%%%%%%5
\subsection{}














%%%%%%%%%%%%%%%%%%%%%%%%%%%%%%%%%%%%%%%%%%%%%%%%%%%%%5
\subsection{ Lorentz 矢量}

\newpage
\subsection{推导:的运动方程}
1.
对于
\begin{equation}
     S=\int_{t_1}^{t_2}{\mathrm{d}t}L
\end{equation}
两边同时变分
\begin{equation}
     \delta S=\int_{t_1}^{t_2}{\mathrm{d}t}\delta L
\end{equation}
2.对于拉氏量
\begin{equation}
    L=L[q_i(t),\dot{q}_i(t)]
\end{equation}
变分计算为
\begin{equation}
     \delta L=\frac{\partial L}{\partial q_i}\delta q_i+\frac{\partial L}{\partial \dot{q}_i}\delta \dot{q}_i
\end{equation}
得到
\begin{equation}
    \delta S=\int_{t_1}^{t_2}{\mathrm{d}t}\left( \frac{\partial L}{\partial q_i}\delta q_i+\frac{\partial L}{\partial \dot{q}_i}\delta \dot{q}_i \right) 
\end{equation}
3.因为求导和变分可交换
\begin{equation}
     \delta \dot{q}_i=\delta \left( \frac{\mathrm{d}}{\mathrm{d}t}q_i \right) =\frac{\mathrm{d}}{\mathrm{d}t}\left( \delta q_i \right) 
\end{equation}
得到
\begin{equation}
     \begin{aligned}
         \delta S&=\int_{t_1}^{t_2}{\mathrm{d}t}\left( \frac{\partial L}{\partial q_i}\delta q_i+\frac{\partial L}{\partial \dot{q}_i}\delta \left( \frac{\mathrm{d}}{\mathrm{d}t}q_i \right) \right) 
\\
&=\int_{t_1}^{t_2}{\mathrm{d}t}\left( \frac{\partial L}{\partial q_i}\delta q_i+\frac{\partial L}{\partial \dot{q}_i}\frac{\mathrm{d}}{\mathrm{d}t}\left( \delta q_i \right) \right) 
     \end{aligned}
\end{equation}
4.求导法则
\begin{equation}
    \frac{\mathrm{d}}{\mathrm{d}t}\left( \frac{\partial L}{\partial \dot{q}_i}\delta q_i \right) =\left( \frac{\mathrm{d}}{\mathrm{d}t}\frac{\partial L}{\partial \dot{q}_i} \right) \delta q_i+\frac{\partial L}{\partial \dot{q}_i}\frac{\mathrm{d}}{\mathrm{d}t}\left( \delta q_i \right) 
\end{equation}
写出
\begin{equation}
    \frac{\partial L}{\partial \dot{q}_i}\frac{\mathrm{d}}{\mathrm{d}t}\left( \delta q_i \right) =\frac{\mathrm{d}}{\mathrm{d}t}\left( \frac{\partial L}{\partial \dot{q}_i}\delta q_i \right) -\left( \frac{\mathrm{d}}{\mathrm{d}t}\frac{\partial L}{\partial \dot{q}_i} \right) \delta q_i
\end{equation}
得到
\begin{equation}
    \begin{aligned}
        \delta S&=\int_{t_1}^{t_2}{\mathrm{d}t}\left[ \frac{\partial L}{\partial q_i}\delta q_i+\frac{\mathrm{d}}{\mathrm{d}t}\left( \frac{\partial L}{\partial \dot{q}_i}\delta q_i \right) -\left( \frac{\mathrm{d}}{\mathrm{d}t}\frac{\partial L}{\partial \dot{q}_i} \right) \delta q_i \right] 
\\
&=\int_{t_1}^{t_2}{\mathrm{d}t}\left( \frac{\partial L}{\partial q_i}-\frac{\mathrm{d}}{\mathrm{d}t}\frac{\partial L}{\partial \dot{q}_i} \right) \delta q_i+\int_{t_1}^{t_2}{\mathrm{d}t}\frac{\mathrm{d}}{\mathrm{d}t}\left( \frac{\partial L}{\partial \dot{q}_i}\delta q_i \right) 
    \end{aligned}
\end{equation}
5.牛顿莱布尼兹法则
\begin{equation}
    \int_{t_1}^{t_2}{\mathrm{d}t}\frac{\mathrm{d}}{\mathrm{d}t}\left( \frac{\partial L}{\partial \dot{q}_i}\delta q_i \right) =\int_{t_1}^{t_2}{\mathrm{d}\left( \frac{\partial L}{\partial \dot{q}_i}\delta q_i \right)}=\frac{\partial L}{\partial \dot{q}_i}\delta q_i|_{t_1}^{t_2}
\end{equation}
得到
\begin{equation}
    \delta S=\int_{t_1}^{t_2}{\mathrm{d}t}\left( \frac{\partial L}{\partial q_i}-\frac{\mathrm{d}}{\mathrm{d}t}\frac{\partial L}{\partial \dot{q}_i} \right) \delta q_i+\frac{\partial L}{\partial \dot{q}_i}\delta q_i|_{t_1}^{t_2}
\end{equation}
6.
\begin{equation}
    \delta q_i(t_1)=\delta q_i(t_2)=0
\end{equation}
写出
\begin{equation}
    \frac{\partial L}{\partial \dot{q}_i}\delta q_i|_{t_1}^{t_2}=\frac{\partial L}{\partial \dot{q}_i}\delta q_i(t_2)-\frac{\partial L}{\partial \dot{q}_i}\delta q_i(t_1)=0
\end{equation}
得到
\begin{equation}
    \delta S=\int_{t_1}^{t_2}{\mathrm{d}t}\left( \frac{\partial L}{\partial q_i}-\frac{\mathrm{d}}{\mathrm{d}t}\frac{\partial L}{\partial \dot{q}_i} \right) \delta q_i
\end{equation}
7.利用最小作用量原理
\begin{equation}
    \delta S=0
\end{equation}
等式右边积分为零
\begin{equation}
    \int_{t_1}^{t_2}{\mathrm{d}t}\left( \frac{\partial L}{\partial q_i}-\frac{\mathrm{d}}{\mathrm{d}t}\frac{\partial L}{\partial \dot{q}_i} \right) \delta q_i=0
\end{equation}
被积函数为零
\begin{equation}
    \frac{\partial L}{\partial q_i}-\frac{\mathrm{d}}{\mathrm{d}t}\frac{\partial L}{\partial \dot{q}_i}=0
\end{equation}

\newpage
\subsection{推导:场的运动方程}
1.
对于
\begin{equation}
    S=\int{\mathrm{d}^4x}\mathcal{L} 
\end{equation}
两边同时变分
\begin{equation}
    \delta S=\int{\mathrm{d}^4x}\delta \mathcal{L} 
\end{equation}
2.对于拉氏量
\begin{equation}
    \mathcal{L} =\mathcal{L} (\Phi _a,\partial _{\mu}\Phi _a)
\end{equation}
变分计算为
\begin{equation}
    \delta \mathcal{L} =\frac{\partial \mathcal{L}}{\partial \Phi _a}\delta \Phi _a+\frac{\partial \mathcal{L}}{\partial \left( \partial _{\mu}\Phi _a \right)}\delta \left( \partial _{\mu}\Phi _a \right) 
\end{equation}
得到
\begin{equation}
    \delta S=\int{\mathrm{d}^4x\left[ \frac{\partial \mathcal{L}}{\partial \Phi _a}\delta \Phi _a+\frac{\partial \mathcal{L}}{\partial \left( \partial _{\mu}\Phi _a \right)}\delta \left( \partial _{\mu}\Phi _a \right) \right]}
\end{equation}
3.因为可交换
\begin{equation}
    \delta \left( \partial _{\mu}\Phi _a \right) =\partial _{\mu}\left( \delta \Phi _a \right) 
\end{equation}
得到
\begin{equation}
    \delta S=\int{\mathrm{d}^4x}\left[ \frac{\partial \mathcal{L}}{\partial \Phi _a}\delta \Phi _a+\frac{\partial \mathcal{L}}{\partial \left( \partial _{\mu}\Phi _a \right)}\partial _{\mu}\left( \delta \Phi _a \right) \right] 
\end{equation}
4.求导法则
\begin{equation}
    \partial _{\mu}\left( \frac{\partial \mathcal{L}}{\partial \left( \partial _{\mu}\Phi _a \right)}\delta \Phi _a \right) =\frac{\partial \mathcal{L}}{\partial \left( \partial _{\mu}\Phi _a \right)}\partial _{\mu}\left( \delta \Phi _a \right) +\left( \partial _{\mu}\frac{\partial \mathcal{L}}{\partial \left( \partial _{\mu}\Phi _a \right)} \right) \delta \Phi _a
\end{equation}
写出
\begin{equation}
    \frac{\partial \mathcal{L}}{\partial \left( \partial _{\mu}\Phi _a \right)}\partial _{\mu}\left( \delta \Phi _a \right) =\partial _{\mu}\left( \frac{\partial \mathcal{L}}{\partial \left( \partial _{\mu}\Phi _a \right)}\delta \Phi _a \right) -\left( \partial _{\mu}\frac{\partial \mathcal{L}}{\partial \left( \partial _{\mu}\Phi _a \right)} \right) \delta \Phi _a
\end{equation}
得到
\begin{equation}
    \begin{aligned}
        \delta S&=\int{\mathrm{d}^4x}\left[ \frac{\partial \mathcal{L}}{\partial \Phi _a}\delta \Phi _a+\partial _{\mu}\left( \frac{\partial \mathcal{L}}{\partial \left( \partial _{\mu}\Phi _a \right)}\delta \Phi _a \right) -\left( \partial _{\mu}\frac{\partial \mathcal{L}}{\partial \left( \partial _{\mu}\Phi _a \right)} \right) \delta \Phi _a \right] 
\\
&=\int{\mathrm{d}^4x}\left[ \frac{\partial \mathcal{L}}{\partial \Phi _a}-\partial _{\mu}\frac{\partial \mathcal{L}}{\partial \left( \partial _{\mu}\Phi _a \right)} \right] \delta \Phi _a+\int{\mathrm{d}^4x}\partial _{\mu}\left[ \frac{\partial \mathcal{L}}{\partial \left( \partial _{\mu}\Phi _a \right)}\delta \Phi _a \right] 
    \end{aligned}
\end{equation}
5.广义斯托克斯,将四维体积转化为三维面积
\begin{equation}
    \int_{\mathcal{V}}{\mathrm{d}^4x}\partial _{\mu}\left[ \frac{\partial \mathcal{L}}{\partial \left( \partial _{\mu}\Phi _a \right)}\delta \Phi _a \right] =\int_{\mathcal{S}}{\mathrm{d}\sigma _{\mu}}\frac{\partial \mathcal{L}}{\partial \left( \partial _{\mu}\Phi _a \right)}\delta \Phi _a
\end{equation}
得到
\begin{equation}
    \delta S=\int{\mathrm{d}^4x}\left[ \frac{\partial \mathcal{L}}{\partial \Phi _a}-\partial _{\mu}\frac{\partial \mathcal{L}}{\partial \left( \partial _{\mu}\Phi _a \right)} \right] \delta \Phi _a+\int_{\mathcal{S}}{\mathrm{d}\sigma _{\mu}}\frac{\partial \mathcal{L}}{\partial \left( \partial _{\mu}\Phi _a \right)}\delta \Phi _a
\end{equation}
6.全空间积分为0
\begin{equation}
    \int_{\mathcal{S}}{\mathrm{d}\sigma _{\mu}}\frac{\partial \mathcal{L}}{\partial \left( \partial _{\mu}\Phi _a \right)}\delta \Phi _a=0
\end{equation}
得到
\begin{equation}
    \delta S=\int{\mathrm{d}^4x}\left[ \frac{\partial \mathcal{L}}{\partial \Phi _a}-\partial _{\mu}\frac{\partial \mathcal{L}}{\partial \left( \partial _{\mu}\Phi _a \right)} \right] \delta \Phi _a
\end{equation}
7.利用最小作用量原理
\begin{equation}
    \delta S=0
\end{equation}
等式右边积分为零
\begin{equation}
    \int{\mathrm{d}^4x}\left[ \frac{\partial \mathcal{L}}{\partial \Phi _a}-\partial _{\mu}\frac{\partial \mathcal{L}}{\partial \left( \partial _{\mu}\Phi _a \right)} \right] \delta \Phi _a=0
\end{equation}
被积函数为零
\begin{equation}
    \frac{\partial \mathcal{L}}{\partial \Phi _a}-\partial _{\mu}\frac{\partial \mathcal{L}}{\partial \left( \partial _{\mu}\Phi _a \right)}=0
\end{equation}


第二种写法是分部积分



%%%%%%%%%%%%%%%%%%%%%%%%%%%%%%%%%%%%%%%%%%%%%%%%%%%%%5
\subsection{Lorentz 张量}





















总结:自然单位制下的麦克斯韦方程组
\begin{equation}
    \begin{aligned}
        \nabla \cdot \mathbf{E}&=\rho 
\\
\nabla \times \mathbf{B}&=\mathbf{J}+\frac{\partial \mathbf{E}}{\partial t}
\\
\nabla \cdot \mathbf{B}&=0
\\
\nabla \times \mathbf{E}&=-\frac{\partial \mathbf{B}}{\partial t}
    \end{aligned}
\end{equation}

Gauss 定律
\begin{equation}
    \nabla \cdot \mathbf{E}=\rho 
\end{equation}
Ampère 环路定律对应的Ampère-Maxwell 方程
\begin{equation}
    \nabla \times \mathbf{B}=\mathbf{J}+\frac{\partial \mathbf{E}}{\partial t}
\end{equation}
Gauss 磁定律
\begin{equation}
    \nabla \cdot \mathbf{B}=0
\end{equation}
Faraday 电磁感应定律对应的 MaxwellFaraday 方程
\begin{equation}
    \nabla \cdot \mathbf{B}=0
\end{equation}

补充:推导
1.1推导
\begin{equation}
    \nabla \cdot \mathbf{E}=\rho \Rightarrow \partial _{\mu}F^{\mu 0}=J^0
\end{equation}
过程
\begin{equation}
    \begin{aligned}
        \nabla \cdot \mathbf{E}&=\rho 
\\
\partial _iE^i&=J^0
\\
-\partial _iF^{0i}&=J^0
\\
\partial _iF^{i0}&=J^0
\\
\partial _0F^{00}+\partial _iF^{i0}&=J^0
\\
\partial _{\mu}F^{\mu 0}&=J^0
    \end{aligned}
\end{equation}
其中






2.1推导
\begin{equation}
    \nabla \times \mathbf{B}=\mathbf{J}+\frac{\partial \mathbf{E}}{\partial t}\Rightarrow \partial _{\mu}F^{\mu i}=J^i
\end{equation}
过程
\begin{equation}
    \begin{aligned}
        \nabla \times \mathbf{B}&=\mathbf{J}+\frac{\partial \mathbf{E}}{\partial t}
\\
\varepsilon ^{ijk}\partial _jB^k&=J^i+\partial _0E^i
\\
\partial _j\varepsilon ^{ijk}B^k&=J^i+\partial _0E^i
\\
-\partial _jF^{ij}&=J^i-\partial _0F^{0i}
\\
\partial _0F^{0i}-\partial _jF^{ij}&=J^i
\\
\partial _0F^{0i}+\partial _jF^{ji}&=J^i
\\
\partial _{\mu}F^{\mu i}&=J^i
$$

    \end{aligned}
\end{equation}




3.1推导
\begin{equation}
    
\end{equation}
过程
\begin{equation}
    \begin{aligned}
        \nabla \times \mathbf{E}&=-\frac{\partial \mathbf{B}}{\partial t}
\\
\varepsilon ^{kmn}\partial _mE^n&=-\partial _0B^k
\\
-\varepsilon ^{kmn}\partial _mF^{0n}&=\frac{1}{2}\varepsilon ^{kmn}\partial _0F^{mn}
\\
\varepsilon ^{kmn}\partial _mF^{0n}+\frac{1}{2}\varepsilon ^{kmn}\partial _0F^{mn}&=0
\\
\varepsilon ^{kij}\varepsilon ^{kmn}\left( \partial _mF^{0n}+\frac{1}{2}\partial _0F^{mn} \right) &=0
\\
\left( \delta ^{im}\delta ^{jn}-\delta ^{in}\delta ^{jm} \right) \left( \partial _mF^{0n}+\frac{1}{2}\partial _0F^{mn} \right) &=0
\\
\delta ^{im}\delta ^{jn}\partial _mF^{0n}-\delta ^{in}\delta ^{jm}\partial _mF^{0n}+\frac{1}{2}\left( \delta ^{im}\delta ^{jn}\partial _0F^{mn}-\delta ^{in}\delta ^{jm}\partial _0F^{mn} \right) &=0
\\
\partial _iF^{0j}-\partial _jF^{0i}+\frac{1}{2}\left( \partial _0F^{ij}-\partial _0F^{ji} \right) &=0
\\
-\partial _iF^{j0}-\partial _jF^{0i}+\frac{1}{2}\left( \partial _0F^{ij}+\partial _0F^{ij} \right) &=0
\\
\partial _0F^{ij}-\partial _iF^{j0}-\partial _jF^{0i}&=0
\\
\partial ^0F^{ij}+\partial ^iF^{j0}+\partial ^jF^{0i}&=0
    \end{aligned}
\end{equation}







4.1推导
\begin{equation}
    \nabla \times \mathbf{E}=-\frac{\partial \mathbf{B}}{\partial t}\Rightarrow \partial ^0F^{ij}+\partial ^iF^{j0}+\partial ^jF^{0i}=0
\end{equation}

\begin{equation}
    \begin{aligned}
        
    \end{aligned}
\end{equation}



1.2推导
\begin{equation}
    \partial _{\mu}F^{\mu 0}=J^0\Rightarrow \nabla \cdot \mathbf{E}=\rho 
\end{equation}
过程
\begin{equation}
    \begin{aligned}
        \partial _{\mu}F^{\mu 0}&=J^0
\\
\,\partial _0F^{00}+\partial _iF^{i0}&=J^0
\\
\partial _iF^{i0}&=J^0
\\
-\partial _iF^{0i}&=J^0
\\
\partial _iE^i&=J^0
\\
\nabla \cdot \mathbf{E}&=\rho 
    \end{aligned}
\end{equation}
其中
\begin{equation}
    \begin{aligned}
        F^{00}&=0
\\
\partial _0F^{00}&=0
    \end{aligned}
\end{equation}




2.2推导
\begin{equation}
    \partial _{\mu}F^{\mu i}=J^i\Rightarrow \nabla \times \mathbf{B}=\mathbf{J}+\frac{\partial \mathbf{E}}{\partial t}
\end{equation}
过程
\begin{equation}
    \begin{aligned}
        \partial _{\mu}F^{\mu i}&=J^i
\\
\partial _0F^{0i}+\partial _jF^{ji}&=J^i
\\
-\partial _0E^i-\partial _j\varepsilon ^{jik}B^k&=J^i
\\
-\partial _0E^i-\varepsilon ^{jik}\partial _jB^k&=J^i
\\
-\partial _0E^i+\varepsilon ^{ijk}\partial _jB^k&=J^i
\\
\varepsilon ^{ijk}\partial _jB^k&=J^i+\partial _0E^i
\\
\nabla \times \mathbf{B}&=\mathbf{J}+\frac{\partial \mathbf{E}}{\partial t}
    \end{aligned}
\end{equation}




3.2推导
\begin{equation}
    \partial ^iF^{jk}+\partial ^jF^{ki}+\partial ^kF^{ij}=0\Rightarrow \nabla \cdot \mathbf{B}=0
\end{equation}
过程
\begin{equation}
    \begin{aligned}
        \partial ^iF^{jk}+\partial ^jF^{ki}+\partial ^kF^{ij}&=0
\\
-\partial _iF^{jk}-\partial _jF^{ki}-\partial _kF^{ij}&=0
\\
\partial _iF^{jk}+\partial _jF^{ki}+\partial _kF^{ij}&=0
\\
\partial _i\varepsilon ^{jkl}B^l+\partial _j\varepsilon ^{kil}B^l+\partial _k\varepsilon ^{ijl}B^l&=0
\\
\varepsilon ^{jkl}\partial _iB^l+\varepsilon ^{kil}\partial _jB^l+\varepsilon ^{ijl}\partial _kB^l&=0
\\
\varepsilon ^{ijk}\left( \varepsilon ^{jkl}\partial _iB^l+\varepsilon ^{kil}\partial _jB^l+\varepsilon ^{ijl}\partial _kB^l \right) &=0
\\
\varepsilon ^{jki}\varepsilon ^{jkl}\partial _iB^l+\varepsilon ^{kij}\varepsilon ^{kil}\partial _jB^l+\varepsilon ^{ijk}\varepsilon ^{ijl}\partial _kB^l&=0
\\
\delta ^{il}\partial _iB^l+\delta ^{jl}\partial _jB^l+\delta ^{kl}\partial _kB^l&=0
\\
3\partial _lB^l&=0
\\
\partial _iB^i&=0
\\
\nabla \cdot \mathbf{B}&=0
    \end{aligned}
\end{equation}







4.2推导
\begin{equation}
    \partial ^0F^{ij}+\partial ^iF^{j0}+\partial ^jF^{0i}=0\Rightarrow \nabla \times \mathbf{E}=-\frac{\partial \mathbf{B}}{\partial t}
\end{equation}
过程
\begin{equation}
    \begin{aligned}
        
    \end{aligned}
\end{equation}








%%%%%%%%%%%%%%%%%%%%%%%%%%%%%%%%%%%%%%%%%%%%%%%%%%%%%5
\subsection{}

















